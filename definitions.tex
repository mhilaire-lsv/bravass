
\section{Definitions}\label{section definitions}



In this section, we introduce general notations and preliminary definitions.

The model we are interested in is (S)WSTS (and later some particular instances, i.e. Timed/Counter Automata for instance).


% \textcolor{red}{Before defining WSTS, need a definition of TS and WQO}


\subsection{Transition systems}


\begin{definition}
A {\em labeled transition system} (LTS for short) is a tuple $T=(S, \Lambda, \rightarrow)$ where 
$S$ is a set of {\em configurations}, $ \Lambda$ is a set of {\em labels}, and 
${\rightarrow} \subseteq S\times \Lambda \times S$ is a 
ternary relation,
denoted as the set of {\em labeled transitions}. 
\end{definition}

We
 prefer to use infix notation and $(s,a ,s')\in {\rightarrow} $ will be abbreviated as
       $s  \xrightarrow{a}  s'$
to represent a transition from configuration $s$ to configuration $s'$ with label $a$. \\

\noindent
Labels can be used to represent the reading of an input, but also to represent an action performed during the transition or conditions that must hold in order to allow the use of the transition.


A {\em path} in a labeled transition system from a {\em source configuration} $s_0$
to a {\em target configuration} $s_n$ is a sequence 
$\pi = s_0 \xrightarrow{a_0 } s_1 \xrightarrow{a_1 } \cdots \xrightarrow{a_{n-1} } s_n$. 
We define the {\em concatenation} $ \pi_1 \pi_2$ of 
two paths $\pi_1$ and $\pi_2$ when the source configuration of $\pi_2$ is equal to the target configuration of $\pi_1$
as expected.
The {\em length} of 
$\pi = s_0 \xrightarrow{a_0 } s_1 \xrightarrow{a_1 } \cdots \xrightarrow{a_{n-1} } s_n$
is defined as $|\pi|=n$. We say the path is {\em labeled} by $a_0 a_1 , \ldots a_{n-1}$.
For all $w \in \Lambda^*$, all $s,s' \in S$, we will write $s \xrightarrow{w } s'$ if there exists a path from $s$ to $s'$ labeled by $w$. 

An {\em infinite path} is an infinite sequence
 $\pi = s_0 \xrightarrow{a_0 } s_1 \xrightarrow{a_1 } \cdots $.
For each infinite (resp. finite) path  $\pi = s_0 \xrightarrow{a_0 } s_1 \xrightarrow{a_1 } \cdots$ 
(resp. $\pi = s_0 \xrightarrow{a_0 } s_1 \xrightarrow{a_1 } \cdots \xrightarrow{a_{n-1}} s_n$)
and $i,j \in \N$ (resp. $i,j \in [0,n]$) with $i<j$ we denote
by $\pi[i,j]$ the path 
$s_i  \xrightarrow{a_i } s_{i+1}  \xrightarrow{a_{i+1} } \cdots  \xrightarrow{a_{j-1} } s_j$
 and by $\pi[i]$ the configuration $s_i$.
As expected, a {\em prefix} of a finite or infinite path $\pi$ is a finite path of the form $\pi[0,j]$, and
a  {\em suffix} of a finite path $\pi$ is a path of the form $\pi[i,n]$.  \\
Given an infinite path $\pi = s_0 \xrightarrow{a_0 } s_1 \xrightarrow{a_1 } \cdots$ let
$\text{\textit{Inf}}(\pi) = \{ s \in S \mid \forall i ~ \exists j > i ~ s_j = s\}$.

\newcommand{\pred}{\textsf{pred}}
\newcommand{\post}{\textsf{post}}
% \renewcommand{\succ}{\textsf{Succ}}


\noindent
The set of {\em successors} of a configuration $s \in S$ is defined as
$\post(s) = \{ s' \in S \mid \exists a \in \Lambda ~ s \xrightarrow{a} s'\}$.
A configuration without successors is called a {\em dead end}.  \\
The set of {\em predecessor} of a configuration $s \in S$ is defined as
$\pred(s) = \{ s' \in S \mid \exists a \in \Lambda ~ s' \xrightarrow{a} s\}$.



A labeled transition system $(S, \Lambda, \rightarrow)$ is {\em deterministic} if for all configurations $s_1, s_2, s_3 \in S$ and all
$a \in \Lambda$,
 $ s_1  \xrightarrow{a} s_{2}$ and  
 $s_1  \xrightarrow{a} s_{3} $ implies $s_2 = s_3 $. \\



\begin{definition}
An {\em (unlabeled) transition system} is a pair $T = (S,\rightarrow )$ where $S$ is a set of 
{\em configurations} and  
$ {\rightarrow} \subseteq S \times S$ is a
binary relation 
on
the set of configurations, denoted as the set of {\em transitions}. 
\end{definition}

We again prefer to use infix notation and write $s \rightarrow s'$ to denote a {\em transition} from configuration $s$ to configuration $s'$ (i.e., $ (s,s') \in  {\rightarrow} $). \newline
Note 
 that an unlabeled transition system can be seen as a labeled transition system where the set of labels consists of only one element. 
Determinism, (infinite) paths, their length, and concatenation in unlabeled transition systems are
then defined as expected.

\textcolor{red}{Thinking about whether or not it is pertinent to have LTS and not only TS. LTS can be usefull for TA because of the use of the guards/time as labels but it may be unecessary.}

We write $\rightarrow^{k}$, $\rightarrow^{+}$, $\rightarrow^{=}$, $\rightarrow^{*}$
for the $k$-step iteration of $\rightarrow$, its transitive closure, its reflexive closure, its reflexive and transitive closure). We use similar notation for $\post$ and $\pred$...

\textcolor{red}{This makes sense for TS but not so much for LTS ...}

A transition system is {\em finitely branching} if all $\post(s)$ are finite. 
\textcolor{blue}{We restrict our attention to finitely branching TSs.}

\textcolor{blue}{Alain: the forward coverability algorithm for infinitely branching TSs.; the backward cov algo may work for essentially finitely branching TSs.}
\textcolor{red}{Not sure that TS induced by TA are finitely branching. Actually I believe they are not, i.e. for instance for a TA with one clock $x$, from a state $q$ and clock $x$ set at $0$, if there is a transition e.g. $(q, x \geq 3, \emptyset, q')$ then the set of successors of 
$(q,0)$ is $\{q'\} \times \{3, 4, 5, \ldots \}$. Need to check where finitely branching appears as an assumption/requirement.}


\subsection{Well-quasi-orderings}

A {\em quasi-ordering} (a qo) is any reflexive and transitive relation $\leq$.

We abbreviate $x \leq y \not\leq x$ by $x < y$.

% A partial ordering (a po) is an antisymmetric qo

Any qo induces an equivalence relation ($x \equiv y$ iff $x \leq y \leq x$).
% and gives rise to a po between the equivalence classes ?
% 	do we need po ?

We now recall a few results from the theory of well-orderings (add reference [...]).


\begin{definition}
 A {\em well-quasi-ordering} (a wqo) is any quasi-ordering $\leq$ (over some set $X$ ) such that, for any infinite sequence $x_0, x_1, x_2, ...$ in $X$, there exist indexes $i \leq j$ with
$x_i \leq  x_j$.
\end{definition}

Notice that a wqo is well-founded, i.e. it admits no infinite strictly decreasing sequence
$x_0 > x_1 > x_2 > \cdots$

% \textcolor{red}{Add lemma about infinite increasing subsequences ?}

\begin{lemma}
(Erd\"os and Rado). Assume $\leq$ is a wqo. Then any infinite sequence contains an infinite increasing subsequence: $x_{i_0} \leq x_{i_1} \leq x_{i_2} \cdots$ (with $i_0 < i_1 < i_2 \cdots$).
\end{lemma}

\iffalse
\begin{proof}
Consider an infinite sequence and the set $M = \{i \in N \mid \forall j > i ~ x_i \not\leq x_j \}$. $M$ cannot
be infinite, otherwise it would lead to an infinite subsequence contradicting the wqo
hypothesis. Thus, $M$ is bounded and any $x_i$ with $i$ beyond $M$ can start an infinite
increasing subsequence.
\end{proof}
\fi


Given $\leq$ a quasi-ordering over some set $X$, an {\em upward-closed set} is any set $I \subseteq X$ such that if $y \geq x$ and $x \in I$ then $y \in I $.
A {\em downward-closed set} is any set $I \subseteq X$ such that if $y \leq x$ and $x \in I$ then $y \in I $. 
To any $A \subseteq X$ we associate
the {\em upward-closure of $A$} 
 $\uparrow A = \{x \in X \mid \exists a \in A ~ y \geq a\}$
 and the 
 {\em downward-closure of $A$} 
 $\downarrow A = \{x \in X \mid \exists a \in A ~ y \leq a\}$. 
We abbreviate $\uparrow \{x\}$ (resp. $\downarrow \{x\}$)
as $\uparrow x$ (resp. $\downarrow x$).


A {\em basis} of an upward-closed set $I$ is a set $I_b$ such that $I = \bigcup_{x \in I_b} \uparrow x$. 
% Higman investigated ordered sets with the finite basis property.


\begin{lemma}{(Higman [40])} 
If $\leq$ is a wqo then any upward-closed $I$ has a
finite basis.
\end{lemma}

% \textcolor{red}{Expliquer ce que c'est une base d'abords.}

\begin{proof}
The set of minimal elements of $I$ is a basis because $\leq$ is well-founded. It
only contains a finite number of non-equivalent elements otherwise they would make
an infinite sequence contradicting the wqo assumption.
\end{proof}



\begin{lemma}  \label{upward-closed stablizes}
If $\leq$ is a wqo then any infinite increasing sequence $I_0 \subseteq I_1 \subseteq I_2 \subseteq \cdots$ of
upward-closed sets eventually stabilizes, i.e. there is a $k \in N$ such that 
$I_k = I_{k+1} = I_{k+2} = \cdots $.
\end{lemma}

\begin{proof}
Assue we have a counter-example.
We extract an infinite subsequence where
inclusion is strict: $I_{n_0} \subsetneq I_{n_1} \subsetneq I_{n_2} \cdots$. Now, for any $i>0$, we can pick some $x_i \in I_{n_i} \setminus I_{n_{i-1}}$. The well-quasi-ordering hypothesis means that the infinite sequence of $x_i$'s
contains an increasing pair $x_i \leq x_j$ for some $i<j$. Because $x_i$ belongs to an upward-
closed set $I_{n_i}$ we have $x_j \in I_{n_i} $, contradicting $x_j \not\in I_{n_{ j - 1}}$.
\end{proof}


% \textcolor{red}{Define WSTS}

% \textcolor{red}{Define SWSTS -  necessary}

\subsection{Well-structured transition systems}


\begin{definition}\cite{DBLP:journals/tcs/FinkelS01}
A {\em (resp. strongly) well-structured transition system} (abbreviated as WSTS)  $(S, \rightarrow, \leq)$
is a transition system $(S, \rightarrow)$
equipped with a wqo ${\leq} \subseteq S \times S$ between states such that  
%  decidable ≤  wqo on S, i.e., for each two given states s, s ′ ∈ S, it is decidable whether s ≤ s ′ . 
the transition relation $ \rightarrow$ is compatible (resp. strongly compatible) with the wqo, i.e., for all 
$s_1, t_1 , s_2 \in S$
	with $s_1 \leq s_2$  and $s_1 \rightarrow t_1$ , there exists 
	$t_2 \in S$ with 
	$t_1 \leq t_2$ and $s_2 \rightarrow^{*} t_2$ 			
				(resp. $s_2 \rightarrow t_2$ ).
\end{definition}

Several families of formal models of processes give rise to WSTSs in a natural way, e.g. Petri nets when inclusion between markings is used as the well-ordering.


% \textcolor{red}{Define 'has effective pred-basis'. Maybe it should be included in WSTS definition, maybe it can be another def. I kind of like the idea of 'effective pred-basis' and 'decidable $\leq$' being independant from the WSTS definition}

\begin{proposition}\cite{DBLP:journals/tcs/FinkelS01}
If $S$ is an WSTS and $I \subseteq S$ is an upward-closed set of states, then $\pred^*(I )$ is upward-
closed.
\end{proposition}

\iffalse
Proof. Assume $s \in \pred^* (I )$. Then $s \rightarrow^* t$ for some $t \in I $. If now $s' \geq s$ then upward-compatibility entails that $s' \rightarrow^* t'$ for some $t' \geq t$. Then $t' \in I$ and $s' \in \pred^*(I )$.
\fi

\begin{proposition}\cite{DBLP:journals/tcs/FinkelS01}
If $S$ is a WSTS with strong monotony and $I \subseteq S$ is an upward-closed set of states, then $\pred(I )$ is upward-closed.
\end{proposition}
\alain{étrange ces deux propositions, l'une suffirait}
\iffalse
Proof. Assume $s \in \pred (I )$. Then $s \rightarrow t$ for some $t \in I $. If now $s' \geq s$ then strong upward-compatibility entails that $s' \rightarrow t'$ for some $t' \geq t$. Then $t' \in I$ and $s' \in \pred(I )$.
\fi
\mathieu{On peut probablement enlever la Proposition $8$ alors. La proposition $9$ étant plus directement pertinente pour les lemmes/preuves (permet d'avoir itérativement $\pred^k(I)$ upward-closed if $I$ upward-closed - for all $k$).}



\begin{definition}\cite{DBLP:journals/iandc/AbdullaCJT00}
A WSTS has {\em effective pred-basis} if there exists an algorithm accepting
any state $s \in S$ and returning $pb(s)$, a finite basis of $\uparrow \pred(\uparrow s)$.
\end{definition}

% \textcolor{red}{Define what an Ideal is. Actually an Ideal is just an upward-closed set, so maybe this just adds some confusion. Anti-ideal just downward closed so again just not that helpful a notation. Maybe have a }


\iffalse
\begin{definition}
A {\em bi-ideal} $I \subseteq S$ is an upward-closed and downward-closed set, i.e
$\uparrow I = I = \downarrow I$.
\end{definition}

"Bi-ideals often represent “control states” as in [cf \%]. "
% Parosh Aziz Abdulla, Karlis Cerans, Bengt Jonsson & Yih-Kuen Tsay (1996): General Decidability Theorems for Infinite-State Systems. In: Proc. LICS 1996, IEEE Computer Society Press, pp. 313–321,

% \textcolor{red}{Probably one can already 'deduce' from this that ideal $I$ and anti-ideal $J$ for resp. good and bad states, in the case of e.g. timed automata would be given by sets of states}
\fi

A downard-closed set $J$ is {\em decidable} if, given $s \in S$, it is decidable whether
$s \in J$. 
\textcolor{red}{Since a downward-closed set does not have an ``upward-basis'' in general, we will demand that membership is decidable.}
\alain{remarque pas claire}
% \textcolor{red}{Do we still demand this ?}
\mathieu{On peut enlever ce point de discussion ici et en discuter de manière plus complête au moment où on discute de SAFE et BAD plus en détail je pense.}

\begin{claim}{(stability of ideals)}
Let $I, J \subseteq S$ be upward-closed. Then the sets $I \cup J$, and $I \cap J$ are upward-closed.
\end{claim}


% \begin{claim}
% Given a finite set $A \subseteq S$ with $I =\uparrow A$, we can compute a finite basis $B$ of $I$.
% \end{claim}

\begin{fact}\label{fact basis}
% Fact 3 ([1]). 
% Parosh Aziz Abdulla, Karlis Cerans, Bengt Jonsson & Yih-Kuen Tsay (1996): General Decidability Theorems for Infinite-State Systems. In: Proc. LICS 1996, IEEE Computer Society Press, pp. 313–321,
For every upward-closed set $I \subseteq S$, there exists a finite basis $B$ of $I$. 
% (ii) Given a finite set $A \subseteq S$ with $I =\uparrow A$, we can compute a finite basis $B$ of $I$.
% \alain{(ii) est bizarre car A est déjà une base finie de $\uparrow A$, et $min(A)$ est la base minimale}
\end{fact}


\begin{definition}{ (index)}. 
If $S$ is a WSTS with strong monotony and $I \subseteq S$  is an upward-closed set and $k \geq 0$, let $I^k= \bigcup_{0 \leq j \leq k} \pred^j(I)$.
The {\em index} $k(I)$ is the
smallest $k_0$ s.t. $I^k = I^{k_0}$ for all $k \geq k_0$.
\end{definition}


If $S$ is a WSTS with strong monotony and $I \subseteq S$ is an upward-closed set, the sets $\pred(I )$, and $\pred^{\geq k}(I )$ for
every $k \geq 0$ are upward-closed, thus Lemma~\ref{upward-closed stablizes} ensures the existence of $k(I)$.

\begin{remark}
$I^{k+1} $ can be rewritten $I^{k+1}= \bigcup_{0 \leq j \leq k+1} \pred^j(I) = 
I \cup \bigcup_{1 \leq j \leq k+1} \pred^j(I) =
I \cup \pred(\bigcup_{0 \leq j \leq k} \pred^j(I))
=  I \cup \pred(I^k)$.
\end{remark}

This ensures the following.

\begin{fact}\label{stop condition}
% Fact 4 (stop condition). 
If $S$ is a WSTS with strong monotony and $I \subseteq S$ is an upward-closed set and $k \geq 0$ s.t. $I^k = I^{k+1}$ , then $I^\ell = I^k$ for all $\ell \geq k$, i.e.,
$k(I) \leq k$. This also implies that $\pred^*(I) = I^k$.
\end{fact}



\begin{lemma}
% Lemma 3 ([1]) 
% Parosh Aziz Abdulla, Karlis Cerans, Bengt Jonsson & Yih-Kuen Tsay (1996): General Decidability Theorems for Infinite-State Systems. In: Proc. LICS 1996, IEEE Computer Society Press, pp. 313–321,
 Given a basis of an upward-closed set $I \subseteq S$, and a state $s$ of an effective strongly well-structured transition
system, we can decide whether we can reach $I$ from $s$.
\end{lemma}

\begin{proof}
We have to show that we can compute a basis of $I^{k+1}$ if we are given a basis of $I^k $. 
Then the
decidability of the stop condition follows directly. Let $B$ be a basis of $I^k$. 
We have
$$I^{k+1} = I \cup \pred(I^k ) = I \cup
\bigcup_{s' \in B}
\pred(\uparrow \{s' \}).$$
\alain{n'est-ce pas $I^{k+1} = I^k \cup \pred(I^k ) = I^k \cup
\bigcup_{s' \in B}
\pred(\uparrow \{s' \}).$ ?}
\mathieu{À strictement parler, c'est  $I^{k+1} = \bigcup_{0 \leq j \leq k+1} \pred^j(I) = 
I \cup \bigcup_{1 \leq j \leq k+1} \pred^j(I) =
I \cup \pred(\bigcup_{0 \leq j \leq k} \pred^j(I))
=  I \cup \pred(I^k)$. Je vais rajouter une remarque au dessus.}

Since $\pred(\uparrow \{s' \}$) is computable for any $s'\in S$ by definition, we obtain a finite generating set of $I^{k+1}$ . By
Fact~\ref{fact basis}, we can compute a basis of $I^{k+1}$.
\end{proof}

%%%
\iffalse
%%%%
%%%%
\subsection{Defining resilience}


\subsubsection{Resilience for general transition systems}

% \textcolor{red}{Ask the question why use a set of propositions for $SAFE$ and $BAD$ rather than use subsets of the set of configurations ? }

We ask whether we can reach a state 
%which satisfies
in 
%
$SAFE$  in a reasonable amount of time whenever we reach a state 
% which satisfies
in
%
$BAD$. 
From this we formulate two resilience problems. First consider the case where the recovery time
is bound by a given natural number $k \geq 0$, i.e., the \emph{explicit resilience problem} for TS.

\problemx{$k$-Resilience problem }
{a transition system $(S,\rightarrow)$, an integer $k$, a state $s \in S$, $SAFE, BAD \subseteq S$ and $SAFE \cap BAD = \emptyset$.}
{$\forall s' \in BAD, ~ s \rightarrow^* s' \implies \exists s'' \in SAFE ~ s' \rightarrow^{\leq k} s''$ ?\newline}

If a system $S$ satisfies the explicit resilience property for an integer $k$, we say that $S$ is $k$-resilient.

% If we assume that the transition system yields infinite sequences of transitions, we can express the property to be evaluated in CTL by s |= AG(BAD → 0≤ j≤k EX j SAFE). 

We can also ask whether there exists such a bound $k$ such that $S$ is $k$-resilient. We call this problem the \emph{bounded resilience problem}.

%
%\problemx{Bounded Resilience problem}
%{a transition system $(S,\rightarrow)$, a state $s \in S$, $SAFE, BAD \subseteq S$ and $SAFE \cap BAD = \emptyset$.}
%{$\exists k \geq 0 ~ \forall s' \in BAD, ~ s \rightarrow^* s' \implies \exists s'' \in SAFE ~ s' \rightarrow^{\leq k} s''$ ?\newline}
%


\problemx{Bounded Resilience problem}
{a transition system $(S,\rightarrow)$, a state $s \in S$, $SAFE, BAD \subseteq S$ and $SAFE \cap BAD = \emptyset$.}
{$\exists k \geq 0  S$ is $k$-resilient. ?\newline}
%%
\fi
%%%


pour $k,k'$ donnés, chercher si $SAFE_{max}$ et $BAD_{max}$ ont un sens ? on peut faire l'union des SAFE, SAFE' et des BAD, BAD' prendre le max des k. Fixer k, chercher et calculer $SAFE_{max}$ et $BAD_{max}$.


\section{Resilience for WSTS}

The $k$-resilience problem is to decide whether from a bad state, is it possible to find a path of length smaller than a given $k$ to reach a safe state. Let us define two sets BAD and SAFE. From all states $b \in BAD$ does there exist a path of length smaller than $k$ that reaches a state $s \in SAFE$ ?

It can be formally stated as follows :

\problemx{uniform $k$-resilience problem}
{A transition system $(S,\rightarrow)$, two sets $SAFE, BAD \subseteq S$.}
{$BAD \longrightarrow^{\leq k} SAFE$ ?\newline}
%
\alain{il faudrait ne pas répéter 3 fois les mêmes imputs pour les 3 pbs: énoncer les 3 uniformes pbs d'un coup avec une fois l'input puis les 3 pbs pour un état s donné sans répéter non plus 3 fois les mêmes inputs}

For most of transition systems, this problem is decidable if both SAFE and BAD are finite. When one of the two sets is infinite, it generally becomes undecidable. So we restrict the framework to WSTS and sets SAFE and BAD will be upward or downward closed. In \cite{DBLP:conf/gg/Ozkan22} BAD is downward closed and SAFE is upward closed. But there are other possible cases: for instance, the well-known mutual exclusion property is often modelized in a $d$-counters machine by the property that a counter $c_{mutex}$ must be bounded by (usually) one. Then, the set $SAFE =  \{c_{mutex} \leq 1\} \times \mathbb{N}^{d-1}$ is downward closed and $BAD =\{c_{mutex} \geq 2\} \times  \mathbb{N}^{d-1} $ is the upward closed complementary of SAFE.
%		

Most decidable properties in well-structured transition systems are given as upward or downward closed sets \cite{DBLP:journals/iandc/AbdullaCJT00, DBLP:journals/tcs/FinkelS01}.
%  Parosh Aziz Abdulla, Karlis Cerans, Bengt Jonsson & Yih-Kuen Tsay (1996): General Decidability Theorems for Infinite-State Systems. In: Proc. LICS 1996, IEEE Computer Society Press, pp. 313–321,
%  Alain Finkel & Philippe Schnoebelen (2001): Well-structured transition systems everywhere! Theor. Comput. Sci. 256(1-2), pp. 63–92, doi:10.1016/S0304-3975(00)00102-X.
Transfering the abstract resilience problems into this framework,
it is therefore reasonable to demand that both propositions, SAFE and BAD, are given by 
upward-closed or downward-closed sets.

\begin{tabular}{ l c r }
   BAD/SAFE & Up & Down \\
   Up & dec ? & dec ? \\
   Down & dec & ? \\
 \end{tabular}

We first assume that the safety property is given by an upward-closed set and the bad condition by a decidable downward-closed set. 
\alain{très discutable. pourquoi pas supposer des conditions plus gnérales ? perd-on la décidabilité ?}
% \textcolor{red}{Seems like a reasonable assumption to me.}

%From these considerations, we formulate instances of the abstract resilience problems for well-
%structured transition systems.

%
%\problemx{(I,J)-$k$-resilience problem for WSTS}
%{A state $s$ of a WSTS $(S,\rightarrow, \leq)$, an effective set $I$ (with a given basis), a set $J$.}
%{$\forall s' \in J ~ (s \rightarrow^* s') \implies \exists s'' \in I ~ s' \rightarrow^{\leq k} s''$ ?\newline}
%
%\problemx{general $k$-resilience problem for WSTS}
%{A state $s$ of a WSTS $(S,\rightarrow, \leq)$, an effective set $I$ (with a given basis).}
%{$\forall s'  ~ (s \rightarrow^* s') \implies \exists s'' \in I ~ s' \rightarrow^{\leq k} s''$ ?\newline}

Again, we first consider the case where the recovery time is bounded by a $k \in \N$.

\problemx{$k$-resilience problem for WSTS}
{A state $s$ of a WSTS $(S,\rightarrow, \leq)$, an upward-closed set $U$ with a given basis, a decidable downward-closed set $D$.}
{$D \longrightarrow^{\leq k} U$ ?  ou $D \cap post^*(s) \longrightarrow^{\leq k} U$ ?  \newline}
%
%  		{$\forall s' ~ (s \rightarrow^* s') \wedge s' \in D  \implies \exists s'' \in U ~ s' \rightarrow^{\leq k} s''$ ?\newline}
%
$D \longrightarrow^{\leq k} U$: $\forall s' \in D \exists s'' \in U ~ s' \rightarrow^{\leq k} s''$ ?\\
%
$D \cap post^*(s) \longrightarrow^{\leq k} U$ ?
%
$\forall s' ~ (s \rightarrow^* s') \wedge s' \in D  \implies \exists s'' \in U ~ s' \rightarrow^{\leq k} s''$ ?
 \alain{ça signifie pour tout $d \in D$ il existe un chemin de longueur $\leq k$ qui arrive dans un $u \in U$. c'est plus simple et plus général et ça ne demande moins de décider l'accessibilité....comment faire pour vérifier pour tous les (D,U) ? sinon ça dépend tjs de l'état intial $s$ et s'il faut vérifier pour tous les $s$....}


%ou plus simple : $D \rightarrow^{\leq k} U$ ? 


%Analogously, we formulate the resilience problem for WSTSs.

\problemx{bounded Resilience problem for WSTS}
{A state $s$ of a WSTS $W=(S,\rightarrow, \leq)$, an upward-closed set $U$ with a given basis, a decidable downward-closed set $D$.}
%{$\exists k \geq 0 ~ \forall s' \in D ~ s \rightarrow^* s' \implies \exists s'' \in U ~ s' \rightarrow^{\leq k} s''$ ?\newline}
{$\exists k \geq 0$ such that $W$ is $k$-resilient. ?\newline}

\problemx{Resilience problem for WSTS}
{A state $s$ of a WSTS $(S,\rightarrow, \leq)$, an upward-closed set $U$ with a given basis, a decidable downward-closed set $D$.}
{$ ~ \forall s' \in D ~ (s \rightarrow^* s') \implies \exists s'' \in U ~ s' \rightarrow^{*} s''$ ?\newline}

%\textcolor{blue}{
%justification des definitions: si on enlève la condition  an upward-closed set $U$ (extended resilience $U$ qq) reachability reduces to resilience (avec $U=\{x\}$ donc indecidable pour les modèles à reachability indécidable.
%}

\begin{remark} 
We require $U$ to be upward-closed, as, in the general case if we lift restrictions on $U$ and $D$, then reachability reduces itself to resilience. Indeed, the particular resilience problem where we take $D = \{s\}$ and $U = \{x\}$ is reachability of $x$ from $s$, and thus generally undecidable for WSTS. 
\end{remark}

\textcolor{blue}{
et encore: renforcer sur l'hypothèses sur I, J tous les deux clos par le haut+bas.
}

\textcolor{blue}{
compute the sets of $(I,J,k)$ st S is resilient.\\
max des I,J possibles \\
}

\begin{remark} 
$k$-resilienc $\leq$ bounded Resilience $\leq$ Resilience.  \textcolor{red}{AVERIFIER !!}
\end{remark}

\alain{examiner les cas I,J un élement, cloture, tout, clos par le bas et par le haut,... ...}

\begin{proposition}\label{reductions}
coverability, $t$-liveness, home state 
%and reachability
 reduce to resilience.
\end{proposition}


\begin{proof}

\begin{itemize}
\item $J=S= \downarrow S$ et $I=\uparrow x$ then this particular resilience is coverability of $x$ from $s$.
%\item $J=\{s\}$ and $I=\{x\}$ this particular resilience is reachability of $x$ from $s$.

\item $t$-liveness  can be expressed as the following formula
$ ~ \forall s' \in S ~ (s \rightarrow^* s') \implies \exists s'' \in U_t ~ s' \rightarrow^{*} s''$ 
where
$U_t=\uparrow \pred(t)$
is the upward closure of the preconditions to use transition $t$.  
The problem reduces itself to the resilience problem
where $J=S$ and $I = \uparrow \pred(t)$.

\item the home state problem can be expressed as $\ldots$
\end{itemize}

%	\mathieu{Ne marche pas (un des deux sens est invalide imho)}
%  \textcolor{red}{  Let $J=\downarrow S = S$ and $I = \uparrow t$ where $t \in S$. Now resilience remains to the following property: $\forall s' \in S,  \exists s'' \in I, ~ s' \rightarrow^{*} s''$. Let us apply this property to LCS with $I = \uparrow (q,\epsilon) = (q, \Sigma^*)$. This property, $\forall s' \in S,  \exists w \in \Sigma^* $ such that $ ~ s' \rightarrow^{*} (q,w)$ expresses that there is an infinite run passing infinitely often through the control-state $q$; this LTL-property has been proved undecidable for Lossy Channel Systems (LCS) \cite{DBLP:conf/icalp/AbdullaJ94}.
% Since LCS are effective WSTS with strong monotony (with an adequate semantics), we conclude.}
%	\mathieu{Ne marche pas (un des deux sens est invalide imho)}
\end{proof}

Since Reset Petri nets are also effective WSTS with strong monotony, we could think to use the undecidability of LTL for Reset Petri nets.

and with effective basis of $post^*$ ???


% \alain{cette dernière propriété fait penser à la $t$-liveness dans un rdPetri ou n'importe quel modèle monotone..qui peut s'exprimer par $ ~ \forall s' \in S ~ (s \rightarrow^* s') \implies \exists s'' \in U_t ~ s' \rightarrow^{*} s''$ ? avec $U_t=\uparrow \pred(t)$=la clôture par le haut des préconditions pour déclencher la transition $t$}





expliquer toutes les variantes de la résilience

\alain{dire juste que SAFE = upward closed et BAD = downward closed.
envisager les 3 autres cas. souvent BAD = upward closed (par ex l'exclusion mutuelle).
4 resilience problem (U,D), (U,U), (D,U), (D,D)}
% \textcolor{red}{In the Özkan paper the input include a basis of $\uparrow post^*(s)$. Not $100$\% sure it is necessary, think we can try to do without this assumption in the input.}

Let $(X,\leq)$ be a wqo and $U$ be an upward closed subset of $X$. We say that $U$ is \emph{computable} if there is an algorithm that computes a finite basis of $U$.


\begin{theorem}\cite{DBLP:conf/gg/Ozkan22,DBLP:journals/corr/abs-2108-00889}
{\sc bounded resilience} is decidable for WSTS $S$ with strong upward compatibility and such that $\uparrow post^*(S)$ is computable.
\end{theorem}

We may immediately generalyse this last result by strenghthenin to \emph{unbounded} resilience. The proof is essentially the same than the previous one.

\begin{corollary}\label{postcomputable}
{\sc resilience} is decidable for WSTS with strong upward compatibility and such that $\uparrow post^*(S)$ is computable.
\end{corollary}


\begin{proof}
From Fact~\ref{stop condition}, there exists $k \in \N$ such that
$\post^*(I) = I^{k_m}$. We can compute this $k_m$ by iteratively computing
$I^{k+1}$ from $I^k$, checking $I^{k+1}=I^k$, 
returning $k_m$ if that is the case.
Then, because $k_m$-resilience is decidable, 
checking $\uparrow post^*(s) \cap J \subseteq I^k = \pred^*(I)$ is,
and resilience is decidable.
% \textcolor{red}{à faire Mathieu}
\end{proof}

The author in \cite{DBLP:conf/gg/Ozkan22}
%\"Ozkan
 goes on to prove that for finite-branching WSTS with strong monotonicity, a basis of $\uparrow \post^*(s)$ is computable for every given state $s$ iff the downward-closed problem is decidable, i.e. given a state $s$ of a WSTS
% in the regarded class 
with strong monotonicity 
and a decidable downward-closed set $J$, it can be decided whether $\exists s' \in J ~ s \to^* s'$ (Proposition~1 in \cite{DBLP:conf/gg/Ozkan22}). 
The characterization is used to show that Petri nets are $\post^*$-effective. It is well-known that Petri nets constitute WSTS with strong monotonicity (Thm.~6.1 in \cite{DBLP:journals/tcs/FinkelS01}).
However, the hypothesis that $\uparrow \post^*$ is computable cannot be tested in the general WSTS framework.

\begin{proposition}
There exist classes of WSTSs with strong upward compatibility for which there don't exist an algorithm computing a basis of $\uparrow \post^*$.
\end{proposition}


\begin{proof}
Let us show that Reset VASS and LCS don't enjoy the property that $\uparrow \post^*$ is computable.
Suppose that one are able to compute a finite basis of $\uparrow post^*$ for Reset VASS. Then, one would be able to decide whether $0$ is reachable in counter $i$ by examining if there is some vector $v$ in the basis such that $v(i)=0$. But reachability of $0$ in a counter $i$ is undecidable for Reset VASS. 
% sont WSTS with strong compatibility. et l'accessibilité est undecidable.
$\uparrow post^*$ n'est pas calculable pour les LCS non plus car $\uparrow post^*= \uparrow \downarrow post^*$ n'est pas calculable ???
\textcolor{red}{à relire par Mathieu}
\end{proof}

il existe des modèles où on peut calculer $\uparrow post^*$: inserted channel systems: on sait calculer $post^*$ qui est rationnel donc on sait calculer $\uparrow post^*$.

remarque si on enlève la strong monotony mais en gardant with effective basis of $post^*$:


\begin{proposition}(à prouver)
 leur algorithme for {\sc bounded resilience} termine mais pas correct for WSTS with upward compatibility, not necessarly strong, and an effective basis of $post^*$ ?
\end{proposition}

\begin{proof}
leur algorithme termine mais pas correct car l'intersection de post et J car alors le lemmme xxxx est faux (contre-exemple)
 \textcolor{red}{à faire Mathieu}
 
\mathieu{En train de chercher un contre exemple pour le Lemme~\ref{Lemma intersection}.} 
 
\end{proof}

\begin{theorem}(à prouver)
{\sc bounded resilience} is undecidable ??? for WSTS with upward compatibility, not necessarly strong, and an effective basis of $post^*$ ???
\end{theorem}

\begin{proof}
 \textcolor{red}{à faire Mathieu}
\end{proof}


si on garde strong mais on enlève effective basis of $post^*$:

\begin{proposition}(à prouver)
{\sc $k$-resilience}, hence {\sc bounded resilience}, is undecidable for effective WSTS with strong upward compatibility but without basis of post* ????
\end{proposition}

\begin{proof}
pas de effective basis of $post^*$. leur algorithme n'est plus un algo. ex reset PN ?  \textcolor{red}{à faire Mathieu}
\end{proof}


Unfortunately, resilience is undecidable for (general) WSTS even with strong monotony.

\begin{theorem}
{\sc resilience} is undecidable for WSTS with strong monotony.
\end{theorem}


\begin{proof} 

Since $t$-liveness is undecidable for reset Petri-Nets that are WSTS with strong monotony, from Proposition \ref{reductions}, we deduce that resilience is undecidable for WSTS with strong monotony.
  
\end{proof}


\begin{corollary}
{\sc Resilience} is undecidable for WSTS.
\end{corollary}

\begin{theorem}
In WSTS with strong monotony, 
the {\sc bounded resilience} problem
is equivalent to the
{\sc resilience} problem.
\end{theorem}

\begin{proof}
strong WSTS

$\pred^*(I) = \pred^{k}(I)$

\end{proof}

\begin{corollary}
Bounded Resilience is undecidable for WSTS with strong monotony.
\end{corollary}

\begin{proof} 
  From
  cf
the two  theorems above. 
\end{proof}



Let us deduce that effective basis of $post^*$ ex reset PN ?

Since VASS are WSTS with strong upward compatibility and since there is an algorithm that computes a finite basis of  $\uparrow post^*(S)$, \cite{DBLP:conf/gg/Ozkan22} deduced that bounded resilience is decidable for VASS, hence resilience is decidable for VASS. \alain{resilience decidable for LCS ? LCS sont-ils strong (quelle sémantique ?) et $\uparrow post^*(S)$ is computable ?}

We may consider context-free grammars.

\begin{corollary}
{\sc resilience} is decidable for finite automata and context-free grammars (seen as transition systems)
\end{corollary}

\begin{proof}

  \textcolor{red}{à faire Mathieu}
Context-free grammars are WSTS \cite{DBLP:journals/tcs/FinkelS01} \alain{monotony ne semble pas strong...}avec $\uparrow post^*$ calculable (because the reachability set of a pushdown machine is an effectively computable regular language) ???? resilience decidable.
\end{proof}


\cite{DBLP:journals/ipl/BouajjaniEFMRWW00}


