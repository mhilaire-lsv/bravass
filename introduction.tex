\section{Introduction}\label{section introduction}


{\bf Context.} In the quite informal paper \cite{DBLP:journals/corr/PrasadZ16}, Prasad and Zuck studied resilience of a process to an adversary and they used the framework of WSTS enjoying both upward and downward monotonies to prove the decidability of a kind of resilience.

{\bf MFCS 2023
Abstract submission deadline:    	 April 24th (AoE) 
Paper submission deadline:    	April 28th (AoE)} \\
%
{\bf CONCUR 2023 Abstract Registration Due 	Apr 24, 2023
Submission Deadline 	May 2, 2023}\\
%
{\bf FSTTCS 2033  Submission deadline: July 14, 2022 AoE (firm)
Notification to authors: September 16, 2022}

trop d'hypotheses trop fortes sur les WSTS

{\bf Our contributions}
We introduce a more general definition of resilience generalising the two given in \cite{DBLP:conf/gg/Ozkan22}.
Surprinsingly, the general undecidability statements were not known neither proved.

\begin{itemize}

\item safe clos haut, bad clos par le bas: resilience = reachability d'un clos par le bas (anti-coverability)

\item safe clos haut : resilience indecidable for WSTS with strong.

\end{itemize}


\begin{center}
	\begin{figure}
			\hspace{0.75cm}
				\label{air control}
\includegraphics[width=0.85\textwidth]{FigureB}
	\caption{A channel system with three automata and four channels. Leftmost and rightmost automata represents aircrafts attempting to land and staying on stand-by in the air while they haven't received permission to land, while the middle one represent the control tower managing the aircrafts landings so that only one aircraft at a time is allowed for landing.}
	\end{figure}
\end{center}

Consider the channel system in Figure~\ref{air control}. It models a scenario in which two aircrafts wants to land in an airport at the same time. Safety requires that only one aircraft at a time attempt to land. The role of the control tower is to ensure this;
there are two possible choices: aircraft A waits for aircraft B to land
or vice-versa. We want ...

% contrôle de trafic aérien, perturbation dans le planning → Bad (par exemple, trop d’avions qui cherchent à atterir en même temps et qui doivent être mis en stand-by au dessus de l’aéroport?) → Safe (trafic fluide à nouveau). K-résilience important car les avions peuvent pas rester indéfiniment en stand-by (re : le carburant est un réactif limitant) 



