





\documentclass{beamer}
%  \usepackage[utf8]{inputenc}
\usepackage[utf8]{inputenc}
  \usetheme{Warsaw}
  \usepackage{dsfont}
  \usepackage{xcolor}
%  \usepackage[dvipsnames]{xcolor}
  \usepackage{tabto}

\usepackage{mathtools}
\usepackage{todonotes}
\usepackage{microtype}

\usepackage{complexity}
\usepackage{amsmath}

\usepackage{stmaryrd}
\usepackage{dsfont}



\usepackage{mathrsfs}
\usepackage{mathalpha}
\usepackage{amsmath}
\usepackage{amsfonts}
\usepackage{multirow}

\usepackage{ textcomp } 

\usepackage{stmaryrd}
\usepackage{wrapfig}


\title[Resilience and Home-Space for WSTS]{Resilience and Home-Space for WSTS}

\author[Alain Finkel ~ Mathieu Hilaire]{Alain Finkel \inst{1} \and Mathieu Hilaire \inst{2}}
\institute[Alain Finkel ~ Mathieu Hilaire]{\inst{1} Université Paris-Saclay, CNRS, ENS Paris-Saclay, LMF, Gif-sur-Yvette, France, Institut Universitaire de France \and %
                      \inst{2} Université Lyon 1, LIRIS, France}

\date{}

\setbeamertemplate{page number in head/foot}[totalframenumber]



\usepackage{pgf}
\usepackage{tikz}
\usetikzlibrary{arrows,automata}
%\usepackage[latin1]{inputenc}

\bibliographystyle{alpha}% the recommnded bibstyle

\newcommand{\Z}{\mathbb{Z}}
\newcommand{\N}{\mathbb{N}}


\colorlet{Mycolor1}{green!10!orange!90!}
\colorlet{Mycolor3}{blue!20!magenta!90!}
\definecolor{Mycolor2}{HTML}{00F9DE}


\newcommand{\pred}{\textsf{pred}}
\newcommand{\post}{\textsf{post}}



\newcommand{\Bad}{\textsf{Bad}}
\newcommand{\Safe}{\textsf{Safe}}


\begin{document}


\maketitle


%vvvvvvvvvvvvvvvvvvvvvvvvvvvvvvvvvvvvvvvvvvvvvvvvvvvvvvvvvvvvvvvvvvvvvvvvvv
  \begin{frame}{Outline}

\tableofcontents
  \end{frame}
  \section{Introduction}
%vvvvvvvvvvvvvvvvvvvvvvvvvvvvvvvvvvvvvvvvvvvvvvvvvvvvvvvvvvvvvvvvvvvvvvvvvv
  \begin{frame}{Motivation}

\begin{columns}[T]
 \column{0.35\textwidth}
 \begin{figure}
 	\hspace{-0.5cm}
\includegraphics[width=1.\textwidth]{fridge}
% \caption{Hasse diagram of some classes of transition systems, together with the decidability (in green) or undecidability (in red) of the resilience problems. Decidability of bounded resilience and $k-$resilience variants are indicated in blue.}
\end{figure} 

 \column{0.65\textwidth}
\phantom{Fridge.png} \\
\phantom{Fridge.png} \\
{\sf Bad} state = temperature $\geq 6^\circ C$ \newline

 {\sf Good} state = temperature $\leq 6^\circ C$ \\

\end{columns}

\pause

\phantom{Fridge.png} \\
Other applications e.g.  nuclear reactor 

\iffalse
  \todo[inline,color=blue!20]{ {\it À l'oral~}
We are usually focused on how to avoid bad states in general but you can see when your fridge the system stops providing refrigerating temperature it's not over yet
If you leave your fridge off for a minute it's not going to have Terrible impact but if you leave it off for more than one hour you're gonna have to throw the things in it away

And I've been talking like your frige at home but it apply to any type of cooling system it can be like refrigerating medical supplies or nuclear reactor.

And this is interesting Bcz now the stuff is not necessarily avoid bad states completely

This lead to concept of resilience
}
\fi

  \end{frame}
%vvvvvvvvvvvvvvvvvvvvvvvvvvvvvvvvvvvvvvvvvvvvvvvvvvvvvvvvvvvvvvvvvvvvvvvvvv
  \begin{frame}{Example: Reset-VASS $V_1$}
  
  
   \begin{center}
 	\begin{figure}
 	\vspace{.06cm}
\includegraphics[width=.50\textwidth]{FigA}
	\end{figure}
\end{center}  

  \end{frame}
%vvvvvvvvvvvvvvvvvvvvvvvvvvvvvvvvvvvvvvvvvvvvvvvvvvvvvvvvvvvvvvvvvvvvvvvvvv
  \begin{frame}{Example: Reset-VASS $V_2$}
  
  
   \begin{center}
 	\begin{figure}
% 	\hspace{-2.cm}
\includegraphics[width=.50\textwidth]{FigB}
	\end{figure}
\end{center}  

  \end{frame}
%vvvvvvvvvvvvvvvvvvvvvvvvvvvvvvvvvvvvvvvvvvvvvvvvvvvvvvvvvvvvvvvvvvvvvvvvvv
  \begin{frame}{Resilience(s)}
  

% TODO

% Define the Six different problems we study/focus on.

% (Takes some room but I think mb having the six on the same page could be better than having three then three state-* on two separate pages ?)
  
 {\small
  
  
{\sc Resilience problems} 

\hspace{-0.5cm}  {\bf INPUT:\ }{A transition system $\mathscr{S}=(S,\rightarrow)$, and a set $\Safe \subseteq S$.}

\hspace{-0.5cm}  {\bf QUESTION:\ } 

\hspace{-0.5cm}  {({\sc resilience problem (RP)}) $\Bad \rightarrow^{*} \Safe$ ?

\hspace{-0.5cm}  ({\sc $k$-resilience problem (kRP)})		$\Bad \rightarrow^{\leq k} \Safe$ ?

\hspace{-0.5cm}  ({\sc bounded resilience problem (BRP)})	$\exists k ~ S \rightarrow^{\leq k} \Safe$ ?\newline}

}

  \end{frame}
%vvvvvvvvvvvvvvvvvvvvvvvvvvvvvvvvvvvvvvvvvvvvvvvvvvvvvvvvvvvvvvvvvvvvvvvvvv
  \begin{frame}{Resilience(s)}
  
 {\small
 
  \hspace{-0.2cm}
{\sc \textcolor{red}{State} resilience problems}

\hspace{-0.5cm}  {\bf INPUT:\ }{A transition system $\mathscr{S}=(S,\rightarrow)$, \textcolor{red}{$s \in S$}, and a set $\Safe \subseteq S$.}

\hspace{-0.5cm}  {\bf QUESTION:\ }

\hspace{-0.5cm}{ ({\sc state-resilience problem (SRP)}) $\textcolor{red}{\post^*(s) } \rightarrow^{*} \Safe$ ?

\hspace{-0.5cm}  ({\sc $k$-state-resilience problem (kSRP)})		$\textcolor{red}{\post^*(s)} \rightarrow^{\leq k} \Safe$ ?

\hspace{-0.5cm} ({\sc bounded-state-resilience problem (BSRP)})	$\exists k ~ \textcolor{red}{\post^*(s)} \rightarrow^{\leq k} \Safe$?\newline}

 }

% \todo[inline,color=blue!20]{{\bf  \tiny (Takes some room but I think mb having the six on the same page could be better than having three then three state-* on two separate pages ?)}} 


  \end{frame}
%vvvvvvvvvvvvvvvvvvvvvvvvvvvvvvvvvvvvvvvvvvvvvvvvvvvvvvvvvvvvvvvvvvvvvvvvvv
	\section{WBTS, WSTS}
  \begin{frame}{}
  
  \begin{definition}[\cite{DBLP:journals/iandc/Finkel90}]
A {\em Well-Behaved Transition System} (WBTS) 
is a FAC\footnote{FAC = all anti-chains are finite}-ordered transition system $\mathscr{S}=(S, \rightarrow, \leq)$ such that   
%the transition relation $ \rightarrow$ is (upward) compatible with $\leq$, i.e., for all 
%$s_1, t_1 , s_2 \in S$
%	with $s_1 \leq s_2$  and $s_1 \rightarrow t_1$, there exists 
%	$t_2 \in S$ with 
%	$t_1 \leq t_2$ and $s_2 \rightarrow^{*} t_2$.
\end{definition}


   \begin{center}
 	\begin{figure}
 	% \hspace{-2.cm}
\includegraphics[width=.25\textwidth]{WSTS_def}
	\end{figure}
\end{center}  



    \end{frame}
%vvvvvvvvvvvvvvvvvvvvvvvvvvvvvvvvvvvvvvvvvvvvvvvvvvvvvvvvvvvvvvvvvvvvvvvvvv
  \begin{frame}
% TODO
%  \todo[inline,color=blue!20]{ {\it À l'oral~} 
% Resilience(s) "obviously"  undecidable in general so restriction $\to$ restriction to WSTS
% $\to$ so we need to define WSTS properly }

\begin{definition}[\cite{DBLP:journals/iandc/Finkel90}]
A {\em Well-Structured Transition System} (WSTS) 
is a well-ordered transition system $\mathscr{S}=(S, \rightarrow, \leq)$ such that   
%the transition relation $ \rightarrow$ is (upward) compatible with $\leq$, i.e., for all 
%$s_1, t_1 , s_2 \in S$
%	with $s_1 \leq s_2$  and $s_1 \rightarrow t_1$, there exists 
%	$t_2 \in S$ with 
%	$t_1 \leq t_2$ and $s_2 \rightarrow^{*} t_2$.
\end{definition}


   \begin{center}
 	\begin{figure}
 	% \hspace{-2.cm}
\includegraphics[width=.25\textwidth]{WSTS_def}
	\end{figure}
\end{center}  

\pause

\begin{theorem}
{\sc Resilience} undecidable for WSTS
\end{theorem}

% \todo[inline,color=blue!20]{    \tiny
% diapo 5: théorème: c'est indécidable dans le cas général des systèmes de transitions
% diapo 5: déjà pour les bien structurés c'est indécidable
% résilience indécidable pour les post effectif WSTS quelque soit safe clos par le haut ou par le bas
% }

%  \todo[inline,color=blue!20]{    \tiny Definition right now is very formal for a slideshow, maybe make it less verbose? }





  \end{frame}
%vvvvvvvvvvvvvvvvvvvvvvvvvvvvvvvvvvvvvvvvvvvvvvvvvvvvvvvvvvvvvvvvvvvvvvvvvv
  \begin{frame}{WBTS Taxonomy}
  
   \begin{center}
 	\begin{figure}
% 	\hspace{-2.cm}
\includegraphics[width=.80\textwidth]{WSTS_taxonomy_large}
	\end{figure}
\end{center}  




  \end{frame}
%vvvvvvvvvvvvvvvvvvvvvvvvvvvvvvvvvvvvvvvvvvvvvvvvvvvvvvvvvvvvvvvvvvvvvvvvvv
	\section{Resiliences}
  \begin{frame}{Results}
  
% Now that everything is somewhat properly defined maybe the picture with all the results ?  
  
   \begin{center}
 	\begin{figure}
 	% \hspace{-2.cm}
 	\vspace{-0.25cm}
\includegraphics[width=1.00\textwidth]{All_results}
% \caption{Hasse diagram of some classes of transition systems, together with the decidability (in green) or undecidability (in red) of the resilience problems. Decidability of bounded resilience and $k-$resilience variants are indicated in blue.}
	\end{figure}
\end{center}  
    
    
        
  \end{frame}
%vvvvvvvvvvvvvvvvvvvvvvvvvvvvvvvvvvvvvvvvvvvvvvvvvvvvvvvvvvvvvvvvvvvvvvvvvv
  \begin{frame}{Results}
 
   \begin{center}
 	\begin{figure}
 	% \hspace{-2.cm}
 	\vspace{-0.25cm}
\includegraphics[width=1.00\textwidth]{resultA}
% \caption{Hasse diagram of some classes of transition systems, together with the decidability (in green) or undecidability (in red) of the resilience problems. Decidability of bounded resilience and $k-$resilience variants are indicated in blue.}
	\end{figure}
\end{center}  

  \end{frame}
%vvvvvvvvvvvvvvvvvvvvvvvvvvvvvvvvvvvvvvvvvvvvvvvvvvvvvvvvvvvvvvvvvvvvvvvvvv
  \begin{frame}{Definitions}
  

$I$ Ideal =
\begin{itemize}
 \item Downward-closed % is any set $D \subseteq X$ such that if $y \leq x$ and $x \in D$ then $y \in D $. 
 ($y \leq x  \wedge x \in I \to y \in I$) 
 \item Directed % i.e. it is nonempty and for every $a,b \in D$, there exists $c \in D$ such that $a \leq c$ and $b \leq c$.
 ($I \neq \emptyset \wedge \forall a,b \in I, \exists c \in I, a \leq c \vee b \leq c$ )
\end{itemize} 

  
\hspace{-2cm}

$S$ effective = 
 \begin{itemize}
\item $s \to t$ decidable 
\item membership decidable 
\item $\leq$ decidable 
\item membership in $Ideals(S)$ decidable 
\item inclusion of ideals decidable
\end{itemize}



  \end{frame}
%vvvvvvvvvvvvvvvvvvvvvvvvvvvvvvvvvvvvvvvvvvvvvvvvvvvvvvvvvvvvvvvvvvvvvvvvvv
  \begin{frame}{Resilience when $\Safe = {\uparrow \Safe}$}
   
% Result A: Théorème 4 


% \begin{theorem}
% Let $\mathscr{S}=(S,\rightarrow, \leq)$ be a $\post$-ideal-effective WBTS with strong compatibility and a set $\Safe = \mathop{\uparrow} \Safe$.
% {\sc Resilience}, {\sc Bounded resilience} 
% and {\sc $k$-resilience} are decidable.
% \end{theorem}


\begin{theorem}
WBTS $\mathscr{S}=(S,\rightarrow, \leq)$ 
\begin{itemize}
\item $\mathop{\downarrow} \post(I)$ computable %$\post$-ideal-effective 
\item \textcolor{red}{strong} compatibility
\item $\Safe = \mathop{\uparrow} \Safe$
\end{itemize}
{\sc Resilience}, {\sc Bounded resilience} 
and {\sc $k$-resilience} decidable.
\end{theorem}


\begin{center}
 	\begin{figure}
 	% \hspace{-2.cm}
\includegraphics[width=.25\textwidth]{WSTS_strong}
	\end{figure}
\end{center}  

       \end{frame}
%vvvvvvvvvvvvvvvvvvvvvvvvvvvvvvvvvvvvvvvvvvvvvvvvvvvvvvvvvvvvvvvvvvvvvvvvvv
  \begin{frame}{Ingredient 1: link with coverability}
  
  \begin{block}{Proposition}
   $s_j$ is coverable from $x$ in $\mathscr{S}$ if and only if $\mathop{\downarrow} s_j$ is coverable (for inclusion) from $\mathop{\downarrow} x$ in $\hat{\mathscr{S}} $ 
 \end{block}
 
\pause

  \begin{block}{ }
The resilience problem can be reduced to the following infinite number of instances of the coverability problem in $\mathscr{S}$: for all $x \in \Bad$ does there exist an $j$ such that $s_j$ is coverable from $x$.  
 \end{block}
 
 \pause

  \begin{block}{ }
This infinite set of coverability questions can be reduced to a \emph{finite} set of coverability questions in the completion $\hat{\mathscr{S}}=(Ideals(S),\rightarrow, \subseteq)$ of $\mathscr{S}=(S,\rightarrow, \leq)$. 
 \end{block}
 


  \end{frame}
%vvvvvvvvvvvvvvvvvvvvvvvvvvvvvvvvvvvvvvvvvvvvvvvvvvvvvvvvvvvvvvvvvvvvvvvvvv
  \begin{frame}{Ingredient 2: the completion}
 
 
 \begin{definition}[\cite{BFM-ic17}]
The \emph{completion}   of a WSTS $\mathscr{S}=(S,\rightarrow, \leq)$ is the associated ordered transition system $\hat{\mathscr{S}}=(Ideals(S),\rightarrow, \subseteq)$ where 
%$Ideals(S)$ is the set of ideals of $S$ and
 $I \rightarrow J$ if $J$ belongs to the finite ideal decomposition of $\mathop{\downarrow} \post_{\mathscr{S}}(I)$. 
% The completion is always \emph{finitely branching} but it is not necessarly a WSTS since $\subseteq$ is not necessarly a wqo. 
%It is proved in  \cite{BFM-ic17} that $\hat{\mathscr{S}}$ is WSTS iff $\mathscr{S}=(S,\rightarrow, \leq)$ is $\omega^2$-WSTS. \\
\end{definition}

\pause

\begin{exampleblock}{\cite[Proposition 30]{BFM-ic17}}
If $x \xrightarrow{k} y$ in $\mathscr{S}$ then for every ideal $I \supseteq \mathop{\downarrow} x$, there exists an ideal $J \supseteq \mathop{\downarrow} y$ such that $I \xrightarrow{k} J$ in $\hat{\mathscr{S}}$.
\end{exampleblock}

\pause

\begin{exampleblock}{\cite[Proposition 29]{BFM-ic17}}
If $I \xrightarrow{k} J$ in $\hat{\mathscr{S}}$ then for every $y \in J$, there exists $x \in I$ and $y' \geq y$ such that $x \xrightarrow{k'} y'$ in $\mathscr{S}$. Moreover, if $\mathscr{S}$ has transitive compatibility then $k’ \geq k$; if $\mathscr{S}$ has strong compatibility then $k’ = k$.
\end{exampleblock}




 
  

  \end{frame}
  %vvvvvvvvvvvvvvvvvvvvvvvvvvvvvvvvvvvvvvvvvvvvvvvvvvvvvvvvvvvvvvvvvvvvvvvvvv
    \begin{frame}{Results}
  
% Now that everything is somewhat properly defined maybe the picture with all the results ?  
  
   \begin{center}
 	\begin{figure}
 	% \hspace{-2.cm}
 	\vspace{-0.25cm}
\includegraphics[width=1.00\textwidth]{All_results}
% \caption{Hasse diagram of some classes of transition systems, together with the decidability (in green) or undecidability (in red) of the resilience problems. Decidability of bounded resilience and $k-$resilience variants are indicated in blue.}
	\end{figure}
\end{center}  
    
    
          \end{frame}
%vvvvvvvvvvvvvvvvvvvvvvvvvvvvvvvvvvvvvvvvvvvvvvvvvvvvvvvvvvvvvvvvvvvvvvvvvv
  \begin{frame}{State resilience}
 
   \begin{center}
 	\begin{figure}
 	% \hspace{-2.cm}
 	\vspace{-0.25cm}
\includegraphics[width=1.00\textwidth]{resultB}
% \caption{Hasse diagram of some classes of transition systems, together with the decidability (in green) or undecidability (in red) of the resilience problems. Decidability of bounded resilience and $k-$resilience variants are indicated in blue.}
	\end{figure}
\end{center}  




  \end{frame}
%vvvvvvvvvvvvvvvvvvvvvvvvvvvvvvvvvvvvvvvvvvvvvvvvvvvvvvvvvvvvvvvvvvvvvvvvvv
  \begin{frame}{Resilience $\ldots$}
  
% Now that everything is somewhat properly defined maybe the picture with all the results ?  
  
   \begin{center}
 	\begin{figure}
 	% \hspace{-2.cm}
 	\vspace{-0.25cm}
\includegraphics[width=1.00\textwidth]{All_results}
% \caption{Hasse diagram of some classes of transition systems, together with the decidability (in green) or undecidability (in red) of the resilience problems. Decidability of bounded resilience and $k-$resilience variants are indicated in blue.}
	\end{figure}
\end{center}  
    
    

  \end{frame}
%vvvvvvvvvvvvvvvvvvvvvvvvvvvvvvvvvvvvvvvvvvvvvvvvvvvvvvvvvvvvvvvvvvvvvvvvvv
  \begin{frame}{Resilience $\ldots$ for VASS}
 
   \begin{center}
 	\begin{figure}
 	% \hspace{-2.cm}
 	\vspace{-0.25cm}
\includegraphics[width=1.00\textwidth]{resultC}
% \caption{Hasse diagram of some classes of transition systems, together with the decidability (in green) or undecidability (in red) of the resilience problems. Decidability of bounded resilience and $k-$resilience variants are indicated in blue.}
	\end{figure}
\end{center}  


  \end{frame}
%vvvvvvvvvvvvvvvvvvvvvvvvvvvvvvvvvvvvvvvvvvvvvvvvvvvvvvvvvvvvvvvvvvvvvvvvvv
  \begin{frame}{Vector Addition System with States}
 
% Définir rapidement les VASS 
 
 \begin{definition} 
A {\em vector addition system with (control-)states (VASS)} in dimension $d$ ($d$-VASS for short) is a finite $\mathds{Z}^d$-labeled directed graph $V = (Q,T)$, where $Q$ is the set of {\em control-states}, and $T \subseteq Q \times \mathds{Z}^d \times Q$ is the set of {\em control-transitions}. 
 \end{definition} 

 \begin{definition} 
A {\em vector addition system (VAS)} in dimension $d$ ($d$-VAS for short) is a $d$-VASS where the set of control-states is a singleton.
 \end{definition} 

  \end{frame}
%vvvvvvvvvvvvvvvvvvvvvvvvvvvvvvvvvvvvvvvvvvvvvvvvvvvvvvvvvvvvvvvvvvvvvvvvvv
  \begin{frame}{Resilience(s) for VASS}

\begin{theorem}\label{SL VASS}
{\sc Resilience} is decidable for VASS, and $\mathbb{Z}-$VASS, when $\Safe$  is a semilinear set.
\end{theorem}

- Resilience $\to$ inclusion d'idéaux

- Bounded resilience and $k$-resilience $\to$ more complex
 
   \end{frame}
%vvvvvvvvvvvvvvvvvvvvvvvvvvvvvvvvvvvvvvvvvvvvvvvvvvvvvvvvvvvvvvvvvvvvvvvvvv
  \begin{frame}{Resilience(s) for VASS}

- Bounded resilience and $k$-resilience $\to$ more complex
 
- Remaque sur les VAS : it does not hold

\begin{block}{Proposition}
{\sc Bounded resilience} and {\sc $k$-resilience} never hold for $d$-VAS when $\Safe = \mathop{\downarrow} \Safe$ and $\Safe \neq \N^d$.
\end{block}


   \end{frame}
%vvvvvvvvvvvvvvvvvvvvvvvvvvvvvvvvvvvvvvvvvvvvvvvvvvvvvvvvvvvvvvvvvvvvvvvvvv
  \begin{frame}{Resilience(s) for VASS}

\begin{theorem}\label{vass down}
{\sc $k$-resilience }  and {\sc Bounded resilience} are decidable for VASS when 
$\Safe = \mathop{\downarrow} \Safe$.
\end{theorem}



  \end{frame}
%vvvvvvvvvvvvvvvvvvvvvvvvvvvvvvvvvvvvvvvvvvvvvvvvvvvvvvvvvvvvvvvvvvvvvvvvvv
  \begin{frame}{Conclusion}
  

How to expand current results ?

- detailed complexity analysis (in VASS \& variations)

- controller/environment framework

- other classes of $\Safe / \Bad$ (e.g. semilinear)


  \end{frame}
%vvvvvvvvvvvvvvvvvvvvvvvvvvvvvvvvvvvvvvvvvvvvvvvvvvvvvvvvvvvvvvvvvvvvvvvvvv
  \begin{frame}
  
  \begin{center}
  Thanks for your attention
  \end{center}
  
  \end{frame}
%vvvvvvvvvvvvvvvvvvvvvvvvvvvvvvvvvvvvvvvvvvvvvvvvvvvvvvvvvvvvvvvvvvvvvvvvvv
  \begin{frame}{References}
      
      {\tiny
\bibliography{bib}
}


  \end{frame}
%vvvvvvvvvvvvvvvvvvvvvvvvvvvvvvvvvvvvvvvvvvvvvvvvvvvvvvvvvvvvvvvvvvvvvvvvvv

 
  
\end{document}
