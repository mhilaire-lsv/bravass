




\subsection{Case: $\Safe=\uparrow \Safe$.}\label{safe-up}

%

We start with the case $\Safe=\uparrow \Safe$, hence $\Bad=\downarrow \Bad$. 
It can be viewed as a generalizeation
 of coverability, as
it asks whether for every element of $\Bad$ it is possible to cover an element of the basis of $\Safe$.



% \mathieu{I'm using the notion of ideally effective-ness for WBTS here}
%
Let us recall that the \emph{completion}  \cite{BFM-ic17} of a WSTS $\mathscr{S}=(S,\rightarrow, \leq)$ is the associated ordered transition system $\hat{\mathscr{S}}=(Ideals(S),\rightarrow, \subseteq)$ where states of $\hat{\mathscr{S}}$ are ideals of $S$ and $I \rightarrow J$ if $J$ belongs to the finite ideal decomposition of $\downarrow \post_{\mathscr{S}}(I)$. The completion is always finitely branching but it is not necessarly WSTS since $\subseteq$ is not necessarly a wqo. $\hat{\mathscr{S}}$ is WSTS iff $\mathscr{S}=(S,\rightarrow, \leq)$ is $\omega^2$-WSTS (intuitively speaking, $(S,\leq)$ must not contain the Rado set). 
A WSTS $\mathscr{S}=(S,\rightarrow,\leq)$ is {\em completion-post-effective} if
it is post-effective, and there exists a Turing Machine $M_\downarrow$ that computes, on input $s \in S$, % some $\hat(e) \in \hat( ... )$ such that
 $\downarrow s$, and some Turing Machine $M_{\uparrow^C}$ that computes, on input
 $\{s_1, s_2, \ldots, s_m\} \in S^m$, 
 the ideal decomposition of $S \setminus \uparrow \{ s_1, s_2, \ldots, s_m\}$.
Coverability is shown decidable  [Theorem 44] in \cite{BFM-ic17} for completion-post-effective $\omega^2$-WSTS (we don't need the pred-basis hypothesis).

Let us recall two other results in \cite{BFM-ic17}. Proposition 30 establishes a strong relation between the runs of a WSTS $\mathscr{S}=(S,\rightarrow, \leq)$ and the runs of its completion $\hat{\mathscr{S}}$. It states that if $x \xrightarrow{k} y$ in $\mathscr{S}$ then for every ideal $I \supseteq \downarrow x$, there exists an ideal $J \supseteq \downarrow y$ such that $I \xrightarrow{k} J$ in $\hat{\mathscr{S}}$. Proposition 29 establishes that if $I \xrightarrow{k} J$ in $\hat{\mathscr{S}}$ then for every $y \in J$, there exists $x \in I$ and $y' \geq y$ such that $x \xrightarrow{k'} y'$ in $\mathscr{S}$. Moreover, if $\mathscr{S}$ has transitive compatibility then $k’ \geq k$; if $\mathscr{S}$ has strong compatibility then $k’ = k$.
%





%
\begin{theorem}\label{down-up}
Let $\mathscr{S}=(S,\rightarrow, \leq)$ be a completion-post-effective $\omega^2$-WSTS with strong compatibility and a set $\Safe = \uparrow \Safe$.
{\sc Resilience}, {\sc Bounded resilience} 
and {\sc $k$-resilience} are decidable.
\end{theorem}

\begin{proof}
Let  $\{b_1,b_2,...,b_m\}$ be the (unique) minimal basis of $\Safe$ and  $\{J_1, J_2,...,J_n\}$ be the ideal decomposition of $\Bad$.
The %uniform 
resilience problem can be reduced to the following infinite number of instances of the coverability problem in $\mathscr{S}$: for all $x \in \Bad$ does there exist an $j$ such that $b_j$ is coverable from $x$. Let us show how this infinite set of coverability questions can be reduced to a \emph{finite} set of coverability questions in the completion $\hat{\mathscr{S}}=(Ideals(S),\rightarrow, \subseteq)$ of $\mathscr{S}=(S,\rightarrow, \leq)$. 

%Proposition 30 in \cite{BFM-icalp14} establishes a strong relation between the runs of a WSTS $\mathscr{S}=(S,\rightarrow, \leq)$ and the runs of its completion $\hat{\mathscr{S}}$. It states that if $x \xrightarrow{k} y$ in $\mathscr{S}$ then for every ideal $I \supseteq \downarrow x$, there exists an ideal $J \supseteq \downarrow y$ such that $I \xrightarrow{k} J$ in $\hat{\mathscr{S}}$. Proposition 29 establishes that if $I \xrightarrow{k} J$ in $\hat{\mathscr{S}}$ then for every $y \in J$, there exists $x \in I$ and $y' \geq y$ such that $x \xrightarrow{*} y'$ in $\mathscr{S}$.

Let us prove that $b_j$ is coverable from $x$ in $\mathscr{S}$ if and only if $\downarrow b_j$ is coverable (for inclusion) from $\downarrow x$ in $\hat{\mathscr{S}}$.
%
Suppose that $b_j$ is coverable from $x$ then there exists a run $x \xrightarrow{k} y \geq b_j$. From Proposition 30, there exist an ideal $J$ and a run $\downarrow x \xrightarrow{k} J$ where $J \supseteq \downarrow y \supseteq \downarrow b_j$ in $\hat{\mathscr{S}}$, hence $\downarrow b_j$ is covered from $\downarrow x$.
Conversely, if $I \xrightarrow{k} J$ in $\hat{\mathscr{S}}$ with $\downarrow b_j \subseteq J$ then 
%	for every $y \in J$, 
there exists $x \in I$ and $y' \geq b_j$ such that $x \xrightarrow{k} y'  \geq b_j$ in $\mathscr{S}$ and then $b_j$ is coverable from $x$ in $\mathscr{S}$.

Hence we obtain: $\mathscr{S}$ is resilient iff for all $i=1,..,n$ and $j= 1,..m$, $\downarrow b_j$ is coverable from ideal $J_i$ in $\hat{\mathscr{S}}$.
%
Let us denote by $k_{i,j}$ the length of a covering sequence that covers $\downarrow b_j$ from $J_i$ in $\hat{\mathscr{S}}$ and let $k_{i,j}\stackrel{\text{def}}{=}\infty$ if $\downarrow b_j$ is not coverable from $J_i$. Let us now define $K_{\mathscr{S}}(\Safe)=\max(k_{i,j} \mid i=1,..,n$ and $j= 1,..m$).
%

%
%	(if all $k_{i,j}$ are finite) else $K=\infty$.
We now have $\mathscr{S}$ is resilient iff $K_{\mathscr{S}}(\Safe)$ is finite iff $\mathscr{S}$ is $K_{\mathscr{S}}(\Safe)$-resilient with $K_{\mathscr{S}}(\Safe)$ finite.

This implies that resilience and bounded resilience are equivalent to coverability.
%
% iff there is a run in $\hat{S}$ from an ideal I to J such that  $\downarrow x \subseteq I$ and $\downarrow b_j \subseteq J$. This is a consequence of  
%Propositions 29 and 30 in \cite{BFM-icalp14} that establish a strong relation between the runs of a WSTS $\mathscr{S}=(S,\rightarrow, \leq)$ with its completion $\hat{S}$.
%' \in I(\Bad)$.
%
%To decide the bounded resilience, we decide $n \times m$ coverability questions: is state $\downarrow b_j$ coverable from ideal $J_i$ ? If all these $n \times m$ coverability questions are positive then we compute $K=\max(k_{i,j} \mid i=1,..,n $ and $j= 1,..m)$ where $k_{i,j}$ is the least length of a sequence that covers  $\downarrow b_j$ from $J_i$.
%
 % \alain{comment trouver les $k_a$ ?} 
%
%  \alain{else $\mathscr{S}$ is not $K$-resilient...et alors qu'en déduit-on ? il pourrait exister un $K' \geq K$ pour lequel $\mathscr{S}$ est K'-resilient....}. 
%else if some of these $n \times m$ coverability questions are negative then resilience
%do not hold and bounded resilience do not hold either.
Since coverability is decidable for completion-post-effective $\omega^2$-WSTS, we deduce that both the 
  resilience problem and the bounded resilience problem are decidable.



Let us now show that the $k$-resilience problem, with $k \in \mathbb{N}$, is also decidable.
Let us denote by $k'_{i,j}$ the \emph{minimal} length of a covering sequence that covers $\downarrow b_j$ from $J_i$ in $\hat{\mathscr{S}}$ if it exists and let $k'_{i,j}\stackrel{\text{def}}{=}\infty$ if $\downarrow b_j$ is not coverable from $J_i$. 
If $\downarrow b_j$ is coverable from $J_i$, we first compute an $k_{i,j}$, and then we compute $k'_{i,j}$ by iteratively checking whether there exists a sequence of length $0,1,...,k_{i,j}-1$ that covers $\downarrow b_j$ from $J_i$ until we find the minimal one which is necessarly smaller (or equal to) than $k_{i,j}$.

Let us now define $K'_{\mathscr{S}}(\Safe)=\max(k'_{i,j} \mid i=1,..,n$ and $j= 1,..m$) and we deduce that  $\mathscr{S}$ is $k$-resilient iff $k \geq K'_{\mathscr{S}}(\Safe)$. \qed
%	If $k <  \min(K,n)$, then we check every path of length smaller
%	than $\min(K,n)$ from the ideals of the decomposition of $\Bad$ in the completion \alain{à expliquer mieux}.
\end{proof}


Remark we did not make use of the 
% hypothesis
property that $\Bad$ is the complement of $\Safe$, simply using
$\Bad=\downarrow \Bad$ and $\Safe=\uparrow \Safe$, thus 
the above results still hold in the more general case where $\Bad$ and $\Safe$ are not complements of each others.


The completion-post-effective hypothesis is needed to decide {\sc Resilience}.
Indeed, consider the family $\{ f_j: \N^2 \to \N^2\}$ of %strongly 
increasing recursive functions from \cite{FMP-ic04} defined as

$ f_j(n,k) = \begin{dcases*} (n,0) & \text{ if } k=0 \text{ and } TM$_j$ \text{ runs for more than } n \text{ steps} \\
		(n,n+k) &\text{ otherwise,} \end{dcases*} $
		
	\noindent	
	where $TM_j$ is the $j$-th Turing machine (in a classical enumeration)
	which moreover begins
by writing the integer $j$ on its tape,	
	and consider additionnally the function $g : \N^2 \to \N^2$ defined by $g(n,k) = (n+1,k)$. 
	The
	transition system 
	$S_j = (\N^2, \{f_j, g\}, \leq)$ 
	is a WSTS
	and
	has the property that
	it is
	$\uparrow (1,1)$-resilient
	iff $TM_j$ halts on input $j$,
	since $(1,1)$ is coverable from $(0,0)$
	iff $TM_j$ halts on input $j$. 
	Hence there is no Turing machine which correctly determines 
	resilience and halts whenever its input is a
	WSTS.
\iffalse

Let us remark that coverability reduce
to resilience when  $\Safe = \uparrow \Safe$ and $\Bad = \downarrow \Bad$,
as $s \to^* \uparrow t$ 
reduces to $\Safe$-resilience for $\Safe = \uparrow t$ and $\Bad = \downarrow s$.
\mathieu{No longer the case here since $\Bad$ needs be the complement of $\Safe$.}
Let us recall that the coverability problem for strongly increasing well-structured nets is undecidable~\cite{FMP-ic04}, hence the strongly increasing well-structured nets are not completion-post-effective hypothesis nor have the pred-basis hypothesis.
%
Moreover, this implies from the reduction that resilience for strongly increasing well-structured nets is  undecidable. 
\fi
Hence resilience for WSTS in general is undecidable.



