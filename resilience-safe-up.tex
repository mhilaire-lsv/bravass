




\subsection{Case: $\Bad=\downarrow \Bad$.}

%
%		cas étudié par les deux articles sur les graphes
%
Let us choose $\Bad$ to be the set of states from which there is no infinite run. If we are in an upward compatible ordered transition system (like VASS, lossy channel systems,...), then $\Bad$ is downward closed.

\subsubsection{Subcase: $\Safe=\downarrow \Safe$}

The well-known mutual exclusion property can be modelized, in a $d$-VASS with $k$ counters, by the property that a special counter $c_{mutex}$ must be bounded by $k \geq 1$ which counts the number of processes that are allowed to be simultaneously in the critical section. Then, the set $\Safe =  \{c_{mutex} \leq k\} \times \mathbb{N}^{d-1}$ is downward closed.
%		and $\Bad =\{c_{mutex} \geq k+1\} \times  \mathbb{N}^{d-1} $ is the upward closed complementary of $\Safe$. 

% In the case \Safe and \Bad are both downward closed, they can both be finite.

\begin{theorem}\label{down-down}
The resilience problem is undecidable for  effective WSTS with  strong  compatibility such that
%	with more than one minimal element, 
$\Safe=\downarrow \Safe$
and $\Bad=\downarrow \Bad$.
\end{theorem}

\begin{proof}
If the set $S$ of a WSTS $\mathscr{S}=(S,\rightarrow, \leq)$ has an unique minimal element $m$, then $m$ belongs to both $\Safe$ and $\Bad$ which contradicts the assumption $\Safe \cap \Bad= \emptyset$. So let us consider the case of a set $S$ with at least two minimal elements $m_1$ and $m_2$.
The problem of whether $m_2$ is reachable from $m_1$ reduces itself to the resilience problem by considering $\Safe=\downarrow m_2 = \{ m_2\}$ and $\Bad=\downarrow m_1 = \{ m_1\}$. By undecidability of the reachability problem for effective WSTS with strong compatibility we conclude.  
\end{proof}

%%%%

\subsubsection{Subcase: $\Safe=\uparrow \Safe$}

example: in a VASS, we may choose $\Safe$ to be the set of states that are not deadlocks, i.e. from which it is always possible to fire a transition. This set of states is upward closed and $\Safe=S-\Bad=\uparrow \Safe$.

%  Parosh Aziz Abdulla, Karlis Cerans, Bengt Jonsson & Yih-Kuen Tsay (1996): General Decidability Theorems for Infinite-State Systems. In: Proc. LICS 1996, IEEE Computer Society Press, pp. 313–321,
%  Alain Finkel & Philippe Schnoebelen (2001): Well-structured transition systems everywhere! Theor. Comput. Sci. 256(1-2), pp. 63–92, doi:10.1016/S0304-3975(00)00102-X.

%Transfering the abstract resilience problems into this framework,
%it is therefore reasonable to demand that both propositions, \Safe and \Bad, are given by 
%upward-closed or downward-closed sets.

Let us recall that the \emph{completion}  \cite{BFM-ic17} of a WSTS $\mathscr{S}=(S,\rightarrow, \leq)$ is the associated ordered transition system $\hat{\mathscr{S}}=(Ideals(S),\rightarrow, \subseteq)$ where states of $\hat{\mathscr{S}}$ are ideals of $S$ and $I \rightarrow J$ if $J$ belongs to the finite ideal decomposition of $\downarrow \post_{\mathscr{S}}(I)$. The completion is always finitely branching but it is not necessarly WSTS since $\subseteq$ is not necessarly a wqo. $\hat{\mathscr{S}}$ is WSTS iff $\mathscr{S}=(S,\rightarrow, \leq)$ is $\omega^2$-WSTS (intuitively speaking, $(S,\leq)$ must not contain the Rado set). Coverability is shown decidable  [Theorem 44] in \cite{BFM-ic17} for completion-post-effective $\omega^2$-WSTS.

Let us recall two other results in \cite{BFM-ic17}. Proposition 30 establishes a strong relation between the runs of a WSTS $\mathscr{S}=(S,\rightarrow, \leq)$ and the runs of its completion $\hat{\mathscr{S}}$. It states that if $x \xrightarrow{k} y$ in $\mathscr{S}$ then for every ideal $I \supseteq \downarrow x$, there exists an ideal $J \supseteq \downarrow y$ such that $I \xrightarrow{k} J$ in $\hat{\mathscr{S}}$. Proposition 29 establishes that if $I \xrightarrow{k} J$ in $\hat{\mathscr{S}}$ then for every $y \in J$, there exists $x \in I$ and $y' \geq y$ such that $x \xrightarrow{k'} y'$ in $\mathscr{S}$. Moreover, if $\mathscr{S}$ has transitive compatibility then $k’ \geq k$; if $\mathscr{S}$ has strong compatibility then $k’ = k$.


\begin{theorem}\label{down-up}
Let $\mathscr{S}=(S,\rightarrow, \leq)$ be a completion-post-effective $\omega^2$-WSTS with strong compatibility and two 
%finite \alain{non, confusion entre ensemble et base}
 sets $\Bad = \downarrow \Bad$ and $\Safe = \uparrow \Safe$.
The resilience problem (RP), the bounded resilience problem (BRP)
and the $k$-resilience problem (kRP) are decidable.
\end{theorem}

\begin{proof}
Let $\{J_1, J_2,...,J_n\}$ be the ideal decomposition of $\Bad$ and $\{b_1,b_2,...,b_m\}$ be the (unique) minimal basis of $\Safe$.
The %uniform 
resilience problem (RP) can be reduced to the following infinite number of instances of the coverability problem in $\mathscr{S}$: for all $x \in \Bad$ does there exist an $j$ such that $b_j$ is coverable from $x$. Let us show how this infinite set of coverability questions can be reduced to a \emph{finite} set of coverability questions in the completion $\hat{\mathscr{S}}=(Ideals(S),\rightarrow, \subseteq)$ of $\mathscr{S}=(S,\rightarrow, \leq)$. 

%Proposition 30 in \cite{BFM-icalp14} establishes a strong relation between the runs of a WSTS $\mathscr{S}=(S,\rightarrow, \leq)$ and the runs of its completion $\hat{\mathscr{S}}$. It states that if $x \xrightarrow{k} y$ in $\mathscr{S}$ then for every ideal $I \supseteq \downarrow x$, there exists an ideal $J \supseteq \downarrow y$ such that $I \xrightarrow{k} J$ in $\hat{\mathscr{S}}$. Proposition 29 establishes that if $I \xrightarrow{k} J$ in $\hat{\mathscr{S}}$ then for every $y \in J$, there exists $x \in I$ and $y' \geq y$ such that $x \xrightarrow{*} y'$ in $\mathscr{S}$.

Let us prove that $b_j$ is coverable from $x$ in $\mathscr{S}$ if and only if $\downarrow b_j$ is coverable (for inclusion) from $\downarrow x$ in $\hat{\mathscr{S}}$.
%
Suppose that $b_j$ is coverable from $x$ then there exists a run $x \xrightarrow{k} y \geq b_j$. From Proposition 30, there exist an ideal $J$ and a run $\downarrow x \xrightarrow{k} J$ where $J \supseteq \downarrow y \supseteq \downarrow b_j$ in $\hat{\mathscr{S}}$, hence $\downarrow b_j$ is covered from $\downarrow x$.
Conversely, if $I \xrightarrow{k} J$ in $\hat{\mathscr{S}}$ with $\downarrow b_j \subseteq J$ then 
%	for every $y \in J$, 
there exists $x \in I$ and $y' \geq b_j$ such that $x \xrightarrow{k} y'  \geq b_j$ in $\mathscr{S}$ and then $b_j$ is coverable from $x$ in $\mathscr{S}$.

Hence we obtain: $\mathscr{S}$ is resilient iff for all $i=1,..,n$ and $j= 1,..m$, $\downarrow b_j$ is coverable from ideal $J_i$ in $\hat{\mathscr{S}}$.
%
Let us denote by $k_{i,j}$ the length of a covering sequence that covers $\downarrow b_j$ from $J_i$ in $\hat{\mathscr{S}}$ and let $k_{i,j}\stackrel{\text{def}}{=}\infty$ if $\downarrow b_j$ is not coverable from $J_i$. Let us now define $K_{\mathscr{S}}(\Safe,\Bad)=max(k_{i,j} \mid i=1,..,n$ and $j= 1,..m$).
%	(if all $k_{i,j}$ are finite) else $K=\infty$.
We now have $\mathscr{S}$ is resilient iff $K_{\mathscr{S}}(\Safe,\Bad)$ is finite iff $\mathscr{S}$ is $K_{\mathscr{S}}(\Safe,\Bad)$-resilient with $K_{\mathscr{S}}(\Safe,\Bad)$ finite.

This implies that resilience and bounded resilience are equivalent to coverability.
%
% iff there is a run in $\hat{S}$ from an ideal I to J such that  $\downarrow x \subseteq I$ and $\downarrow b_j \subseteq J$. This is a consequence of  
%Propositions 29 and 30 in \cite{BFM-icalp14} that establish a strong relation between the runs of a WSTS $\mathscr{S}=(S,\rightarrow, \leq)$ with its completion $\hat{S}$.
%' \in I(\Bad)$.
%
%To decide the bounded resilience, we decide $n \times m$ coverability questions: is state $\downarrow b_j$ coverable from ideal $J_i$ ? If all these $n \times m$ coverability questions are positive then we compute $K=max(k_{i,j} \mid i=1,..,n $ and $j= 1,..m)$ where $k_{i,j}$ is the least length of a sequence that covers  $\downarrow b_j$ from $J_i$.
%
 % \alain{comment trouver les $k_a$ ?} 
%
%  \alain{else $\mathscr{S}$ is not $K$-resilient...et alors qu'en déduit-on ? il pourrait exister un $K' \geq K$ pour lequel $\mathscr{S}$ est K'-resilient....}. 
%else if some of these $n \times m$ coverability questions are negative then resilience
%do not hold and bounded resilience do not hold either.
Since coverability is decidable for completion-post-effective $\omega^2$-WSTS, we deduce that both the 
  resilience problem (RP) and the bounded resilience problem (BRP) are decidable.

\iffalse
	\end{proof}

	\begin{theorem}\label{k-down-up}
	Let $\mathscr{S}=(S,\rightarrow, \leq)$ be a completion-post-effective $\omega^2$-WSTS 
	with strong compatibility and the predbasis hypothesis, and two finite sets: $\Bad$ 
	and $\Safe$.
	The  $k$-resilience problem (kRP) is decidable 
	%	for $k \geq min(K,n)$ where $n$ satisfies $ \uparrow \pred^n(\Safe)=  \uparrow 	\pred^*(\Safe)$. 
	\textcolor{red}{CONJECTURE for now}
	 \end{theorem}

	\begin{proof}

We begin to compute $K$ and $n$ such that $ \uparrow \pred^n(\Safe)=  \uparrow \pred^*(\Safe)$.
If $K=\infty$ then $\mathscr{S}$ is not resilient for Safe and Bad.
Now, if $k \geq min(K,n)$, we conclude that is $k$-resilient; 
%	if moreover $\mathscr{S}=(S,\rightarrow, \leq)$ has the predbasis hypothesis, 
%	and then kRP is decidable.
\fi

Let us now show that the $k$-resilience problem (kRP), with $k \in \mathbb{N}$, is also decidable.
Let us denote by $k'_{i,j}$ the \emph{minimal} length of a covering sequence that covers $\downarrow b_j$ from $J_i$ in $\hat{\mathscr{S}}$ if it exists and let $k'_{i,j}\stackrel{\text{def}}{=}\infty$ if $\downarrow b_j$ is not coverable from $J_i$. 
If $\downarrow b_j$ is coverable from $J_i$, we first compute an $k_{i,j}$, and then we compute $k'_{i,j}$ by iteratively checking whether there exists a sequence of length $0,1,...,k_{i,j}-1$ that covers $\downarrow b_j$ from $J_i$ until we find the minimal one which is necessarly smaller (or equal to) than $k_{i,j}$.

Let us now define $K'_{\mathscr{S}}(\Safe,\Bad)=max(k'_{i,j} \mid i=1,..,n$ and $j= 1,..m$) and we deduce that  $\mathscr{S}$ is $k$-resilient iff $k \geq K'_{\mathscr{S}}(\Safe,\Bad)$.

%	If $k <  min(K,n)$, then we check every path of length smaller
%	than $min(K,n)$ from the ideals of the decomposition of $\Bad$ in the completion \alain{à expliquer mieux}.
\end{proof}

%





