

\section{Applications section}


\subsection{VASS, 2-VASS, semilinear VASS, Lossy Channel Systems, Integer et Continuous VASS, Regular Channel Systems}

\begin{theorem}{}
Resilience is decidable for semilinear VASS, LCM, Integer et Continuous VASS when $\Safe$ and $\Bad$ are semilinear sets.
\end{theorem}

se déduit du papier Schnoebelen RP 2010, page 8

\begin{theorem}{}
Resilience is decidable for Regular Channel Systems when $\Safe$ and $\Bad$ are regular languages.
\end{theorem} 

open for VASS ???


\subsection{Timed Automata}

\textcolor{red}{Should be defined in a later 'application section' once we start writing any proof, for now I leave it there} 

\renewcommand{\A}{\mathcal{A}}
\newcommand{\B}{\mathcal{B}}
\renewcommand{\C}{\mathcal{C}}
\newcommand{\Const}{\mathsf{Consts}}
\newcommand{\Conf}{\mathsf{Conf}}
\newcommand{\guards}{{\textsc{Guards}}}

% \subsubsection{Guards, Clocks}

A {\em guard} over a finite set of clocks $\Omega$ 
is a comparison of the form
$\omega \bowtie c$, where $ \omega \in \Omega$, $c \in \N$,
and $\bowtie\in\{<,\leq,=,\geq,>\}$.
%
We denote by $\guards(\Omega)$ the {\em set of guards} over the set of 
clocks $\Omega$.
The {\em size} %$|g|$
 of a guard 
$g=\omega \bowtie c$ is defined as %:
$|g|=\log(c)$.
A {\em clock valuation} is a function from $\Omega$ to $\N$;
we write $\vec{0}$ to denote the clock valuation $\omega \mapsto 0$
whenever the set $\Omega$ is clear from the context.
For each clock valuation $v$ and each $t\in\N$ we denote
by $v+t$ the clock valuation $\omega \mapsto v(\omega)+t$.
%
For each guard $g=\omega \bowtie c$ with $c\in\N$,
we write $v\models g$ if $v(\omega)\bowtie c$.

\iffalse
We define an {\em empty guard} $g_\epsilon$ over a non-empty finite set of clocks
$\Omega$ and to be of the form $\omega \geq 0$ for some 
$\omega \in \Omega$. In particular, we
defined $g_\epsilon$ such that for all $v \in \N^\Omega$ 
we have
$v \models g_\epsilon$, hence $g_\epsilon$ can be used as a guard that is always true. 
\fi

% \subsubsection{Timed automata}


A timed automaton is a finite automaton extended with a finite set of clocks $\Omega$ that all progress at the same rate and that can individually be reset to zero. Moreover, every transition is labeled by a guard over 
$\Omega$  and by a set of clocks to be reset. \\

\par\noindent\ignorespacesafterend
Formally, a {\em timed automaton} ({\em TA} for short) is a tuple
$\A=(Q,\Omega,R,q_{init},F)$, where
\begin{samepage}
\begin{itemize}
	\item $Q$ is a non-empty finite {\em set of states}, 
	\item $\Omega$ is a non-empty finite {\em set of clocks},
	\item $R \subseteq Q\times\G(\Omega)\times \mathscr{P}(\Omega) \times Q$
	is a finite {\em set of  rules},
	\item $q_{init}\in Q$ is an {\em initial  state}, and 
	\item $F\subseteq Q$ is a {\em set of final states}.
\end{itemize}
\end{samepage}

\par\noindent\ignorespacesafterend
We also refer to $\A$ as an $n$-TA if $|\Omega| = n$. 
The {\em size} of $\A$ is defined as%:
$$
|\A| \ = \ |Q|+|\Omega|+|R|+\sum_{(q,g,U,q')\in R}|g|.
$$
Let 
$\Const(\A) = \{ c\in\N \mid \exists(q,g,U,q')\in R, \ \exists \omega \in \Omega, \bowtie\in\{<,\leq,=,\geq,>\} : g = \omega \bowtie c \}$ denote the 
set of constants that appear in the guards of the rules of $\A$.

By $\Conf(\A)=Q\times\N^\Omega$ we denote the set of
{\em configurations} of $\A$. 
%We prefer however to denote a configuration in $\Const(\A)$ by $q(v)$ instead of $(q,v)$.\\
We prefer however to abbreviate a configuration	%	 in $\Conf(\A)$	
$(q,v)$ by $q(v)$.


\begin{samepage}
%\begin{definition}
A TA $\A=(Q,\Omega,R,q_{init},F)$ induces the labeled transition system 
$T_{\A} =  (\Conf(\A), \Lambda_{\A}, \rightarrow_{\A})$
where $ \Lambda_{\A} = R \times \N $
and 
where $ \rightarrow_{\A}$ is defined such that, 
for all $(\delta,t)\in R\times\N$ with  	$\delta = (q,g,U,q')\in R$,
for all $q(v), q'(v') \in \Conf(\A)$,
$q(v)\xrightarrow{\delta,t}_{\A} q'(v')$ if
	$v+t\models g$, 
	 $v'(u)=0$ for all $u \in U$ and $v'(\omega)=v(\omega)+t$ for all 
	$\omega \in \Omega \setminus U$.
%\end{definition}
\end{samepage}

A {\em run} from $q_0(v_0)$ to $q_n(v_n)$ in $\A$ is a path in the transition system $T_{\A}$, that is,
a sequence 
$\pi = q_0(v_0)\xrightarrow{\delta_1,t_1}_{\A}q_1(v_1)\cdots\xrightarrow{\delta_n,t_n}_{\A}q_n(v_n)$;
it is called {\em reset-free} if for all $i \in \{1,\ldots,n\}$,
 $\delta_i = (g_i,\emptyset)$ for some guard $g_i$.


We say $\pi$ is {\em accepting} if $q_0(v_0) = q_{init}(\vec{0})$ and $q_n \in F$. 
\iffalse
We say {\em reachability holds} for the TA $\A$ 
if there exists an accepting run. %
% if there is a run in $\A$ from $q_{init}(\vec{0})$   to some configuration $q(v)$ for some $q\in F$, and $v\in\N^\Omega$.
%We refer to Figure~\ref{example pta} for an instance of a PTA for which reachability holds.
\fi


It is worth mentioning that there are further modes of time valuations and guards which exist in the literature, we refer
to \cite{Andre19} for a recent overview. 
%
% \mh{Comment on difference between continuous and discrete time}
Notably, we consider in this article only the case of timed automata over discrete time. It is worth mentioning that in
the case of timed automata over continuous time (i.e. with clocks having values in $\R_{\geq 0}$),
% However, for parametric timed automata with closed (i.e., non-strict) clock constraints and parameters restricted to ranging over integers, 1 standard digitisation techniques apply [HMP92, OW03], reducing the reachability problem over dense time to discrete (integer) time.
techniques~\cite{HenzingerMP92,OuaknineW03} exist for reducing the reachability problem to discrete time in the case of closed (i.e. non-strict) clock constraints ranging over integers. \\




\problemx{TA $k$-resilience problem}
{A state $q$ of a TA $(Q, X, \Delta)$, a set $SAFE \subseteq Q$, a set $BAD \subseteq Q$.}
{$\forall q' \in BAD \forall v,v' \in \N^X ~ (q(v) \rightarrow^* q'(v')) \implies \exists q'' \in SAFE \exists v'' \in \N^X ~ q'(v') \rightarrow^{\leq k} q''(v'')$ ?\newline}


Analogously, we formulate the bounded resilience problem for WSTSs.


\problemx{TA bounded resilience problem}
{A state $q$ of a TA $(Q, X, \Delta)$, a set $SAFE \subseteq Q$, a set $BAD \subseteq Q$.}
{$\exists k \geq 0 ~ \forall q' \in BAD \forall v,v' \in \N^X ~ (q(v) \rightarrow^* q'(v')) \implies \exists q'' \in SAFE \exists v'' \in \N^X ~ q'(v') \rightarrow^{\leq k} q''(v'')$ ?\newline}

\textcolor{red}{I think there can be a discussion to be had here about how to quantify on the clock valuations}

\textcolor{red}{Here one thing that could be interesting to try to formalize is: how to enforce that the time that passes is less than $k$, rather than the number of transitions. This is tricky to deal with I find but it should be more doable if for instance we use one counter automata, where the counter effect of the sequence can be quantified more explicitly I suppose ?
But here you could also use a kinda special clock $x$ that is reset when you enter $BAD$ and is not reset between a state in $BAD$ and a state in $SAFE$, you could check that $x < k$.}

\textcolor{red}{... I guess if you use $0/1$-TA then the problems become closer one to another ? Also of note is that $0/1$-TA induces transition systems with bounded branching, so I guess it may be interesting to investigate these first ?}

A {\em $0/1$ timed automaton } ({\em $0/1$-TA} for short) is a tuple
$$\B=(Q,X, \Delta_0, \Delta_1, q_{init}, F),$$
\par\noindent\ignorespacesafterend
 where
$\B_i=(Q,X, R_i, q_{init}, F)$ is a TA for all $i \in \{0,1\}$.
For simplicity we define its {\em size}
as $|\B|=|\B_0|+|\B_1|$.
We analogously denote the constants of $\B$ 
by $\Const(\B)$ and its configurations by  $\Conf(\B)$.

\begin{samepage}
%\begin{definition}
A $0/1$ timed automaton $\B=(Q,X,R_0,R_1,q_{init},F)$ 
induces the labeled transition system 
$T_{\B} = (\Conf(\B), \lambda_{\B}, \rightarrow_{\B}) $
where $ \lambda_{\B} = (R_0 \cup R_1) \times \{ 0,1\}$
	and where $ \rightarrow_{\B}$
	is defined such that
	for all $q(z), q'(z') \in \Conf(\B)$, 
	for all $(\delta,i) \in \lambda_{\B}$
	with $\delta  = (q,g,U,q')\in R_i$
	$q(v)\xrightarrow{\delta,i}_{\B} q'(v')$ if
	$v+i \models g$, 
	$v'(u)=0$ for all $u \in U$ and $v'(\omega)=v(\omega)+ i$ for all $\omega \in \Omega
	\setminus U$. 
%\end{definition}
\end{samepage}



As expected, we write $q(v)\xrightarrow{\delta,i}_{\B}q'(v')$ if 
$q(v)\xrightarrow{\delta,i}_{\B}q'(v')$ for some 
$i\in\{0,1\}$, and some $\delta \in R_i$.




\subsection{One-Counter Automata}

\textcolor{red}{Should be defined in a later 'application section' once we start writing any proof, for now I leave it there} 


\problemx{OCA $k$-resilience problem}
{A state $q$ of a OCA $(Q, \Delta)$, a set $SAFE \subseteq Q$, a set $BAD \subseteq Q$.}
{$\forall q' \in BAD \forall n,n' \in \N ~ (q(n) \rightarrow^* q'(n')) \implies \exists q'' \in SAFE \exists n'' \in \N ~ q'(n') \rightarrow^{\leq k} q''(n'')$ ?\newline}



\problemx{OCA bounded resilience problem}
{A state $q$ of a OCA $(Q, \Delta)$, a set $SAFE \subseteq Q$, a set $BAD \subseteq Q$.}
{$\exists k \geq 0 ~ \forall q' \in BAD \forall n,n' \in \N ~ (q(n) \rightarrow^* q'(n')) \implies \exists q'' \in SAFE \exists n'' \in \N ~ q'(n') \rightarrow^{\leq k} q''(n'')$ ?\newline}



%
%
%\subsection{Vector Addition System with States or PN}
%
%\textcolor{red}{Should be defined in a later 'application section' once we start writing any proof, for now I leave it there} 
%



