
\section{Resilience for downward-WSTS}

Let us recall the definition of downward-WSTS.

\begin{definition} \cite{DBLP:journals/tcs/FinkelS01}
A \emph{downward-WSTS} is an ordered transition system $(S, \rightarrow, \leq)$
such that $\leq$ is a wqo and
%  decidable ≤  wqo on S, i.e., for each two given states s, s ′ ∈ S, it is decidable whether s ≤ s ′ . 
the transition relation $ \rightarrow$ is downward-compatible with $\leq$, i.e., for all  transition $s_1 \rightarrow t_1$
with $s_1, t_1 , s_2 \in S$
	and $s_2 \leq s_1$, there exists a transition $s_2 \rightarrow^{*} t_2$ with $t_2 \leq t_1$. 		
\end{definition}

%		enlever l'hypothèse finitely branching ?
A downward-WSTS has the \emph{reflexive} compatibility if for all transition  $s_1 \rightarrow t_1$ with
$s_1, t_1 , s_2 \in S$ and $s_2 \leq s_1$, either $t_1 \leq s_2$ or there exists a transition $s_2 \rightarrow t_2$ with $t_2 \leq t_1$.\\

Downward-WSTS with reflexive compatibility enjoy a nice reachability property.

\begin{theorem}[Proposition 5.4 in \cite{DBLP:journals/tcs/FinkelS01}]
%	Since by Proposition 5.4 in \cite{DBLP:journals/tcs/FinkelS01}, we know that  
For finitely branching downward-WSTS $(S, \rightarrow, \leq)$, with reflexive compatibility, decidable $\leq$ and effective $post$, a finite basis of $\uparrow post^*(s_0)$ is computable for any $s_0 \in S$.
\end{theorem}

%	Let us say that a downward-WSTS is \emph{effective}
Using Corollary \ref{postcomputable}, we deduce that resilience is decidable for finitely branching downward-WSTS with reflexive compatibility and with strong upward compatibility, with a decidable $\leq$ and an effective $post$.

%%%%%
\alain{peut-on faire mieux ? enlever des hypothèses ?}

Let us establish a general Lemma on downward-closed and upward-closed sets.

\begin{lemma}\label{Lemma intersection}
Let $A \subseteq S$, $J \subseteq S$ be a downward-closed set and $I \subseteq S$ be an upward-closed set. 
Then $A \cap J \subseteq I$  iff $ (\uparrow  A) \cap J \subseteq I$.
\end{lemma}


\begin{proof}
Let us suppose that $A \cap J \subseteq I$. Then ${\uparrow (A \cap J)} \subseteq {\uparrow I} = I$.
Let us show that $({\uparrow A}) \cap J \subseteq {\uparrow (A \cap J)}$.
Let $x \in ({\uparrow A}) \cap J$, then there exists $a \in A$ such that $x \geq a$.
Since $x \in J$ and $J$ is downward-closed, we also have $a \in J$.
Hence $a \in A \cap J$ and then $x \in { \uparrow (A \cap J)}$.
In the other direction,
since $A \subseteq {\uparrow A}$, the inclusion
$({\uparrow  A}) \cap J \subseteq I$ implies
$A \cap J \subseteq ({\uparrow  A}) \cap J \subseteq I$.
\end{proof}

Let us apply the previous Lemma with $A=post^*(s)$.

\begin{corollary}
For all $k \in \N$ and $s \in S$,
$ post^*(s)\cap J \subseteq I^k $  iff $ (\uparrow  post^*(s)) \cap J \subseteq I^k$. 
% et travailler avec $\downarrow post^*(s)$ à la place de $\uparrow post^*(s)$.
\end{corollary}

% La question deviens: quel $k$ pour que
% $\downarrow post^*(s) \cap J \subseteq I^k$.
% Ou alors, à $k$ fixé, est-ce que
% $\downarrow post^*(s) \cap J \subseteq I^k$.



Assume $k$ is fixed for now.

{\bf Explanations on the two procedures.}
Procedure 1 enumerates inductive invariants in some fixed order $D_1$ , $D_2$ , . . . , i.e. upward closed subsets $D_i \subseteq S$ such that $\uparrow post(D_i ) \subseteq D_i$. \alain{write U for upward closed subset and not D as for downward...}
Every inductive invariant $D_i$ is an “over-approximation” of $\uparrow post^*(s)$ if it contains $s$.
(on énumère des sur-approximations de la cloture par le haut de $post*(s)$ par leur bases finies).
Each “over-approximation” $D_i$ is given by its basis $b(D_i)$. Notice that, by standard monotonicity, $\uparrow post^*(s)$ is such an inductive invariant and may
eventually be found.
% \alain{attention D est clos par le bas et $\uparrow post^*(s)$ est clos par le haut, je ne vois pas pourquoi $\uparrow post^*(s) \subseteq D$}
% \mathieu{Oui effectivement les D devraient être clos par le haut, je corrige ça à la prochaine update.}

Procedure 1 stops when it finds a basis $b(D)$ of an invariant $D$ such that
$b(D)  \cap J \subseteq I^k$.  Since $b(D)$ is finite and $J$ is decidable, we can
directly compute $b(D)  \cap J$.
% We can compute a basis of $I^{k+1}$ if we have a basis of $I^k$.
We can compute a basis
of $I^{k+1}$ if we have a basis of $I^k$. %This follows by the proof of Lemma 3.
Due to the Lemma ????, 
$b(D)  \cap J \subseteq I^k$ implies
$D  \cap J \subseteq I^k$.
Hence
$b(D)  \cap J \subseteq I^k$ implies
$\uparrow post^*(s) \cap J \subseteq D  \cap J \subseteq I^k$.
(since $ \uparrow post^*(s)  \subseteq D$).



The second procedure iteratively computes
$post^{\leq n}(s) \cap J$
until it finds an element
not in $ I^k$.


This result is not in xxx and don't use the problematic hypothsis on $post^*$.
\alain{ben si car downward-WSTS + qqhypothèses (voir plus haut) ont un $ \uparrow post^*(s)$ calculable !
peut-on faire sauter qq hypothèses ?  finitely branching ??? with reflexive compatibility ????}

% Et une fois que l'on a ça, on peut écrire

% Ce serait une procédure qui pourrait par exemple calculer $post^m(s) \cap J $ jusqu'à trouver un élément pas dans $I^k$ ?

% Càd on calcule les éléments de $post^m(s)$ un à un, pour chacun on vérifie s’il est dans $J$ puis s’il y est on vérifie s’il est dans $I^k$ ?

% Il faut aussi une base de $J$ et que $I^k$ soit décidable, mais ça a l'air faisable j'ai l'impression...




\begin{figure}
\fbox{\parbox[t][3.cm][c]{6cm}{
$\phantom{a}$\\
(1)\phantom{aaaaa} $i \leftarrow 0$\\
(2)\phantom{aaaaa}\textbf{while} $\neg( \uparrow post^*(D_i) \subseteq D_i $
			 and $ s \in D_i$
			 and $ b(D)  \cap J \subseteq I^k  )$ \textbf{loop} \newline
	(3)\phantom{aaaaaa}$\phantom{aaaa} i \leftarrow i +1$ \newline
(4)\phantom{aaaaa}\textbf{end loop} \newline
(5)\phantom{aaaaa}\textbf{return} $\text{\textit{false}}$ \newline
	}
}
	\caption{\textbf{Procedure RES:} enumerates inductive invariants to find an inclusion certificate.}\label{procedure1}
\end{figure}




\begin{figure}
\fbox{\parbox[t][3.cm][c]{6cm}{
$\phantom{a}$\\
(1)\phantom{aaaaa} $D \leftarrow \{ s \} $\\
(2)\phantom{aaaaa}\textbf{while} $D \cap J \subseteq I^k $ \textbf{loop} \newline
	(3)\phantom{aaaaaa}$\phantom{aaaa} D \leftarrow D \cup post(D)$ \newline
(4)\phantom{aaaaa}\textbf{end loop} \newline
(5)\phantom{aaaaa}\textbf{return} $\text{\textit{false}}$ \newline
	}
}
	\caption{\textbf{Procedure NOT-RES:} searches for a non-inclusion certificate.}\label{procedure2}
\end{figure}


We show that these two procedures allow to solve the $k$-resilience.

\begin{proposition}
The procedures $RES$ and $NOT-RES$ have the following property.
\begin{itemize}
%
\item $k$-resilience holds if and only if Procedure $RES$ terminates.
%
\item $k$-resilience do not hold if and only if Procedure $NOT-RES$ terminates.
%
\end{itemize}
\end{proposition}


\begin{proof}

\begin{enumerate}

\item By a simple induction, it can be shown that $\uparrow post^*(D) \subseteq D$ for every inductive invariant $D$. 
If Procedure 1 terminates, then
$post^*(s) \cap J \subseteq \uparrow post^*(s) \cap J \subseteq D  \cap J \subseteq I^k$
which implies that $k$-resilience holds.

It remains to show that Procedure 1 terminates whenever $k$-resilience holds. To do so, it suffices to prove that $\uparrow post^*(s)$ is an inductive invariant. Indeed, this implies that
$\uparrow post^*(s)$ is eventyally found by Procedure 1 when $k$-resilience holds. 

Formally, let us show that $\uparrow post(\uparrow post^*(s)) \subseteq \uparrow post^*(s)$.
Let $b \in \uparrow post(\uparrow post^*(s))$ 
there exists $a', a, b$ such that
$s \rightarrow^* a'$,
$a' \leq a$,
$a \rightarrow b'$,
and
$b' \leq b$.
% By monotonicity, there exists b'' ≥ b' such that a → b'' . Therefore, x → ∗ b'' and b' ≥ b, hence b ∈ ↓ Post ∗ (x).
By downward compatibility 
there exists $b'' \leq b'$ such that $a \rightarrow b'' $. Therefore, $x \rightarrow^* b''$ and $b' \geq b$, hence $b \in \downarrow post^*(x)$.
%\alain{où est cette hypothèse downward compatibility  ?}
\item Procedure 2 computes
 $post^{\leq n}(s) \cap J$
 until it finds an element not in $ I^k$.

If Procedure 2 terminates, then
$k$-resilience does not hold.
It remains to show that Procedure 2 terminates whenever $k$-resilience does not hold.
Assume $ post^*(s) \cap J \not\subseteq I^k$, then there exists $a \in post^*(s) \cap J$ such that $a \not\in I^k$. Since $post^*(s) = \bigcup_{n} post^{\leq n}(s)$, then 
$a \in post^*(s)$
implies
there exists
$n_a$
such that
$a \in post^{\leq n_a}(s)$.
Hence,  $post^{\leq n_a}(s) \cap J$ contains an element not in 
$I^k$,
and Procedure 2 terminates.
\end{enumerate}

\end{proof}


\begin{theorem}
{\sc $k$-resilience} is decidable for effective \alain{pas défini}WSTS with downward/strong upward compatibility.
%	with strong monotony.
\end{theorem}
\alain{les resultatts et preuves ne sont pas faciles à lire: faire un seul niveau de parentheses}
\begin{proof}
%			\begin{proof}

Assume  $(S, \rightarrow, \leq)$ is a WSTS with downward/upward compatibility, $J$ is a decidable downward-closed subset of $S$, and $I$ is an upward-closed set with a given basis.
We define inductively
$I^{k} = \bigcup_{0 \leq j \leq k} \pred^j(I)$. Note that for all $k \in \N$, $I^k$ is upward-closed due to
the strongly upward compatibility of $(S, \rightarrow, \leq)$.

% Si on inverse les types de propriétés (downward closed pour $I = SAFE$, upward closed pour $J= BAD$) alors on peut écrire un lemme symmétrique du Lemme $4$:

The $k$-resilience property can be expressed as the formula
$ post^*(s) \cap J \subseteq I^k$. In order to decide whether the inclusion holds, we execute two procedures in parallel, one trying to prove $ post^*(s)\cap J \subseteq I^k$ 
and one looking for a counter example.

%    In order to certify inclusion in $I^k$, we need to work with finite representations.
%    The next lemma \alain{where ? the previous lemmas ? numbers ?} 
%    uses that $I$ and $J$ are upward- and downward-closed, respectively.
\end{proof}


\begin{corollary}
{\sc bounded resilience} and {\sc resilience} are decidable for WSTS with downward/strong upward compatibility.
\alain{+effective hypothesis}
\end{corollary}

\begin{proof}{sketch}

Iteratively check whether $k$-resilience holds. 
If this is the case, return $k_{min} = k$. Otherwise check whether 
$I^{k+1} = I^k$. If so, return $-1$ (false), otherwise
continue.
The stop condition is decidable
and by Fact~\ref{stop condition} also sufficient. 
\end{proof}

\alain{explorer d'autres hypothèses...}
%
%{\sc resilience} is decidable for WSTS with strong upward compatibility and k-resilience.
%
%jouer avec les hypothèses downward/upward compatibility


