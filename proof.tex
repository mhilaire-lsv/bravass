
\section{Decidability}

\textcolor{red}{Pour l'instant je regarde la décidabilité du problème "symmétrique" c'est à dire que j'ai inversé toutes les propriétés et demandé à ce que SWSTS soit downward-compatible. Je pense ça doit être possible de faire les choses autrement ...}

\begin{theorem}
{\sc SWSTS $k$-resilience} is decidable.
\end{theorem}

\begin{proof}{sketch}

Assume  $(S, \rightarrow, \leq)$ is a SWSTS with downward compatibility, $I$ is a decidable downward-closed subset of $S$, and $J$ is an upward-closed set with a given basis.

We define inductively
$I^{k+1} = I \cup pre(I^k )$. Note that for all $k \in \N$, $I^k$ is downward-closed due to
the strongly downward compatibility of $(S, \rightarrow, \leq)$.

% Si on inverse les types de propriétés (downward closed pour $I = SAFE$, upward closed pour $J= BAD$) alors on peut écrire un lemme symmétrique du Lemme $4$:

The $k$-resilience property can be expressed as the formula
$ post^*(s) \cap J \subseteq I^k$. In order to decide whether the inclusion holds, we execute two procedures in parallel, one trying to prove $ post^*(s)\cap J \subseteq I^k$ 
and one looking for a counter example.

In order to certify inclusion in $I^k$, we need to work with finite representations.
The next lemma uses that $I'$ and $J$ are downward- and upward-closed, respectively.

\begin{lemma}
Let $A \subseteq S$ be a set, $J \subseteq S$ upward-closed and $I' \subseteq S$ downward-closed. 
Then $A \cap J \subseteq I' \leftrightarrow (\downarrow  A) \cap J \subseteq I'$.
\end{lemma}

% Et une fois que l'on a ça, on peut écrire

\begin{corollary}
For all $k \in \N$,
$ post^*(s)\cap J \subseteq I^k \leftrightarrow (\downarrow  post^*(s)) \cap J \subseteq I^k$. 
% et travailler avec $\downarrow post^*(s)$ à la place de $\uparrow post^*(s)$.
\end{corollary}

% La question deviens: quel $k$ pour que
% $\downarrow post^*(s) \cap J \subseteq I^k$.
% Ou alors, à $k$ fixé, est-ce que
% $\downarrow post^*(s) \cap J \subseteq I^k$.


Assume $k$ is fixed for now.

Procedure 1 enumerates inductive invariants in some fixed order $D_1$ , $D_2$ , . . . , i.e. downward closed subsets $D_i \subseteq S$ such that $\downarrow Post(D_i ) \subseteq D_i$. 
Every inductive invariant $D_i$ is an “over-approximation” of $post^*(s)$ if it contains $s$.
(on énumère des sur-approximations de la cloture par le bas de $post*(s)$ par leur bases finies).
Each “over-approximation” $D_i$ is given by its basis $b(D_i)$. Notice that, by standard monotonicity, $\downarrow post^*(s)$ is such an inductive invariant and may
eventually be found.


Procedure 1 stops when it finds a basis $b(D)$ of an invariant $D$ such that
$b(D)  \cap J \subseteq I^k$.  Since $b(D)$ is finite and we know the basis for $J$, we can
directly compute $b(D)  \cap J$.
% We can compute a basis of $I^{k+1}$ if we have a basis of $I^k$.
Due to the Lemma, 
$b(D)  \cap J \subseteq I^k$ implies
$D  \cap J \subseteq I^k$.
Hence
$b(D)  \cap J \subseteq I^k$ implies
$\downarrow post^*(s) \cap J \subseteq D  \cap J \subseteq I^k$.
(since $D$ contains $post^*(s)$).



The second procedure iteratively computes
$post^{\leq n}(s) \cap J$
until it finds an element
not in $ I^k$.

% Ce serait une procédure qui pourrait par exemple calculer $post^m(s) \cap J $ jusqu'à trouver un élément pas dans $I^k$ ?

% Càd on calcule les éléments de $post^m(s)$ un à un, pour chacun on vérifie s’il est dans $J$ puis s’il y est on vérifie s’il est dans $I^k$ ?

% Il faut aussi une base de $J$ et que $I^k$ soit décidable, mais ça a l'air faisable j'ai l'impression...




\begin{figure}
\fbox{\parbox[t][3.cm][c]{6cm}{
$\phantom{a}$\\
(1)\phantom{aaaaa} $i \leftarrow 0$\\
(2)\phantom{aaaaa}\textbf{while} $\neg( \downarrow post^*(D_i) \subseteq D_i $
			 and $ s \in D_i$
			 and $ b(D)  \cap J \subseteq I^k  )$ \textbf{loop} \newline
	(3)\phantom{aaaaaa}$\phantom{aaaa} i \leftarrow i +1$ \newline
(4)\phantom{aaaaa}\textbf{end loop} \newline
(5)\phantom{aaaaa}\textbf{return} $\text{\textit{false}}$ \newline
	}
}
	\caption{\textbf{Procedure 1:} enumerates inductive invariants to find an inclusion certificate.}\label{procedure1}
\end{figure}




\begin{figure}
\fbox{\parbox[t][3.cm][c]{6cm}{
$\phantom{a}$\\
(1)\phantom{aaaaa} $D \leftarrow \{ s \} $\\
(2)\phantom{aaaaa}\textbf{while} $D \cap J \subseteq I^k $ \textbf{loop} \newline
	(3)\phantom{aaaaaa}$\phantom{aaaa} D \leftarrow D \cup post(D)$ \newline
(4)\phantom{aaaaa}\textbf{end loop} \newline
(5)\phantom{aaaaa}\textbf{return} $\text{\textit{false}}$ \newline
	}
}
	\caption{\textbf{Procedure 2:} searches for a non-inclusion certificate.}\label{procedure2}
\end{figure}

\newpage


We show that these two procedures are correct:

\end{proof}



