
\section{Decidability}

\begin{proof}{sketch}


Si on inverse les types de propriétés (downward closed pour $I = SAFE$, upward closed pour $J= BAD$) alors on peut écrire un lemme symmétrique du Lemme $4$:

\begin{lemma}
Let $A \subseteq S$ be a set, $J \subseteq S$ upward-closed and $I' \subseteq S$ downward-closed. 
Then $A \cap J \subseteq I' \leftrightarrow (\downarrow  A) \cap J \subseteq I'$.
\end{lemma}

Et une fois que l'on a ça, on peut écrire
$ post^*(s)\cap J \subseteq I' \leftrightarrow (\downarrow  post^*(s)) \cap J \subseteq I'$ 
et travailler avec $\downarrow post^*(s)$ à la place de $\uparrow post^*(s)$.

La question deviens: quel $k$ pour que
$\downarrow post^*(s) \cap J \subseteq I^k$.
Ou alors, à $k$ fixé, est-ce que
$\downarrow post^*(s) \cap J \subseteq I^k$.

Il faut aussi une condition pour s'assurer que $I^k$ soit bien downward-closed.

Où
$I^{k+1} = I \cup pre(I^k )$.

On vas imaginer temporairement qu'on arriver à s'assurer $I^k$ downward-closed. 
(Peut-être on doit faire la supposition de SWSTS strongly downward compatible à la place, pour avoir ça ?)

On reviens sur $\downarrow post^*(s) \cap J \subseteq I^k$.

C'est là peut être qu'on peut s'inspirer de l'algo forward.

Enumerates inductive invariants in some fixed order $D_1$ , $D_2$ , . . . , i.e. downward closed subsets $D_i \subseteq X$ such that $\downarrow Post(D_i ) \subseteq D_i$. [...]
Every inductive invariant $D_i$ is an “over-approximation” of $post^*(x)$ if it contains $x$.

On a un nombre fini de bornes supérieures pour $D_i$ on peut considèrer qu’elles consistent en une base de $D_i$ ? On énumère les $D_i$ en énumérant leurs bases directement ?

Ensuite, si jamais	$Base(D_1)  \cap J \subseteq I^k$.

alors

$D_1  \cap J \subseteq I^k$	(par le lemme)

et donc

$\downarrow post*(s) \cap J \subseteq D_1  \cap J \subseteq I^k$

(puisque $D_1$ est une sur-approximation).

Ce qui voudrais dire que on a bien la propriété de résilience.


Le problème évidement c'est que cette procédure ne termine peut être pas enfin on n'a aucune guarrantie à priori. À ce moment là il faudrait sans doute faire une autre procédure en parallèle, comme dans l'algo forward, et qui elle terminerais si on a {\em pas} la propriété de résilience.

 La propriété de $k$-résilience peut être résumée par la formule: $post^*(s) \cap J \subseteq I^k$.

Donc la question ce serait est-ce que, à $k$ fixé ($k$ fixé pour l'instant on vas dire) on aurais une procédure qui termine pour montrer que ce n'est pas le cas ?

Ce serait une procédure qui pourrait par exemple calculer $post^m(s) \cap J $ jusqu'à trouver un élément pas dans $I^k$ ?

Càd on calcule les éléments de $post^m(s)$ un à un, pour chacun on vérifie s’il est dans $J$ puis s’il y est on vérifie s’il est dans $I^k$ ?

Il faut aussi une base de $J$ et que $I^k$ soit décidable, mais ça a l'air faisable j'ai l'impression...


\textcolor{red}{Ça me perturbe un peu quand même de devoir inverser toutes les propriétés et avoir SWSTS downward-compatible, je pense ça doit être possible de faire les choses mieux...}



\end{proof}



