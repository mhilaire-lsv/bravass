
\section{Decidability}

\begin{theorem}
{\sc $k$-resilience} is decidable for effective WSTS with strong monotony.
\end{theorem}


\begin{proof}

Assume  $(S, \rightarrow, \leq)$ is a SWSTS with upward compatibility, $J$ is a decidable downward-closed subset of $S$, and $I$ is an upward-closed set with a given basis.

We define inductively
$I^{k} = \bigcup_{0 \leq j \leq k} pre^j(I)$. Note that for all $k \in \N$, $I^k$ is upward-closed due to
the strongly upward compatibility of $(S, \rightarrow, \leq)$.

% Si on inverse les types de propriétés (downward closed pour $I = SAFE$, upward closed pour $J= BAD$) alors on peut écrire un lemme symmétrique du Lemme $4$:

The $k$-resilience property can be expressed as the formula
$ post^*(s) \cap J \subseteq I^k$. In order to decide whether the inclusion holds, we execute two procedures in parallel, one trying to prove $ post^*(s)\cap J \subseteq I^k$ 
and one looking for a counter example.

In order to certify inclusion in $I^k$, we need to work with finite representations.
The next lemma uses that $I$ and $J$ are upward- and downward-closed, respectively.

\begin{lemma}
Let $A \subseteq S$ be a set, $J \subseteq S$ downward-closed and $I \subseteq S$ upward-closed. 
Then $A \cap J \subseteq I \leftrightarrow (\uparrow  A) \cap J \subseteq I$.
\end{lemma}

% Et une fois que l'on a ça, on peut écrire

\begin{corollary}
For all $k \in \N$,
$ post^*(s)\cap J \subseteq I^k \leftrightarrow (\uparrow  post^*(s)) \cap J \subseteq I^k$. 
% et travailler avec $\downarrow post^*(s)$ à la place de $\uparrow post^*(s)$.
\end{corollary}

% La question deviens: quel $k$ pour que
% $\downarrow post^*(s) \cap J \subseteq I^k$.
% Ou alors, à $k$ fixé, est-ce que
% $\downarrow post^*(s) \cap J \subseteq I^k$.



Assume $k$ is fixed for now.

Procedure 1 enumerates inductive invariants in some fixed order $D_1$ , $D_2$ , . . . , i.e. downward closed subsets $D_i \subseteq S$ such that $\uparrow post(D_i ) \subseteq D_i$. 
Every inductive invariant $D_i$ is an “over-approximation” of $\uparrow post^*(s)$ if it contains $s$.
(on énumère des sur-approximations de la cloture par le haut de $post*(s)$ par leur bases finies).
Each “over-approximation” $D_i$ is given by its basis $b(D_i)$. \textcolor{red}{Notice that, by standard monotonicity, $\uparrow post^*(s)$ is such an inductive invariant and may
eventually be found.}


Procedure 1 stops when it finds a basis $b(D)$ of an invariant $D$ such that
$b(D)  \cap J \subseteq I^k$.  Since $b(D)$ is finite and $J$ is decidable, we can
directly compute $b(D)  \cap J$.
% We can compute a basis of $I^{k+1}$ if we have a basis of $I^k$.
We can compute a basis
of $I^{k+1}$ if we have a basis of $I^k$. %This follows by the proof of Lemma 3.
Due to the Lemma, 
$b(D)  \cap J \subseteq I^k$ implies
$D  \cap J \subseteq I^k$.
Hence
$b(D)  \cap J \subseteq I^k$ implies
$\uparrow post^*(s) \cap J \subseteq D  \cap J \subseteq I^k$.
(since $D$ contains $ \uparrow post^*(s)$).



The second procedure iteratively computes
$post^{\leq n}(s) \cap J$
until it finds an element
not in $ I^k$.

% Ce serait une procédure qui pourrait par exemple calculer $post^m(s) \cap J $ jusqu'à trouver un élément pas dans $I^k$ ?

% Càd on calcule les éléments de $post^m(s)$ un à un, pour chacun on vérifie s’il est dans $J$ puis s’il y est on vérifie s’il est dans $I^k$ ?

% Il faut aussi une base de $J$ et que $I^k$ soit décidable, mais ça a l'air faisable j'ai l'impression...




\begin{figure}
\fbox{\parbox[t][3.cm][c]{6cm}{
$\phantom{a}$\\
(1)\phantom{aaaaa} $i \leftarrow 0$\\
(2)\phantom{aaaaa}\textbf{while} $\neg( \uparrow post^*(D_i) \subseteq D_i $
			 and $ s \in D_i$
			 and $ b(D)  \cap J \subseteq I^k  )$ \textbf{loop} \newline
	(3)\phantom{aaaaaa}$\phantom{aaaa} i \leftarrow i +1$ \newline
(4)\phantom{aaaaa}\textbf{end loop} \newline
(5)\phantom{aaaaa}\textbf{return} $\text{\textit{false}}$ \newline
	}
}
	\caption{\textbf{Procedure 1:} enumerates inductive invariants to find an inclusion certificate.}\label{procedure1}
\end{figure}




\begin{figure}
\fbox{\parbox[t][3.cm][c]{6cm}{
$\phantom{a}$\\
(1)\phantom{aaaaa} $D \leftarrow \{ s \} $\\
(2)\phantom{aaaaa}\textbf{while} $D \cap J \subseteq I^k $ \textbf{loop} \newline
	(3)\phantom{aaaaaa}$\phantom{aaaa} D \leftarrow D \cup post(D)$ \newline
(4)\phantom{aaaaa}\textbf{end loop} \newline
(5)\phantom{aaaaa}\textbf{return} $\text{\textit{false}}$ \newline
	}
}
	\caption{\textbf{Procedure 2:} searches for a non-inclusion certificate.}\label{procedure2}
\end{figure}

\newpage


We show that these two procedures are correct:


\begin{enumerate}

\item $k$-resilience holds if, and only if, Procedure 1 terminates.

\item $k$-resilience do not hold if, and only if, Procedure 2 terminates.

\end{enumerate}

Proof:

\begin{enumerate}

\item By a simple induction, it can be shown that $\uparrow post^*(D) \subseteq D$ for every inductive invariant $D$. 
If Procedure 1 terminates, then
$post^*(s) \cap J \subseteq \uparrow post^*(s) \cap J \subseteq D  \cap J \subseteq I^k$
which implies that $k$-resilience holds.

It remains to show that Procedure 1 terminates whenever $k$-resilience holds. To do so, it suffices to prove that $\uparrow post^*(s)$ is an inductive invariant. Indeed, this implies that
$\uparrow post^*(s)$ is eventyally found by Procedure 1 when $k$-resilience holds. 

Formally, let us show that $\uparrow post(\uparrow post^*(s)) \subseteq \uparrow post^*(s)$.
Let $b \in \uparrow post(\uparrow post^*(s))$ 
there exists $a', a, b$ such that
$s \rightarrow^* a'$,
$a' \leq a$,
$a \rightarrow b'$,
and
$b' \leq b$.
% By monotonicity, there exists b'' ≥ b' such that a → b'' . Therefore, x → ∗ b'' and b' ≥ b, hence b ∈ ↓ Post ∗ (x).
\textcolor{red}{By downward compatibility 
there exists $b'' \leq b'$ such that $a \rightarrow b'' $. Therefore, $x \rightarrow^* b''$ and $b' \geq b$, hence $b \in \downarrow post^*(x)$.}
\alain{où est cette hypothèse downward compatibility  ?}
\item Procedure 2 computes
 $post^{\leq n}(s) \cap J$
 until it finds an element not in $ I^k$.

If Procedure 2 terminates, then
$k$-resilience does not hold.
It remains to show that Procedure 2 terminates whenever $k$-resilience does not hold.
Assume $ post^*(s) \cap J \not\subseteq I^k$, then there exists $a \in post^*(s) \cap J$ such that $a \not\in I^k$. Since $post^*(s) = \bigcup_{n} post^{\leq n}(s)$, then 
$a \in post^*(s)$
implies
there exists
$n_a$
such that
$a \in post^{\leq n_a}(s)$.
Hence,  $post^{\leq n_a}(s) \cap J$ contains an element not in 
$I^k$,
and Procedure 2 terminates.
\end{enumerate}





\end{proof}


\begin{theorem}
{\sc $k$-resilience} is undecidable for effective WSTS ???
\end{theorem}

\begin{theorem}
{\sc resilience} is decidable for WSTS with strong monotony.
\end{theorem}

\begin{proof}{sketch}

Iteratively check whether $k$-resilience holds. 
If this is the case, return $k_{min} = k$. Otherwise check whether 
$I^{k+1} = I^k$. If so, return $-1$ (false), otherwise
continue.
The stop condition is decidable
and by Fact 4 also sufficient. 
\alain{Fact 4 don't exist}
\end{proof}

