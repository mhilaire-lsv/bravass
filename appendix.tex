

\section{Reset-VASS and state-resilience}\label{appendix}


% \section{Proof of Theorem~\ref{liveness reset}}\label{appendix}

% Recall Theorem~\ref{liveness reset}.


%\begin{proposition}\label{liveness reset}
% The  zero reachability problem can be reduced to the $t$-liveness problem in reset-VASS.
%\end{proposition}


Let us recall the undecidable \cite{araki1976PN} decision problem of {\em zero-reachability} in reset-VASS, which consists in, given a
reset-VASS $V=(Q,T)$, and $p(\textbf{u}) \in Q \times \N^d$,
deciding whether $\exists q \in Q ~ p(\textbf{u}) \to^* q(\textbf{0})$.
We describe therein how to adjust Theorem~5.5 from \cite{peterson1981petri} reducing the problem to {\sc $t$-liveness}. \\

\noindent
\textbf{Proposition~\ref{liveness reset}}
%\label{liveness reset}
{\em The  zero reachability problem can be reduced to {\sc $t$-liveness}.}



\begin{proof}


We wish to determine if $q(\textbf{0})$ is reachable in a reset-VASS $V_1$
with
initial vector $\textbf{u}$ and initial \alain{control-}state $q_0$.\alain{incohérent: control-state !, notation $q_0(\textbf{u})$  à utiliser}
To do so, we construct a reset-VASS $V_2$
with initial vector $\textbf{v}$ and initial state $q'_0$ \alain{idem}
which is live if and only if 
$q(\textbf{0})$
is reachable from $q_0(\textbf{u})$.


The new reset-VASS is constructed by adding four new vector coordinate
and
$|V_1|+4$ new transitions for every control-state.

We first modify all transitions to include a $-1$ on the first additional vector coordinate and a $+1$ on the second additional vector coordinate. We also add a transition with only $+1$ on the first additional vector coordinate and $-1$ on the second one.

% The reset PN $C_2$ is constructed from $C_1$ by the addition of two places $r_1$ and $r_2$ and $|P_1| +2$ transitions $s$, $\{ s_p | p \in P_1 \}$ and $s'$.

% We first modify all transitions of $T_1$ to include $r_1$ as both an input and an output.

The initial vector $\textbf{v}$ is constructed from $\textbf{u}$ by having at least $1$ on the first two additional vector coordinate.
\alain{un dessin aiderait...}

% The initial marking $\mu_2$ will include a token in $r_1$. 

These two vector coordinate serve to mark that the run is ongoing. As long as they are both non-zero, the adjusted transitions of $V_1$ are live and can be used normally. Thus any vector which is reachable in $V_1$ is also reachable% [in $V_2$]
.

% The place $r_1$ is a 'run' place; as long as the token remains in $r_1$ the transitions of $T_1$ can fire normally.
% Thus any marking which is reachable in the places of $P_1$ is also reachable in $C_2$.


We add an additional transition $s$ which reset the first two additional vector coordinate from control-state $q$.
% Transition $s$ is defined to have $r_1$ as an input and a null output.
This allows to disable the transitions of $V_1$ and to "freeze" the vector of $V_1$.
% This allows the token in $r_1$ to be removed, disabling all transitions in $T_1$ and "freezing" the marking of $P_1$.

% Note that all transitions of $T_1$ are in conflict and, by construction if not by definition, no more than one transition can fire at a time.
\alain{donne des noms, numéros,...}
These first two additional vector coordinate together with this additional transition allow
to reach any reachable vector and control-state $q$ and then to freeze the configuration.

% The place $r_1$ and transition $s$ allow the net $C_1$ to reach any reachable marking and then for $s$ to fire and freeze the net at that marking.

Now we need to see if that vector is zero.
% Now we need to see if that marking is zero.
\alain{structurer la preuve pour la rendre lisible, nommer les objets, faire des paragraphes,...}
We introduce two new additional vector coordinate and $d$ new transitions which have
$-1$ on the $d$-th vector coordinate and $+1$ on this penultimate coordinate. 

% We introduce a new place $r_2$ and new transitions $s_p$ which have $p$ as input and $r_2$ as output.

If $q(\textbf{0})$ is reachable, then these $d$ new transitions are not live. Everything is dead if $s$ fires in $q(\textbf{0})$.

% If $p$ can ever become zero, $s_p$ is not live. The entire net is dead if $s$ fires in the marking zero.

If we can always have some vector coordinate be not zero, then we can always fire some transition to add $+1$ to the penultimate additional coordinate.
% If we can always have some $p_i$ not be zero, then we can always fire some $s_{p_i}$, putting a token in $r_2$.
In this case we can add to the first additional coordinate again and assure that all transitions are live.
% In this case we must put a token back in $r_1$ and assure that all transitions in $C_2$ are live.
We must be sure that $V_2$ is live even if $V_1$ is {\em not} live.
% We must be sure that $C_2$ is live even if $C_1$ is {\em not} live.
% This is accomplished by a transition $s'$ which  "floods" the net $C_2$ with tokens, assuring that every transition is live if a token is ever put in $r_2$. Transition $s'$ has $r_2$ as its input and every place of $C_2$ as output
This is accomplished by a transition $s'$ which arbitrarily increments every other vector coordinate 
while decrementing the penultimate one and, additionally, a transition which 
substract from the last additional vector coordinate and add to the penultimate one. 


Now if $q(\textbf{0})$ is reachable from $q_0(\textbf{u})$ in $V_1$, then $V_2$ can also reach a similar
vector $q(0,0,0,\ldots, 0, 1,0,0,0)$ by executing the same sequence of transitions.
Then $s$ can freeze the $V_1$ subset,
% Now if $0$ is reachable in $R(C_1, \mu_1)$  then the net $C_2$ can also reach this marking in the place of $P_1$ by executing the same sequence of transition firings. Then $s$ can fire, freezing the $C_1$ subset.
ending up in $q(0^{d+4})$, and $V_2$ is not live.
% Since $mu(p_i) = 0$ for all $p_i \in P_1$, no transition $s_{p_i}$ with $p_i \in P_1$ can fire, and $C_2$ is dead.
Thus if $q(\textbf{0})$ is reachable then $V_2$ is not live.


Conversely, if $V_2$ is not live, then a vector is reachable in which the last two coordinate are equal to $0$ and there is no reachable vector in which either is equal to at least one.
% Conversely, if $C_2$ is not live, then a marking $\mu$ must be reachable in which  $\mu(r_2) = 0$ and there is no reachable state in which $r_2$ has a token (in particular, since we do not allow token removal from $r_2$, the marking $\mu$ must be reached in a sequence of transitions that do not place any token in $r_2$.).
If it is not possible to have a strictly positive integer in one of the last two coordinates, then the transitions are not live, and the
$d$-dimensional subset of the current vector must be zero.
% If $r_2$ has no token and cannot get any, then the transitions $s_p$ are not live, and the markings of $p$ for $p \in P_1$ must be zero. 
% Thus if $C_2$ is not live then a marking is reachable in which the marking of each $p$ in $P_1$ is zero. 
Thus if $V_2$ is not live then 
$q(\textbf{0})$ is reachable.
\end{proof}








