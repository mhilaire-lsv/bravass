
\section{Proof of Theorem~\ref{liveness reset}}\label{appendix}

Recall Theorem~\ref{liveness reset}.

\mathieu{À refaire avec des Reset VASS}

\begin{theorem*}[Adjusting Theorem~5.5 from \cite{peterson1981petri}]
The  Reset Petri Net zero reachability problem can be reduced to the liveness problem.
\end{theorem*}

Let us describe therein how to adjust Theorem~5.5 from \cite{peterson1981petri}.

\begin{proof}
If we wish to determine if $0 \in R(C_1, \mu_1)$ in any
reset Petri net $C_1 = (P_1, T_1, I_1, O_1, R_1 )$ (where $R_1(t) =$ places that are reset when t fires off), with initial marking $\mu_1$, we construct a
reset petri net $C_2 = (P_2, T_2, I_2, O_2, R_2 )$ with initial marking
$\mu_2$ which is live if and only if the zero marking is not reachable from $\mu_1$.

The reset PN $C_2$ is constructed from $C_1$ by the addition of two places $r_1$ and $r_2$
and $|P_1| +2$ transitions $s$, $\{ s_p | p \in P_1 \}$ and $s'$.

We first modify all transitions of $T_1$ to include $r_1$ as both an input and an output.
The initial marking $\mu_2$ will include a token in $r_1$. 


The place $r_1$ is a 'run' place; as long as the token remains in $r_1$ the transitions of $T_1$ can fire normally.
Thus any marking which is reachable in the places of $P_1$ is also reachable in $C_2$.



Transition $s$ is defined to have $r_1$ as an input and a null output.
This allows the token in $r_1$ to be removed, disabling all transitions in $T_1$
and "freezing" the marking of $P_1$.

Note that all transitions of $T_1$ are in conflict and, by construction if not by definition, no more than one transition can fire at a time.

The place $r_1$ and transition $s$ allow the net $C_1$ to reach any reachable marking and then for $s$ to fire and freeze the net at that marking.

Now we need to see if that marking is zero.

We introduce a new place $r_2$ and new transitions $s_p$ which have $p$ as input and $r_2$ as output.

If $p$ can ever become zero, $s_p$ is not live. The entire net is dead if
$s$ fires in the marking zero.

If we can always have some $p_i$ not be zero, then we can always fire some $s_{p_i}$, putting a token in $r_2$.

In this case we must put a token back in $r_1$ and assure that all transitions in $C_2$ are live.
We must be sure that $C_2$ is live even if $C_1$ is {\em not} live.
This is accomplished by a transition $s'$ which  "floods" the net $C_2$ with tokens, assuring that every transition is live if a token is ever put in $r_2$.
Transition $s'$ has $r_2$ as its input and every place of $C_2$ as output

Now if $0$ is reachable
in $R(C_1, \mu_1)$ 
then the net $C_2$ can also reach this marking
in the place of $P_1$ by executing the same sequence of transition firings.
Then $s$ can fire, freezing the $C_1$ subset.
Since $mu(p_i) = 0$ for all $p_i \in P_1$, no transition $s_{p_i}$ with $p_i \in P_1$ can fire, and $C_2$ is dead.
Thus if $0$ is reachable then $C_2$ is not live.

Conversey, if $C_2$ is not live, then a marking $\mu$ must be reachable in which 
$\mu(r_2) = 0$ and there is no reachable state in which $r_2$ has a token
(in particular, since we do not allow token removal from $r_2$, the marking $\mu$ must be reached in a sequence of transitions that do not place any token in $r_2$.)
% (Rappel: tant que il y a un token quelque part, on peut mettre un token en r_2
% et r_2 commence sans token)
If $r_2$ has no token and cannot get any, then the transitions $s_p$ are not live, and the markings of $p$ for $p \in P_1$ must be zero. 
Thus if $C_2$ is not live then a marking is reachable in which the marking of each $p$ in $P_1$ is zero. 

\end{proof}


\section{More Appendix thing if necessary}\label{appendix B}
