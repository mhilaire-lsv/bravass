
%%%
\iffalse
%%%%
%%%%
\subsection{Defining resilience}


\subsubsection{Resilience for general transition systems}


We ask whether we can reach a state 
%which satisfies
in 
%
$\Safe$  in a reasonable amount of time whenever we reach a state 
% which satisfies
in
%
$\Bad$. 
From this we formulate two resilience problems. First consider the case where the recovery time
is bound by a given natural number $k \geq 0$, i.e., the \emph{explicit resilience problem} for TS.

\problemx{$k$-Resilience problem }
{a transition system $(S,\rightarrow)$, an integer $k$, a state $s \in S$, $\Safe, \Bad \subseteq S$ and $\Safe \cap \Bad = \emptyset$.}
{$\forall s' \in \Bad, ~ s \rightarrow^* s' \implies \exists s'' \in \Safe ~ s' \rightarrow^{\leq k} s''$ ?\newline}

If a system $S$ satisfies the explicit resilience property for an integer $k$, we say that $S$ is $k$-resilient.

% If we assume that the transition system yields infinite sequences of transitions, we can express the property to be evaluated in CTL by s |= AG(\Bad → 0≤ j≤k EX j \Safe). 

We can also ask whether there exists such a bound $k$ such that $S$ is $k$-resilient. We call this problem the \emph{bounded resilience problem}.

%
%\problemx{Bounded Resilience problem}
%{a transition system $(S,\rightarrow)$, a state $s \in S$, $\Safe, \Bad \subseteq S$ and $\Safe \cap \Bad = \emptyset$.}
%{$\exists k \geq 0 ~ \forall s' \in \Bad, ~ s \rightarrow^* s' \implies \exists s'' \in \Safe ~ s' \rightarrow^{\leq k} s''$ ?\newline}
%


\problemx{Bounded Resilience problem}
{a transition system $(S,\rightarrow)$, a state $s \in S$, $\Safe, \Bad \subseteq S$ and $\Safe \cap \Bad = \emptyset$.}
{$\exists k \geq 0  \mathscr{S}$ is $k$-resilient. ?\newline}
%%

%%%


pour $k,k'$ donnés, chercher si $\Safe_{max}$ et $\Bad_{max}$ ont un sens ? on peut faire l'union des \Safe, \Safe' et des \Bad, \Bad' prendre le max des k. Fixer k, chercher et calculer $\Safe_{max}$ et $\Bad_{max}$.

%
%\alain{dire juste que \Safe = upward closed et \Bad = downward closed.
%envisager les 3 autres cas. souvent \Bad = upward closed (par ex l'exclusion mutuelle).
%4 resilience problem (U,D), (U,U), (D,U), (D,D)}

\fi




\newcommand{\Bad}{\textsf{Bad}}
\newcommand{\Safe}{\textsf{Safe}}



\section{Resilience for WSTS}


The resilience problem (resp. the $k$-resilience problem) in a transition system is to decide whether from a bad state, there exists a path (resp. a path of length smaller than $k$) that reaches a safe state. Let us formalize these properties.

\problemx{resilience problem (RP)}
{A transition system $\mathscr{S}=(S,\rightarrow)$ and two sets $\Safe, \Bad \subseteq S$.}
{$\Bad \longrightarrow^{*} \Safe$ ?\newline}
%
%\alain{il faudrait ne pas répéter 3 fois les mêmes imputs pour les 3 pbs: énoncer les 3 uniformes pbs d'un coup avec une fois l'input puis les 3 pbs pour un état s donné sans répéter non plus 3 fois les mêmes inputs}

\problemx{$k$-resilience problem (kRP)}
{A transition system $\mathscr{S}=(S,\rightarrow), k \in \mathbb{N}$ and two sets $\Safe, \Bad \subseteq S$.}
{$\Bad \longrightarrow^{\leq k} \Safe$ ?\newline}

We add a third problem that decides whether there exists an $k$ such  that the system is $k$-resilient.
%
\problemx{bounded resilience problem (BRP)}
{A transition system $\mathscr{S}=(S,\rightarrow)$ and two sets $\Safe, \Bad \subseteq S$.}
%{$\exists k \geq 0 ~ \forall s' \in D ~ s \rightarrow^* s' \implies \exists s'' \in U ~ s' \rightarrow^{\leq k} s''$ ?\newline}
{$\exists k \geq 0$ such that $\mathscr{S}$ is %uniformely
 $k$-resilient ?\newline}

\alain{trouver les \Bad~ et \Safe~ maximum tels que S est resilient. est-ce vrai que si S est $(B_i,D_i)$-resilient alors S est $(\cap, \cup B_i,D_i)$-resilient ? on pourrait décider que \Safe=complement de \Bad~ et donc on aurait un seul ensemble (et son complement)}

These three reachability problems are decidable for finite transition systems but undecidable for (general) infinite-state transition systems. 
So we restrict our framework to the class of infinite-state WSTS. Since most of decidable properties in WSTS are given as effective upward or downward closed sets \cite{DBLP:journals/iandc/AbdullaCJT00, DBLP:journals/tcs/FinkelS01}, we consider upward closed or downward closed sets $\Safe$ and $\Bad$.
\alain{on peut penser à des ensembles $\Bad$ definis dans une logique booleennne sur les clos par le bas, haut, +...}

For instance, the well-known mutual exclusion property is often modelized in a $d$-counters machine by the property that a counter $c_{mutex}$ must be bounded by (usually) one. Then, the set $\Safe =  \{c_{mutex} \leq 1\} \times \mathbb{N}^{d-1}$ is downward closed and $\Bad =\{c_{mutex} \geq 2\} \times  \mathbb{N}^{d-1} $ is the upward closed complementary of $\Safe$. In \cite{DBLP:conf/gg/Ozkan22}, the authors considered that $\Bad$ is always downward closed and $\Safe$ is always upward closed.
%		
RP is decidable for lossy counter machines (LCM) with $\Safe$ and $\Bad$ semilinear sets as a consequence of results in Section 3.4 of \cite{DBLP:conf/rp/Schnoebelen10}. In particular, the existing proofs of resilience use the decidability of reachability; but we wish to decide resilience for models with undecidable reachability.
%
In what follows, we will always suppose that a downward closed set $D$ is given by its finite set of ideals $I_D=\{J_1, J_2,...,J_n\}$ satisfying $D=\downarrow D = \downarrow (J_1 \cup J_2 \cup..\cup J_n)$ and that an upward closed set $U$ is given by its finite set $Min_U$ of its minimal elements satisfying $U=\uparrow Min_U$.

Let $U,V$ be two upward-closed sets given by its set of minimal element and $D,E$ be two downward-closed sets given by its ideal decomposition. All the four  inclusions $U \subseteq V$,  $U \subseteq D$, $D \subseteq U$ and $D \subseteq E$ are decidable provided that the ordering is decidable and sets are recursive.


\begin{proposition}[Reformulation]\label{reformulation}
$\mathscr{S}=(S,\rightarrow,\leq)$ is %uniformely 
(\Bad,\Safe)-resilient iff $\Bad \subseteq \pred^*(\Safe)$.\\
$\mathscr{S}=(S,\rightarrow,\leq)$ is %uniformely 
(\Bad,\Safe)-$k$-resilient iff $\Bad \subseteq \pred^k(\Safe)$.
\end{proposition}


\iffalse

\begin{proposition}\label{general}
\textcolor{red}{
$\mathscr{S}=(S,\rightarrow,\leq)$ is %uniformely 
(\Bad,\Safe)-bounded-resilient iff $\mathscr{S}=(S,\rightarrow,\leq)$ is %uniformely 
(\Bad,\Safe)-resilient (CONJECTURE FOR NOW)
}
\end{proposition}

\begin{proof}
Resilient means that ...
\end{proof}

\fi


\subsection{Case: $\Safe=\uparrow \Safe$ and $\Bad=\uparrow \Bad$.}




\begin{theorem}\label{up-up}
The resilience problem is decidable for WSTS with effective predbasis, $\Safe=\uparrow \Safe$
and $\Bad=\uparrow \Bad$.

\end{theorem}


\begin{proof}
% Let us first solve if $\mathscr{S}=(S,\rightarrow,\leq)$ is (\Bad,\Safe)-resilient. 
Since $\Safe=\uparrow \Safe$ and
$\mathscr{S}=(S,\rightarrow,\leq)$ is a WTSTS,  $\pred^*(\Safe)=\uparrow \pred^*(\Safe)$ and $\pred^*(\Safe)$ admits a finite basis $B_{\pred^*(\Safe)}$. Since $\mathscr{S}=(S,\rightarrow,\leq)$ is a WSTS  with effective predbasis, we may compute a finite basis of $\pred^*(\Safe)$ with the backward coverability algorithm. 
Since $\Bad$  is upward closed, there exists a finite basis $B_{\Bad}$ such that $\Bad = \uparrow B_{\Bad}$. Moreover $ \uparrow B_{\Bad} \subseteq \uparrow B_{\pred^*(\Safe)}$ iff for every $b \in B_{\Bad}$, there is a $s \in B_{\pred^*(\Safe)}$ such that $s \leq b$,
hence the resilience problem is decidable.
\end{proof}

\begin{proposition}
In WSTS with strong compatibility and effective predbasis,  $\Safe=\uparrow \Safe$, the bounded resilience problem is equivalent to the resilience problem.
\end{proposition}

\begin{proof}
Since $\Safe=\uparrow \Safe$ and
$\mathscr{S}=(S,\rightarrow,\leq)$ is a WTSTS with strong %upward-
compatibility, then $\pred^n(\Safe)= \uparrow~\pred^n(\Safe)$ for all $n \in \N$,
and there exists $n_0 \in \N$ such that 
$\pred^{n_0}(\Safe) = \uparrow \pred^{n_0}(\Safe) = \uparrow \pred^*(\Safe) = \pred^*(\Safe)$.
Hence the equivalence.
\end{proof}

\begin{corollary}\label{B-up-up}
The bounded resilience problem is decidable for WSTS with effective predbasis,
strong compatibility
 $\Safe=\uparrow \Safe$
and $\Bad=\uparrow \Bad$.
\end{corollary}


\begin{theorem}\label{k-up-up}
The % three %uniform 
$k$-resilience problem is decidable for WSTS with effective predbasis, strong %upward-
compatibility, $\Safe=\uparrow \Safe$
and $\Bad=\uparrow \Bad$.
%	and \Bad is upward closed or downward closed.
\end{theorem}

\begin{proof}
Since $\Safe=\uparrow \Safe$ and
$\mathscr{S}=(S,\rightarrow,\leq)$ is a WTSTS with strong %upward-
compatibility, then $\pred^k(\Safe)= \uparrow~\pred^k(\Safe)$ for all $k \in \N$, and $\pred^k(\Safe)$ admits a finite basis $B_{\pred^k(\Safe)}$. From a similar reasoning as above, the $k$-resilience problem is decidable.\alain{réserver k pour le k-résilience, il y a d'autres lettres...n, m, p,...}
\end{proof}




\subsection{Case: $\Safe=\uparrow \Safe$ and $\Bad=\downarrow \Bad$.}

%
%		cas étudié par les deux articles sur les graphes
%


%  Parosh Aziz Abdulla, Karlis Cerans, Bengt Jonsson & Yih-Kuen Tsay (1996): General Decidability Theorems for Infinite-State Systems. In: Proc. LICS 1996, IEEE Computer Society Press, pp. 313–321,
%  Alain Finkel & Philippe Schnoebelen (2001): Well-structured transition systems everywhere! Theor. Comput. Sci. 256(1-2), pp. 63–92, doi:10.1016/S0304-3975(00)00102-X.

%Transfering the abstract resilience problems into this framework,
%it is therefore reasonable to demand that both propositions, \Safe and \Bad, are given by 
%upward-closed or downward-closed sets.

The \emph{completion}  \cite{BFM-ic17} of a WSTS $\mathscr{S}=(S,\rightarrow, \leq)$ is the associated ordered transition system $\hat{\mathscr{S}}=(Ideals(S),\rightarrow, \subseteq)$ where states of $\hat{\mathscr{S}}$ are ideals of $S$ and $I \rightarrow J$ if $J$ belongs to the finite ideal decomposition of $\downarrow \post_{\mathscr{S}}(I)$. The completion is always finitely branching but it is not necessarly WSTS since $\subseteq$ is not necessarly a wqo. $\hat{\mathscr{S}}$ is WSTS iff $\mathscr{S}=(S,\rightarrow, \leq)$ is $\omega^2$-WSTS (intuitively speaking, $(S,\leq)$ must not contain the Rado set). Coverability is shown decidable  [Theorem 44] in \cite{BFM-ic17} for completion-post-effective $\omega^2$-WSTS.

Let us recall two other results in \cite{BFM-ic17}. Proposition 30 establishes a strong relation between the runs of a WSTS $\mathscr{S}=(S,\rightarrow, \leq)$ and the runs of its completion $\hat{\mathscr{S}}$. It states that if $x \xrightarrow{k} y$ in $\mathscr{S}$ then for every ideal $I \supseteq \downarrow x$, there exists an ideal $J \supseteq \downarrow y$ such that $I \xrightarrow{k} J$ in $\hat{\mathscr{S}}$. Proposition 29 establishes that if $I \xrightarrow{k} J$ in $\hat{\mathscr{S}}$ then for every $y \in J$, there exists $x \in I$ and $y' \geq y$ such that $x \xrightarrow{k'} y'$ in $\mathscr{S}$. Moreover, if $\mathscr{S}$ has transitive compatibility then $k’ \geq k$; if $\mathscr{S}$ has strong compatibility then $k’ = k$.


\begin{theorem}\label{down-up}
Let $\mathscr{S}=(S,\rightarrow, \leq)$ be a completion-post-effective $\omega^2$-WSTS with strong compatibility and two 
%finite \alain{non, confusion entre ensemble et base}
 sets $\Bad = \downarrow \Bad$ and $\Safe = \uparrow \Safe$.
The resilience problem (RP) and the bounded resilience problem (BRP) are decidable.
\end{theorem}

\begin{proof}
Let $\{J_1, J_2,...,J_n\}$ be the ideal decomposition of $\Bad$ and $B_{\Safe}=\{b_1,b_2,...,b_m\}$ be the minimal basis of $\Safe$.
The %uniform 
resilience problem (RP) is equivalent to an infinite number of instances of the coverability problem : for all $x \in \Bad$ does there exist an $j$ such that $b_j$ is coverable from $x$. This infinite set of coverability questions can be reduced to a finite number of instances of the coverability problem in the completion $\hat{\mathscr{S}}=(Ideals(S),\rightarrow, \subseteq)$ of $\mathscr{S}=(S,\rightarrow, \leq)$.

%Proposition 30 in \cite{BFM-icalp14} establishes a strong relation between the runs of a WSTS $\mathscr{S}=(S,\rightarrow, \leq)$ and the runs of its completion $\hat{\mathscr{S}}$. It states that if $x \xrightarrow{k} y$ in $\mathscr{S}$ then for every ideal $I \supseteq \downarrow x$, there exists an ideal $J \supseteq \downarrow y$ such that $I \xrightarrow{k} J$ in $\hat{\mathscr{S}}$. Proposition 29 establishes that if $I \xrightarrow{k} J$ in $\hat{\mathscr{S}}$ then for every $y \in J$, there exists $x \in I$ and $y' \geq y$ such that $x \xrightarrow{*} y'$ in $\mathscr{S}$.

Let us show that $b_j$ is coverable from $x$ in $\mathscr{S}$ iff $\downarrow b_j$ is coverable (for inclusion) from $\downarrow x$ in $\hat{\mathscr{S}}$.

Suppose that $b_j$ is coverable from $x$ then there exists a run $x \xrightarrow{k} y \geq b_j$. From Proposition 30, there exist an ideal $J$ and a run $\downarrow x \xrightarrow{k} J$ where $J \supseteq \downarrow y \supseteq \downarrow b_j$ in $\hat{\mathscr{S}}$, hence $b_j$ is covered from $\downarrow x$.
Conversely, if $I \xrightarrow{k} J$ in $\hat{\mathscr{S}}$ with $\downarrow b_j \subseteq J$ then 
%	for every $y \in J$, 
there exists $x \in I$ and $y' \geq b_j$ such that $x \xrightarrow{k} y'  \geq b_j$ in $\mathscr{S}$ and then $b_j$ is coverable from $x$ in $\mathscr{S}$.
%
This proves that resilience and bounded resilience are equivalent since, here, resilience implies bounded resilience (bounded resilience always implies resilience).
%
% iff there is a run in $\hat{S}$ from an ideal I to J such that  $\downarrow x \subseteq I$ and $\downarrow b_j \subseteq J$. This is a consequence of  
%Propositions 29 and 30 in \cite{BFM-icalp14} that establish a strong relation between the runs of a WSTS $\mathscr{S}=(S,\rightarrow, \leq)$ with its completion $\hat{S}$.
%' \in I(\Bad)$.

To decide the  bounded resilience problem (BRP), we decide $n \times m$ coverability questions: is state $b_j$ coverable from ideal $J_i$ ? If all these $n \times m$ coverability questions are positive then we compute $K=max(k_{i,j} \mid i=1,..,m $ and $j= 1,..m)$ where $k_{i,j}$ is the least length of a sequence that covers  $b_j$ from $J_i$.
\alain{peut-être definir $K_{ \mathscr{S}}(Safe,Bad)$ en dehors de la preuve puisqu'on l'utilise ailleurs (théorème suivant) comme suit: If all these $n \times m$ coverability questions are positive then  $K(Safe,Bad)=max(k_{i,j})$ else $K=\infty$}
 % \alain{comment trouver les $k_a$ ?} 
 by iteratively checking the different possibles values of every $k_{i,j}$ until we find the minimal one,
 and we deduce that  $\mathscr{S}$ is $K$-resilient,
%  \alain{else $\mathscr{S}$ is not $K$-resilient...et alors qu'en déduit-on ? il pourrait exister un $K' \geq K$ pour lequel $\mathscr{S}$ est K'-resilient....}. 
else if some of these $n \times m$ coverability questions are negative then resilience
do not hold and bounded resilience do not hold either.
 Since coverability is decidable for completion-post-effective $\omega^2$-WSTS, we have shown that both the 
  resilience problem (RP) and the  
  bounded resilience problem (BRP) are decidable.
\end{proof}


\begin{theorem}\label{k-down-up}
<<<<<<< HEAD
Let $\mathscr{S}=(S,\rightarrow, \leq)$ be a completion-post-effective $\omega^2$-WSTS with strong compatibility and the predbasis hypothesis, and two finite sets: $\Bad$ and $\Safe$.
The  $k$-resilience problem (kRP) is decidable 
%	for $k \geq min(K,n)$ where $n$ satisfies $ \uparrow \pred^n(\Safe)=  \uparrow \pred^*(\Safe)$. 
\textcolor{red}{CONJECTURE for now}
 \end{theorem}

\begin{proof}
We begin to compute $K$ and $n$ such that $ \uparrow \pred^n(\Safe)=  \uparrow \pred^*(\Safe)$.
If $K=\infty$ then $\mathscr{S}$ is not resilient for Safe and Bad.
Now, if $k \geq min(K,n)$, we conclude that is $k$-resilient; 
%	if moreover $\mathscr{S}=(S,\rightarrow, \leq)$ has the predbasis hypothesis, 
%	and then kRP is decidable.
	If $k <  min(K,n)$, then we can answer the question by checking every paths of length smaller
	than $min(K,n)$ from the ideals of the decomposition of $\Bad$ in the completion \alain{à expliquer mieux}.
\end{proof}

%




\subsection{Case: $\Safe=\downarrow \Safe$ and $\Bad=\downarrow \Bad$.}

% In the case \Safe and \Bad are both downward closed, they can both be finite.

\begin{theorem}\label{down-down}
The resilience problem is undecidable for WSTS with more than one minimal element, 
$\Safe=\downarrow \Safe$
and $\Bad=\downarrow \Bad$.
\end{theorem}

\begin{proof}
In the case of a WSTS with at least two minimal elements $0_1$ and $0_2$, the problem of whether $0_2$ is reachable from $0_1$ reduces itself to the resilience problem, hence its undecidability.  
\end{proof}

In the case of a WSTS with only one minimal element,  
% (ex: Petri Nets and their «Zero») 
then by downward closure the minimal element is present in both $\Safe$ and $\Bad$ and hence resilience always hold.






%%%%%%%%
\subsection{Case: $\Safe=\downarrow \Safe$ and $\Bad=\uparrow \Bad$.}
%
%		cas exclusion mutuelle
%

\begin{theorem}\label{down-up}
\textcolor{red}{
The resilience problem is undecidable for WSTS with more than one minimal element, 
$\Safe=\downarrow \Safe$
and $\Bad=\uparrow \Bad$.
}
\end{theorem}

\begin{proof}
TBD
\end{proof}


\subsection{Case Synthesis}

\mathieu{En rouge quand il s'agit de conjectures}



\begin{center}
\begin{tabular}{ | l | c | c | c | r |}
\hline   \Safe~\Bad & $\uparrow$~ $\uparrow$~ & $\uparrow$~ $\downarrow$~ & $\downarrow$~ $\downarrow$~ & $\downarrow$~ $\uparrow$~ \\ \hline
   RP & Decidable (Theorem~\ref{up-up}) & Decidable (Theorem~\ref{down-up}) & Undecidable (Theorem~\ref{down-down}) & \textcolor{red}{Undecidable} \\ \hline
   BRP & Decidable (Corollary~\ref{B-up-up}) &  Decidable (Theorem~\ref{down-up}) & ?? & ?? \\ \hline
      kRP & Decidable (Theorem~\ref{k-up-up}) & \textcolor{red}{Decidable} (Theorem~\ref{k-down-up}) & ?? & ?? \\ \hline
 \end{tabular}
\end{center}




%We first assume that the safety property is given by an upward-closed set and the bad condition by a decidable downward-closed set. 
% \textcolor{red}{Seems like a reasonable assumption to me.}

%From these considerations, we formulate instances of the abstract resilience problems for well-
%structured transition systems.


