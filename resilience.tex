
\section{Resilience for WSTS}


In this section we introduce the concept of resilience, bounded-resilience and $k$-resilience for well structured transition systems. We show that the resilience problems are decidable
for  completion-post-effective $\omega^2$-WSTS with strong compatibility
in the case $\Bad$ is downward-closed and $\Safe $ is upward-closed,
and that resilience is decidable for downward-compatible ideal-effective WSTS
when $\Safe$ is downward-closed and $\Bad $ is upward-closed. 
We show that the three resilience problems are however undecidable for WSTS with
strong compatibility in general.


In a transition system $\mathscr{S}=(S,\rightarrow)$, we consider two subsets of states $\Safe, \Bad \subseteq S$ such that $\Safe \cap  \Bad = \emptyset$ and $\Safe ,  \Bad \neq \emptyset$.
%	The property of mutual exclusion is often modelised with $\Safe = \downarrow \Safe$.
The \emph{resilience problem} (resp. the \emph{$k$-resilience problem}) for $(\mathscr{S},\Safe,\Bad)$ is to decide whether from \emph{any} state in $\Bad$, \emph{there exists} a path (resp. a path of length smaller than or equal to $k$) that reaches a state in $\Safe$. We use the notation $\Bad \longrightarrow^{*} \Safe$ (resp. $\Bad \longrightarrow^{\leq k} \Safe$) for $\forall x \in \Bad, \exists y \in \Safe$ such that $x \longrightarrow^{*} y$ (resp.  $\forall x \in \Bad, \exists y \in \Safe$ such that $x \longrightarrow^{\leq k} y$). In our framework, $\Safe, \Bad \subseteq S$  are possibly infinite but they must admit a computable finite representation : for example, downward-closed sets and upward-closed sets in wqo and semilinear sets in $\mathbb{N}^d$ have finite representations. 


{\bf Related problems.} 
A set of configurations $H$ is an {\em home-space} for a set of configurations $X$ if every configuration reachable from $X$ can reach $H$. Resilience as introduced above essentially asks whether $\Safe$ is a home-space for $\Bad$, whereas $k$-resilience and bounded-resilience focus more on the length of the paths from $\Bad$ to $\Safe$. The home-space problem is decidable for
Petri nets when $X$ and $H$ are both semilinear sets~\cite{DBLP:journals/corr/abs-2207-02697}.
To these author's knowledge, there are no general results for home-space in WSTS.

% \mathieu{vérifier au passage que c'est vrai qu'il y a pas de résultats home-space WSTS}

Resilience and the home-space problem are also linked to the 
model-checking of the ``from-all'' formula $\forall s \in X~ \exists t \in Y~ s \to^* t$
which has been shown decidable for Lossy Counter Machine
when $X$ and $Y$ are semi-linear sets~\cite{DBLP:conf/rp/Schnoebelen10}.
Other decidable formulae include ``one-to-one'' ($\exists s \in  X ~ \exists t \in  Y ~ s \to^* 
 t$), and ``all-to-same'' ($\exists t \in  Y ~ \forall s \in  X ~ s \to^*  t$),
whereas ``one-to-all'' ($\exists s \in  X ~ \forall t \in  Y ~ s \to^*  t$), 
``all-to-all'' ($\forall s \in  X ~ \forall t \in  Y ~ s \to^*  t$)
  and ``to-all'' ($\forall t \in  Y  ~ \exists s \in  X ~ s \to^*  t$) are undecidable (again, for Lossy Counter Machines). 
  


Let us formalize three resilience problems.


\problemx{resilience problem (RP)}
{A transition system $\mathscr{S}=(S,\rightarrow)$ and two sets $\Safe, \Bad \subseteq S$.}
{$\Bad \longrightarrow^{*} \Safe$ ?\newline}
%
%\alain{il faudrait ne pas répéter 3 fois les mêmes imputs pour les 3 pbs: énoncer les 3 uniformes pbs d'un coup avec une fois l'input puis les 3 pbs pour un état s donné sans répéter non plus 3 fois les mêmes inputs}

%We add a third problem that decides whether there exists an $k$ such  that the system is $k$-resilient.
%

\problemx{$k$-resilience problem (kRP)}
{A transition system $\mathscr{S}=(S,\rightarrow), k \in \mathbb{N}$ and two sets $\Safe, \Bad \subseteq S$.}
{$\Bad \longrightarrow^{\leq k} \Safe$ ?\newline}

\problemx{bounded resilience problem (BRP)}
{A transition system $\mathscr{S}=(S,\rightarrow)$ and two sets $\Safe, \Bad \subseteq S$.}
%{$\exists k \geq 0 ~ \forall s' \in D ~ s \rightarrow^* s' \implies \exists s'' \in U ~ s' \rightarrow^{\leq k} s''$ ?\newline}
{$\exists k \geq 0$ such that $\mathscr{S}$ is %uniformely
 $k$-resilient ?\newline}


\begin{example}
Let us consider a VASS with only one state $q$, one counter, and one transition that substracts $-1$ from the counter, with $\Safe = \downarrow q(0)$, $\Bad = \uparrow q(1)$,
then resilience hold (one always ends up in the positive integers) but not bounded resilience: for every bound $k$ there is an element of $\Bad$ e.g. $q(-k-1)$ which necessitate at least $k+1$ steps before it can reach $\Safe$. 
\end{example}


% \begin{remark}[Reformulation]\label{reformulation}
% $\mathscr{S}=(S,\rightarrow,\leq)$ is %uniformely 
% (\Safe, \Bad)-resilient iff $\Bad \subseteq \pred^*(\Safe)$.\\
% $\mathscr{S}=(S,\rightarrow,\leq)$ is %uniformely 
% (\Safe, \Bad)-$k$-resilient iff $\Bad \subseteq \pred^{\leq k}(\Safe)$.
% \end{remark}


  
These three resilience problems are decidable for finite transition systems but undecidable for (general) infinite-state transition systems. So we restrict our framework to the class of infinite-state WSTS. Since most of decidable properties in WSTS rely on the computation of upward or downward-closed sets \cite{DBLP:journals/iandc/AbdullaCJT00, DBLP:journals/tcs/FinkelS01}, we consider upward-closed or downward-closed sets $\Safe$ and $\Bad$. In \cite{DBLP:journals/corr/abs-2108-00889}, the authors considered that $\Bad$ is downward-closed and $\Safe$ is upward-closed.

Surprinsingly, we have not found decidability results about the resilience problems for WSTS.
%		
We deduce 
% \textcolor{red}{that the three resilience problems are decidable for all four pairs of sets $\Safe$ and $\Bad$ downward-closed and upward-closed sets.}
% \mathieu{changer ça}
decidability results for the four pairs of sets $\Safe$ and $\Bad$ downward-closed and upward-closed as seen in Section~\ref{synthesis}.
In particular, the existing proofs of resilience use the decidability of reachability; but we wish to decide resilience for models with undecidable reachability.
%
%	 and ideals are recursive 
Our result will rely on coverability instead of reachability. We moreover show undecidability when coverability is not decidable in the case $\Safe = \uparrow \Safe$. 






While considering a sequence of configurations, every prefix is always 
either in the case that the latest visited $\Bad$ configuration has already been followed by a $\Safe$ configuration
or in the case that it still has to. 
Thus, in context, there are no true ``neutral'' configurations: either a neutral configuration is contextually ``$\Safe$'' because it follows a visit in a $\Safe$ configuration, either it is contextually ``$\Bad$'' because it follows a visit in a $\Bad$ configuration and has, accordingly, to reach a $\Safe$ one.
Hence we will consider not only $\Bad$ and $\Safe$ disjoint but
% w.l.o.g.
	also
$\Bad$ complement of $\Safe$,
in which case
if $\Bad$ is downward-closed then $\Safe$ is upward-closed and vice-versa.
Cases where $\Bad$ or $\Safe$ are not complements of each other, and share closure properties, i.e. are both downward-closed or both upward-closed,
have also been studied, 
see 
Appendix~\ref{case down down}
and
Appendix~\ref{case up up} for results.








\subsection{Case: $\Bad=\downarrow \Bad$.}

%
%		cas étudié par les deux articles sur les graphes
%
Let us choose $\Bad$ to be the set of states from which there is no infinite run. If we are in an upward compatible ordered transition system (like VASS, lossy channel systems,...), then $\Bad$ is downward closed.

\subsubsection{Subcase: $\Safe=\downarrow \Safe$}

The well-known mutual exclusion property can be modelized, in a $d$-VASS with $k$ counters, by the property that a special counter $c_{mutex}$ must be bounded by $k \geq 1$ which counts the number of processes that are allowed to be simultaneously in the critical section. Then, the set $\Safe =  \{c_{mutex} \leq k\} \times \mathbb{N}^{d-1}$ is downward closed.
%		and $\Bad =\{c_{mutex} \geq k+1\} \times  \mathbb{N}^{d-1} $ is the upward closed complementary of $\Safe$. 

% In the case \Safe and \Bad are both downward closed, they can both be finite.

\begin{theorem}\label{down-down}
The resilience problem is undecidable for  effective WSTS with  strong  compatibility such that
%	with more than one minimal element, 
$\Safe=\downarrow \Safe$
and $\Bad=\downarrow \Bad$.
\end{theorem}

\begin{proof}
If the set $S$ of a WSTS $\mathscr{S}=(S,\rightarrow, \leq)$ has an unique minimal element $m$, then $m$ belongs to both $\Safe$ and $\Bad$ which contradicts the assumption $\Safe \cap \Bad= \emptyset$. So let us consider the case of a set $S$ with at least two minimal elements $m_1$ and $m_2$.
The problem of whether $m_2$ is reachable from $m_1$ reduces itself to the resilience problem by considering $\Safe=\downarrow m_2 = \{ m_2\}$ and $\Bad=\downarrow m_1 = \{ m_1\}$. By undecidability of the reachability problem for effective WSTS with strong compatibility we conclude.  
\end{proof}

%%%%

\subsubsection{Subcase: $\Safe=\uparrow \Safe$}

example: in a VASS, we may choose $\Safe$ to be the set of states that are not deadlocks, i.e. from which it is always possible to fire a transition. This set of states is upward closed and $\Safe=S-\Bad=\uparrow \Safe$.

%  Parosh Aziz Abdulla, Karlis Cerans, Bengt Jonsson & Yih-Kuen Tsay (1996): General Decidability Theorems for Infinite-State Systems. In: Proc. LICS 1996, IEEE Computer Society Press, pp. 313–321,
%  Alain Finkel & Philippe Schnoebelen (2001): Well-structured transition systems everywhere! Theor. Comput. Sci. 256(1-2), pp. 63–92, doi:10.1016/S0304-3975(00)00102-X.

%Transfering the abstract resilience problems into this framework,
%it is therefore reasonable to demand that both propositions, \Safe and \Bad, are given by 
%upward-closed or downward-closed sets.

Let us recall that the \emph{completion}  \cite{BFM-ic17} of a WSTS $\mathscr{S}=(S,\rightarrow, \leq)$ is the associated ordered transition system $\hat{\mathscr{S}}=(Ideals(S),\rightarrow, \subseteq)$ where states of $\hat{\mathscr{S}}$ are ideals of $S$ and $I \rightarrow J$ if $J$ belongs to the finite ideal decomposition of $\downarrow \post_{\mathscr{S}}(I)$. The completion is always finitely branching but it is not necessarly WSTS since $\subseteq$ is not necessarly a wqo. $\hat{\mathscr{S}}$ is WSTS iff $\mathscr{S}=(S,\rightarrow, \leq)$ is $\omega^2$-WSTS (intuitively speaking, $(S,\leq)$ must not contain the Rado set). Coverability is shown decidable  [Theorem 44] in \cite{BFM-ic17} for completion-post-effective $\omega^2$-WSTS.

Let us recall two other results in \cite{BFM-ic17}. Proposition 30 establishes a strong relation between the runs of a WSTS $\mathscr{S}=(S,\rightarrow, \leq)$ and the runs of its completion $\hat{\mathscr{S}}$. It states that if $x \xrightarrow{k} y$ in $\mathscr{S}$ then for every ideal $I \supseteq \downarrow x$, there exists an ideal $J \supseteq \downarrow y$ such that $I \xrightarrow{k} J$ in $\hat{\mathscr{S}}$. Proposition 29 establishes that if $I \xrightarrow{k} J$ in $\hat{\mathscr{S}}$ then for every $y \in J$, there exists $x \in I$ and $y' \geq y$ such that $x \xrightarrow{k'} y'$ in $\mathscr{S}$. Moreover, if $\mathscr{S}$ has transitive compatibility then $k’ \geq k$; if $\mathscr{S}$ has strong compatibility then $k’ = k$.


\begin{theorem}\label{down-up}
Let $\mathscr{S}=(S,\rightarrow, \leq)$ be a completion-post-effective $\omega^2$-WSTS with strong compatibility and two 
%finite \alain{non, confusion entre ensemble et base}
 sets $\Bad = \downarrow \Bad$ and $\Safe = \uparrow \Safe$.
The resilience problem (RP), the bounded resilience problem (BRP)
and the $k$-resilience problem (kRP) are decidable.
\end{theorem}

\begin{proof}
Let $\{J_1, J_2,...,J_n\}$ be the ideal decomposition of $\Bad$ and $\{b_1,b_2,...,b_m\}$ be the (unique) minimal basis of $\Safe$.
The %uniform 
resilience problem (RP) can be reduced to the following infinite number of instances of the coverability problem in $\mathscr{S}$: for all $x \in \Bad$ does there exist an $j$ such that $b_j$ is coverable from $x$. Let us show how this infinite set of coverability questions can be reduced to a \emph{finite} set of coverability questions in the completion $\hat{\mathscr{S}}=(Ideals(S),\rightarrow, \subseteq)$ of $\mathscr{S}=(S,\rightarrow, \leq)$. 

%Proposition 30 in \cite{BFM-icalp14} establishes a strong relation between the runs of a WSTS $\mathscr{S}=(S,\rightarrow, \leq)$ and the runs of its completion $\hat{\mathscr{S}}$. It states that if $x \xrightarrow{k} y$ in $\mathscr{S}$ then for every ideal $I \supseteq \downarrow x$, there exists an ideal $J \supseteq \downarrow y$ such that $I \xrightarrow{k} J$ in $\hat{\mathscr{S}}$. Proposition 29 establishes that if $I \xrightarrow{k} J$ in $\hat{\mathscr{S}}$ then for every $y \in J$, there exists $x \in I$ and $y' \geq y$ such that $x \xrightarrow{*} y'$ in $\mathscr{S}$.

Let us prove that $b_j$ is coverable from $x$ in $\mathscr{S}$ if and only if $\downarrow b_j$ is coverable (for inclusion) from $\downarrow x$ in $\hat{\mathscr{S}}$.
%
Suppose that $b_j$ is coverable from $x$ then there exists a run $x \xrightarrow{k} y \geq b_j$. From Proposition 30, there exist an ideal $J$ and a run $\downarrow x \xrightarrow{k} J$ where $J \supseteq \downarrow y \supseteq \downarrow b_j$ in $\hat{\mathscr{S}}$, hence $\downarrow b_j$ is covered from $\downarrow x$.
Conversely, if $I \xrightarrow{k} J$ in $\hat{\mathscr{S}}$ with $\downarrow b_j \subseteq J$ then 
%	for every $y \in J$, 
there exists $x \in I$ and $y' \geq b_j$ such that $x \xrightarrow{k} y'  \geq b_j$ in $\mathscr{S}$ and then $b_j$ is coverable from $x$ in $\mathscr{S}$.

Hence we obtain: $\mathscr{S}$ is resilient iff for all $i=1,..,n$ and $j= 1,..m$, $\downarrow b_j$ is coverable from ideal $J_i$ in $\hat{\mathscr{S}}$.
%
Let us denote by $k_{i,j}$ the length of a covering sequence that covers $\downarrow b_j$ from $J_i$ in $\hat{\mathscr{S}}$ and let $k_{i,j}\stackrel{\text{def}}{=}\infty$ if $\downarrow b_j$ is not coverable from $J_i$. Let us now define $K_{\mathscr{S}}(\Safe,\Bad)=max(k_{i,j} \mid i=1,..,n$ and $j= 1,..m$).
%	(if all $k_{i,j}$ are finite) else $K=\infty$.
We now have $\mathscr{S}$ is resilient iff $K_{\mathscr{S}}(\Safe,\Bad)$ is finite iff $\mathscr{S}$ is $K_{\mathscr{S}}(\Safe,\Bad)$-resilient with $K_{\mathscr{S}}(\Safe,\Bad)$ finite.

This implies that resilience and bounded resilience are equivalent to coverability.
%
% iff there is a run in $\hat{S}$ from an ideal I to J such that  $\downarrow x \subseteq I$ and $\downarrow b_j \subseteq J$. This is a consequence of  
%Propositions 29 and 30 in \cite{BFM-icalp14} that establish a strong relation between the runs of a WSTS $\mathscr{S}=(S,\rightarrow, \leq)$ with its completion $\hat{S}$.
%' \in I(\Bad)$.
%
%To decide the bounded resilience, we decide $n \times m$ coverability questions: is state $\downarrow b_j$ coverable from ideal $J_i$ ? If all these $n \times m$ coverability questions are positive then we compute $K=max(k_{i,j} \mid i=1,..,n $ and $j= 1,..m)$ where $k_{i,j}$ is the least length of a sequence that covers  $\downarrow b_j$ from $J_i$.
%
 % \alain{comment trouver les $k_a$ ?} 
%
%  \alain{else $\mathscr{S}$ is not $K$-resilient...et alors qu'en déduit-on ? il pourrait exister un $K' \geq K$ pour lequel $\mathscr{S}$ est K'-resilient....}. 
%else if some of these $n \times m$ coverability questions are negative then resilience
%do not hold and bounded resilience do not hold either.
Since coverability is decidable for completion-post-effective $\omega^2$-WSTS, we deduce that both the 
  resilience problem (RP) and the bounded resilience problem (BRP) are decidable.

\iffalse
	\end{proof}

	\begin{theorem}\label{k-down-up}
	Let $\mathscr{S}=(S,\rightarrow, \leq)$ be a completion-post-effective $\omega^2$-WSTS 
	with strong compatibility and the predbasis hypothesis, and two finite sets: $\Bad$ 
	and $\Safe$.
	The  $k$-resilience problem (kRP) is decidable 
	%	for $k \geq min(K,n)$ where $n$ satisfies $ \uparrow \pred^n(\Safe)=  \uparrow 	\pred^*(\Safe)$. 
	\textcolor{red}{CONJECTURE for now}
	 \end{theorem}

	\begin{proof}

We begin to compute $K$ and $n$ such that $ \uparrow \pred^n(\Safe)=  \uparrow \pred^*(\Safe)$.
If $K=\infty$ then $\mathscr{S}$ is not resilient for Safe and Bad.
Now, if $k \geq min(K,n)$, we conclude that is $k$-resilient; 
%	if moreover $\mathscr{S}=(S,\rightarrow, \leq)$ has the predbasis hypothesis, 
%	and then kRP is decidable.
\fi

Let us now show that the $k$-resilience problem (kRP), with $k \in \mathbb{N}$, is also decidable.
Let us denote by $k'_{i,j}$ the \emph{minimal} length of a covering sequence that covers $\downarrow b_j$ from $J_i$ in $\hat{\mathscr{S}}$ if it exists and let $k'_{i,j}\stackrel{\text{def}}{=}\infty$ if $\downarrow b_j$ is not coverable from $J_i$. 
If $\downarrow b_j$ is coverable from $J_i$, we first compute an $k_{i,j}$, and then we compute $k'_{i,j}$ by iteratively checking whether there exists a sequence of length $0,1,...,k_{i,j}-1$ that covers $\downarrow b_j$ from $J_i$ until we find the minimal one which is necessarly smaller (or equal to) than $k_{i,j}$.

Let us now define $K'_{\mathscr{S}}(\Safe,\Bad)=max(k'_{i,j} \mid i=1,..,n$ and $j= 1,..m$) and we deduce that  $\mathscr{S}$ is $k$-resilient iff $k \geq K'_{\mathscr{S}}(\Safe,\Bad)$.

%	If $k <  min(K,n)$, then we check every path of length smaller
%	than $min(K,n)$ from the ideals of the decomposition of $\Bad$ in the completion \alain{à expliquer mieux}.
\end{proof}

%









%%%%%%%%
\subsection{Case: $\Safe=\downarrow \Safe$.}\label{safe-down}
%
%		cas unbounded counters
%

Let us now consider the case $\Safe=\downarrow \Safe$ hence $\Bad=\uparrow \Bad$.
It is of interest to note this case can be linked to the problem of mutual exclusion.
Indeed the well-known mutual exclusion property can be modelized, in a $d$-VASS with $d$ counters, by the property that a special counter $c_{mutex}$ must be bounded by $k \geq 1$ which counts the (maximal) number of processes that are allowed to be simultaneously in the critical section. Then, the set $\Safe =  \{c_{mutex} \leq k\} \times \mathbb{N}^{d-1}$ is downward-closed		and $\Bad =\{c_{mutex} \geq k+1\} \times  \mathbb{N}^{d-1} $ is the upward-closed complementary of $\Safe$. 
%  indeed, one can choose $\Bad$ as the set of states from which the counter $c_{mutex}$ is not bounded by $k$, and $\Safe$ to be the downward-closed complement of $\Bad$.

%	 In the case $\Safe=\downarrow \Safe$, our contribution consist in the following:



%
\begin{theorem}\label{up-down}
{\sc Resilience}  is decidable for ideal-effective WSTS with 
$\Safe=\downarrow \Safe$
and
the additional hypothesis that
for all downward-closed set $D \subseteq S$, the set $\pred^*(D)$ is downward-closed.
% \alain{est-ce que  $\pred(D)$ is downward closed avec D downward closed serait suffisant ? si l'ordre est omega2 alors closbas+inclusion est wqo donc il faudrait juste que la suite croissante $pred^n(D)+le reste$ converge...}
\end{theorem}

\begin{proof}
By hypothesis $\pred^*(\Safe)$ is downward-closed, since $\Safe$ is downward-closed.
%  Let $\{J_1, J_2,...,J_n\}$ be the ideal decomposition of $\pred^*(\Safe)$ and $\{b_1,b_2,...,b_m\}$ be the (unique) minimal basis of $\Bad$.
The resilience problem can be reformulated as 
% $\uparrow \{b_1,b_2,...,b_m\} \subseteq J_1 \cup J_2 \cup \cdots \cup J_n$.
$\Bad \subseteq  \pred^*(\Safe)$.
Since $\mathscr{S}=(S,\rightarrow, \leq)$ is ideally effective, we can compute intersections of upward- or downward-closed 
%\alain{non définis}
 subsets.
Hence we can compute the intersection of
% $\uparrow \{b_1,b_2,...,b_m\}$
$\Bad$
and
$S \setminus \pred^*(\Safe)$,
which are both upward-closed.
Since
% $\uparrow \{b_1,b_2,...,b_m\} \subseteq J_1 \cup J_2 \cup \cdots \cup J_n$
$\Bad \subseteq \pred^*(\Safe)$
can be reformulated as
$\Bad \cap (S \setminus \pred^*(\Safe)) = \emptyset$
the resilience problem is decidable. \qed
\end{proof}

Let us recall that a system $\mathscr{S}=(S,\rightarrow, \leq)$ is \emph{downward compatible} \cite{DBLP:journals/tcs/FinkelS01} if
for all $s_1, s_2, t_1 \in S$ with
$s_2 \leq s_1$ and $s_1 \to t_1$
there
exists $t_2 \in S$ with
$t_2 \leq t_1$ and $s_2 \to^* t_2$.

\begin{corollary}
{\sc Resilience} is decidable for ideal-effective downward-compatible WSTS with 
$\Safe=\downarrow \Safe$.
\end{corollary}

\begin{proof}


Let $D$ be a downward-closed subset of $S$
and let $x \in \downarrow \pred^*(D)$.
By downward closure, there exists
$y \in \pred^*(D)$ 
such that $x \leq y$.
By definition of $\pred^*(D)$ then there exists 
$d \in D$, $m\geq 0$ and $(a_i)_{0 \leq i \leq m+1} \in S^{m+2}$ such that
$y = a_0 \to a_1 \to a_2 \to \cdots \to a_m \to a_{m+1} = d$.

By downward compatibility $a_0 \to a_1$
implies that there exists $a'_1 \in S$ such that $a'_1 \leq a_1$ and
$x \to^* a'_1$.
More generally $a_i \to a_{i+1}$ and
$a'_i\leq a_i$ implies the existence of $a'_{i+1} \in S$ with $a'_{i+1} \leq a_{i+1}$ and
$a'_i \to^* a'_{i+1}$,
and, by induction,
 $x \to^* a'_1 \to^* \cdots \to^* a'_{m} \to^* a'_{m+1} = d'$
with $d' \leq d$.
Since
$d'$ 
% \alain{qui appartient à qui ?}
belongs to $D$ by downward closure of $D$, $x \in \pred^*(D)$. \qed
\end{proof}

% \begin{proposition}
In the case
of a ideal-effective WSTS 
where
the additional hypothesis that
for all downward-closed set $D \subseteq S$, the set $\pred^*(D)$ is downward-closed
is not met,
the above construction
can provide a proof
of non-resilience
i.e. when
$\Bad \cap (S \setminus \downarrow\pred^*(\Safe)) \neq \emptyset$
then
$\Bad \not\subseteq \downarrow\pred^*(\Safe)$
and hence
$\Bad \not\subseteq \pred^*(\Safe)$.
When $\Bad \cap (S \setminus \downarrow\pred^*(\Safe)) = \emptyset$
however
it is not enough to conclude.

% \alain{indecidabilite en enlevant une ou des hypotheses: The resilience problem is undecidable for ideal-effective WSTS}

% \mathieu{
% Home-state indécidable pour les minsky machine (preuve très rapide: M à $2$ compteurs s'arrête ssi M' visite $(0,0,0)$ depuis $\uparrow(0,0,1)$, où M' est M mais avec un troisième compteur qui commence à $1$ et qui simule M jusqu'à ce que M s'arrête et à ce moment là décroit les deux premiers compteurs puis le troisième et atteint $(0,0,0)$).
% or les minsky machine peuvent être simulées par les reset petri nets, on réduit le home-state des minsky machine au home-state des reset petri nets, ce qui donne l'indécidabilité du home-state pour les WSTS en général.}


In the case
of a ideal-effective WSTS 
where the hypothesis that
{for all downward-closed set $D \subseteq S$, the set $\pred^*(D)$ is downward-closed}
is not met, the problem is undecidable.
 Indeed, it is 
undecidable for reset-VAS whether zero (the vector containing only zeroes, also denoted~$ \textbf{0}$) is a home-state for $\N^d \setminus \textbf{0}$, where it is a particular case of resilience with $\Safe = \downarrow \textbf{0}$. This stems from the fact that it is undecidable for Minsky machines with more than one counter, whether zero is a home-state. % See Appendix~\ref{HS-Minsk} for a more detailed construction.


\iffalse
	% 	From Appendix
	%
% \subsection{Home-state is undecidable for Minsky machine, reset-VASS and WSTS}\label{HS-Minsk}
	%
% We mentionned in Section~$3$ that {\sc Resilience} is undecidable for WSTS in general when $\Safe = \downarrow \Safe$ and $\Bad = \uparrow \Bad$. Let us provide a more detailed undecidability proof here.

First, let us recall the home-space problem

\problemx{Home-space}
{A transition system $\mathscr{S}=(S,\rightarrow)$ a states $s_0 \in S$ and a subset $H \subseteq S$.}
{$\post^*(s_0) \subseteq \pred^*(H)$ ?\newline}

% whether a state $s$ is such that, for all state $s_0$ there exists a path from $s_0$ to $s$.
% post ∗ (S ) ⊆ pred ∗ (s_0)
\fi

Home-space is undecidable for Minsky machine with at least $3$ counters, since it is undecidable in the particular case where $H$ is a singleton.
This stems from the fact $2$-counter Minsky machine termination is undecidable~\cite{Min61,Min67}.
From a $2$-counter Minsky machine $M$, one can construct a $3$-counter Minsky machine $M'$ 
such that machine $M$ terminates for all inputs iff machine $M'$ can reach $(0,0,0)$ from any input with at least $1$ on its third counter. We build $M'$ to simulate $M$ until it reaches a control-state indicative of termination, then lower the first two counters until they reach $0$, then, and only then, finally lower the third counter until it reaches $0$.
Remark that in the construction, from any input with at least $1$ on its third counter, it is not possible to reach $(1,0,0)$, $(0,1,0)$ or $(1,1,0)$.
% Home-state indécidable pour les Minsky machine (preuve très rapide: M à $2$ compteurs s'arrête ssi M' visite $(0,0,0)$ depuis $\uparrow(0,0,1)$, où M' est M mais avec un troisième compteur qui commence à $1$ et qui simule M jusqu'à ce que M s'arrête et à ce moment là décroit les deux premiers compteurs puis le troisième et atteint $(0,0,0)$).
% or les minsky machine peuvent être simulées par les reset petri nets, on réduit le home-state des minsky machine au home-state des reset petri nets, ce qui donne l'indécidabilité du home-state pour les WSTS en général.
Based on this construction, the problem of deciding whether the downward closed set $\downarrow (1,1,0)$ is reachable in a $3$-counter Minsky machine
from any input with at least $1$ on its third counter is undecidable. 
Hence {\sc Resilience} is undecidable for Minsky machines.


Executions of Minsky machines can be simulated by reset-VASS~\cite{araki1976PN}. 
% Let us recall more formally what a reset-VASS is.
Reset-VASS extend the basic VASS model with special “reset
transitions” that set to $0$ some coordinates in the vector. Let us recall their definition here.
\begin{definition}
A {\em reset-VASS} in dimension $d$ %(reset-VASS for short)
 is a finite 
labeled directed graph $V = (Q,T)$, where $Q$ is the set of {\em control-states}, 
$T \subseteq Q \times Op \times Q$
is the set of {\em control-transitions}, and $Op = \{ add(\textbf{z}) \mid \textbf{z} \in \mathds{Z}^d\} \cup 
		\{ reset(i) \mid i \in \{1,\ldots,d\} \}$.
\end{definition}

Again $Q \times \N^d$
 denotes the set of configurations of $V$.
For every configurations $p(\textbf{u}), q(\textbf{v}) \in Q \times \N^d$ and every control-transition $t$ we write
$p(\textbf{u}) \xrightarrow{t} q(\textbf{v})$ when 
\begin{samepage}\begin{itemize}
\item  $t = (q,add(\textbf{z}),q') \in T$
% then for all $\textbf{u} \in \N^d$ such that  
% $\textbf{u}+\textbf{z} \geq 0$
% $q(\textbf{u}) \xrightarrow{\textbf{z}} q'(\textbf{u}+\textbf{z})$,
and $\textbf{u}+\textbf{z} = \textbf{v} \geq 0$,
\item $t = (q,reset(\gamma),q') \in T$ 
% then for all $\textbf{u} \in \N^d$ 
% $q(\textbf{u}) \xrightarrow{z} q'(\textbf{u}')$,  where 
and
$\textbf{v}[\gamma] = 0$, and $\textbf{v}[\gamma'] = \textbf{u}[\gamma']$ for all $\gamma' \in \{1,\ldots, d\} \setminus \gamma$.
\end{itemize} \end{samepage}

It is well known that reset-VASS are WSTS~\cite{dufourd1998reset}. 
Since Reset-VASS can simulate executions of a Minsky machine, {\sc Resilience} is undecidable for reset-VASS and hence for WSTS in general as well.





Remark we did not make use of the
% hypothesis
property $\Bad$ complement of $\Safe$, simply 
$\Bad=\uparrow \Bad$ and $\Safe=\downarrow \Safe$, thus 
the above results still hold in the more general case where $\Bad$ and $\Safe$ are not complement of each others.









{\bf Synthesis of the main decidability results}\label{synthesis}

% \mathieu{En rouge quand il s'agit de conjectures}


% \alain{simplifier le tableau en enlevant les deux dernières colonnes}

\begin{center}
\begin{tabular}{ | l | c | c | c | r |}
\hline   \Safe~\Bad %& $\uparrow$~ $\uparrow$~ 
		& $\uparrow$~ $\downarrow$~ 
		 & $\downarrow$~ $\uparrow$~ 
		 %& $\downarrow$~ $\downarrow$~
 \\ \hline
   RP %& Decidable (Thm~\ref{up-up}) 
   	& Decidable (Thm~\ref{down-up})  
   		 & Decidable (Thm~\ref{up-down}) 
   	%	 & Undecidable (Thm~\ref{down-down})
    \\ \hline
   BRP %& Decidable (Corollary~\ref{B-up-up}) 
   &  Decidable (Thm~\ref{down-up}) 
   		 & ?? 
   	%	 & ??
    \\ \hline
      kRP %& Decidable (Thm~\ref{k-up-up}) 
      & Decidable (Thm~\ref{down-up}) 
      		& ?? 
      	%	& ??
       \\ \hline
 \end{tabular}
\end{center}




%We first assume that the safety property is given by an upward-closed set and the bad condition by a decidable downward-closed set. 
% \textcolor{red}{Seems like a reasonable assumption to me.}

%From these considerations, we formulate instances of the abstract resilience problems for well-
%structured transition systems.


