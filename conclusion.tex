



\section{conclusion}


We have extended the state-resilience problem introduced in~\cite{DBLP:journals/corr/abs-2108-00889,DBLP:conf/gg/Ozkan22}.
% The problem [] is []. 
%In [] the problem was introduced. We expand on the notion. 
We complemented previous results by providing some undecidability proofs for resilience and state-resilience in general. We also exhibited classes of WSTS with decidable resilience, namely, 
completion-post-effective $\omega^2$-WSTS with strong compatibility (in the case $\Bad = \downarrow \Bad$ and $\Safe = \uparrow \Safe$) and 
downward-compatible
ideal-effective WSTS
(in the case $\Bad = \uparrow \Bad$ and $\Safe = \downarrow \Safe$).
We additionnally generalized decidability results from~\cite{DBLP:journals/corr/abs-2108-00889,DBLP:conf/gg/Ozkan22}.

Several questions still remain.
% We will analyse in detail the complexities of the different resilience. resilience for other computable models like stack automata,...
For instance, we have been concerned with decidability only, and a detailed complexity analysis of the different resilience problems still remains to be done. 
One other angle of attack not studied here is synthesis of sets $\Bad$ and/or $\Safe$ for resilience,
for instance, given a set $\Safe$, find the maximal upward-closed (resp. downward-closed) subset 
$\Bad$ so that a system is $(\Safe,\Bad)$-resilient. 
% \alain{trouver les \Bad~ et \Safe~ maximum tels que S est resilient. est-ce vrai que si S est $(B_i,D_i)$-resilient alors S est $(\cap, \cup B_i,D_i)$-resilient ?}

One could also extend upon the classes of set $\Bad$ and $\Safe$ considered. As with semilinear sets for VASS, one could study resilience for sets defined in a boolean logic on upward and downward-closed subsets. Finally, while we mention VASS, but a more detailed analysis of the resilience problems could be also done for other computational models such as pushdown automata, one-counter automata or timed automata.

% \alain{on peut penser à des ensembles $\Bad$ definis dans une logique booleennne sur les clos par le bas, haut, +...}

%
%
%\subsection{Vector Addition System with States or PN}
%
%\textcolor{red}{Should be defined in a later 'application section' once we start writing any proof, for now I leave it there} 
%


