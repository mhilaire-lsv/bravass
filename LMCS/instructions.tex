\documentclass{lmcs} %%% last changed 2014-08-20

%% mandatory lists of keywords
\keywords{MANDATORY list of keywords}

%% read in additional TeX-packages or personal macros here:
%% e.g. \usepackage{tikz}
\usepackage{hyperref}
%%\input{myMacros.tex}
%% define non-standard environments BEYOND the ones already supplied
%% here, for example
\theoremstyle{plain}\newtheorem{satz}[thm]{Satz} %\crefname{satz}{Satz}{S\"atze}
%% Do NOT replace the proclamation environments lready provided by
%% your own.

\def\eg{{\em e.g.}}
\def\cf{{\em cf.}}

%% due to the dependence on amsart.cls, \begin{document} has to occur
%% BEFORE the title and author information:

\begin{document}

\title[Instructions]{Instructions for Authors\\How to prepare papers
  for LMCS using \texorpdfstring{\MakeLowercase{\texttt{lmcs.cls}}}{lmcs.cls}\rsuper*\\Version of
  2022-04-01}
\titlecomment{{\lsuper*}OPTIONAL comment concerning the title, \eg,
  if a variant or an extended abstract of the paper has appeared elsewhere.}
\thanks{thanks, optional.}	%optional

% affiliations are numbered automatically with a, b, c (see below)
% use the optional argument to indicate the affiliation(s) of each author
% omit the argument if there is only one author, or only one affiliation
\author[A.~Name1]{Alice Name1}[a]
\author[B.~Name2]{Bob Name2}[a,b]
\author[J.~Name3]{Josiah S.~Carberry\lmcsorcid{0000-0002-1825-0097}}[a]

% affiliation 1 (automatically numbered a)
\address{University 1, address1}	%optional
% write emails for all authors having that affiliation
\email{name1@email1, name2@email1, name3@email1}  %optional

% affiliation 2 (automatically numbered b)
\address{University 2, address2}	%optional
\email{name2@email2}  %optional

%% etc.

%% required for running head on odd and even pages, use suitable
%% abbreviations in case of long titles and many authors:

%%%%%%%%%%%%%%%%%%%%%%%%%%%%%%%%%%%%%%%%%%%%%%%%%%%%%%%%%%%%%%%%%%%%%%%%%%%

%% the abstract has to PRECEDE the command \maketitle:
%% be sure not to issue the \maketitle command twice!

\begin{abstract}
  \noindent The abstract has to precede the maketitle command.  Be
  sure not to issue the maketitle command twice!  In the abstract,
  mathematical expressions must be kept to the absolute minimum.
  Otherwise it should consist of plain ASCII text, without
  \TeX-commands, including explicit references using the
  \texttt{\textbackslash cite} command.  Presently we are not able to
  automatically extract an abstract containing such data and reliably
  turn it into html code.  If you cannot meet these criteria, it is
  your responsibility to provide us with an html-version of your
  abstract.  Please keep the abstract fairy short to prevent it from
  spilling over to the second page!
\end{abstract}

\maketitle

%% start the paper here:
\section*{Introduction}\label{S:one}

  Logical Methods in Computer Science is a community effort. It is run
  by scientists like you who devote their time and effort to make this
  a high-class open access journal that is free of cost for readers
  as well as authors.  To minimize the extra work for the layout
  editor and to ensure smooth and fast publication of accepted
  articles, authors are asked to strictly adhere to the instructions
  for preparing their final version given in this document, which
  takes the form of a sample paper.

  These revised instructions distill the experience of running the
  Journal for several years.  They address the most time-consuming
  aspects of getting articles into publishable shape.  While most
  articles require only minimal intervention, as the Journal's volume
  increased, so did the amount of time needed to reformat
  non-compliant articles.

\subsection*{\TeX-nical matters}

  Please be aware that the class-file {\bf lmcs.cls} supplied to
  authors will be replaced by the Journal's master class-file before
  publication of your article.  Hence it is not necessary, and is in
  fact counterproductive, to emulate the appearance of published
  articles by means of your personal macros.  Submissions not using
  {\bf lmcs.cls} will be returned to the authors, as the reformatting
  that usually results from changing the class-file is usually too
  extensive and requires the original authors' intervention.

  What authors \emph{can} do to help the layout editor is to make
  their \TeX- source compatible with the {\bf hyperref}-package, which
  is included by the master class-file.  In particular, care should be
  taken to use the \texttt{\textbackslash texorpdfstring} macro for
  mathematical expressions in section or subsection headings (see the hyperref
  documentation for details: \url{https://ctan.org/pkg/hyperref}).

  Authors must not (1) use unsupported fonts (like the
  \texttt{times}-package or the \texttt{txfonts}-package), (2) change
  the numbering style for theorems and definitions and the like, \eg,
  by redefining the already provided proclamation environments for
  Theorems, Propositions, Lemmata, Corollaries etc.\ (you can add
  further environments, but those should comply with the default
  numbering style), and (3) use the \texttt{\textbackslash sloppy}
  option globally.  If it is impossible to achieve good line breaks by
  other means once the article is finished (reformulating a sentence,
  changing the word order, etc.), one can use \texttt{\textbackslash
    sloppy} as a last resort \emph{locally} in a paragraph.

  Using lengthy mathematical expressions inside running text can lead
  to ugly breaks within formulae, even without producing overfull
  hboxes.  If this is a persistent problem in your paper, please
  consider using more displayed formulae, or changing your notation.

  The use of different macro-packages for the purpose of creating
  diagrams or other graphical displays is strongly discouraged.  In
  the past that has led to papers that required different ways of
  processing to display the graphics of one type or the other, but
  could not easily be made to correctly display both types of graphics
  simultaneously.  As a rule of thumb, as long as {\bf pdflatex}
  correctly processes your paper, you should be in good shape. (Users
  of the {\bf pstricks}-package and those used to including external
  eps-files should transform the resulting PostScript files to pdf.)

  Please be aware that the proofs may display different vertical
  spacing in general, in particular different page breaks than the
  version originally submitted.  If adjustments are deemed to be
  necessary by the authors, they can be implemented on the basis of
  the proofs, in collaboration with the layout editor, as a last step
  before final publication.

\subsection*{Matters of convenience}

  Please submit {\bf only one file} containing the
  TeX-source of your paper!  It is a major inconvenience when certain
  changes have to be applied in several files separately.  Of course,
  we understand that separating a TeX source into several files has
  advantages during the creation of a paper, but please combine all
  parts into a single file for your submission. You personal macros
  can of course be contained in a separate file, as can be external
  graphics.  For the latter a dedicated subdirectory is required.


\subsection*{Matters of style}

  Your article should start with an introduction.  This is the place
  to employ mathematical notation and give references, as opposed to
  the abstract.  It is up to the authors to decide, whether to assign
  a section number to the introduction or not.

\section{Multiple authors}

  In papers with multiple authors several points need to be mentioned.
  Do not worry about footnote signs that will link author $n$ to
  address $n$ and the optional thanks $n$.  This will be taken care of
  by the layout editor.  Even if authors share an affiliation and part
  of an email address, they should follow the strict scheme outlined
  above and list their data individually.  The layout editor will
  notice duplication of data and can then arrange for more
  space-efficient formatting.  Alternatively, Authors can write ``same
  data as Author n'' into some field to alert the layout editor.
  Unfortunately, so far we have not been able to devise a system that
  automatically takes care of these issues.  But once the layout
  editor is made aware of some duplication, he can take some fairly
  simple measures to adjust the format accordingly.  Placing the
  responsibility on the layout editor insures that these formatting
  issues are handled uniformly in different papers and that the
  authors do not have to second-guess the Journal's policy.

\section{Use of  Definitions and Theorems etc.}

  The numbering scheme for proclamations (Theorems, Definitions, etc.)
  uses the section number followed by the number of the current
  proclamation.  There are no different ``numbering threads'' for the
  various types of proclamations, as then the relative position of,
  \eg, Theorem 2.7 relative to, \eg, Definition 2.9 would not be clear.

\begin{defi}\label{D:first}
  This is a definition.
\end{defi}

  Please use the supplied proclamation environments (as well as
  LaTeX's cross-referencing facilities), or extend them in the spirit
  of the given ones, if necessary (\cf~\autoref{T:big} below).
  Refrain from replacing the Journal's proclamation macros by your own
  constructs, especially do not change the numbering scheme: all
  proclamations are to be numbered consecutively!

\subsection{First Subsection}

  This is a test of subsectioning.  It works like numbering of
  paragraphs but is not linked with the numbering of theorems.

\begin{satz}\label{T:big}
  This is a sample for a proclamation environment that can be added
  along with your personal macros, \emph{in case the supplied
    environments do not suffice}.  Please refrain from substituting
  other environments for the supplied ones.  We distinguish those
\begin{itemize}
\item with italicised text:
\begin{itemize}
\item\emph{\texttt{\textbackslash
    begin\{thm\}}\dots\texttt{\textbackslash end\{thm\}}} for a theorem;
\item\emph{\texttt{\textbackslash
    begin\{lem\}}\dots\texttt{\textbackslash end\{lem\}}} for a lemma;
\item\emph{\texttt{\textbackslash
    begin\{cor\}}\dots\texttt{\textbackslash end\{cor\}}} for a corollary;
\item\emph{\texttt{\textbackslash
    begin\{prop\}}\dots\texttt{\textbackslash end\{prop\}}} for a proposition;
\item\emph{\texttt{\textbackslash
    begin\{asm\}}\dots\texttt{\textbackslash end\{asm\}}} for an
  assumption;
\end{itemize}
\item  and those with normal roman text:
\begin{itemize}
\item\emph{\texttt{\textbackslash
    begin\{defi\}}\dots\texttt{\textbackslash end\{defi\}}} for a definition;
\item\emph{\texttt{\textbackslash
    begin\{conv\}}\dots\texttt{\textbackslash end\{conv\}}} for a convention;
\item\emph{\texttt{\textbackslash
    begin\{conj\}}\dots\texttt{\textbackslash end\{conj\}}} for a conjecture;
\item\emph{\texttt{\textbackslash
    begin\{fact\}}\dots\texttt{\textbackslash end\{fact\}}} for facts;
\item\emph{\texttt{\textbackslash
    begin\{algo\}}\dots\texttt{\textbackslash end\{algo\}}} for algorithms;
\item\emph{\texttt{\textbackslash
    begin\{pty\}}\dots\texttt{\textbackslash end\{pty\}}} for a property;
\item\emph{\texttt{\textbackslash
    begin\{clm\}}\dots\texttt{\textbackslash end\{clm\}}} for a claim;
\item\emph{\texttt{\textbackslash
    begin\{nota\}}\dots\texttt{\textbackslash end\{nota\}}} for a notation;
\item\emph{\texttt{\textbackslash
    begin\{exa\}}\dots\texttt{\textbackslash end\{exa\}}} for a single example;
\item\emph{\texttt{\textbackslash
    begin\{exas\}}\dots\texttt{\textbackslash end\{exas\}}} for a list of examples;
\item\emph{\texttt{\textbackslash
    begin\{rem\}}\dots\texttt{\textbackslash end\{rem\}}} for single remarks;
\item\emph{\texttt{\textbackslash
    begin\{rems\}}\dots\texttt{\textbackslash end\{rems\}}} for a list of remarks;
\item\emph{\texttt{\textbackslash
    begin\{prob\}}\dots\texttt{\textbackslash end\{prob\}}} for a
  single problem;
\item\emph{\texttt{\textbackslash
  begin\{probs\}}\dots\texttt{\textbackslash end\{probs\}}} for a list
  of problems;
\item\emph{\texttt{\textbackslash
    begin\{oprob\}}\dots\texttt{\textbackslash end\{oprob\}}} for a single open problem;
\item\emph{\texttt{\textbackslash
    begin\{oprobs\}}\dots\texttt{\textbackslash end\{oprobs\}}} for a
  list of open problems;
\item\emph{\texttt{\textbackslash begin\{obs\}}\dots\texttt{\textbackslash
  end\{obs\}}} for a single observation;
\item\emph{\texttt{\textbackslash begin\{obss\}}\dots\texttt{\textbackslash
  end\{obss\}}} for a list of observations;
\item\emph{\texttt{\textbackslash
    begin\{qu\}}\dots\texttt{\textbackslash end\{qu\}}} for a single question;
\item\emph{\texttt{\textbackslash
    begin\{qus\}}\dots\texttt{\textbackslash end\{qus\}}} for a list of questions.
\end{itemize}
\end{itemize}
  The present new environment \emph{\texttt{\textbackslash
    begin\{satz\}}\dots\texttt{\textbackslash end\{satz\}}} was defined
  by
\begin{center}
  \emph{\texttt{\textbackslash theoremstyle\{plain\}\textbackslash newtheorem\{satz\}[thm]\{Satz\}}}
\end{center}
\end{satz}

\begin{proof} You can use the familiar \texttt{\textbackslash
    begin\{proof\}}\dots\texttt{\textbackslash end\{proof\}}
  construction.  Please do not insert a blank line before
  \texttt{\textbackslash end\{proof\}}, as this moves the box to a new
  line.

  In case a proof ends with an itemization,
  please issue the command \texttt{\textbackslash qedhere} at the end
  of the final item, \emph{before} calling \texttt{\textbackslash
    end\{enumerate\}} (or similar) and \texttt{\textbackslash end\{proof\}}.
  Otherwise the end-of-proof box is put on a separate line
  following the last item, which looks awkward, unless the last line
  is too full to accommodate the box.

  For options how to handle proofs ending in a displayed multi-line
  equation or formula, see
  \href{http://tex.stackexchange.com/questions/101929/qed-or-qedhere-at-the-end-of-split-environment}{this
    discussion}.
\end{proof}

\begin{cor}\label{C:big}
  If no proof is given, \texttt{lmcs.cls} provides Paul Taylor's
  end-of-proof box \emph{\texttt{\textbackslash qed}} to conclude a
  proclamation (Theorem, Proposition, Lemma, Corollary).  Please do
  not redefine \emph{\texttt{\textbackslash qed}}!\qed
\end{cor}

\section{Itemization}\label{S:item}
  \texttt{lmcs.cls} provides the familiar environments
\begin{enumerate}
\item\texttt{\textbackslash
  begin\{itemize\}}\dots\texttt{\textbackslash end\{itemize\}} (see
 \autoref{T:big} above)
\item\texttt{\textbackslash
  begin\{enumerate\}}\dots\texttt{\textbackslash end\{enumerate\}}
  (see this listing)
\item\texttt{\textbackslash
    begin\{description\}}\dots\texttt{\textbackslash end\{description\}}
\end{enumerate}
  in a form based on the \texttt{enumitem}-package, version 3.5.2
  (please update, if you have an earlier version).  This offers
  considerable simplifications, both for authors and the Layout
  Editor.  Modifying the spacing of these environments is strongly
  discouraged.  If you wish to change the labels, please consult the
  documentation of the \texttt{enumitem}-package.  A simple example is
  found at the end of this document.

  When proclamations or proofs start with an itemization without
  preceding text, two possibilities exist:

\begin{thm}\label{T:m}\hfill  %% \hfill pushes the first item to a new line
\begin{enumerate}
\item
  Issuing an {\em\texttt{\textbackslash hfill}}-command before the
  beginning of the list environment will push the first item to a new
  line, like in this case.
\item
  This is the second item.
\end{enumerate}
\end{thm}

\proof\hfill  %% \hfill pushes the first item to a new line
\begin{enumerate}
\item
  The same behavior occurs in proofs; to start the first item on a
  new line an explicit \texttt{\textbackslash hfill}-command is necessary.
\item
  The journal's bibliography style is \texttt{alphaurl}, as exemplified
  by this citation \cite{koslowski:mib}. Please use bibtex, i.e.\ supply a
  \verb|.bib| file with the references in bibtex format along with your submission. \qed
\end{enumerate}

  \noindent We strongly recommend using this variant since it produces
  rather orderly output.  The space-saving variant, in contrast, can
  look quite awful, \cf~\autoref{T:en} below.  Please notice that
  this paragraph is not indented, since it is following a proclamation
  that ended with a list environment.  This can be achieved by
  starting the paragraph directly after the end of that environment,
  without inserting a blank line, or by explicit use of the
  noindent-command at the beginning of the paragraph.  The effect
  indentation may have after a list environment is demonstrated after
  the proof of \autoref{T:en}.

\begin{thm}\label{T:en} %% without \hfill the first item is indented

\begin{enumerate}%[\em(1)]
\item
  Without the \emph{\texttt{\textbackslash hfill}}-command the first item
  starts in the same line as the title for the proclamation.
\item
  This may be useful when space needs to be conserved, but not in an
  electronic journal.
\end{enumerate}
\end{thm}

\proof %% without \hfill the first item is indented
\begin{enumerate}%[(1)]
\item
  As you can see, the second option produces a somewhat unpleasant effect.
\item
  Hence we would urge authors to use the first variant.  Perhaps a
  \TeX-guru can help us to make that the default, without the need for
  the \texttt{\textbackslash hfill}-command.\qed
\end{enumerate}

  Here we started a new paragraph without suppressing its
  indentation.  This adds to the rather disorienting appearance
  produced by not turning off the space-saving measures built into
  amsart.cls, on which this style is based.  Please do issue the
  \hbox{\textbackslash noindent} command in such situations, just as
  after the proof of \autoref{T:m} above.

\section*{Acknowledgment}
  \noindent The authors wish to acknowledge fruitful discussions with
  A and B.

  %% the following bibliography is gererated manually for the sake of brevity
  %% only; please use a separate .bib file in your submission

\begin{thebibliography}{Kos97}

\bibitem[Kos97]{koslowski:mib}
J{\"u}rgen Koslowski.
\newblock Monads and interpolads in bicategories.
\newblock {\em Theory Appl. Categ.}, 3(8):182--212, 1997.

\end{thebibliography}

\appendix
\section{}
  Here is a check-list to be completed before submitting the paper to
  LMCS:
\begin{itemize}[label=$\triangleright$]
\item your submission includes the latest version of lmcs.cls, that is, it does
  not rely on the version of the class file provided by arXiv
\item the text of your submission is contained in a single file,
  except for macros and graphics
\item your graphics use only one format
\item you have employed the Journal's original proclamation environments,
  or suitable extensions thereof
\item you have loaded the hyperref package
\item you have \emph{not} loaded the times package
\item you have not routinely adjusted vertical spacing manually by issuing
  \texttt{\textbackslash vspace} or \texttt{\textbackslash vskip} commands
\item you have used the command \texttt{\textbackslash sloppy} only
  locally and in emergency cases
\item your displayed equations use the
  \texttt{\textbackslash[\dots\textbackslash]} construct
\item your abstract only contains as few math-expressions as possible and no
  references
\item your references are supplied in bibtex format in a separate \verb|.bib| file
\end{itemize}

  This listing also shows how to override the default bullet $\bullet$
  of the \texttt{itemize}-envronment by a different symbol, in this
  case \texttt{\textbackslash triangleright}.
\end{document}
