

%%%%%%%%
\subsection{Case: $\Safe=\downarrow \Safe$.}
%
%		cas unbounded counters
%

Let us now consider the case $\Bad=\uparrow \Bad$ and $\Safe=\downarrow \Safe$.
It is of interest to note this case can be linked to the problem of mutual exclusion.
Indeed the well-known mutual exclusion property can be modelized, in a $d$-VASS with $k$ counters, by the property that a special counter $c_{mutex}$ must be bounded by $k \geq 1$ which counts the number of processes that are allowed to be simultaneously in the critical section. Then, the set $\Safe =  \{c_{mutex} \leq k\} \times \mathbb{N}^{d-1}$ is downward-closed		and $\Bad =\{c_{mutex} \geq k+1\} \times  \mathbb{N}^{d-1} $ is the upward-closed complementary of $\Safe$. 
%  indeed, one can choose $\Bad$ as the set of states from which the counter $c_{mutex}$ is not bounded by $k$, and $\Safe$ to be the downward-closed complement of $\Bad$.

 In the case $\Safe=\downarrow \Safe$, our contribution consist in the following:



%
\begin{theorem}\label{up-down}
The resilience problem is decidable for ideal-effective WSTS with 
$\Safe=\downarrow \Safe$
and $\Bad=\uparrow \Bad$
and
the additional hypothesis that
for all downward-closed set $D \subseteq S$, the set $\pred^*(D)$ is downward-closed.
% \alain{est-ce que  $\pred(D)$ is downward closed avec D downward closed serait suffisant ? si l'ordre est omega2 alors closbas+inclusion est wqo donc il faudrait juste que la suite croissante $pred^n(D)+le reste$ converge...}
\end{theorem}

\begin{proof}
By hypothesis $\pred^*(\Safe)$ is downward-closed, since $\Safe$ is downward-closed.
%  Let $\{J_1, J_2,...,J_n\}$ be the ideal decomposition of $\pred^*(\Safe)$ and $\{b_1,b_2,...,b_m\}$ be the (unique) minimal basis of $\Bad$.
The resilience problem can be reformulated as 
% $\uparrow \{b_1,b_2,...,b_m\} \subseteq J_1 \cup J_2 \cup \cdots \cup J_n$.
$\Bad \subseteq  \pred^*(\Safe)$.
Since $\mathscr{S}=(S,\rightarrow, \leq)$ is ideally effective, we can compute intersections of upward- or downward-closed 
%\alain{non définis}
 subsets.
Hence we can compute the intersection of
% $\uparrow \{b_1,b_2,...,b_m\}$
$\Bad$
and
$S \setminus \pred^*(\Safe)$,
which are both upward-closed.
Since
% $\uparrow \{b_1,b_2,...,b_m\} \subseteq J_1 \cup J_2 \cup \cdots \cup J_n$
$\Bad \subseteq \pred^*(\Safe)$
can be reformulated as
$\Bad \cap (S \setminus \pred^*(\Safe)) = \emptyset$
the resilience problem is decidable.
\end{proof}

Let us recall that a system $\mathscr{S}=(S,\rightarrow, \leq)$ is downward compatible if
for all $s_1, s_2, t_1 \in S$ with
$s_2 \leq s_1$ and $s_1 \to t_1$
there
exists $t_2 \in S$ with
$t_2 \leq t_1$ and $s_2 \to^* t_2$.

\begin{corollary}
The resilience problem is decidable for ideal-effective downward-compatible WSTS with 
$\Safe=\downarrow \Safe$
and $\Bad=\uparrow \Bad$.
\end{corollary}

\begin{proof}


Let $D$ be a downward-closed subset of $S$
and let $x \in \downarrow \pred^*(D)$.
By downward closure, there exists
$y \in \pred^*(D)$ 
such that $x \leq y$.
By definition of $\pred^*(D)$ then there exists 
$d \in D$, $m\geq 0$ and $(a_i)_{0 \leq i \leq m+1} \in S^{m+2}$ such that
$y = a_0 \to a_1 \to a_2 \to \cdots \to a_m \to a_{m+1} = d$.

By downward compatibility $a_0 \to a_1$
implies that there exists $a'_1 \in S$ such that $a'_1 \leq a_1$ and
$x \to^* a'_1$.
More generally $a_i \to a_{i+1}$ and
$a'_i\leq a_i$ implies the existence of $a'_{i+1} \in S$ with $a'_{i+1} \leq a_{i+1}$ and
$a'_i \to^* a'_{i+1}$,
and, by induction,
 $x \to^* a'_1 \to^* \cdots \to^* a'_{m} \to^* a'_{m+1} = d'$
with $d' \leq d$.
Since
$d'$ 
% \alain{qui appartient à qui ?}
belongs to $D$ by downward closure of $D$, $x \in \pred^*(D)$.
\end{proof}

% \begin{proposition}
In the case
of a ideal-effective WSTS 
where
the additional hypothesis that
for all downward-closed set $D \subseteq S$, the set $\pred^*(D)$ is downward-closed
is not met,
the above construction
can provide a proof
of non-resilience
i.e. when
$\Bad \cap (S \setminus \downarrow\pred^*(\Safe)) \neq \emptyset$
then
$\Bad \not\subseteq \downarrow\pred^*(\Safe)$
and hence
$\Bad \not\subseteq \pred^*(\Safe)$.
When $\Bad \cap (S \setminus \downarrow\pred^*(\Safe)) = \emptyset$
however
it is not enough to conclude.

% \alain{indecidabilite en enlevant une ou des hypotheses: The resilience problem is undecidable for ideal-effective WSTS}

% \mathieu{
% Home-state indécidable pour les minsky machine (preuve très rapide: M à $2$ compteurs s'arrête ssi M' visite $(0,0,0)$ depuis $\uparrow(0,0,1)$, où M' est M mais avec un troisième compteur qui commence à $1$ et qui simule M jusqu'à ce que M s'arrête et à ce moment là décroit les deux premiers compteurs puis le troisième et atteint $(0,0,0)$).
% or les minsky machine peuvent être simulées par les reset petri nets, on réduit le home-state des minsky machine au home-state des reset petri nets, ce qui donne l'indécidabilité du home-state pour les WSTS en général.}


Without the additional hypothesis that
for all downward-closed set $D \subseteq S$, the set $\pred^*(D)$ is downward-closed, the problem becomes undecidable. Indeed, it is 
undecidable for Reset 
%Petri-nets
VAS whether zero (the vector containing only zeroes, also denoted~$ \textbf{0}$) is a home-state, where it is a particular case of resilience with $\Safe = \downarrow \textbf{0}$, $\Bad$ complement of $\Safe$. This stems from the fact that it is undecidable for Minsky machines with more than one counter, whether zero is a home-state. See Appendix~\ref{HS-Minsk} for a more detailed construction.


Remark we did not make use of the hypothesis $\Bad$ complement of $\Safe$, simply 
$\Bad=\uparrow \Bad$ and $\Safe=\downarrow \Safe$, thus 
the above results still hold in the more general case where $\Bad$ and $\Safe$ are not complement of each others.


