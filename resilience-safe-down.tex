



%%%%%%%%
\subsection{Case: $\Safe=\downarrow \Safe$ and $\Bad=\uparrow \Bad$.}
%
%		cas exclusion mutuelle
%
example: mutual exclusion.

\begin{theorem}
The resilience problem is decidable for ideal-effective WSTS with 
$\Safe=\downarrow \Safe$
and $\Bad=\uparrow \Bad$
and
the additional hypothesis that
for all $D \subseteq S$ downward-closed, $\pred^*(D)$ is downward-closed.
% \alain{est-ce que  $\pred(D)$ is downward closed avec D downward closed serait suffisant ? si l'ordre est omega2 alors closbas+inclusion est wqo donc il faudrait juste que la suite croissante $pred^n(D)+le reste$ converge...}
\end{theorem}

\begin{proof}
By hypothesis $\pred^*(\Safe)$ is downward-closed, since $\Safe$ is downward-closed.
 Let $\{J_1, J_2,...,J_n\}$ be the ideal decomposition of $\pred^*(\Safe)$ and $\{b_1,b_2,...,b_m\}$ be the (unique) minimal basis of $\Bad$.
The resilience problem can be reformulated as 
 $\uparrow \{b_1,b_2,...,b_m\} \subseteq J_1 \cup J_2 \cup \cdots \cup J_n$.
Since $\mathscr{S}=(S,\rightarrow, \leq)$ is ideally effective, we can compute intersections of closed subsets.
Hence we can compute the intersection of
$\uparrow \{b_1,b_2,...,b_m\}$
and
$S \setminus \pred^*(\Safe)$.
Since
$\uparrow \{b_1,b_2,...,b_m\} \subseteq J_1 \cup J_2 \cup \cdots \cup J_n$
can be reformulated as
$\uparrow \{b_1,b_2,...,b_m\} \cap (S \setminus \pred^*(\Safe)) = \emptyset$
the resilience problem is decidable.
\end{proof}

\begin{corollary}
The resilience problem is decidable for ideal-effective downward-compatible WSTS with 
$\Safe=\downarrow \Safe$
and $\Bad=\uparrow \Bad$.
\end{corollary}

\begin{proof}
Downward compatibility of $\mathscr{S}=(S,\rightarrow, \leq)$ 
means that
for all $s_1, s_2, t_1 \in S$ with
$s_2 \leq s_1$ and $s_1 \to t_1$
there
exists
$t_2 \leq t_1$ such that $s_2 \to^* t_2$.

Let $D$ be a downward-closed subset of $S$
and let $x \in \downarrow \pred^*(D)$.
By design there exists
$y \in \pred^*(D)$ 
such that $x \leq y$
and $d \in D$ such that
$y \to a_1 \to a_2 \to \cdots \to a_m \to d$.

By downward compatibility

$y \to a_1$

implies

$x \to^* a'_1$

with 
$a'_1 \leq a_1$.

Then

$a_i \to a_{i+1}$

implies the existence of $a'_{i+1} \leq a_{i+1}$ such that

$a'_i \to^* a'_{i+1}$,

and by induction

 $x \to^* a'_1 \to^* \cdots \to^* a'_{m} \to^* d'$

where

$d' \leq d$ belongs to $D$ by downward closure of $D$.

Hence $x \in \pred^*(D)$.
\end{proof}

% \begin{proposition}
In the case









\subsection{Case: $\Safe=\downarrow \Safe$ and $\Bad=\downarrow \Bad$.}

example: ?

% In the case \Safe and \Bad are both downward closed, they can both be finite.

\begin{theorem}\label{down-down}
The resilience problem is undecidable for WSTS with
%	with more than one minimal element, 
$\Safe=\downarrow \Safe$
and $\Bad=\downarrow \Bad$.
\end{theorem}

\begin{proof}
In the case of a WSTS with at least two minimal elements $m_1$ and $m_2$, the problem of whether $m_2$ is reachable from $m_1$ reduces itself to the resilience problem, hence its undecidability.  
\mathieu{dévellopper un peu la preuve, $m_2$ minimal de $\Safe$, $m_1$ minimal de $\Bad$...}
\end{proof}

In the case of a WSTS with only one minimal element,  
% (ex: Petri Nets and their «Zero») 
then by downward closure the minimal element is present in both $\Safe$ and $\Bad$ and hence resilience always hold.



