
% This is samplepaper.tex, a sample chapter demonstrating the
% LLNCS macro package for Springer Computer Science proceedings;
% Version 2.21 of 2022/01/12
%
\documentclass[runningheads]{llncs}
%
\usepackage[T1]{fontenc}
% T1 fonts will be used to generate the final print and online PDFs,
% so please use T1 fonts in your manuscript whenever possible.
% Other font encondings may result in incorrect characters.
%
\usepackage{graphicx}
% Used for displaying a sample figure. If possible, figure files should
% be included in EPS format.
%
% If you use the hyperref package, please uncomment the following two lines
% to display URLs in blue roman font according to Springer's eBook style:
%\usepackage{color}
%\renewcommand\UrlFont{\color{blue}\rmfamily}
%

\bibliographystyle{splncs04}% the mandatory bibstyle
\usepackage{booktabs}   %% For formal tables:
                        %% http://ctan.org/pkg/booktabs
\usepackage{subcaption} %% For complex figures with subfigures/subcaptions
                        %% http://ctan.org/pkg/subcaption



\usepackage{mathtools}
\usepackage{todonotes}
\usepackage{microtype}

\usepackage{complexity}
\usepackage{amsmath}

\usepackage{stmaryrd}
\usepackage{dsfont}


% \usepackage{MnSymbol} % %
\usepackage{mathrsfs}
\usepackage{mathalpha}
\usepackage{amsmath}
\usepackage{amsfonts}


\usepackage{ textcomp } 

\usepackage{stmaryrd}
\usepackage{wrapfig}


%

%\newcommand{\pred}{\textsf{pred}}
%\newcommand{\post}{\textsf{post}}
% \renewcommand{\succ}{\textsf{Succ}}

\newcommand{\problemx}[3]{
	\vspace{0.2cm}
\par\noindent\underline{\sc#1}\par\nobreak\vskip.2\baselineskip
\begingroup\clubpenalty10000\widowpenalty10000
\setbox0\hbox{\bf INPUT:\ }\setbox1\hbox{\bf QUESTION:\ }
\dimen0=\wd0\ifnum\wd1>\dimen0\dimen0=\wd1\fi
\vskip-\parskip\noindent
\hbox to\dimen0{\box0\hfil}\hangindent\dimen0\hangafter1\ignorespaces#2\par
\vskip-\parskip\noindent
\hbox to\dimen0{\box1\hfil}\hangindent\dimen0\hangafter1\ignorespaces#3\par
\endgroup
	\vspace{-0.2cm}
}

\newcounter{claimcounter}
\setcounter{claimcounter}{0}
\newtheorem{subclaim}{Subclaim}{}
\newtheorem{fact}{Fact}{}
% \newtheorem*{theorem*}{Theorem}


\makeatletter



\renewcommand{\poly}{\mathrm{poly}}


\newcommand{\alain}[1]{\todo[inline,color=red!20]{{\bf AF:} #1}}
\newcommand{\mathieu}[1]{\todo[inline,color=blue!20]{{\bf MH:} #1}}



\newcommand{\pred}{\textsf{pred}}
\newcommand{\post}{\textsf{post}}
% \renewcommand{\succ}{\textsf{Succ}}

\newcommand{\Bad}{\textsf{Bad}}
\newcommand{\Safe}{\textsf{Safe}}


\begin{document}
%
\title{Resilience and Home-Space for WSTS} 
%
%\titlerunning{Abbreviated paper title}
% If the paper title is too long for the running head, you can set
% an abbreviated paper title here
%
\author{Alain Finkel\thanks{Institut Universitaire de France} and Mathieu Hilaire\thanks{This work was partly done while the two authors were supported by the Agence Nationale de la Recherche grant BraVAS (ANR-17-CE40-0028).}}
% This research has been supported by ANR programme BraVAS (ANR-17-CE40-0028).
%
\authorrunning{A. Finkel, M. Hilaire.}
% First names are abbreviated in the running head.
% If there are more than two authors, 'et al.' is used.
%
\institute{Université Paris-Saclay, CNRS, ENS Paris-Saclay, LMF, Gif-sur-Yvette, France %\email{hilaire@lsv.fr}
}
%
\maketitle              % typeset the header of the contribution
%


%\titlerunning{Resiliency} 
%
%\author{Alain Finkel}
%{Université Paris-Saclay, CNRS, ENS Paris-Saclay, LMF, Gif-sur-Yvette, France}
%{finkel@lsv.fr}{}{This work was partly done while the author was supported by the Agence Nationale de la Recherche grant BraVAS (ANR-17-CE40-0028)).}



% \ccsdesc{Theory of computation~Automata over infinite objects}
% \ccsdesc[500]{Theory of computation~Automata extensions}




% \category{} 

% \relatedversion{}
% \relatedversiondetails{Full Version}{}

%\acknowledgements{The author would like to thank.}





\begin{abstract}
	

We investigate and expand upon the notion of resilience as defined by Okan Özkan and Nick Würdemann~\cite{DBLP:journals/corr/PrasadZ16,DBLP:journals/corr/abs-2108-00889,DBLP:conf/gg/Ozkan22}, which asks whether a system can reach a safe state in a bounded number of steps whenever it reached a bad state. We introduce a more general definition of resilience and we define three decidability questions for each of type of resilience: The (state-)resilience problem (RP), the bounded (state-)resilience problem (BRP)
and the $k$-(state-)resilience problem (kRP). The (general) resilience problem 
% is the following: let $L_1,L_2 \subseteq S$ be two subsets of states, S is $(L_1,L_2)$-resilient if $\post^*(L_1)	\subseteq \pred^*(L_2)$. 
asks whether a set of safe elements $\Safe$ is reachable for all elements in a set of bad elements $\Bad$. 
We remark that it is an instance of the home space problem defined in the framework of VASS.
State resilience only consider bad elements that are reachable from an initial state, while bounded-resilience and $k$-resilience ask whether safe is reachable in a bounded number of step and $k$ or less steps respectively.

In a well-structured framework we show that resilience and state-resilience problems are both undecidable for WSTS with strong compatibility. Moreover, state-resilience,
bounded-state-resilience and
 $k$-state-resilience
are undecidable for strongly upward-compatible WSTS with effective pred-basis. 

On the positive side, we find classes of WSTS that have a \emph{decidable} resilience. The resilience problem (RP), the bounded resilience problem (BRP)
and the $k$-resilience problem (kRP) are decidable for completion-post-effective $\omega^2$-WSTS with strong compatibility and two sets $\Bad = \downarrow \Bad$ and $\Safe = \uparrow \Safe$, and moreover the resilience problem is decidable for ideal-effective WSTS with 
$\Safe=\downarrow \Safe$
and $\Bad=\uparrow \Bad$
and
the additional hypothesis that
for all downward-closed set $D \subseteq S$, the set $\pred^*(D)$ is downward-closed.

We then generalize the main theorem of \cite{DBLP:journals/corr/abs-2108-00889,DBLP:conf/gg/Ozkan22} by relaxing the strong compatibility hypothesis: {\sc State-resilience} is decidable for 
 WSTS with effective 
$\uparrow$ $\post^*$ basis
when
$\Safe=\uparrow \Safe$
and $\Bad=\downarrow \Bad$. However, we show that removing the effective 
$\uparrow$ $\post^*$ basis hypothesis leads to undecidability, but we found another effective hypothesis: {\sc State-resilience} is decidable for ideal-effective WSTS with downward and upward compatibilities, and {\sc $k$-state-resilience} and {\sc bounded-state-resilience} are decidable for ideal-effective WSTS with strong downward compatibility.

Finally, we study the resilience for VASS, where most problems are shown decidable.







\end{abstract}

\keywords{Verification, Resilience, Well-structured transition systems, Vector addition system with states}


\alain{factoriser remerciements à ANR (fait), ajouter mon affiliation à IUF}
\newcommand{\LCM}{\mathsf{LCM}}
\newcommand{\LOGSPACE}{\mathsf{LOGSPACE}}
\newcommand{\MSO}{\mathsf{MSO}}
\newcommand{\SO}{\mathsf{SO}}

 \newcommand{\N}{\mathds{N}}



	\section{Introduction}\label{section introduction}


{\bf Context.} In the quite informal paper \cite{DBLP:journals/corr/PrasadZ16}, Prasad and Zuck studied resilience of a process to an adversary and they used the framework of WSTS enjoying both upward and downward monotonies to prove the decidability of a kind of resilience.

MFCS 2023 \\
Abstract submission deadline:    	April 24th (AoE) \\
Paper submission deadline:    	April 28th (AoE)

trop d'hypotheses trop fortes sur les WSTS

{\bf Our contributions}
We introduce a more general definition of resilience generalising the two given in \cite{DBLP:conf/gg/Ozkan22}.
Surprinsingly, the general undecidability statements were not known neither proved.

\begin{itemize}

\item safe clos haut, bad clos par le bas: resilience = reachability d'un clos par le bas (anti-coverability)

\item safe clos haut : resilience indecidable for WSTS with strong.

\end{itemize}


\section{Definitions}\label{section definitions}



In this section, we introduce general notations and preliminary definitions.






\problemx{$k$-Resilience problem}
{A state $s$ of a WSTS $(S,,)$, an ideal $I$ with a given basis, a decidable anti-ideal $J$.}
{$\forall s' \in J ~ (s \rightarrow^* s') \to \exists s'' \in I ~ s' \rigtarrow^{\leq k} s''$ ?\newline}


\problemx{Bounded Resilience problem}
{A state $s$ of a WSTS $(S,,)$, an ideal $I$ with a given basis, a decidable anti-ideal $J$.}
{$\exists k \geq 0 ~ \forall s' \in J ~ (s \rightarrow^* s') \to \exists s'' \in I ~ s' \rigtarrow^{\leq k} s''$ ?\newline}






%%%
\iffalse
%%%%
%%%%
\subsection{Defining resilience}


\subsubsection{Resilience for general transition systems}


We ask whether we can reach a state 
%which satisfies
in 
%
$\Safe$  in a reasonable amount of time whenever we reach a state 
% which satisfies
in
%
$\Bad$. 
From this we formulate two resilience problems. First consider the case where the recovery time
is bound by a given natural number $k \geq 0$, i.e., the \emph{explicit resilience problem} for TS.

\problemx{$k$-Resilience problem }
{a transition system $(S,\rightarrow)$, an integer $k$, a state $s \in S$, $\Safe, \Bad \subseteq S$ and $\Safe \cap \Bad = \emptyset$.}
{$\forall s' \in \Bad, ~ s \rightarrow^* s' \implies \exists s'' \in \Safe ~ s' \rightarrow^{\leq k} s''$ ?\newline}

If a system $S$ satisfies the explicit resilience property for an integer $k$, we say that $S$ is $k$-resilient.

% If we assume that the transition system yields infinite sequences of transitions, we can express the property to be evaluated in CTL by s |= AG(\Bad → 0≤ j≤k EX j \Safe). 

We can also ask whether there exists such a bound $k$ such that $S$ is $k$-resilient. We call this problem the \emph{bounded resilience problem}.

%
%\problemx{Bounded Resilience problem}
%{a transition system $(S,\rightarrow)$, a state $s \in S$, $\Safe, \Bad \subseteq S$ and $\Safe \cap \Bad = \emptyset$.}
%{$\exists k \geq 0 ~ \forall s' \in \Bad, ~ s \rightarrow^* s' \implies \exists s'' \in \Safe ~ s' \rightarrow^{\leq k} s''$ ?\newline}
%


\problemx{Bounded Resilience problem}
{a transition system $(S,\rightarrow)$, a state $s \in S$, $\Safe, \Bad \subseteq S$ and $\Safe \cap \Bad = \emptyset$.}
{$\exists k \geq 0  \mathscr{S}$ is $k$-resilient. ?\newline}
%%

%%%


pour $k,k'$ donnés, chercher si $\Safe_{max}$ et $\Bad_{max}$ ont un sens ? on peut faire l'union des \Safe, \Safe' et des \Bad, \Bad' prendre le max des k. Fixer k, chercher et calculer $\Safe_{max}$ et $\Bad_{max}$.

%
%\alain{dire juste que \Safe = upward closed et \Bad = downward closed.
%envisager les 3 autres cas. souvent \Bad = upward closed (par ex l'exclusion mutuelle).
%4 resilience problem (U,D), (U,U), (D,U), (D,D)}

\fi




\newcommand{\Bad}{\textsf{Bad}}
\newcommand{\Safe}{\textsf{Safe}}



\section{Resilience for WSTS}

In a transition system $\mathscr{S}=(S,\rightarrow)$, we consider two subsets of states $\Safe, \Bad \subseteq S$ such that $\Safe \cap  \Bad = \emptyset$.
%	The property of mutual exclusion is often modelised with $\Safe = \downarrow \Safe$.
The \emph{resilience problem} (resp. the \emph{$k$-resilience problem}) for $(\mathscr{S},\Safe,\Bad)$ is to decide whether from \emph{any} state in $\Bad$, \emph{there exists} a path (resp. a path of length smaller than or equal to $k$) that reaches a state in $\Safe$. We use the notation $\Bad \longrightarrow^{*} \Safe$ (resp. $\Bad \longrightarrow^{\leq k} \Safe$) for $\forall x \in \Bad, \exists y \in \Safe$ such that $x \longrightarrow^{*} y$ (resp.  $\forall x \in \Bad, \exists y \in \Safe$ such that $x \longrightarrow^{\leq k} y$). In our framework, $\Safe, \Bad \subseteq S$  are possibly infinite but they must admit a computable finite representation : for example, downward-closed sets and upward-closed sets in wqo and semilinear sets in $\mathbb{N}^d$ have finite representations. 
%	If $\Safe = S \backslash \Bad$, then $\Safe$ is downward-closed iff $\Bad$ is upward-closed. 
Let us formalize three resilience problems.

\problemx{resilience problem (RP)}
{A transition system $\mathscr{S}=(S,\rightarrow)$ and two sets $\Safe, \Bad \subseteq S$.}
{$\Bad \longrightarrow^{*} \Safe$ ?\newline}
%
%\alain{il faudrait ne pas répéter 3 fois les mêmes imputs pour les 3 pbs: énoncer les 3 uniformes pbs d'un coup avec une fois l'input puis les 3 pbs pour un état s donné sans répéter non plus 3 fois les mêmes inputs}

%We add a third problem that decides whether there exists an $k$ such  that the system is $k$-resilient.
%

\problemx{$k$-resilience problem (kRP)}
{A transition system $\mathscr{S}=(S,\rightarrow), k \in \mathbb{N}$ and two sets $\Safe, \Bad \subseteq S$.}
{$\Bad \longrightarrow^{\leq k} \Safe$ ?\newline}

\problemx{bounded resilience problem (BRP)}
{A transition system $\mathscr{S}=(S,\rightarrow)$ and two sets $\Safe, \Bad \subseteq S$.}
%{$\exists k \geq 0 ~ \forall s' \in D ~ s \rightarrow^* s' \implies \exists s'' \in U ~ s' \rightarrow^{\leq k} s''$ ?\newline}
{$\exists k \geq 0$ such that $\mathscr{S}$ is %uniformely
 $k$-resilient ?\newline}


\alain{dessous: à dire mais pas sous forme de proposition}
\begin{proposition}[Reformulation]\label{reformulation}
$\mathscr{S}=(S,\rightarrow,\leq)$ is %uniformely 
(\Bad,\Safe)-resilient iff $\Bad \subseteq \pred^*(\Safe)$.\\
$\mathscr{S}=(S,\rightarrow,\leq)$ is %uniformely 
(\Bad,\Safe)-$k$-resilient iff $\Bad \subseteq \pred^k(\Safe)$.
\end{proposition}

These three resilience problems are decidable for finite transition systems but undecidable for (general) infinite-state transition systems. So we restrict our framework to the class of infinite-state WSTS. Since most of decidable properties in WSTS rely on the computation of upward or downward closed sets \cite{DBLP:journals/iandc/AbdullaCJT00, DBLP:journals/tcs/FinkelS01}, we consider upward closed or downward closed sets $\Safe$ and $\Bad$.

For instance, the well-known mutual exclusion property is often modelized in a $d$-counters machine by the property that a counter $c_{mutex}$ must be bounded by (usually) one. Then, the set $\Safe =  \{c_{mutex} \leq 1\} \times \mathbb{N}^{d-1}$ is downward closed and $\Bad =\{c_{mutex} \geq 2\} \times  \mathbb{N}^{d-1} $ is the upward closed complementary of $\Safe$. 

In \cite{DBLP:conf/gg/Ozkan22}, the authors considered that $\Bad$ is downward closed and $\Safe$ is upward closed.
%		
RP is decidable for lossy counter machines (LCM) with $\Safe$ and $\Bad$ semilinear sets as a consequence of results in Section 3.4 of \cite{DBLP:conf/rp/Schnoebelen10}. We deduce that the three resilience problems are decidable for all four pairs of sets $\Safe$ and $\Bad$ downward closed and upward closed sets.

In particular, the existing proofs of resilience use the decidability of reachability; but we wish to decide resilience for models with undecidable reachability.
%

%	 and ideals are recursive 
\alain{???}.




\iffalse

\begin{proposition}\label{general}
\textcolor{red}{
$\mathscr{S}=(S,\rightarrow,\leq)$ is %uniformely 
(\Bad,\Safe)-bounded-resilient iff $\mathscr{S}=(S,\rightarrow,\leq)$ is %uniformely 
(\Bad,\Safe)-resilient (CONJECTURE FOR NOW)
}
\end{proposition}

\begin{proof}
Resilient means that ...
\end{proof}

\fi







\subsection{Case: $\Safe=\uparrow \Safe$.}

%

We start with the case $\Safe=\uparrow \Safe$ hence $\Bad=\downarrow \Bad$. 
It can be viewed as a generalizeation
 of coverability, as
it asks whether for every element of $\Bad$ it is possible to cover an element of the basis of $\Safe$.


% Let us choose $\Bad$ to be the set of states from which there is no infinite run. If we are in an upward compatible ordered transition system (like VASS, lossy channel systems,...), then $\Bad$ is downward-closed.
% \alain{ce qui précède est faux et pas convaincant: un vass qui seulement décroit ses compteurs}

%%%%

\iffalse
As a
example: in a VASS, we may choose $\Safe$ to be the set of states that are not deadlocks, i.e. from which it is always possible to fire a transition. This set of states is upward-closed and $\Safe=S \setminus \Bad=\uparrow \Safe$.\alain{mal rédigé}
\fi

%  Parosh Aziz Abdulla, Karlis Cerans, Bengt Jonsson & Yih-Kuen Tsay (1996): General Decidability Theorems for Infinite-State Systems. In: Proc. LICS 1996, IEEE Computer Society Press, pp. 313–321,
%  Alain Finkel & Philippe Schnoebelen (2001): Well-structured transition systems everywhere! Theor. Comput. Sci. 256(1-2), pp. 63–92, doi:10.1016/S0304-3975(00)00102-X.

%Transfering the abstract resilience problems into this framework,
%it is therefore reasonable to demand that both propositions, \Safe and \Bad, are given by 
%upward-closed or downward-closed sets.

%


% \mathieu{I'm using the notion of ideally effective-ness for WBTS here}
%
Let us recall that the \emph{completion}  \cite{BFM-ic17} of a WSTS $\mathscr{S}=(S,\rightarrow, \leq)$ is the associated ordered transition system $\hat{\mathscr{S}}=(Ideals(S),\rightarrow, \subseteq)$ where states of $\hat{\mathscr{S}}$ are ideals of $S$ and $I \rightarrow J$ if $J$ belongs to the finite ideal decomposition of $\downarrow \post_{\mathscr{S}}(I)$. The completion is always finitely branching but it is not necessarly WSTS since $\subseteq$ is not necessarly a wqo. $\hat{\mathscr{S}}$ is WSTS iff $\mathscr{S}=(S,\rightarrow, \leq)$ is $\omega^2$-WSTS (intuitively speaking, $(S,\leq)$ must not contain the Rado set). 
A WSTS $\mathscr{S}=(S,\rightarrow,\leq)$ is {\em completion-post-effective} if
it is post-effective, and there exists a Turing Machine $M_\downarrow$ that computes, on input $s \in S$, % some $\hat(e) \in \hat( ... )$ such that
 $\downarrow s$, and some Turing Machine $M_{\uparrow^C}$ that computes, on input
 $\{s_1, s_2, \ldots, s_m\} \in S^m$, 
 the ideal decomposition of $S \setminus \uparrow \{ s_1, s_2, \ldots, s_m\}$.
Coverability is shown decidable  [Theorem 44] in \cite{BFM-ic17} for completion-post-effective $\omega^2$-WSTS (we don't need the pred-basis hypothesis).

Let us recall two other results in \cite{BFM-ic17}. Proposition 30 establishes a strong relation between the runs of a WSTS $\mathscr{S}=(S,\rightarrow, \leq)$ and the runs of its completion $\hat{\mathscr{S}}$. It states that if $x \xrightarrow{k} y$ in $\mathscr{S}$ then for every ideal $I \supseteq \downarrow x$, there exists an ideal $J \supseteq \downarrow y$ such that $I \xrightarrow{k} J$ in $\hat{\mathscr{S}}$. Proposition 29 establishes that if $I \xrightarrow{k} J$ in $\hat{\mathscr{S}}$ then for every $y \in J$, there exists $x \in I$ and $y' \geq y$ such that $x \xrightarrow{k'} y'$ in $\mathscr{S}$. Moreover, if $\mathscr{S}$ has transitive compatibility then $k’ \geq k$; if $\mathscr{S}$ has strong compatibility then $k’ = k$.
%





%
\begin{theorem}\label{down-up}
Let $\mathscr{S}=(S,\rightarrow, \leq)$ be a completion-post-effective $\omega^2$-WSTS with strong compatibility and two 
%finite \alain{non, confusion entre ensemble et base}
 sets $\Bad = \downarrow \Bad$ and $\Safe = \uparrow \Safe$.
The resilience problem (RP), the bounded resilience problem (BRP)
and the $k$-resilience problem (kRP) are decidable.
\end{theorem}

\begin{proof}
Let $\{J_1, J_2,...,J_n\}$ be the ideal decomposition of $\Bad$ and $\{b_1,b_2,...,b_m\}$ be the (unique) minimal basis of $\Safe$.
The %uniform 
resilience problem (RP) can be reduced to the following infinite number of instances of the coverability problem in $\mathscr{S}$: for all $x \in \Bad$ does there exist an $j$ such that $b_j$ is coverable from $x$. Let us show how this infinite set of coverability questions can be reduced to a \emph{finite} set of coverability questions in the completion $\hat{\mathscr{S}}=(Ideals(S),\rightarrow, \subseteq)$ of $\mathscr{S}=(S,\rightarrow, \leq)$. 

%Proposition 30 in \cite{BFM-icalp14} establishes a strong relation between the runs of a WSTS $\mathscr{S}=(S,\rightarrow, \leq)$ and the runs of its completion $\hat{\mathscr{S}}$. It states that if $x \xrightarrow{k} y$ in $\mathscr{S}$ then for every ideal $I \supseteq \downarrow x$, there exists an ideal $J \supseteq \downarrow y$ such that $I \xrightarrow{k} J$ in $\hat{\mathscr{S}}$. Proposition 29 establishes that if $I \xrightarrow{k} J$ in $\hat{\mathscr{S}}$ then for every $y \in J$, there exists $x \in I$ and $y' \geq y$ such that $x \xrightarrow{*} y'$ in $\mathscr{S}$.

Let us prove that $b_j$ is coverable from $x$ in $\mathscr{S}$ if and only if $\downarrow b_j$ is coverable (for inclusion) from $\downarrow x$ in $\hat{\mathscr{S}}$.
%
Suppose that $b_j$ is coverable from $x$ then there exists a run $x \xrightarrow{k} y \geq b_j$. From Proposition 30, there exist an ideal $J$ and a run $\downarrow x \xrightarrow{k} J$ where $J \supseteq \downarrow y \supseteq \downarrow b_j$ in $\hat{\mathscr{S}}$, hence $\downarrow b_j$ is covered from $\downarrow x$.
Conversely, if $I \xrightarrow{k} J$ in $\hat{\mathscr{S}}$ with $\downarrow b_j \subseteq J$ then 
%	for every $y \in J$, 
there exists $x \in I$ and $y' \geq b_j$ such that $x \xrightarrow{k} y'  \geq b_j$ in $\mathscr{S}$ and then $b_j$ is coverable from $x$ in $\mathscr{S}$.

Hence we obtain: $\mathscr{S}$ is resilient iff for all $i=1,..,n$ and $j= 1,..m$, $\downarrow b_j$ is coverable from ideal $J_i$ in $\hat{\mathscr{S}}$.
%
Let us denote by $k_{i,j}$ the length of a covering sequence that covers $\downarrow b_j$ from $J_i$ in $\hat{\mathscr{S}}$ and let $k_{i,j}\stackrel{\text{def}}{=}\infty$ if $\downarrow b_j$ is not coverable from $J_i$. Let us now define $K_{\mathscr{S}}(\Safe,\Bad)=\max(k_{i,j} \mid i=1,..,n$ and $j= 1,..m$).
%

%
%	(if all $k_{i,j}$ are finite) else $K=\infty$.
We now have $\mathscr{S}$ is resilient iff $K_{\mathscr{S}}(\Safe,\Bad)$ is finite iff $\mathscr{S}$ is $K_{\mathscr{S}}(\Safe,\Bad)$-resilient with $K_{\mathscr{S}}(\Safe,\Bad)$ finite.

This implies that resilience and bounded resilience are equivalent to coverability.
%
% iff there is a run in $\hat{S}$ from an ideal I to J such that  $\downarrow x \subseteq I$ and $\downarrow b_j \subseteq J$. This is a consequence of  
%Propositions 29 and 30 in \cite{BFM-icalp14} that establish a strong relation between the runs of a WSTS $\mathscr{S}=(S,\rightarrow, \leq)$ with its completion $\hat{S}$.
%' \in I(\Bad)$.
%
%To decide the bounded resilience, we decide $n \times m$ coverability questions: is state $\downarrow b_j$ coverable from ideal $J_i$ ? If all these $n \times m$ coverability questions are positive then we compute $K=\max(k_{i,j} \mid i=1,..,n $ and $j= 1,..m)$ where $k_{i,j}$ is the least length of a sequence that covers  $\downarrow b_j$ from $J_i$.
%
 % \alain{comment trouver les $k_a$ ?} 
%
%  \alain{else $\mathscr{S}$ is not $K$-resilient...et alors qu'en déduit-on ? il pourrait exister un $K' \geq K$ pour lequel $\mathscr{S}$ est K'-resilient....}. 
%else if some of these $n \times m$ coverability questions are negative then resilience
%do not hold and bounded resilience do not hold either.
Since coverability is decidable for completion-post-effective $\omega^2$-WSTS, we deduce that both the 
  resilience problem (RP) and the bounded resilience problem (BRP) are decidable.

\iffalse
	\end{proof}

	\begin{theorem}\label{k-down-up}
	Let $\mathscr{S}=(S,\rightarrow, \leq)$ be a completion-post-effective $\omega^2$-WSTS 
	with strong compatibility and the predbasis hypothesis, and two finite sets: $\Bad$ 
	and $\Safe$.
	The  $k$-resilience problem (kRP) is decidable 
	%	for $k \geq \min(K,n)$ where $n$ satisfies $ \uparrow \pred^n(\Safe)=  \uparrow 	\pred^*(\Safe)$. 
	\textcolor{red}{CONJECTURE for now}
	 \end{theorem}

	\begin{proof}

We begin to compute $K$ and $n$ such that $ \uparrow \pred^n(\Safe)=  \uparrow \pred^*(\Safe)$.
If $K=\infty$ then $\mathscr{S}$ is not resilient for Safe and Bad.
Now, if $k \geq min(K,n)$, we conclude that is $k$-resilient; 
%	if moreover $\mathscr{S}=(S,\rightarrow, \leq)$ has the predbasis hypothesis, 
%	and then kRP is decidable.
\fi

Let us now show that the $k$-resilience problem (kRP), with $k \in \mathbb{N}$, is also decidable.
Let us denote by $k'_{i,j}$ the \emph{minimal} length of a covering sequence that covers $\downarrow b_j$ from $J_i$ in $\hat{\mathscr{S}}$ if it exists and let $k'_{i,j}\stackrel{\text{def}}{=}\infty$ if $\downarrow b_j$ is not coverable from $J_i$. 
If $\downarrow b_j$ is coverable from $J_i$, we first compute an $k_{i,j}$, and then we compute $k'_{i,j}$ by iteratively checking whether there exists a sequence of length $0,1,...,k_{i,j}-1$ that covers $\downarrow b_j$ from $J_i$ until we find the minimal one which is necessarly smaller (or equal to) than $k_{i,j}$.

Let us now define $K'_{\mathscr{S}}(\Safe,\Bad)=\max(k'_{i,j} \mid i=1,..,n$ and $j= 1,..m$) and we deduce that  $\mathscr{S}$ is $k$-resilient iff $k \geq K'_{\mathscr{S}}(\Safe,\Bad)$.
%	If $k <  \min(K,n)$, then we check every path of length smaller
%	than $\min(K,n)$ from the ideals of the decomposition of $\Bad$ in the completion \alain{à expliquer mieux}.
\end{proof}

% \alain{indecidabilite en enlevant une ou des hypotheses: The resilience problem (RP), the bounded resilience problem (BRP) and the $k$-resilience problem (kRP) are undecidable for WSTS with strong compatibility}
%


Without the completion-post-effective hypothesis nor the pred-basis hypothesis, coverability 
becomes undecidable. For instance the coverability problem for strongly increasing well-structured nets is undecidable~\cite{FMP-ic04}. 
Coverability reduce
to resilience when $\Bad = \downarrow \Bad$ and  $\Safe = \uparrow \Safe$,
as $s \to^* \uparrow t$ 
reduces to $(\Safe,\Bad)$-resilience for $\Bad = \downarrow s$ and $\Safe = \uparrow t$.
Hence, without the completion-post-effective hypothesis, 
{\sc Resilience} itself is undecidable.

Remark we did not make use of the hypothesis that $\Bad$ is the complement of $\Safe$, simply using
$\Bad=\downarrow \Bad$ and $\Safe=\uparrow \Safe$, thus 
the above results still hold in the more general case where $\Bad$ and $\Safe$ are not complements of each others.






%%%%%%%%
\subsection{Case: $\Safe=\downarrow \Safe$.}
%
%		cas unbounded counters
%

Let us now consider the case $\Safe=\downarrow \Safe$ hence $\Bad=\uparrow \Bad$.
It is of interest to note this case can be linked to the problem of mutual exclusion.
Indeed the well-known mutual exclusion property can be modelized, in a $d$-VASS with $d$ counters, by the property that a special counter $c_{mutex}$ must be bounded by $k \geq 1$ which counts the (maximal) number of processes that are allowed to be simultaneously in the critical section. Then, the set $\Safe =  \{c_{mutex} \leq k\} \times \mathbb{N}^{d-1}$ is downward-closed		and $\Bad =\{c_{mutex} \geq k+1\} \times  \mathbb{N}^{d-1} $ is the upward-closed complementary of $\Safe$. 
%  indeed, one can choose $\Bad$ as the set of states from which the counter $c_{mutex}$ is not bounded by $k$, and $\Safe$ to be the downward-closed complement of $\Bad$.

%	 In the case $\Safe=\downarrow \Safe$, our contribution consist in the following:



%
\begin{theorem}\label{up-down}
{\sc Resilience}  is decidable for ideal-effective WSTS with 
$\Safe=\downarrow \Safe$
and
the additional hypothesis that
for all downward-closed set $D \subseteq S$, the set $\pred^*(D)$ is downward-closed.
% \alain{est-ce que  $\pred(D)$ is downward closed avec D downward closed serait suffisant ? si l'ordre est omega2 alors closbas+inclusion est wqo donc il faudrait juste que la suite croissante $pred^n(D)+le reste$ converge...}
\end{theorem}

\begin{proof}
By hypothesis $\pred^*(\Safe)$ is downward-closed, since $\Safe$ is downward-closed.
%  Let $\{J_1, J_2,...,J_n\}$ be the ideal decomposition of $\pred^*(\Safe)$ and $\{b_1,b_2,...,b_m\}$ be the (unique) minimal basis of $\Bad$.
The resilience problem can be reformulated as 
% $\uparrow \{b_1,b_2,...,b_m\} \subseteq J_1 \cup J_2 \cup \cdots \cup J_n$.
$\Bad \subseteq  \pred^*(\Safe)$.
Since $\mathscr{S}=(S,\rightarrow, \leq)$ is ideally effective, we can compute intersections of upward- or downward-closed 
%\alain{non définis}
 subsets.
Hence we can compute the intersection of
% $\uparrow \{b_1,b_2,...,b_m\}$
$\Bad$
and
$S \setminus \pred^*(\Safe)$,
which are both upward-closed.
Since
% $\uparrow \{b_1,b_2,...,b_m\} \subseteq J_1 \cup J_2 \cup \cdots \cup J_n$
$\Bad \subseteq \pred^*(\Safe)$
can be reformulated as
$\Bad \cap (S \setminus \pred^*(\Safe)) = \emptyset$
the resilience problem is decidable.
\end{proof}

Let us recall that a system $\mathscr{S}=(S,\rightarrow, \leq)$ is \emph{downward compatible} if
for all $s_1, s_2, t_1 \in S$ with
$s_2 \leq s_1$ and $s_1 \to t_1$
there
exists $t_2 \in S$ with
$t_2 \leq t_1$ and $s_2 \to^* t_2$.

\begin{corollary}
{\sc Resilience} is decidable for ideal-effective downward-compatible WSTS with 
$\Safe=\downarrow \Safe$.
\end{corollary}

\begin{proof}


Let $D$ be a downward-closed subset of $S$
and let $x \in \downarrow \pred^*(D)$.
By downward closure, there exists
$y \in \pred^*(D)$ 
such that $x \leq y$.
By definition of $\pred^*(D)$ then there exists 
$d \in D$, $m\geq 0$ and $(a_i)_{0 \leq i \leq m+1} \in S^{m+2}$ such that
$y = a_0 \to a_1 \to a_2 \to \cdots \to a_m \to a_{m+1} = d$.

By downward compatibility $a_0 \to a_1$
implies that there exists $a'_1 \in S$ such that $a'_1 \leq a_1$ and
$x \to^* a'_1$.
More generally $a_i \to a_{i+1}$ and
$a'_i\leq a_i$ implies the existence of $a'_{i+1} \in S$ with $a'_{i+1} \leq a_{i+1}$ and
$a'_i \to^* a'_{i+1}$,
and, by induction,
 $x \to^* a'_1 \to^* \cdots \to^* a'_{m} \to^* a'_{m+1} = d'$
with $d' \leq d$.
Since
$d'$ 
% \alain{qui appartient à qui ?}
belongs to $D$ by downward closure of $D$, $x \in \pred^*(D)$.
\end{proof}

% \begin{proposition}
In the case
of a ideal-effective WSTS 
where
the additional hypothesis that
for all downward-closed set $D \subseteq S$, the set $\pred^*(D)$ is downward-closed
is not met,
the above construction
can provide a proof
of non-resilience
i.e. when
$\Bad \cap (S \setminus \downarrow\pred^*(\Safe)) \neq \emptyset$
then
$\Bad \not\subseteq \downarrow\pred^*(\Safe)$
and hence
$\Bad \not\subseteq \pred^*(\Safe)$.
When $\Bad \cap (S \setminus \downarrow\pred^*(\Safe)) = \emptyset$
however
it is not enough to conclude.

% \alain{indecidabilite en enlevant une ou des hypotheses: The resilience problem is undecidable for ideal-effective WSTS}

% \mathieu{
% Home-state indécidable pour les minsky machine (preuve très rapide: M à $2$ compteurs s'arrête ssi M' visite $(0,0,0)$ depuis $\uparrow(0,0,1)$, où M' est M mais avec un troisième compteur qui commence à $1$ et qui simule M jusqu'à ce que M s'arrête et à ce moment là décroit les deux premiers compteurs puis le troisième et atteint $(0,0,0)$).
% or les minsky machine peuvent être simulées par les reset petri nets, on réduit le home-state des minsky machine au home-state des reset petri nets, ce qui donne l'indécidabilité du home-state pour les WSTS en général.}


Without the additional hypothesis that
for all downward-closed set $D \subseteq S$, the set $\pred^*(D)$ is downward-closed, the problem becomes undecidable.

 Indeed, it is 
undecidable for reset-VAS whether zero (the vector containing only zeroes, also denoted~$ \textbf{0}$) is a home-state for $\N^d \setminus \textbf{0}$, where it is a particular case of resilience with $\Safe = \downarrow \textbf{0}$, $\Bad$ complement of $\Safe$. This stems from the fact that it is undecidable for Minsky machines with more than one counter, whether zero is a home-state. % See Appendix~\ref{HS-Minsk} for a more detailed construction.


	% 	From Appendix
	%
% \subsection{Home-state is undecidable for Minsky machine, reset-VASS and WSTS}\label{HS-Minsk}
	%
% We mentionned in Section~$3$ that {\sc Resilience} is undecidable for WSTS in general when $\Safe = \downarrow \Safe$ and $\Bad = \uparrow \Bad$. Let us provide a more detailed undecidability proof here.

First, let us recall the home-state problem

\problemx{Home-state}
{A transition system $\mathscr{S}=(S,\rightarrow)$ and a state $s \in S$.}
{$\post^*(S) \subseteq \pred^*(\{s\})$ ?\newline}

% whether a state $s$ is such that, for all state $s_0$ there exists a path from $s_0$ to $s$.
% post ∗ (S ) ⊆ pred ∗ (s_0)


Home-state is undecidable for Minsky machine with $3+$ counters.
This stems from the fact $2$-counter Minsky machine termination is undecidable~\cite{Min61, Min67}.
From a $2$-counter Minsky machine $M$, one can construct a $3$-counter Minsky machine $M'$ 
such that the Minsky machine $M$ terminates for all inputs iff the Minsky machine $M'$ can reach $(0,0,0)$ from any input with at least $1$ on its third counter. We build $M'$ to simulate $M$ until it reaches a control state indicative of termination, then lower the first two counters until they reach $0$, then, and only then, finally lower the third counter until it reaches $0$.
% Home-state indécidable pour les Minsky machine (preuve très rapide: M à $2$ compteurs s'arrête ssi M' visite $(0,0,0)$ depuis $\uparrow(0,0,1)$, où M' est M mais avec un troisième compteur qui commence à $1$ et qui simule M jusqu'à ce que M s'arrête et à ce moment là décroit les deux premiers compteurs puis le troisième et atteint $(0,0,0)$).
% or les minsky machine peuvent être simulées par les reset petri nets, on réduit le home-state des minsky machine au home-state des reset petri nets, ce qui donne l'indécidabilité du home-state pour les WSTS en général.
Thus {\sc Home-state} is undecidable for Minsky machines. Since $\post^*(S) \subseteq \pred^*(\{s\})$ is equivalent to $S \rightarrow^* \{s\}$,  {\sc Home-state} is a particular instance of {\sc Resilience}. Hence {\sc Resilience} is undecidable for Minsky machines.


Executions of Minsky machines can be simulated by reset-VASS~\cite{araki1976PN}. 
% Let us recall more formally what a reset-VASS is.
Reset-VASS extend the basic VASS model with special “reset
transitions” that resets (set to $0$) some coordinates in the vector. Let us recall their definition here.
\begin{definition}
A {\em reset-VASS} in dimension $d$ %(reset-VASS for short)
 is a finite 
labeled directed graph $V = (Q,T)$, where $Q$ will be referred to as the {\em control-states} of $V$, and where 
$T \subseteq Q \times Op \times Q$
 will be referred to as the {\em control-transitions} of $V$,
where $Op = \{ add(\textbf{z}) \mid \textbf{z} \in \mathds{Z}^d\} \cup 
		\{ reset(i) \mid i \in \{1,\ldots,d\} \}$.
\end{definition}

Again $Q \times \N^d$
 denotes the set of configurations of $V$.
For every configurations $p(\textbf{u}), q(\textbf{v}) \in Q \times \N^d$ and every control-transition $t$ we write
$p(\textbf{u}) \xrightarrow{t} q(\textbf{v})$ when 
\begin{samepage}\begin{itemize}
\item  $t = (q,add(\textbf{z}),q') \in T$
% then for all $\textbf{u} \in \N^d$ such that  
% $\textbf{u}+\textbf{z} \geq 0$
% $q(\textbf{u}) \xrightarrow{\textbf{z}} q'(\textbf{u}+\textbf{z})$,
and $\textbf{u}+\textbf{z} = \textbf{v} \geq 0$,
\item $t = (q,reset(\gamma),q') \in T$ 
% then for all $\textbf{u} \in \N^d$ 
% $q(\textbf{u}) \xrightarrow{z} q'(\textbf{u}')$,  where 
and
$\textbf{v}[\gamma] = 0$, and $\textbf{v}[\gamma'] = \textbf{u}[\gamma']$ for all $\gamma' \in \{1,\ldots, d\} \setminus \gamma$.
\end{itemize} \end{samepage}

It is well known that reset-VASS are WSTS~\cite{dufourd1998reset}. 
Since Reset-VASS can simulate executions of a Minsky machine, {\sc Resilience} is undecidable for reset-VASS and hence for WSTS in general as well.





Remark we did not make use of the hypothesis $\Bad$ complement of $\Safe$, simply 
$\Bad=\uparrow \Bad$ and $\Safe=\downarrow \Safe$, thus 
the above results still hold in the more general case where $\Bad$ and $\Safe$ are not complement of each others.





\subsection{Case Synthesis}

% \mathieu{En rouge quand il s'agit de conjectures}



\begin{center}
\begin{tabular}{ | l | c | c | c | r |}
\hline   \Safe~\Bad & $\uparrow$~ $\uparrow$~ & $\uparrow$~ $\downarrow$~ & $\downarrow$~ $\uparrow$~ & $\downarrow$~ $\downarrow$~ \\ \hline
   RP & Decidable (Theorem~\ref{up-up}) & Decidable (Theorem~\ref{down-up})  & Decidable (Theorem~\ref{up-down}) & Undecidable (Theorem~\ref{down-down}) \\ \hline
   BRP & Decidable (Corollary~\ref{B-up-up}) &  Decidable (Theorem~\ref{down-up}) & ?? & ?? \\ \hline
      kRP & Decidable (Theorem~\ref{k-up-up}) & Decidable (Theorem~\ref{down-up}) & ?? & ?? \\ \hline
 \end{tabular}
\end{center}


\alain{trouver les \Bad~ et \Safe~ maximum tels que S est resilient. est-ce vrai que si S est $(B_i,D_i)$-resilient alors S est $(\cap, \cup B_i,D_i)$-resilient ?}
%
\alain{on peut penser à des ensembles $\Bad$ definis dans une logique booleennne sur les clos par le bas, haut, +...}


%We first assume that the safety property is given by an upward-closed set and the bad condition by a decidable downward-closed set. 
% \textcolor{red}{Seems like a reasonable assumption to me.}

%From these considerations, we formulate instances of the abstract resilience problems for well-
%structured transition systems.








\section{State-resilience}


Resilience is a strong property since it implies that from every element of $\Bad$ there must exist a path to $\Safe$. However, when one considers a system with an initial state $s_0$, it could be sufficient to ask that only from $\Bad \cap \post^*(s_0)$, there must exist a path to $\Safe$. 
In the {\bf Flight Controller System} example for instance, $(q_{A,1}, q_{M,4}, q_{B,0})(\textbf{0})$
belongs to $\Bad$, but is not reachable from $(q_{A,0}, q_{M,0}, q_{B,0})(\textbf{0})$,
hence it may seem unnecessary to check for the existence of a path to $\Safe$ from this element.
Since we still consider $\Bad$ and $\Safe$ complements of each others, this is akin to asking whether from $\post^*(s_0)$ there always exists a path to $\Safe$. 
The three previous problems become:




\problemx{State-resilience problem (SRP)}
{A transition system $\mathscr{S}=(S,\rightarrow)$, $s \in S$ and two sets $\Safe \subseteq S$.}
% {$\Bad \cap \post^*(s)  \rightarrow^{*} \Safe $ ? \newline}
{$\post^*(s)  \rightarrow^{*} \Safe $ ? \newline}

\problemx{$k$-state-resilience problem (kSRP)}
{A transition system $\mathscr{S}=(S,\rightarrow)$, $s \in S$ and two sets $\Safe \subseteq S$.}
% { $\Bad \cap \post^*(s) \longrightarrow^{\leq k} \Safe$ ?  \newline}
{ $\post^*(s) \longrightarrow^{\leq k} \Safe$ ?  \newline}

\problemx{bounded-state-resilience problem (BSRP)}
{A transition system $\mathscr{S}=(S,\rightarrow)$, $s \in S$ and two sets $\Safe \subseteq S$.}
{$\exists k \geq 0$ such that $\mathscr{S}$ is $k$-state-resilient ?\newline}



\iffalse
\problemx{State resilience problems}
{A transition system $\mathscr{S}=(S,\rightarrow)$, $s \in S$, two sets $\Safe, \Bad \subseteq S$, $k \in \mathbb{N}$.}
{ ({\sc state-resilience problem (SRP)}) $\post^*(s) \longrightarrow^{*} \Safe$ ?\newline  
({\sc $k$-state-resilience problem (kSRP)})		$\post^*(s) \longrightarrow^{\leq k} \Safe$ ?\newline
({\sc bounded-state-resilience problem (BSRP)})	$\exists k \geq 0$ such that $\mathscr{S}$ is 
 $k$-resilient ?\newline}
\fi


Since these problems are undecidable for general infinite-state transition systems, we restrict our study to WSTS.
As in the Section~3, we study decidability results for $\Safe$ downward-closed and upward-closed. 


\subsection{Case: $\Safe = \uparrow \Safe$% and $\Bad = \downarrow \Bad$
}

We start with the case $\Safe = \uparrow \Safe$ hence $\Bad = \downarrow \Bad$.
%
Unfortunately, in this case {\sc State-resilience} is undecidable for (general) WSTS even with strong upward-compatibility.
This stems from the fact that it is undecidable in the particular case of reset-VASS,
where  $t$-liveness is both undecidable and 
 reducible to {\sc State-resilience}. This undecidability result furthermore implies the undecidability of the other two state resilience problems by straightforward reductions.


\begin{theorem}\label{srp up down}
{\sc State-resilience},
{\sc Bounded-state-resilience} and
{\sc $k$-state-resilience}
are undecidable for strongly compatible WSTS with effective pred-basis
when
$\Safe=\uparrow \Safe$.
\end{theorem}


\begin{proof}
{\sc State-resilience} itself is undecidable % from Proposition~\ref{reductions} 
since 
{\sc $t$-liveness} is undecidable in reset-VASS
and reducible to {\sc State-resilience}.
Additionally, in WSTS with strong compatibility and effective pred-basis, {\sc Bounded-state-resilience} is
reducible to {\sc $k$-state-resilience}:
since $\Safe=\uparrow \Safe$ and
$\mathscr{S}=(S,\rightarrow,\leq)$ is a WSTS with strong %upward-
compatibility, then $\pred^{\leq n}(\Safe)= \uparrow~\pred^{\leq n}(\Safe)$ for all $n \in \N$,
and there exists $n_0 \in \N$ such that 
$\pred^{\leq n_0}(\Safe) = \uparrow \pred^{\leq n_0}(\Safe) = \uparrow \pred^*(\Safe) = \pred^*(\Safe)$.
We compute 
$n_0$, then iteratively check whether $k$-state-resilience 
hold for $k$ from $0$ to $n_0$.  
Furthermore, in WSTS with strong compatibility and effective pred-basis,  $\Safe=\uparrow \Safe$, {\sc Bounded-state-resilience} is equivalent to {\sc State-resilience},
since 
$\pred^{\leq n_0}(\Safe) = \uparrow \pred^{\leq n_0}(\Safe) = {\uparrow \pred^*(\Safe)} = \pred^*(\Safe)$.
Hence the undecidability of {\sc Bounded-state-resilience}
and 
 {\sc $k$-state-resilience}. 
\end{proof}






On the positive side, let us recall a result about {\sc Bounded-state-resilience} decidability (called resilience in \cite{DBLP:conf/gg/Ozkan22,DBLP:journals/corr/abs-2108-00889}).
\begin{theorem}\cite{DBLP:conf/gg/Ozkan22,DBLP:journals/corr/abs-2108-00889}\label{ref ozkan}
{\sc Bounded-state-resilience} and {\sc $k$-state-resilience} are decidable for WSTS $S$ with strong compatibility and such that $\uparrow \post^*(s)$ is computable for $s \in S$
when
$\Safe=\uparrow \Safe$.
\end{theorem}

We may immediately generalyse this last result by strengthening to \emph{unbounded} {\sc State-resilience}. The proof is essentially the same than the previous one.

\begin{corollary}\label{postcomputable}
{\sc State-resilience} is decidable for WSTS with strong compatibility and such that $\uparrow \post^*(s)$ is computable for $s \in S$
when
$\Safe=\uparrow \Safe$.
\end{corollary}

\begin{proof}
%From Fact~\ref{stop condition},
Since $\mathscr{S}$ is a WSTS there exists $n_0 \in \N$ such that
$\pred^*(\Safe) =  \pred^{\leq n_0}(\Safe)$. We can compute this $n_0$ by iteratively computing 
%\alain{mauvaises notations: $k, k_m, Safe^k,...$ n'a aucun sens...}
$\pred^{\leq n+1}(\Safe)$ from $\pred^{\leq n}(\Safe)$, checking 
$\pred^{\leq n+1}(\Safe) = \pred^{\leq n}(\Safe)$, 
returning $n$ if that is the case.
Then, because {\sc $n_0$-state-resilience} is decidable, 
checking $\uparrow \post^*(s) %\cap \Bad
 \subseteq \pred^{\leq n}(\Safe) = \pred^*(\Safe)$ is,
and {\sc State-resilience} is decidable.
\end{proof}

The proof of Theorem~\ref{ref ozkan} rely on the computability of $\uparrow \post^*(s)$ and on the following lemma.

\begin{lemma}\label{Lemma intersection}
Let $A \subseteq S$, $D \subseteq S$ be a downward-closed set and $U \subseteq S$ be an upward-closed set. 
Then $A \cap D \subseteq U$  iff $ (\uparrow  A) \cap D \subseteq U$.
\end{lemma}


\begin{proof}
Let us suppose that $A \cap D \subseteq U$. Then ${\uparrow (A \cap D)} \subseteq {\uparrow U} = U$.
Let us show that $({\uparrow A}) \cap D \subseteq {\uparrow (A \cap D)}$.
Let $x \in ({\uparrow A}) \cap D$, then there exists $a \in A$ such that $x \geq a$.
Since $x \in D$ and $D$ is downward-closed, we also have $a \in D$.
Hence $a \in A \cap D$ and then $x \in { \uparrow (A \cap D)}$.
In the other direction,
since $A \subseteq {\uparrow A}$, the inclusion
$({\uparrow  A}) \cap D \subseteq U$ implies
$A \cap D \subseteq ({\uparrow  A}) \cap D \subseteq U$.
\end{proof}




The computability of $\uparrow \post^*(s)$ however seems a strong hypothesis. What are the WSTS for which $\uparrow \post^*(s)$ is computable for $s \in S$ ?
Ozkan \cite{DBLP:conf/gg/Ozkan22} argues that it is precisely the WSTS for which the following problem is decidable.

\problemx{Downward-reachability problem}
{A transition system $\mathscr{S}=(S,\rightarrow)$, $s \in S$ and a downward-closed set $D
\subseteq S$.}
{$s  \to^* D$? \newline}

%\"Ozkan
\begin{proposition}[Proposition 1 in \cite{DBLP:conf/gg/Ozkan22}]\label{post*}
For finite-branching WSTS%with strong compatibility
, a basis of $\uparrow \post^*(s)$ is computable for every state $s$ iff the downward-reachability problem is decidable.
\end{proposition}


The idea behind the proof is the following. For deciding whether a downward-closed set $D$ is reachable from $s$, one checks whether
$B_{\uparrow \post^*(s)} \cap D = \emptyset$, 
where $B_{\uparrow \post^*(s)}$ is a basis of $\uparrow \post^*(s)$,
%\alain{rapeler $B_{\uparrow \post^*(s)}$} 
 that is equivalent to $\post^*(s)\cap D = \emptyset$ by
Lemma~\ref{Lemma intersection}. For the converse direction, one computes the sequence of upward-closed sets
$U_n = \uparrow \post^{\leq n}(s)$ until it becomes stationnary. 
Decidability of downward-reachability leads to the decidability of the following stop condition:
asking whether $S \setminus U_n$ is reachable from $s$.





More concretely,
VASS for instance are $\uparrow \post^*$-effective WSTS \cite{DBLP:journals/corr/abs-2108-00889}. 
It is well-known that 
VASS are WSTS with strong compatibility and since there is an algorithm that computes a finite basis of  $\uparrow \post^*(s)$, \cite{DBLP:conf/gg/Ozkan22} deduced that {\sc Bounded-state-resilience} is decidable for VASS.
Hence {\sc State-resilience} is decidable for %both Petri nets and 
VASS.
However, the hypothesis that $\uparrow \post^*$ is computable cannot be tested in the general WSTS framework.  \iffalse Moreover, we may show:

\begin{proposition}
There exist classes of WSTSs with strong 
 compatibility for which there doesn't exist an algorithm computing a basis of $\uparrow \post^*$.
\end{proposition}


\begin{proof}
Let us show that reset-VASS, that are effective WSTSs with strong compatibility, don't enjoy the property that $\uparrow \post^*$ is computable.
Suppose that one are able to compute a finite basis of $\uparrow \post^*$ for reset-VASS. 
Then, {\sc State-resilience} would be decidable for reset-VASS, which leads to a contradiction.
\end{proof} \fi
Moreover there exist classes of WSTSs with strong 
 compatibility for which there doesn't exist an algorithm computing a basis of $\uparrow \post^*$, such as reset-VASS.

Keeping the $\uparrow \post^*$ effectiveness hypothesis but loosening the strong compatibility one still yields some decidability result for the general {\sc State-resilience}. Using the same proof structure as Theorem 1 from \cite{DBLP:journals/corr/abs-2108-00889} we obtain:


\begin{theorem}\label{post srp}
{\sc State-resilience} is decidable for 
 WSTS with effective 
$\uparrow$ $\post^*$ basis
when
$\Safe=\uparrow \Safe$.
\end{theorem}


\begin{proof}
Let $B_{\uparrow \post^*(s)}$ 
%\alain{notation fluctuante}
 be a basis of $\uparrow \post^*(s)$, $B_\Safe$ a basis of $\Safe$.
By applying Lemma~\ref{Lemma intersection} twice, we obtain
\[ \post^*(s) %\cap \Bad
 \subseteq \pred^*(\Safe) \text{ iff } B_{\uparrow \post^*(s)} %\cap \Bad 
 \subseteq \pred^*(\Safe)\]
Since $B_{\uparrow \post^*(s)}$ is finite and we can compute a basis of $\pred^*(\Safe)$ from $B_\Safe$, we can check that $B_{\uparrow \post^*(s)} %\cap \Bad
 \setminus \pred^*(\Safe) = \emptyset$. 
\end{proof}





However when removing strong compatibility, precision is lost.
Since $\pred(\uparrow \Safe)$ is not necessarily upward-closed, it is possible to have 
 $\uparrow \post^* (s) \cap S \not\subseteq \pred(\Safe)$,
despite having 
$\post^* (s) \cap S \subseteq \pred( \Safe)$.
In such a case the algorithm from
\cite{DBLP:conf/gg/Ozkan22} would deduce that $1$-rechable-resilience does not hold,
which is incorrect.

Thus in case of a WSTS with an effective basis of $\uparrow \post^*$ and (not strong) compatibility, we don't know the decidability status of {\sc $k$-state-resilience} and 
{\sc Bounded-state-resilience}. 


{\bf Results synthesis in the case $\Safe = \uparrow \Safe$}
\begin{center}
\begin{tabular}{ | l | c | c | c | c |}
\hline  Hypothesis & strong %upward 
			compatibility ~ & $\uparrow \post^*$ effective 
			& strong %upward 
					compatibility \textit{+} $\uparrow \post^*$ effective \\ \hline
   SRP & Undecidable (Thm~\ref{srp up down}) & Decidable (Thm~\ref{post srp})  & Decidable (Thm~\ref{postcomputable})\\ \hline
   BSRP & Undecidable (Thm~\ref{srp up down}) &  ??  & Decidable (Thm~\ref{ref ozkan}) \\ \hline
      kSRP & Undecidable (Thm~\ref{srp up down}) & ?? & Decidable (Thm~\ref{ref ozkan}) \\ \hline
 \end{tabular}
\end{center}


\subsection{Case: $\Safe = \downarrow \Safe$% and $\Bad = \uparrow \Bad$
}


We now consider the case $\Safe = \downarrow \Safe$ hence $\Bad = \uparrow \Bad$.
%
Unfortunately the {\sc State-resilience} problem is undecidable for WSTS in the case $\Safe = \downarrow \Safe$. As for resilience in WSTS when $\Safe = \downarrow \Safe$, it stems from undeciability of the corresponding problem in Minsky machines, the executions of which can be simulated by reset-VASS, as seen in Appendix~\ref{HS-Minsk}. 

\begin{theorem}\label{srp down up}
{\sc State-resilience} is undecidable for effective WSTS with  strong  compatibility 
when
$\Safe=\downarrow \Safe$.
\end{theorem}



Despite this, it is possible to yield positive results. Indeed, in many ways the case where $\Safe = \downarrow \Safe$
is symmetrical to the case $\Safe = \uparrow \Safe$.
%
For instance one can write the following lemma:

\begin{lemma}(Symmetrical from Lemma~\ref{Lemma intersection})\label{Lemma intersection 2}
Let $A \subseteq S$, $D \subseteq S$ be a downward-closed set and $U \subseteq S$ be an upward-closed set. 
Then $A \cap U \subseteq D$  iff $ (\downarrow  A) \cap U \subseteq D$.
\end{lemma}

\begin{proof}
Let us suppose that $A \cap U \subseteq D$. Then ${\downarrow (A \cap U)} \subseteq {\downarrow D} = D$. Let us show that $({\downarrow A}) \cap U \subseteq {\downarrow (A \cap U)}$. Let $x \in ({\downarrow A}) \cap U$, then there exists $a \in A$ such that $x \leq a$. Since $x \in U$ and $U$ is upward-closed, we also have $a \in U$. Hence $a \in A \cap U$ and then $x \in { \downarrow (A \cap U)}$. In the other direction, since $A \subseteq {\downarrow A}$, the inclusion $({\downarrow  A}) \cap U \subseteq D$ implies $A \cap U \subseteq ({\uparrow  A}) \cap U \subseteq D$.
\end{proof}


In the case of a WSTS with \emph{downward} compatibility, not necessarily strong,
then $\Safe$ downward-closed implies $\pred^*(\Safe)$ downward-closed and
Lemma~\ref{Lemma intersection 2} can be used to show that
if $\Safe = \downarrow \Safe$,
then
$\post^*(s) %\cap \Bad 
\subseteq \pred^*(\Safe)$  iff $ (\downarrow  \post^*(s)) 
%\cap \Bad 
\subseteq \pred^*(\Safe)$.


\begin{theorem}\label{downward srp}
{\sc State-resilience} is decidable for ideal-effective WSTS with downward and upward compatibilities,
$\Safe = \downarrow \Safe$.
\end{theorem}

\begin{proof}
In order to decide whether $\post^{\leq n}(s) %\cap \Bad
 \subseteq \pred^*(\Safe)$, we execute two procedures in parallel,
one looking for a resilience certificate and one looking for a non-resilience certificate.
Procedure 1 enumerates inductive invariants in some fixed order $D_1$ , $D_2$ , $\ldots$ , i.e. downward-closed subsets $D_i \subseteq S$ such that $\downarrow \post(D_i ) \subseteq D_i$. 
Every inductive invariant $D_i$ is an “over-approximation” of $\downarrow \post^*(s)$ if it contains $s$.
Notice that, by 
upward compatibility, $\downarrow \post^*(s)$ is such an inductive invariant and may eventually be found.

Procedure 1 stops when it finds an invariant $D$ such that
$D  %\cap \Bad
 \subseteq \pred^*(\Safe)$. 
Indeed
$D % \cap \Bad 
\subseteq  \pred^*(\Safe)$ implies
$\downarrow \post^*(s) % \cap \Bad
 \subseteq  \pred^*(\Safe)$
since $ \downarrow \post^*(s)  \subseteq D$.

The second procedure iteratively computes
$\post^{\leq n}(s) %\cap \Bad
$
until it finds an element
not in $\pred^*(\Safe)$.
\end{proof}



Strong downward compatibility implies furthermore the decidability
of {\sc $k$-state-resilience} and {\sc bounded-state-resilience}.

\begin{corollary}\label{downward brp}
{\sc $k$-state-resilience} and {\sc bounded-state-resilience} are decidable for ideal-effective WSTS with strong downward compatibility,
$\Safe = \downarrow \Safe$.
\end{corollary}






% 
\section{Decidability for WSTS with downward/strong upward compatibility}



\begin{lemma}
Let $A \subseteq S$, $J \subseteq S$ downward-closed and $I \subseteq S$ upward-closed. 
Then $A \cap J \subseteq I$  iff $ (\uparrow  A) \cap J \subseteq I$.
\end{lemma}


\begin{proof}

If 
$A \cap J \subseteq I$,
then
${\uparrow (A \cap J)} \subseteq {\uparrow I} = I$.
However
$({\uparrow A}) \cap J \subseteq {\uparrow (A \cap J)}$.
Indeed if $x$ in $({\uparrow A} \cap J)$,
$x$ is in $\uparrow A$
so there exists $a \in A$ such that $x \geq a$.
However $x$ is also in $J$, meaning $a$ is in $J$.
Meaning $a$ is in $(A \cap J)$
and $x \in { \uparrow (A \cap J)}$.

In the other direction,
since $A \subseteq {\uparrow A}$,
$({\uparrow  A}) \cap J \subseteq I$ implies
$A \cap J \subseteq ({\uparrow  A}) \cap J \subseteq I$.
\end{proof}


\begin{corollary}
For all $k \in \N$,
$ post^*(s)\cap J \subseteq I^k $  iff $ (\uparrow  post^*(s)) \cap J \subseteq I^k$. 
% et travailler avec $\downarrow post^*(s)$ à la place de $\uparrow post^*(s)$.
\end{corollary}

% La question deviens: quel $k$ pour que
% $\downarrow post^*(s) \cap J \subseteq I^k$.
% Ou alors, à $k$ fixé, est-ce que
% $\downarrow post^*(s) \cap J \subseteq I^k$.



Assume $k$ is fixed for now.

Procedure 1 enumerates inductive invariants in some fixed order $D_1$ , $D_2$ , . . . , i.e. upward closed subsets $D_i \subseteq S$ such that $\uparrow post(D_i ) \subseteq D_i$. 
Every inductive invariant $D_i$ is an “over-approximation” of $\uparrow post^*(s)$ if it contains $s$.
(on énumère des sur-approximations de la cloture par le haut de $post*(s)$ par leur bases finies).
Each “over-approximation” $D_i$ is given by its basis $b(D_i)$. Notice that, by standard monotonicity, $\uparrow post^*(s)$ is such an inductive invariant and may
eventually be found.
% \alain{attention D est clos par le bas et $\uparrow post^*(s)$ est clos par le haut, je ne vois pas pourquoi $\uparrow post^*(s) \subseteq D$}
% \mathieu{Oui effectivement les D devraient être clos par le haut, je corrige ça à la prochaine update.}

Procedure 1 stops when it finds a basis $b(D)$ of an invariant $D$ such that
$b(D)  \cap J \subseteq I^k$.  Since $b(D)$ is finite and $J$ is decidable, we can
directly compute $b(D)  \cap J$.
% We can compute a basis of $I^{k+1}$ if we have a basis of $I^k$.
We can compute a basis
of $I^{k+1}$ if we have a basis of $I^k$. %This follows by the proof of Lemma 3.
Due to the Lemma, 
$b(D)  \cap J \subseteq I^k$ implies
$D  \cap J \subseteq I^k$.
Hence
$b(D)  \cap J \subseteq I^k$ implies
$\uparrow post^*(s) \cap J \subseteq D  \cap J \subseteq I^k$.
(since $D$ contains $ \uparrow post^*(s)$).



The second procedure iteratively computes
$post^{\leq n}(s) \cap J$
until it finds an element
not in $ I^k$.


This result is not in xxx and don't use the problematic hypothsis on $post^*$.

\begin{theorem}
{\sc $k$-resilience} is decidable for effective WSTS with downward/strong upward compatibility.
%	with strong monotony.
\end{theorem}


\begin{proof}

Assume  $(S, \rightarrow, \leq)$ is a WSTS with downward/upward compatibility, $J$ is a decidable downward-closed subset of $S$, and $I$ is an upward-closed set with a given basis.

We define inductively
$I^{k} = \bigcup_{0 \leq j \leq k} \pred^j(I)$. Note that for all $k \in \N$, $I^k$ is upward-closed due to
the strongly upward compatibility of $(S, \rightarrow, \leq)$.

% Si on inverse les types de propriétés (downward closed pour $I = SAFE$, upward closed pour $J= BAD$) alors on peut écrire un lemme symmétrique du Lemme $4$:

The $k$-resilience property can be expressed as the formula
$ post^*(s) \cap J \subseteq I^k$. In order to decide whether the inclusion holds, we execute two procedures in parallel, one trying to prove $ post^*(s)\cap J \subseteq I^k$ 
and one looking for a counter example.

In order to certify inclusion in $I^k$, we need to work with finite representations.
The next lemma uses that $I$ and $J$ are upward- and downward-closed, respectively.


% Et une fois que l'on a ça, on peut écrire

% Ce serait une procédure qui pourrait par exemple calculer $post^m(s) \cap J $ jusqu'à trouver un élément pas dans $I^k$ ?

% Càd on calcule les éléments de $post^m(s)$ un à un, pour chacun on vérifie s’il est dans $J$ puis s’il y est on vérifie s’il est dans $I^k$ ?

% Il faut aussi une base de $J$ et que $I^k$ soit décidable, mais ça a l'air faisable j'ai l'impression...




\begin{figure}
\fbox{\parbox[t][3.cm][c]{6cm}{
$\phantom{a}$\\
(1)\phantom{aaaaa} $i \leftarrow 0$\\
(2)\phantom{aaaaa}\textbf{while} $\neg( \uparrow post^*(D_i) \subseteq D_i $
			 and $ s \in D_i$
			 and $ b(D)  \cap J \subseteq I^k  )$ \textbf{loop} \newline
	(3)\phantom{aaaaaa}$\phantom{aaaa} i \leftarrow i +1$ \newline
(4)\phantom{aaaaa}\textbf{end loop} \newline
(5)\phantom{aaaaa}\textbf{return} $\text{\textit{false}}$ \newline
	}
}
	\caption{\textbf{Procedure 1:} enumerates inductive invariants to find an inclusion certificate.}\label{procedure1}
\end{figure}




\begin{figure}
\fbox{\parbox[t][3.cm][c]{6cm}{
$\phantom{a}$\\
(1)\phantom{aaaaa} $D \leftarrow \{ s \} $\\
(2)\phantom{aaaaa}\textbf{while} $D \cap J \subseteq I^k $ \textbf{loop} \newline
	(3)\phantom{aaaaaa}$\phantom{aaaa} D \leftarrow D \cup post(D)$ \newline
(4)\phantom{aaaaa}\textbf{end loop} \newline
(5)\phantom{aaaaa}\textbf{return} $\text{\textit{false}}$ \newline
	}
}
	\caption{\textbf{Procedure 2:} searches for a non-inclusion certificate.}\label{procedure2}
\end{figure}

\newpage


We show that these two procedures are correct:


\begin{enumerate}

\item $k$-resilience holds if, and only if, Procedure 1 terminates.

\item $k$-resilience do not hold if, and only if, Procedure 2 terminates.

\end{enumerate}

Proof:

\begin{enumerate}

\item By a simple induction, it can be shown that $\uparrow post^*(D) \subseteq D$ for every inductive invariant $D$. 
If Procedure 1 terminates, then
$post^*(s) \cap J \subseteq \uparrow post^*(s) \cap J \subseteq D  \cap J \subseteq I^k$
which implies that $k$-resilience holds.

It remains to show that Procedure 1 terminates whenever $k$-resilience holds. To do so, it suffices to prove that $\uparrow post^*(s)$ is an inductive invariant. Indeed, this implies that
$\uparrow post^*(s)$ is eventyally found by Procedure 1 when $k$-resilience holds. 

Formally, let us show that $\uparrow post(\uparrow post^*(s)) \subseteq \uparrow post^*(s)$.
Let $b \in \uparrow post(\uparrow post^*(s))$ 
there exists $a', a, b$ such that
$s \rightarrow^* a'$,
$a' \leq a$,
$a \rightarrow b'$,
and
$b' \leq b$.
% By monotonicity, there exists b'' ≥ b' such that a → b'' . Therefore, x → ∗ b'' and b' ≥ b, hence b ∈ ↓ Post ∗ (x).
By downward compatibility 
there exists $b'' \leq b'$ such that $a \rightarrow b'' $. Therefore, $x \rightarrow^* b''$ and $b' \geq b$, hence $b \in \downarrow post^*(x)$.
%\alain{où est cette hypothèse downward compatibility  ?}
\item Procedure 2 computes
 $post^{\leq n}(s) \cap J$
 until it finds an element not in $ I^k$.

If Procedure 2 terminates, then
$k$-resilience does not hold.
It remains to show that Procedure 2 terminates whenever $k$-resilience does not hold.
Assume $ post^*(s) \cap J \not\subseteq I^k$, then there exists $a \in post^*(s) \cap J$ such that $a \not\in I^k$. Since $post^*(s) = \bigcup_{n} post^{\leq n}(s)$, then 
$a \in post^*(s)$
implies
there exists
$n_a$
such that
$a \in post^{\leq n_a}(s)$.
Hence,  $post^{\leq n_a}(s) \cap J$ contains an element not in 
$I^k$,
and Procedure 2 terminates.
\end{enumerate}

\end{proof}



\begin{theorem}
{\sc bounded resilience} and {\sc resilience} is decidable for WSTS with downward/strong upward compatibility.
\end{theorem}

\begin{proof}{sketch}

Iteratively check whether $k$-resilience holds. 
If this is the case, return $k_{min} = k$. Otherwise check whether 
$I^{k+1} = I^k$. If so, return $-1$ (false), otherwise
continue.
The stop condition is decidable
and by Fact~\ref{stop condition} also sufficient. 
\end{proof}

{\sc resilience} is decidable for WSTS with strong upward compatibility and k-resilience.

jouer avec les hypothèses downward/upward compatibility






\section{Solving resilience}


\noindent
\subsection{Computing the maximal resilient subsystem}

Let $\mathscr{S}=(S, \rightarrow, \leq)$ be WSTS and $X \subseteq S$ be an upward-closed set given by its finite minimal basis $B_X$. We say that $\mathscr{S}=(S, \rightarrow, \leq)$ is $X$-resilient (resp. $(s_0,X)$-resilient) if $S  \xrightarrow{*} X$ (resp. $\post^*(s_0) \xrightarrow{*} X$) is satisfied.  We may verify that  $\mathscr{S}=(S, \rightarrow, \leq)$ is $X$-resilient iff  $\mathscr{S}=(S, \rightarrow, \leq)$ is $\pred^*(X)$-resilient. The \emph{maximal $X$-resilient subsystem} of $(S,\rightarrow,\leq)$ is ....
\mathieu{ne pas oublier de mettre une définition ici}
Let us remark that given a WSTS $\mathscr{S}=(S, \rightarrow, \leq)$, one may construct many \emph{guard} systems where every transition $s \rightarrow s'$ of $\mathscr{S}$ is guarded by a (monotone) formula $\phi$ defined by the grammar $\phi ::= s \geq u \mid s' \geq u \mid \phi \vee \phi \mid \phi \wedge \phi$ where $u \in S$. Since guards restrict the original runs but don't create new ones, every guarded transition system $\mathscr{S'}$ is a subsystem of $\mathscr{S}$. Since these formula are upward-compatible, the guarded transition systems $\mathscr{S'}=(S, \rightarrow', \leq)$ are still WSTS.
%
For example, VASS allows such guards.
  %	\lor s' \geq x_2  	\lor...	\lor s' \geq x_p$ (facile dans les VASS et modèles habituels).

\begin{theorem}{}
Given an effective WSTS with effective pred-basis and an upward-closed set $X$, the maximal resilient subsystem is a computable WSTS.
\end{theorem}

\begin{proof}
Since $X \subseteq S$ is upward-closed and $\mathscr{S}=(S, \rightarrow, \leq)$ is an effective WSTS with effective pred-basis, we compute the minimal basis of $\pred^*(X)=\{x_1,x_2,...,x_p\}$. 
%
%
Then we add the following guard $s' \geq x_1  	\vee s' \geq x_2  	\vee...	\vee s' \geq x_p$ to every transition $s \rightarrow s'$ of $\mathscr{S}$ and we define  $s \rightarrow' s'$ as $(s' \geq x_1  	\vee s' \geq x_2  	\vee...	\vee s' \geq x_p) \wedge s \rightarrow s'$ ; this insures that the new (guarded) WSTS $\mathscr{S'}=(S, \rightarrow', \leq,s_0)$ is $\pred^*(X)$-resilient and then it is also the maximal resilient sub-WSTS of $\mathscr{S}=(S, \rightarrow, \leq)$.

%		l'état atteint $s'$ est au dessus des minimaux de $\pred^*(X)$, ainsi par (upward) compatibility on sera toujours capable d'aller dans $X$. 
\end{proof}
%
%
%\noindent
%{\bf Vector addition system with states}

% Définir VASS

Let us recall that a {\em vector addition system with states (VASS)} in dimension $d$ ($d$-VASS for short) is a finite $\mathds{Z}^d$-labeled directed graph $V = (Q,T)$, where $Q$ is the set of {\em control-states}, and where $T \subseteq Q \times \mathds{Z}^d \times Q$ is the set of {\em control-transitions}. 
% The {\em size} of $V$ is defined as $|V|=|Q|+|T|*d*|log(||T||)$ where $||T||$ denotes the absolue value of the largest number that appears in $T$, i.e. $||T|| = max\{ ||\textbf{z}||: (p,\textbf{z},q) \in T\}$.
%
Subsetquently, $Q \times \N^d$ is the set of configurations of the transition system associated with $V$.
For all configurations $p(\textbf{u}), q(\textbf{v}) \in Q \times \N^d$ and for every control-transition $t = (p, \textbf{z}, q)$ we write $p(\textbf{u}) \xrightarrow{t} q(\textbf{v})$ whenever $\textbf{v} = \textbf{u} + \textbf{z} \geq \textbf{0}$
%
\iffalse \mathieu{Defining it like this makes it a LTS rather than a unlabeled TS - maybe talk about how we can 'forget' the labels to obtain an unlabeled TS ?}
\alain{les VASS ne sont pas vraiment étiquetés, sauf si on veut le faire, car on peut dire que les VASS sont définis à partir d'un nombre fini de control-transitions  $t = (p, \textbf{z}, q)$ générant une infinité de transtions notées $p(\textbf{u}) \xrightarrow{t} q(\textbf{v})$ puisque pour tout s, $p(\textbf{u+s}) \xrightarrow{t} q(\textbf{v+s})$} \fi
When in the context of a $d$-VASS, we denote $0^d$ by $\textbf{0}$.


Pour les VASS, je crois qu'on a une borne (2-exp) à la taille de $\pred^*(X)$ et à la taille de ses éléments \cite{DBLP:conf/rp/BozzelliG11} et c'est directement implémentable, donc en particulier, pour tout VASS, il existe un autre VASS de taille 2-exp plus grand qui a le plus grand sous-comportement restant dans $X$ .
Je crois que ça résoud aussi un pb résolu dans Valk et Jantzen 1985 \cite{DBLP:journals/acta/ValkJ85} mais en utilisant le graphe de karp et Miller donc avec complexité ackerman.... Dans ce cas, ça permettrait de résoudre autrement les premiers des 4 pbs dans \cite{DBLP:journals/acta/ValkJ85} avec une complexité 2-exp plutôt qu'ackerman. problèmes: marquage T-bloqué, dead, T-continual ?.

\begin{theorem}{}
Given a VASS and $X$ upward-closed, the maximal $X$-resilient subsystem is a computable VASS of size 2-exp.
\end{theorem}

applications to many upward-closed sets $X$:
$m$ is $\hat{T}$-blocked if no $t \in \hat{T}$ is fireable in $\post^*(m)$.
$m$ is dead if the reachability tree from $m$ is finite.
$m$ is bounded if the reachability set from $m$ is finite.
$m$ is $\hat{T}$-continual if there is an unfinite run from $m$ such that all transitions in $\hat{T}$ appear infinitely often.


\iffalse
\alain{ me semble maintenant inutile
%
\begin{definition}{ (index)}. 
If $\mathscr{S}$ is a WSTS with strong compatibility and $U \subseteq S$  is upward-closed and $k \geq 0$, let $U_k= \bigcup_{0 \leq n \leq k} \pred^n(U)$.
The {\em index} $k(U)$ is the
smallest $k_0$ s.t. $U_k = U_{k_0}$ for all $k \geq k_0$.
\end{definition}
%
%
If $\mathscr{S}$ is a WSTS with strong compatibility and $U \subseteq S$ is an upward-closed set, the sets $\pred(U )$, and $\pred^{\geq k}(U )$ for
every $k \geq 0$ are upward-closed, thus Lemma~\ref{upward-closed stablizes} ensures the existence of $k(U)$.
%
\begin{remark}
$U^{k+1} $ can be rewritten $U^{k+1}= \bigcup_{0 \leq j \leq k+1} \pred^j(U) = 
U \cup \bigcup_{1 \leq j \leq k+1} \pred^j(U) =
U \cup \pred(\bigcup_{0 \leq j \leq k} \pred^j(U))
=  U \cup \pred(U^k)$.
\end{remark}

This ensures the following.

\begin{fact}\label{stop condition}
% Fact 4 (stop condition). 
If $\mathscr{S}$ is a WSTS with strong compatibility and $U \subseteq S$ is an upward-closed set and $k \geq 0$ s.t. $U^k = U^{k+1}$ , then $U^\ell = U^k$ for all $\ell \geq k$, i.e.,
$k(U) \leq k$. This also implies that $\pred^*(U) = U^k$.
\end{fact}


\begin{lemma}
% Lemma 3 ([1]) 
% Parosh Aziz Abdulla, Karlis Cerans, Bengt Jonsson & Yih-Kuen Tsay (1996): General Decidability Theorems for Infinite-State Systems. In: Proc. LICS 1996, IEEE Computer Society Press, pp. 313–321,
 Given a basis of an upward-closed set $U \subseteq S$, and a state $s$ of an effective strongly WSTS, we can decide whether $U  \xrightarrow {*}{} s$.
%	we can reach $U$ from $s$.
\end{lemma}

\begin{proof}
We have to show that we can compute a basis of $U^{k+1}$ if we are given a basis of $U^k $. 
Then the
decidability of the stop condition follows directly. Let $B$ be a basis of $U^k$. 
We have
$$U^{k+1} = U \cup \pred(U^k ) = U \cup
\bigcup_{s' \in B}
\pred(\uparrow \{s' \}).$$

Since a finite basis of $\pred(\uparrow \{s' \}$) is computable for any $s'\in S$ by definition, we obtain a finite generating set of $U^{k+1}$ . By
Fact~\ref{fact basis}, we can compute a basis of $U^{k+1}$.
\end{proof}}


\fi






The two previous problems are well-known decidable.

\alain{probablement enlever toute la suite et garder seulement un point sur les VASS}

In this section, we study the general resilience problem.

\subsection{Pushdown automata}

%We may consider context-free grammars.

\begin{theorem}
The three {\sc resilience} problems are decidable for pushdown automata with $\Safe$ and $\Bad$ regular languages.
\end{theorem}

\begin{proof}
Recall that resilience is equivalent to  $\Bad \subseteq \pred^*(\Safe)$. For every pushdown automaton $A=(Q,\Sigma,...)$ and every regular language $L \subseteq Q \times \Sigma^*$, the set $\pred^*(L)$ is a computable regular language \cite{DBLP:journals/ipl/BouajjaniEFMRWW00}. Since $\Bad$ and $\Safe$ are both regular, we deduce that
%
%			Moreover, $post^*(\Bad)$ is also a computable regular language \alain{pourquoi ?}   $post^*(L)$ and
 the inclusion $\Bad \subseteq \pred^*(\Safe)$ is decidable and resilience too.
%
%The kRP is decidable: % let us compute $ \pred^{\leq k}(\Safe)$ (in polynomial time). 
Similarly, since $ \pred^{\leq k}(\Safe)$ is a computable regular language,
we deduce 
that the inclusion $\Bad \subseteq \pred^{\leq k}(\Safe)$ is decidable and $k$-resilience too.
Since $k$-resilience is decidable, bounded-resilience also is;
we can iteratively check whether $\Bad \subseteq \pred^{\leq k}(\Safe)$,
if this is the case, $k$-resilience holds, and we stop with a positive,
else we check whether $ \pred^{\leq k}(\Safe) =  \pred^{*}(\Safe)$,
if this is the case, then bounded-resilience do not hold,
else we continue.
\end{proof}
%
Let us remark that $\uparrow \post^*(q,w)$ is also a computable regular language, but we cannot directly apply previous results on WSTS since pushdown automata are not WSTS for the usual orderings (subword, prefix).
Context-free grammars are WSTS \cite{DBLP:journals/tcs/FinkelS01} for the subword ordering and $\uparrow \post^*$ is computable but context-free grammars are not strongly compatible.

%  \textcolor{red}{à faire Mathieu}

%
%



\subsection{Regular fifo automata}

Fifo automata and systems of communicating finite-state machines (CFSMs) are essentially finite automata that communicate through fifo channels (or queues). For the sake of simplicity, we consider automata with an unique fifo channel (that have the power of turing machines). In 1986, Pachl introduced in \cite{} the property for such systems to have a regular reachability set (or even a regular relation) and he showed that this gives a semi-algorithm to test non-reachability (with a regular inductive invariant to obtain a witness of non-reachability). Let $(q,w)$ be an initial state and $(q',w')$ be a state. Since $\post^*(q,w)$ is known to be regular (even if we don't know how to compute it), we may enumerate all regular languages $L \subseteq Q \times \Sigma^*$ that contain $(q,w)$ and such that $\post(L) \subseteq L$ (inductive invariant). We know that if $(q',w') \not\in \post^*(q,w)$ then there exists, at least, a regular language $L$ such that $\post^*(q,w) \subseteq L$ and $(q',w') \not\in L$, since $\post^*(q,w)$ is such inductive invariant. This strategy provides a semi-algorithm for non-reachability.
As reachability is recursively enumerable, this provides an algorithm to solve reachability for regular fifo automata.

Let us say that a fifo automaton is \emph{post-regular} (resp. \emph{pred-regular}) if $\post^*(q,w)$ (resp. $\pred^*(q,w)$ ) is regular for all $(q,w) \in Q \times \Sigma^*$. The  fifo automaton is \emph{effectively} post-regular (resp. pred-regular) if the regular language $\post^*(q,w)$ (resp. $\pred^*(q,w)$) is computable. Reachability is (immediately) decidable for both effective post-regular or effective pred-regular fifo automata.
\alain{montrer que post-regular (pred-regular) sur des (q,w) implique post-regular  (pred-regular) sur des langages rationnels L....ça doit marcher en prenant l'expression rationnelle de L...peut-on montrer que pred-regular n'equivaut pas a post-regular ? contre-exemples ? c'est le cas pour les LCM où pred ok mais pas post}
%
%
%Let us say that a fifo automaton is \emph{regular} if the reachability relation is regular for all $(q,w)$; in this case, the fifo automaton is also both post-regular and pred-regular.
%
Now, we just observe that $\Bad \subseteq \pred^*(\Safe)$ is decidable when $\pred^*(\Safe)$ is a computable regular language and $\Bad$ a regular language: 

\begin{theorem}{}
Resilience is decidable for effective pred-regular fifo automata when $\Safe$ and $\Bad$ are regular languages.
\end{theorem} 

%		\begin{proof}
%		$\Bad \subseteq \pred^*(\Safe)$ is decidable because $\pred^*(\Safe)$ is a computable regular language.
%		\alain{prover que $\pred^*(\Safe)$ is a computable regular language}
%		\end{proof}
%
%			Since lossy channel systems have (non computable) regular reachability sets, we imediately deduce that.

Since for  lossy channel systems $\pred^*(L)$ with $L=\uparrow L$ is a computable (upward-closed) regular language, we deduce :

\begin{corollary}
Resilience is decidable for lossy channel systems with $\Safe=\uparrow \Safe$ and any regular language $\Bad$.
\end{corollary}

Another proof is possible for LCS since LCS are WSTS with a computable $\uparrow \post^*(S)$ \alain{à prouver et vérifier les autres hypothèses}
%
%%%
\subsection{LCM, VASS, semilinear VASS, Integer et Continuous VASS}


\begin{theorem}{}
Resilience is decidable for  lossy counter machines, semilinear VASS, Integer VASS when $\Safe$ and $\Bad$ are semilinear sets.
\end{theorem}

\begin{proof}
Resilience is decidable for lossy counter machines with $\Safe$ and $\Bad$ semilinear sets as a consequence of Theorem 3.6 in \cite{DBLP:conf/rp/Schnoebelen10} that implies that $\pred^*(\Safe)$ is a computable semilinear set. Hence since the inclusion between two semilinear sets is decidable, we deduce that  $\Bad \subseteq \pred^*(\Safe)$ is decidable.

%		semilinear VASS.
 \alain{le plus simple serait que $\pred^*(L)$ is semilinear because $\post^*(s)$ is semilinear , peut-être en inversant le VASS V en V'on aurait pred*(V)=post*(V') hence $post^*(L)$ is also semilinear ? hence pred ????}

%		Integer VASS
Recall that reachability is NP-complete for integer VASS. We consider semilinear sets in $\mathbb{Z}^d$.

\end{proof}

We also deduce immediately that resilience is decidable for $2$-VASS that have a semilinear reachability relation.

\begin{theorem}{}
Resilience is decidable for continuous VASS when $\Safe$ and $\Bad$ are definable in the existential theory of
the rationals with addition and order.
\end{theorem}


\begin{proof}
The reachability relation of continuous VASS is definable by a sentence of linear size in the existential theory of
the rationals with addition and order whose complexity is EXPSPACE. Hence, $\Bad \subseteq \pred^*(\Safe)$ is also decidable in EXPSPACE. \alain{and may be in coNP-complete. Recall that reachability is P-complete for continuous VASS}
\end{proof}


open for VASS ???


%		\section{Control of a WSTS to be resilient}



\mathieu{Tableau résumé de la situation pour les VASS}

\begin{center}
\begin{tabular}{ | l | c | c | c | r |}
\hline   \Safe~\Bad 
		& $\uparrow$~ $\downarrow$~ 
		 & $\downarrow$~ $\uparrow$~ 

 \\ \hline
   RP  
   	& Decidable (Thm~\ref{down-up})  
   		 & ??

    \\ \hline
   BRP  
   &  Decidable (Thm~\ref{down-up}) 
   		 & Decidable (never hold ?)

    \\ \hline
      kRP  
      & Decidable (Thm~\ref{down-up}) 
      		& Decidable (never hold ?)

       \\ \hline
   sRP  
   	& Decidable (Thm~\ref{post srp})
   		 & ??


    \\ \hline
   BsRP  
   &  Decidable (Thm~\ref{ref ozkan})
   		 & ?? 

    \\ \hline
      ksRP   
      & Decidable (Thm~\ref{ref ozkan})
      		& ?? 

       \\ \hline

\end{tabular}
\end{center}


\begin{proposition}
{\sc Bounded resilience} and {\sc $k$-resilience} never hold for vector addition system when $\Safe = \downarrow \Safe$ and $\Bad = \uparrow \Bad$
\end{proposition}


\begin{proof}
Consider a given $k \in \N$.
Consider $\Bad \subseteq \N^d$ upward-closed and $\Safe \subseteq \N^d$ downward-closed.
Consider a given Vector addition system, which has a finite number of transitions.
Let us call $c_{\max}$ the maximal constant in a coordinate of a transition.

$\Bad$ admits a finite basis $B_\Bad$.
Consider $\textbf{v}_{\Bad}$ obtained by summing all members of the basis of $\Bad$ and then consider 
$\textbf{u}_k = \textbf{v}_{\Bad} + (k+1) \cdot (c_{\max}, c_{\max}, \ldots, c_{\max})$.

All configurations reachable from $\textbf{u}_k$ in $k$ or less steps are above $ \textbf{v}_{\Bad} $
and thus, are in $\Bad$, by upward-closedness.

Hence  $\Safe$ is not reachable from $\textbf{u}_k$ in $k$ or less steps  and {$k$-resilience} does not hold.
Since the reasoning hold for all $k \in \N$, {\sc Bounded-resilience} does not hold.
\end{proof}

\mathieu{
En train d'essayer de réfléchir voir si ça s'adapte aussi aux VASS.}

%
\iffalse
%
\subsection{Timed Automata}

\textcolor{red}{Should be defined in a later 'application section' once we start writing any proof, for now I leave it there} 

\renewcommand{\A}{\mathcal{A}}
\newcommand{\B}{\mathcal{B}}
\renewcommand{\C}{\mathcal{C}}
\newcommand{\Const}{\mathsf{Consts}}
\newcommand{\Conf}{\mathsf{Conf}}
\newcommand{\guards}{{\textsc{Guards}}}

% \subsubsection{Guards, Clocks}

A {\em guard} over a finite set of clocks $\Omega$ 
is a comparison of the form
$\omega \bowtie c$, where $ \omega \in \Omega$, $c \in \N$,
and $\bowtie\in\{<,\leq,=,\geq,>\}$.
%
We denote by $\guards(\Omega)$ the {\em set of guards} over the set of 
clocks $\Omega$.
The {\em size} %$|g|$
 of a guard 
$g=\omega \bowtie c$ is defined as %:
$|g|=\log(c)$.
A {\em clock valuation} is a function from $\Omega$ to $\N$;
we write $\vec{0}$ to denote the clock valuation $\omega \mapsto 0$
whenever the set $\Omega$ is clear from the context.
For each clock valuation $v$ and each $t\in\N$ we denote
by $v+t$ the clock valuation $\omega \mapsto v(\omega)+t$.
%
For each guard $g=\omega \bowtie c$ with $c\in\N$,
we write $v\models g$ if $v(\omega)\bowtie c$.

\iffalse
We define an {\em empty guard} $g_\epsilon$ over a non-empty finite set of clocks
$\Omega$ and to be of the form $\omega \geq 0$ for some 
$\omega \in \Omega$. In particular, we
defined $g_\epsilon$ such that for all $v \in \N^\Omega$ 
we have
$v \models g_\epsilon$, hence $g_\epsilon$ can be used as a guard that is always true. 
\fi

% \subsubsection{Timed automata}


A timed automaton is a finite automaton extended with a finite set of clocks $\Omega$ that all progress at the same rate and that can individually be reset to zero. Moreover, every transition is labeled by a guard over 
$\Omega$  and by a set of clocks to be reset. \\

\par\noindent\ignorespacesafterend
Formally, a {\em timed automaton} ({\em TA} for short) is a tuple
$\A=(Q,\Omega,R,q_{init},F)$, where
\begin{samepage}
\begin{itemize}
	\item $Q$ is a non-empty finite {\em set of states}, 
	\item $\Omega$ is a non-empty finite {\em set of clocks},
	\item $R \subseteq Q\times\G(\Omega)\times \mathscr{P}(\Omega) \times Q$
	is a finite {\em set of  rules},
	\item $q_{init}\in Q$ is an {\em initial  state}, and 
	\item $F\subseteq Q$ is a {\em set of final states}.
\end{itemize}
\end{samepage}

\par\noindent\ignorespacesafterend
We also refer to $\A$ as an $n$-TA if $|\Omega| = n$. 
The {\em size} of $\A$ is defined as%:
$$
|\A| \ = \ |Q|+|\Omega|+|R|+\sum_{(q,g,U,q')\in R}|g|.
$$
Let 
$\Const(\A) = \{ c\in\N \mid \exists(q,g,U,q')\in R, \ \exists \omega \in \Omega, \bowtie\in\{<,\leq,=,\geq,>\} : g = \omega \bowtie c \}$ denote the 
set of constants that appear in the guards of the rules of $\A$.

By $\Conf(\A)=Q\times\N^\Omega$ we denote the set of
{\em configurations} of $\A$. 
%We prefer however to denote a configuration in $\Const(\A)$ by $q(v)$ instead of $(q,v)$.\\
We prefer however to abbreviate a configuration	%	 in $\Conf(\A)$	
$(q,v)$ by $q(v)$.


\begin{samepage}
%\begin{definition}
A TA $\A=(Q,\Omega,R,q_{init},F)$ induces the labeled transition system 
$T_{\A} =  (\Conf(\A), \Lambda_{\A}, \rightarrow_{\A})$
where $ \Lambda_{\A} = R \times \N $
and 
where $ \rightarrow_{\A}$ is defined such that, 
for all $(\delta,t)\in R\times\N$ with  	$\delta = (q,g,U,q')\in R$,
for all $q(v), q'(v') \in \Conf(\A)$,
$q(v)\xrightarrow{\delta,t}_{\A} q'(v')$ if
	$v+t\models g$, 
	 $v'(u)=0$ for all $u \in U$ and $v'(\omega)=v(\omega)+t$ for all 
	$\omega \in \Omega \setminus U$.
%\end{definition}
\end{samepage}

A {\em run} from $q_0(v_0)$ to $q_n(v_n)$ in $\A$ is a path in the transition system $T_{\A}$, that is,
a sequence 
$\pi = q_0(v_0)\xrightarrow{\delta_1,t_1}_{\A}q_1(v_1)\cdots\xrightarrow{\delta_n,t_n}_{\A}q_n(v_n)$;
it is called {\em reset-free} if for all $i \in \{1,\ldots,n\}$,
 $\delta_i = (g_i,\emptyset)$ for some guard $g_i$.


We say $\pi$ is {\em accepting} if $q_0(v_0) = q_{init}(\vec{0})$ and $q_n \in F$. 
\iffalse
We say {\em reachability holds} for the TA $\A$ 
if there exists an accepting run. %
% if there is a run in $\A$ from $q_{init}(\vec{0})$   to some configuration $q(v)$ for some $q\in F$, and $v\in\N^\Omega$.
%We refer to Figure~\ref{example pta} for an instance of a PTA for which reachability holds.
\fi


It is worth mentioning that there are further modes of time valuations and guards which exist in the literature, we refer
to \cite{Andre19} for a recent overview. 
%
% \mh{Comment on difference between continuous and discrete time}
Notably, we consider in this article only the case of timed automata over discrete time. It is worth mentioning that in
the case of timed automata over continuous time (i.e. with clocks having values in $\R_{\geq 0}$),
% However, for parametric timed automata with closed (i.e., non-strict) clock constraints and parameters restricted to ranging over integers, 1 standard digitisation techniques apply [HMP92, OW03], reducing the reachability problem over dense time to discrete (integer) time.
techniques~\cite{HenzingerMP92,OuaknineW03} exist for reducing the reachability problem to discrete time in the case of closed (i.e. non-strict) clock constraints ranging over integers. \\




\problemx{TA $k$-resilience problem}
{A state $q$ of a TA $(Q, X, \Delta)$, a set $SAFE \subseteq Q$, a set $BAD \subseteq Q$.}
{$\forall q' \in BAD \forall v,v' \in \N^X ~ (q(v) \rightarrow^* q'(v')) \implies \exists q'' \in SAFE \exists v'' \in \N^X ~ q'(v') \rightarrow^{\leq k} q''(v'')$ ?\newline}


Analogously, we formulate the bounded resilience problem for WSTSs.


\problemx{TA bounded resilience problem}
{A state $q$ of a TA $(Q, X, \Delta)$, a set $SAFE \subseteq Q$, a set $BAD \subseteq Q$.}
{$\exists k \geq 0 ~ \forall q' \in BAD \forall v,v' \in \N^X ~ (q(v) \rightarrow^* q'(v')) \implies \exists q'' \in SAFE \exists v'' \in \N^X ~ q'(v') \rightarrow^{\leq k} q''(v'')$ ?\newline}

\textcolor{red}{I think there can be a discussion to be had here about how to quantify on the clock valuations}

\textcolor{red}{Here one thing that could be interesting to try to formalize is: how to enforce that the time that passes is less than $k$, rather than the number of transitions. This is tricky to deal with I find but it should be more doable if for instance we use one counter automata, where the counter effect of the sequence can be quantified more explicitly I suppose ?
But here you could also use a kinda special clock $x$ that is reset when you enter $BAD$ and is not reset between a state in $BAD$ and a state in $SAFE$, you could check that $x < k$.}

\textcolor{red}{... I guess if you use $0/1$-TA then the problems become closer one to another ? Also of note is that $0/1$-TA induces transition systems with bounded branching, so I guess it may be interesting to investigate these first ?}

A {\em $0/1$ timed automaton } ({\em $0/1$-TA} for short) is a tuple
$$\B=(Q,X, \Delta_0, \Delta_1, q_{init}, F),$$
\par\noindent\ignorespacesafterend
 where
$\B_i=(Q,X, R_i, q_{init}, F)$ is a TA for all $i \in \{0,1\}$.
For simplicity we define its {\em size}
as $|\B|=|\B_0|+|\B_1|$.
We analogously denote the constants of $\B$ 
by $\Const(\B)$ and its configurations by  $\Conf(\B)$.

\begin{samepage}
%\begin{definition}
A $0/1$ timed automaton $\B=(Q,X,R_0,R_1,q_{init},F)$ 
induces the labeled transition system 
$T_{\B} = (\Conf(\B), \lambda_{\B}, \rightarrow_{\B}) $
where $ \lambda_{\B} = (R_0 \cup R_1) \times \{ 0,1\}$
	and where $ \rightarrow_{\B}$
	is defined such that
	for all $q(z), q'(z') \in \Conf(\B)$, 
	for all $(\delta,i) \in \lambda_{\B}$
	with $\delta  = (q,g,U,q')\in R_i$
	$q(v)\xrightarrow{\delta,i}_{\B} q'(v')$ if
	$v+i \models g$, 
	$v'(u)=0$ for all $u \in U$ and $v'(\omega)=v(\omega)+ i$ for all $\omega \in \Omega
	\setminus U$. 
%\end{definition}
\end{samepage}



As expected, we write $q(v)\xrightarrow{\delta,i}_{\B}q'(v')$ if 
$q(v)\xrightarrow{\delta,i}_{\B}q'(v')$ for some 
$i\in\{0,1\}$, and some $\delta \in R_i$.




\subsection{One-Counter Automata}

\textcolor{red}{Should be defined in a later 'application section' once we start writing any proof, for now I leave it there} 


\problemx{OCA $k$-resilience problem}
{A state $q$ of a OCA $(Q, \Delta)$, a set $SAFE \subseteq Q$, a set $BAD \subseteq Q$.}
{$\forall q' \in BAD \forall n,n' \in \N ~ (q(n) \rightarrow^* q'(n')) \implies \exists q'' \in SAFE \exists n'' \in \N ~ q'(n') \rightarrow^{\leq k} q''(n'')$ ?\newline}



\problemx{OCA bounded resilience problem}
{A state $q$ of a OCA $(Q, \Delta)$, a set $SAFE \subseteq Q$, a set $BAD \subseteq Q$.}
{$\exists k \geq 0 ~ \forall q' \in BAD \forall n,n' \in \N ~ (q(n) \rightarrow^* q'(n')) \implies \exists q'' \in SAFE \exists n'' \in \N ~ q'(n') \rightarrow^{\leq k} q''(n'')$ ?\newline}

%
\fi
%

\section{conclusion}


\alain{trouver les \Bad~ et \Safe~ maximum tels que S est resilient. est-ce vrai que si S est $(B_i,D_i)$-resilient alors S est $(\cap, \cup B_i,D_i)$-resilient ?}
%
\alain{on peut penser à des ensembles $\Bad$ definis dans une logique booleennne sur les clos par le bas, haut, +...}


%
%
%\subsection{Vector Addition System with States or PN}
%
%\textcolor{red}{Should be defined in a later 'application section' once we start writing any proof, for now I leave it there} 
%





 	



\section{conclusion}

We will analyse in detail the complexities of the different resilience. resilience for other computable models like stack automata,...

\alain{trouver les \Bad~ et \Safe~ maximum tels que S est resilient. est-ce vrai que si S est $(B_i,D_i)$-resilient alors S est $(\cap, \cup B_i,D_i)$-resilient ?}
%
\alain{on peut penser à des ensembles $\Bad$ definis dans une logique booleennne sur les clos par le bas, haut, +...}


%
%
%\subsection{Vector Addition System with States or PN}
%
%\textcolor{red}{Should be defined in a later 'application section' once we start writing any proof, for now I leave it there} 
%







\bibliography{bibliography/bib}


\appendix





\section{Resilience when $\Safe$ and $\Bad$ have the same closure properties}

We relax now the hypothesis that $\Bad$ is the complement of $\Safe$. In this appendix it is thus possible for $\Safe$ and $\Bad$ to share closure properties, i.e. for them to be both downward-closed or upward-closed. We will now explore both possiblities.


\subsection{Case: $\Bad=\downarrow \Bad$ and $\Safe=\downarrow \Safe$}\label{case down down}

% In the case \Safe and \Bad are both downward-closed, they can both be finite.
We start with the case $\Safe = \downarrow \Safe$ and $\Bad = \downarrow \Bad$.
%
Unfortunately, in this case the resilience problem is undecidable.

\begin{theorem}\label{down-down}
{\sc Resilience} is undecidable for  effective WSTS with  strong  compatibility such that
%	with more than one minimal element, 
$\Safe=\downarrow \Safe$
and $\Bad=\downarrow \Bad$.
\end{theorem}

\begin{proof}
If the set $S$ of a WSTS $\mathscr{S}=(S,\rightarrow, \leq)$ has an unique minimal element $m$, then $m$ belongs to both $\Safe$ and $\Bad$ which contradicts the assumption $\Safe \cap \Bad= \emptyset$. So let us consider the case of a set $S$ with at least two minimal elements $m_1$ and $m_2$.
The problem of whether $m_2$ is reachable from $m_1$ reduces itself to the resilience problem by considering $\Safe=\downarrow m_2 = \{ m_2\}$ and $\Bad=\downarrow m_1 = \{ m_1\}$. By undecidability of the reachability problem for effective WSTS with strong compatibility we conclude.  
\end{proof}


\subsection{Case: $\Bad=\uparrow \Bad$ and $\Safe=\uparrow \Safe$.}\label{case up up}


% \alain{2 clos par le haut ont une intersection non vide....}

We now consider the case where $\Safe = \uparrow \Safe$ and $\Bad$ is not the complement of 
$\Safe$ and we'd want $\Bad$ to have the same closure properties as $\Safe$.
We still however maintain the hypothesis that $\Bad \cap \Safe = \emptyset$, thus the conditions 
$\Safe = \uparrow \Safe$ and $\Bad = \uparrow \Bad$ would lead to a contradiction, since two upward-closed sets have not an empty intersection. We consider instead the case where $\Bad$ is as upward-closed
as it can be without intersecting with $\Safe$,
i.e. 
such that there exists an upward-closed set $U$
such that $\Bad = U \setminus \Safe$.



\begin{theorem}\label{up-up}
{\sc Resilience} is decidable for effective WSTS with effective pred-basis, $\Safe=\uparrow \Safe$
and $\Bad= U \setminus \Safe$ with $U = \uparrow U$.
\end{theorem}


\begin{proof}
Since $\Safe=\uparrow \Safe$ is upward-closed, there exists a finite basis $B_{ \Safe}$ such that $\uparrow \Safe = \uparrow B_{\Safe}$. 
Moreover since $\mathscr{S}=(S,\rightarrow,\leq)$ is a WSTS,  $\pred^*(\Safe)=\uparrow \pred^*(\Safe)$ and $\pred^*(\Safe)$ admits a finite basis $B_{\pred^*(\Safe)}$. Since $\mathscr{S}=(S,\rightarrow,\leq)$ is a WSTS  with effective pred-basis, we may compute a finite basis of $\pred^*(\Safe)$ with the backward coverability algorithm. 
Since $U$  is upward-closed, there exists a finite basis $B_{U}$ such that $U = \uparrow B_{U}$. % Moreover $ \uparrow B_{\Bad} \subseteq \uparrow B_{\pred^*(\Safe)}$ iff for every $b \in B_{\Bad}$, there is a $s \in B_{\pred^*(\Safe)}$ such that $s \leq b$.
Since $\Bad = U \setminus \Safe$, $\Bad \subseteq \uparrow B_{\pred^*(\Safe)}$ iff for every $b \in B_{U}$, there is either $s \in B_{\pred^*(\Safe)}$ such that $s \leq b$, either $t \in B_{\Safe}$ such that $t\leq b$,
% si on est dans up Bad, il faut que ou bien on soit dans pred(Safe) ou bien on soit dans Safe
hence the resilience problem is decidable.
\end{proof}

\begin{proposition}
In WSTS with strong compatibility and effective pred-basis,  $\Safe=\uparrow \Safe$, the {\sc Bounded resilience} is equivalent to {\sc Resilience}.
\end{proposition}

\begin{proof}
Since $\Safe=\uparrow \Safe$ and
$\mathscr{S}=(S,\rightarrow,\leq)$ is a WSTS with strong %upward-
compatibility, then $\pred^{\leq n}(\Safe)= \uparrow~\pred^{\leq n}(\Safe)$ for all $n \in \N$,
% \alain{$\pred^n$ ou $\pred^{\leq n}$ ? même question pour post. et où est-ce défini ?}
and there exists $n_0 \in \N$ such that 
$\pred^{\leq n_0}(\Safe) = \uparrow \pred^{\leq n_0}(\Safe) = \uparrow \pred^*(\Safe) = \pred^*(\Safe)$.
Hence the equivalence.
\end{proof}

\begin{corollary}\label{B-up-up}
{\sc Bounded resilience}  is decidable for WSTS with effective pred-basis,
strong compatibility
 $\Safe=\uparrow \Safe$
and $\Bad=U \setminus \Safe$ with $U = \uparrow U$.
\end{corollary}


\begin{theorem}\label{k-up-up}
{\sc $k$-resilience}  is decidable for WSTS with effective pred-basis, strong %upward-
compatibility, $\Safe=\uparrow \Safe$
and $\Bad=U \setminus \Safe$  with $U = \uparrow U$.
%	and \Bad is upward-closed or downward-closed.
\end{theorem}

\begin{proof}
Since $\Safe=\uparrow \Safe$ is upward-closed, there exists a finite basis $B_{ \Safe}$ such that $\uparrow \Safe = \uparrow B_{\Safe}$. 
Moreover since $\mathscr{S}=(S,\rightarrow,\leq)$ is a WSTS with strong (upward) compatibility,  $\pred^{\leq n}(\Safe)=\uparrow \pred^{\leq n}(\Safe)$ for all $n\in \N$ and $\pred^{\leq k}(\Safe)$ admits a finite basis $B_{\pred^{\leq k}(\Safe)}$. % Since $\mathscr{S}=(S,\rightarrow,\leq)$ is a WSTS  with effective pred-basis, we may compute a finite basis of $\pred^{\leq k}(\Safe)$ with the backward coverability algorithm. 
Since $U$  is upward-closed, there exists a finite basis $B_{U}$ such that $U = \uparrow B_{U}$. % Moreover $ \uparrow B_{\Bad} \subseteq \uparrow B_{\pred^{\leq k}(\Safe)}$ iff for every $b \in B_{\Bad}$, there is a $s \in B_{\pred^*(\Safe)}$ such that $s \leq b$.
Since $\Bad = U \setminus \Safe$, $\Bad \subseteq \uparrow B_{\pred^{\leq k}(\Safe)}$ iff for every $b \in B_{U}$, there is either $s \in B_{\pred^{\leq k}(\Safe)}$ such that $s \leq b$, either $t \in B_{\Safe}$ such that $t\leq b$,
% si on est dans up Bad, il faut que ou bien on soit dans pred(Safe) ou bien on soit dans Safe
hence the resilience problem is decidable.
\end{proof}













\end{document}















