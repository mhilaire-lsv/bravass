
\documentclass[a4paper,UKenglish,cleveref, autoref, thm-restate]{lipics-v2021}

\bibliographystyle{plainurl}% the mandatory bibstyle
\usepackage{booktabs}   %% For formal tables:
                        %% http://ctan.org/pkg/booktabs
\usepackage{subcaption} %% For complex figures with subfigures/subcaptions
                        %% http://ctan.org/pkg/subcaption

\usepackage{mathtools}
\usepackage{todonotes}
\usepackage{microtype}

\usepackage{complexity}
\usepackage{amsmath}
\usepackage{dsfont}

\usepackage{stmaryrd}
\usepackage{MnSymbol}
\usepackage{graphicx}




\newcommand{\problemx}[3]{
	\vspace{0.2cm}
\par\noindent\underline{\sc#1}\par\nobreak\vskip.2\baselineskip
\begingroup\clubpenalty10000\widowpenalty10000
\setbox0\hbox{\bf INPUT:\ }\setbox1\hbox{\bf QUESTION:\ }
\dimen0=\wd0\ifnum\wd1>\dimen0\dimen0=\wd1\fi
\vskip-\parskip\noindent
\hbox to\dimen0{\box0\hfil}\hangindent\dimen0\hangafter1\ignorespaces#2\par
\vskip-\parskip\noindent
\hbox to\dimen0{\box1\hfil}\hangindent\dimen0\hangafter1\ignorespaces#3\par
\endgroup
	\vspace{-0.2cm}
}

\newcounter{claimcounter}
\setcounter{claimcounter}{0}
\newtheorem{subclaim}{Subclaim}{}
\newtheorem*{theorem*}{Theorem}

\makeatletter

\usepackage{ textcomp } 

\usepackage{stmaryrd}
\usepackage{MnSymbol}
\usepackage{graphicx}

\usepackage{wrapfig}


\newcommand\sg[1]{\todo[inline,size=\scriptsize]{#1 - \textbf{Stefan}}}
\newcommand\mh[1]{\todo[inline,size=\scriptsize]{#1 - \textbf{Mathieu}}}
\newcommand\sgchanged[1]{{\color{red}{'1}}}
\newcommand\mhchanged[1]{{\color{blue}{'1}}}


\renewcommand{\A}{\mathcal{A}}
\newcommand{\B}{\mathcal{B}}
\renewcommand{\C}{\mathcal{C}}
\newcommand{\T}{\mathcal{T}}
\renewcommand{\G}{\mathcal{G}}
\renewcommand{\O}{\mathcal{O}}
\renewcommand{\R}{\mathbb{R}}
\renewcommand{\H}{\mathcal{H}}
\newcommand{\Z}{\mathbb{Z}}
\newcommand{\N}{\mathbb{N}}
\newcommand{\Time}{\mathbb{T}}
\newcommand{\Const}{\mathsf{Consts}}
\newcommand{\Conf}{\mathsf{Conf}}
\newcommand{\PTA}{\textsc{PTA}}
\newcommand{\MSO}{\textsc{MSO}}


\renewcommand{\poly}{\mathrm{poly}}

\newcommand{\Op}{\mathsf{Op}}
\newcommand{\win}{\textsc{Win}}



\newcommand{\semi}[1]{\xRightarrow{'1}}




\title{Parity games in pushdown systems with pebbles} 

\titlerunning{Parity games in pushdown systems with pebbles} 

% \author{Stefan G\"oller}{University of Kassel \and School of Electrical Engineering and Computer Science \and Kassel, Germany }{stefan.goeller@uni-kassel.de}{}{The author was supported by the Agence Nationale de la Recherche grant no.  ANR-17-CE40-0010.}
\author{Mathieu Hilaire}{Université Paris-Saclay\and
CNRS\and
ENS Paris-Saclay\and
Laboratoire Méthodes Formelles (LMF)\and
Gif-sur-Yvette, France
}{hilaire@lsv.fr}{}{This work was partly done while the author was supported by the 
Agence Nationale de la Recherche grant no.  ANR-17-CE40-0010.}

\authorrunning{M. Hilaire} 

\Copyright{M. Hilaire} 

%\ccsdesc[500]{Theory of computation~Timed and hybrid models}
%\ccsdesc{Theory of computation~Automata over infinite objects}
\ccsdesc[500]{Theory of computation~Automata extensions}

\keywords{Parity Games, Computational Complexity, Pushdown systems, Pebble Alternating Two Way Automata}



\category{} 

\relatedversion{}
% \relatedversiondetails{Full Version}{}

\acknowledgements{The author would like to thank Benedikt Bollig and Stefan G\"oller for helpful discussions and feedback.}


\ArticleNo{70}
\begin{document}

\maketitle


\begin{abstract}
	We investigate the decidability and complexity of
	parity games over an extension of pushdown systems.
	More specifically, we consider pebble pushdown systems, in
	which pebbles can be used to remember some stack content already visited,
	and compared against.
	We determine decidability of the problem, and provide a nonelementary lower bound. 
\end{abstract}

\section{Introduction}

\subsubsection*{Background}



Stack-based automata are a popular formalism for modeling the sequential behavior of computer programs. 
Pushdown systems are probably the most studied models of stack-based automata.
They are
finite automata 
 extended with a stack that can be manipulated by pushing or popping stack symbols from some finite set, and can use the top of the stack to decide which transition to take next.
\textcolor{black}{To underline the fact that we are concerned with the graph of configurations and not the language recognized, we use the name pushdown system rather than pushdown automaton.}
Pushdown systems can be used to model the behavior of recursive programs and,
as a result, a variety of 
 program analysis questions can be reduced to decision problems for games on pushdown systems.
They have applications for instance in inter-procedural control-flow analysis of recursive programs \cite{esparza1999automata, reps2005weighted}.





Two player games can be used in particular to
 represent the
interaction of a program with some environment, with the first player representing the program while the second player represents the environment. A winning condition then expresses a property required to hold however the environment behaves. Finding a winning strategy for the first player
thus provides some way 
 to ensure that the desired 
property, as expressed by the winning condition,
always holds \cite{arnold2003games}.


Games on the graphs of pushdown systems have been extensively studied.
It is known in particular from \cite{walukiewicz1996pushdown} that deciding the winner of a parity game on a graph of a pushdown system is an ${\sf EXPTIME}$-complete problem.
Several generalizations of the pushdown systems 
have been studied in depths, including
tree-pushdown systems \cite{guessarian1983pushdown}, ordered
multi-pushdown systems \cite{breveglieri1996multi, atig2012model}, annotated higher-order pushdown systems \cite{maslov1976multilevel, broadbent2012saturation}, and
strongly normed multi-pushdown systems \cite{czerwinski2012reachability}.




We consider here a different approach to extending pushdown systems, by allowing them to test the stack content against precise values.
Pushdown systems which can test the stack content against regular-expressions are
equivalent to pushdown systems,
so pushdown systems which can test the stack content against specific words are too.
We consider instead checks against unspecified word values that can be instantiated one after the other,
providing the pushdown systems the ability to
store information on some stack contents for later comparisons.



Hence we propose in this paper to extend pushdown systems with a set of pebbles.
Compared to pebble word automata, instead of using pebbles
as markings on their input, pebble pushdown systems have the ability to lift or drop pebbles
on their universe of stack contents. Dropping a pebble records the current stack content with an available pebble until it is lifted. Lifting a pebble then makes it available again.


A pebble automaton is a two-way finite state automaton that uses a fixed, finite number of
pebbles that it can drop on, and lift from 
words, using them as markers.
Pebble automata recognize regular languages only, provided the life times of the pebbles, i.e. the times between dropping a pebble and lifting it again, are properly nested
\cite{globerman1996complexity, engelfriet1999tree}.
Automata with nested pebbles were also introduced
for tree-walking automata. 
It is known that tree-walking automata do not recognize all regular tree languages \cite{bojanczyk2008tree}, the main trouble being that the tree-walking automata get lost rather easily.
Using pebbles is a remedy against getting lost along a tree, but 
the unrestricted use of pebbles leads to a class of tree languages much larger than the
regular tree languages, in fact to all tree languages in $\NSPACE(log ~ n)$.
Thus, in both pebble word automata and pebble tree-walking automata, the placement of the pebble follows a strict stack discipline, and our model will follow this strict stack discipline too.



One motivation to the study of pebble pushdown systems is their ability to compare a stack content against some later stack content.
Similar type of storage for later comparison has already been used for instance
in register pushdown systems. Introduced in \cite{murawski2017reachability}
 as an extension of pushdown systems that 
 can keep data values in
 a set of registers and a stack,
register pushdown systems
enable
manipulation of
data values from an infinite domain,
and have applications for instance for
malware detection and XML schema
checking \cite{senda2021forward, senda2021ltl}.





Another closely related formalism is that of alternating tree automata with nested pebbles, as we 
consider in more details in Section 4. The approach of extending tree automata with pebbles was used in \cite{milo2000typechecking} to show that the XML typechecking problem is decidable, and in \cite{karhumaki2012jewels} to recognize first-order logic on trees. In both cases however, the input trees considered are finite, while 
simulation of plays in pebble pushdown graphs
requires the use of pebble alternating tree automata working on infinite trees
since the stack of a pebble pushdown system is in general unbounded.






The aim of the present paper is to study the decidability and complexity of
deciding
the winner of a parity game on a graph of a pushdown system with  
pebbles.




\subsubsection*{Contributions}


Our main result states that
deciding
the winner of a parity game on a graph of a pebble pushdown system
is decidable but nonelementary.
The contribution is two-fold. 




	For the nonelementary lower bound,
we reduce the $\FO$ satisfiability problem on words, known to be nonelementary
from \cite{Sto74}, to the
problem of deciding whether a player has a winning strategy. 

	For the main contribution, we use pebble alternating two-way tree automata to solve pebble pushdown parity games.
Alternating two-way tree automata with parity winning conditions have been used in the past to solve parity games on pushdown graphs \cite{cachat2002two}. We show that pebble alternating two-way tree automata can similarily be used to solve parity games on pebble pushdown graphs.
We then prove  decidability 
by 
a reduction
 to the problem of model-checking $\MSO$ on a well chosen structure, whose decidability is
 guaranteed by
Muchnik's Theorem~\cite{Wal96}.



\subsubsection*{Overview}

In Section 2, we provide general notations and preliminary definitions. Section 3 will deal with the introduction of pebble pushdown games.
Section 4 is devoted to using pebble alternating tree automata with parity acceptance condition to solve pebble pushdown system parity games. We leave the proof of decidability itself  for Section 5.
Section 6 at last shows a nonelementary lower bound for the problem.




\section{Definitions}


\newcommand{\LCM}{\mathsf{LCM}}
\newcommand{\LOGSPACE}{\mathsf{LOGSPACE}}
\renewcommand{\MSO}{\mathsf{MSO}}
\newcommand{\SO}{\mathsf{SO}}

By $\Z $ we denote the {\em integers} and by $\N=\{0,1,\ldots\}$ we denote the {\em naturals}.
% For every $a,b\in\Z$ with $a\leq b$ we define $[a,b]=\{k\in\Z\mid a\leq k\leq b\}$.
% For every $n\geq 1$ we define $n\Z=\{n\cdot z\mid z\in\Z\}$.
% For every number $n\in\N$ we define $\log(n)=\min\{i+1\mid i\in\N, n\leq 2^i\}$, which is the smallest number of bits necessary to write down $n$ in binary.
For every finite alphabet $A$, $A^*$ is the set of finite
words with letters in  $A$, $A^\omega$ is the set of infinite words with letters in $A$, and
%by
 $A^\infty = A^\omega \cup A^*$. We denote the empty word in $A^*$ by $\epsilon$.
%For all $a\in A$ and all $w\in A^*$ let $|w|_a$ denote the number of occurrences of the letter $a$ in $w$.
%
% For every finite set $M\subset\N\setminus\{0\}$ let $\LCM(M)=\min\{n\geq 1\mid \forall m\in M\setminus\{0\}: m|n\}$ denote the least common multiple of the elements in $M$. 
% For any $j \in \N$ let $\LCM(j)=\LCM([1,j])$ denote the least common multiple of the numbers $\{1,\ldots,j\}$.

 For any two sets $X$ and $S$, let $X^S$ denote the set of all functions from $S$ to $X$.
For any set $S$ let 
%$\mathscr{P}(S) = \{ X \mid X \subseteq S \}$
$\mathcal{P}(S) = \{ X \mid X \subseteq S \}$
denote the power set of $S$.
%
A {\em partial function} $f$ from a set $X$ to a set $Y$ is a function defined on a subset $C$ of $X$ (possibly $X$ itself) with output values in $Y$. We extend the domain of a partial function to the whole set
by considering it as returning the bottom element $\bot$ when it is undefined.




A {\em graph} $G=(V,E)$ is a finite set of graph vertices $V$ with a set of graph edges $E \subseteq V \times V$. 
% Two graphs $G = (V,E)$ and $G'=(V',E')$ with sets of graph vertices $V=V'=V_n=\{1,2,...,n\}$ are said to be {\em isomorphic} if there is a permutation $\rho$ of $V_n$ such that $(u,v)$ is in the set of graph edges $E$ iff $(\rho(u),\rho(v))$ is in the set of graph edges $E'$.
%%%
%
For any set $V$, and a given infinite sequence $\pi = v_0 v_1 \ldots \in V^\omega$ of elements of $V$, we define the set 
$\text{\textit{Inf}}(\pi)$ of elements occurring infinitely often in $\pi$ as
$\text{\textit{Inf}}(\pi) = \{ v \in V \mid \forall i, \exists j > i, ~ v_j = v \}$. 






\subsection{Logics}

We now briefly review some standard definitions from mathematical logic.

\begin{samepage}
\begin{definition}
 A {\em vocabulary} $\tau$ is a set of relational symbols (denoted $P_1, P_2, \ldots $), each of which has a specified
 arity. A symbol $P \in \tau$ is called monadic if its arity is one, i.e., if it is used to
 denote sets.
%
A {\em $\tau$-structure} %(also called a {\em model})
%
$$ \mathfrak{A} = (A,	( P^{\mathfrak{A}} )_{P \in \tau})$$
%
consists of a set $A$ together with an interpretation of
 each $k$-ary relational symbol $P$ from $\tau$ as a $k$-ary relation on $A$; that is, a
set $P^{\mathfrak{A}} \subseteq A^k$.
% A structure $\mathfrak{A}$ is called finite if $A$ and $\sigma$ are finite sets. 
% The {\em universe} of a structure is typically denoted by a Roman letter corresponding to the name of the structure; that is, the universe of $\mathfrak{A}$ is the set $A$, the universe of $\mathfrak{B}$ is $B$, and so on. We shall also occasionally write $a \in \mathfrak{A}$ instead of $a \in A$.\newline
%
\end{definition}
\end{samepage}


%\noindent As an illustration, if $\tau$ consists of a single relational symbol $E$ of arity $2$, graphs are possible $\tau$-structures. 




% Next, we define the  monadic second-order logic, or $\MSO$.

Next, we define monadic second-order formulas over a vocabulary $\tau$, or $\MSO[\tau]$ for short. %We define free variables, and the semantics of $\MSO(\tau)$ formulas.

\begin{samepage}
\begin{definition}\label{MSO}
We assume countably infinite sets of first-order variables {\em Var}
and of second-order variables {\em VAR}. First-order variables
will be typically denoted by $x, y, z, \ldots,$ with subscripts and superscripts,
whereas second-order variables will be typically denoted by $X, Y, Z, \ldots ,$
using subscripts and superscripts as well. We
inductively define %terms and 
formulas of the monadic second-order logic
over vocabulary $\tau$ %as follows: 
by the grammar:
$$ \phi :=  P(x_1 , \ldots , x_k) \mid x_1 = x_2 \mid X(x) 
		\mid \phi \vee \phi \mid \phi \wedge \phi \mid \neg \phi 
		\mid \exists x ~ \phi \mid  \forall x ~ \phi
		\mid \exists X ~ \phi \mid  \forall X ~ \phi		$$
where $P \in \tau$ is a $k$-ary relational symbol, $X \in \text{VAR}$, and $x, x_1, \ldots, x_k \in \text{Var}$.
\end{definition}
\end{samepage}

A formula that does not use existential ($\exists$) nor universal ($\forall$) quantifiers
is called {\em quantifier-free}.
%
% Given a set of formulas $S$, formulas constructed from formulas in $S$ using only the Boolean connectives $\vee$, $\wedge$, and $\neg$ are called {\em Boolean combinations} of formulas in $S$.
%
We shall use the standard shorthand $\phi  \to \psi $ for $ \neg \phi \vee \psi $ 
and $\phi \leftrightarrow \psi $ for
$(\phi  \to \psi ) \wedge (\psi  \to \phi )$. 


If $\overrightarrow{x}$ is the tuple of all the free first-order variables of $\phi$,
and $\overrightarrow{X}$ is the tuple of all the free second-order variables of $\phi$ 
we write $\phi(\overrightarrow{x},\overrightarrow{X})$. A {\em sentence}
is a formula without free variables. We use the standard semantic of $\MSO$,
see Appendix~\ref{MSO appendix} for a more thorough induction over the formula.\\





Concerning the logic $\MSO$, we are interested in the following decision problem.

\problemx{$\MSO[\tau]$ model-checking}
{A $\tau$-structure $\mathfrak{A}$ and a $\MSO[\tau]$-sentence $\phi$.}
{Does $\mathfrak{A} \models \phi $ ?\newline}


\textcolor{black}{We define the problem in a general abstract manner, instead of focusing on particular representation for the input, since we will consider many structures of infinite size.}



{\em First-order logic over a vocabulary $\tau$} ($\FO[\tau]$ for short) is the fragment of $\MSO$ which is restricted to first-order quantifiers and variables: quantification is permitted only over individuals, and no second-order variables are used. 







\subsection{Parity Games}

We first recall the definition of  two  player parity  games.
Two players, player $0$ and player $1$, move along an arena $\mathcal{G}=(V_0,V_1,E,\Omega)$ which is a labelled graph, composed of
two  disjoint  sets of vertices, $V_0$ and $V_1$ respectively associated with player $0$ and player $1$, 
a set of edges $E \subseteq V \times V$, where $V=V_0 \uplus V_1$,
% We require that the out-degree of each vertex is at least one. This ensures that every finite path in (V, E) can be prolonged.
and a mapping $\Omega:V \to \{0,1, \ldots, m\}$,
$m < \omega$, which
assigns a priority to each vertex. %Then the {\em initialized game} $(\mathcal{G}, v_I)$ is given with an initial vertex $v_I \in V$.



A {\em partial play} of $\mathcal{G}$ is a sequence of vertices $v_0 v_1 \ldots \in V^\infty$ connected by the transition relation, i.e. such that $(v_i, v_{i+1}) \in E$.


A {\em play} of $\mathcal{G}$ is a partial play that is {\em maximal} in the following sense: if it is finite, then the last vertex $v_k$ is such that for all $v' \in V$, $(v_k, v') \notin E$. Else, the play is an infinite sequence $ \pi  = v_0 v_1 \ldots \in  V^{\omega} $.


We consider min-parity games: in order to find out who wins a play, we need to take a
look at the positions which are reached infinitely often and compute their priorities using 
$\Omega$.
Player $0$ wins the play $\pi$  iff  $\min( \{ \Omega(v) \mid v \in \text{\textit{Inf}}( \pi) \})$ is even, else, player $1$ wins. The winner of a finite play is the player whose opponent is unable to move, i.e. the winner of a play such that the last vertex of the play is in $V_{0}$ (resp. $V_1$) is player $1$ (resp. player $0$). 



A {\em strategy} $\sigma_i : V^* V_i \to V$ for player $i \in \{ 0,1\}$ is a function that, given a sequence of vertices $ w v$  with $w\in V^*$ and $v \in V_i$,
provides, if possible, a successor $v'$ such that $(v,v') \in E$.
A strategy is called {\em memoryless} if its output only depends on the final vertex of the sequence $v \in V_i$, i.e. if there exists a function $f_i : V_i \to V$ such
that for all $w\in V^*$, $v \in V_i$, we have $\sigma_i(wv) = f_i(v)$.

A partial play $\pi$ is {\em consistent} with a strategy $\sigma_i$ if sequences of vertices that end in $V_i$ along this play all have successors according to the strategy. Given a starting vertex $v_I$,   a strategy $\sigma_0$ for player $0$ and a strategy $\sigma_1$ for player $1$, the play
starting with $v_I$ that is consistent with both strategies is unique and is called the {\em resulting play} of $\sigma_0$ and $\sigma_1$ starting from $v_I$. It is 
denoted by $\pi(v_I, \sigma_0, \sigma_1)$.


A strategy $\sigma_i$ for player $i$ from position $v_I \in V$ is a {\em winning strategy from $v_I$} if 
no matter which
strategy $\sigma_{1-i}$ the other player pursues,
% the resulting play  starting from $v_I$  
%  results in 
% player $i$ winning.
player $i$ wins the play $\pi(v_I, \sigma_0, \sigma_1)$.


The above definitions are essentially the same for finite and infinite arenas.
%
%
In this paper we are going to concern ourselves with parity games over potentially infinitely large arenas, \textcolor{black}{thus we define the following decision problem in a general abstract manner. In particular, we do not discuss representation of the input}. 

\begin{samepage}
\problemx{Solving parity game}
{An arena $\mathcal{G}=(V_0,V_1,E,\Omega)$, an initial vertex $v_I \in V$.}
{Does player $0$ have a winning strategy from $v_I$ in the parity game $\mathcal{G}$?\newline}
\end{samepage}

One important property of parity games is that of {\em memoryless determinacy}: for a parity game and an initial vertex $v_I \in V$, one of the players has a memoryless winning strategy from $v_I$ \cite{zielonka1998infinite}.




\section{Pebble pushdown games}

In this section, we extend pushdown systems with a set of pebbles. 
Pebble pushdown systems will have the ability to lift or drop pebbles on their universe of stack contents according to a strict stack discipline. A \textit{drop} simply records the current stack content with a fresh pebble (such a pebble should be available) and a \textit{lift} pops the last dropped pebble. One can think of a pebble as a register that can store a stack content.




\begin{definition}
An $n$-pebble pushdown system ($n$-PPDS) is a tuple $\mathcal{Z} = (Q, \Gamma,  \Delta )$
where:
\begin{itemize}
\item $Q$ is a {\em finite set of control states},
\item $\Gamma$ is a {\em finite stack alphabet},
% \item $\$ \notin W$ is an {\em initial stack symbol},
\item  $\Delta   \subseteq  Q  \times (\Gamma \biguplus \{ \$ \})  \times \{ 0, 1, \ldots, n\} \times \mathcal{P}(\{ 1, \ldots, n\}) \times Q  \times (\Gamma^* \cup \{ \! \! \text{\textit{lift}}, \text{\textit{drop}}\})$ is a {\em finite transition relation}, where ${\$ \notin \Gamma}$ is some bottom of stack symbol.
\end{itemize}
\end{definition}


A transition in an $n$-pebble pushdown system works like a transition in a pushdown system that additionally manages the $n$ pebbles.
The idea of a transtion $(q, a, i, S, q', \theta) \in \Delta$
is that, if the automaton $\mathcal{Z}$ is in state $q$ with pebbles $1,\ldots, i$ dropped \---- or without pebble dropped if $i = 0$ \---- with top stack symbol $a$, and stack content which corresponds to the stack contents of the pebbles from $S$ and only these pebbles, then
$\mathcal{Z}$ goes to state 
$q'$ and makes modifications to the stack or the pebbles according to
$\theta$.



A {\em pebble set} of $\mathcal{Z}$ is a set $P \subseteq \{ 1, \ldots, n\}$. For a stack alphabet $\Gamma$, a 
{\em $P$-pebble assignment} is a function which maps each $j \in P$ to a word in $\$\Gamma^*$.
For $0 \leq i \leq n$, an {\em $i$-configuration} is a tuple $(q, w, f )$, where 
$q$ is a state,
$w \in \$\Gamma^*$, and
$f$ a $\{ 1, \ldots , i \}$-pebble assignment \---- $f$ is the $\emptyset$-pebble assignment in case $i=0$. 
We call  $q$ the current state, $w$ the current stack, and
$f$ the current pebble assignment. 
We also write $(q,w, w_{1} , \ldots , w_{i} )$ if 
$f(j) = w_j$, for each
$j \leq i$.


The set of configurations is the union of all $i$-configurations for $0 \leq i \leq n$. The transition graph of $\mathcal{Z}$ is $(V,E)$, where $V$ is the set of configurations. %and $E$ is such that
Moreover, 
for all $(q, a, i, S, q', \theta) \in \Delta$, with $a \in \Gamma$ (respectively $ \! a = \$ $),
 for all words $w \in \$ \Gamma^*$, and for all $\{ 1, \ldots , i \}$-pebble assignments 
 
 \begin{samepage}
\noindent
 $f$ such that $f(j) = wa$ (respectively $f(j)= \$ $) for each
$j \in S$ 	and
 $f(j) \neq wa$ (resp. $f(j) \neq \$ $) for each $j \in \{ 1, \ldots , i \} \setminus S$,
%the following holds
we include in $E$ the following transitions:
\begin{itemize}
\item if $\theta \in \Gamma^*$, 
$((q,wa,f), (q',w\theta, f)) \in E $ \\
(respectively, $((q,\$,f), (q',\$\theta, f)) \in E $),
\item if $ \theta = \text{\textit{drop}}$, 
$((q,wa,f), (q',wa, f')) \in E $  \\
(respectively, $((q,\$,f), (q',\$, f')) \in E $),\\
where $f'$ is the $\{ 1, \ldots , i, i+1 \}$-pebble assignment such that
$f'(j) = f(j)$ for each
$j \leq i$, and $f'(i+1) = wa$ (respectively, $f'(i+1) = \$$), and
\item if $\theta = \text{\textit{lift}}$, and $ i \in S $%$i>0$
, 
$((q,wa,f), (q',wa, f')) \in E $ \\ 
(respectively, $((q,\$,f), (q',\$, f')) \in E $),\\
where $f'$ is the $\{ 1, \ldots , i - 1 \}$-pebble assignment such that
$f'(j) = f(j)$ for each
$j < i$.
\end{itemize}
\end{samepage}


This defines a graph  $(V,E) $. 
%
To obtain
a parity game, it remains to define the sets $V_0$ and $V_1$ associating the vertices
to the two players, and the priorities of the configurations. One fixes a disjoint
union
 $Q = Q_0  \biguplus Q_1 $. Then 
 $V_0$ is the set of configurations with current state in $Q_0$ and 
 $V_1$ is the set of configurations with current state in $Q_1$.
The mapping  $\Omega$ is first
defined on $Q$, then  $\Omega (q,w,f) =  \Omega (q)$,  for all words $ w  \in \$ \Gamma^*$, all pebble assignments $f$, and 
$q  \in  Q$. So the player and the
priority only depend on the control states of $\mathcal{Z}$. 
This defines the {\em $n$-pebble pushdown game} $ \mathcal{G}_{\mathcal{Z}, Q_0,Q_1,\Omega}$.
The starting configuration is defined if we also fix an initial state 
$q_I  \in  Q$ as $v_I = (q_I, \$, f_I)$, where $f_I$ is the $\emptyset$-pebble assignment. \newline



The main problem we are interested in in this article is the following decision problem.

\problemx{Solving pebble pushdown parity game}
{A pebble pushdown system $\mathcal{Z}= (Q,\Gamma, \Delta )$, a disjoint
union $Q = Q_0  \biguplus Q_1 $, a priority mapping $\Omega$, an initial state $q_I \in Q$% and initial stack symbol $w_I \in W$
.}
{Does player $0$ have a winning strategy from $(q_I,\$, f_I)$ in the  parity game
$ \mathcal{G}_{\mathcal{Z}, Q_0,Q_1,\Omega}$ ?}
%$(\mathcal{G}_{\mathcal{Z}}, \Omega)$ ?\newline}









\section{Tree Automata}

In this section, we use pebble alternating two-way tree automata to solve pebble pushdown parity games. 
We introduce first alternating two-way tree automata, then pebble alternating two-way tree automata,
then explain the reduction. Decidability questions concerning pebble alternating two-way tree automata
are dealt with in Section~\ref{section decidability}.



\subsection{Alternating two-way tree automata}

Given a finite set $\Gamma$ of directions, a $\Gamma$-tree is a prefix closed set 
$T  \subseteq \Gamma^*$,
i.e., if $x.d  \in  T$, where 
$x  \in  \Gamma^*$ and $d  \in  \Gamma $, then $x  \in  T $. 
We will sometimes forget the “.” of the concatenation. 
The elements of $T$ are called {\em nodes}, the
 empty word  $\epsilon$
% word $\$ $
  is the {\em root} of $T $. For every $x.d  \in  T$, $d  \in  \Gamma$ the node $x$ is the
unique {\em parent} of $x.d$, and $x.d$ is a {\em $d$-child} of $x$. The {\em direction} of a node $x.d$
$( \neq   \epsilon )$ 
%$(\neq \$ )$
is $d$. 
The full infinite tree is 
$T = \Gamma^*$. A {\em path (branch)} of a tree $T$ is
a sequence  $\beta   \in  T^{\infty}$  such that 
$ \beta  = u_0 u_1 \ldots u_n$ or 
$ \beta  = u_0 u_1 u_2 \ldots$, where
$u_0 =  \epsilon$  and
 $\forall i < n,  \exists d  \in  \Gamma$, $u_{i+1} = u_i.d$. 
 A path can be finite or infinite.



For a finite set $X$, $\mathcal{B}^+ (X)$ is the set of positive Boolean formulas
over $X$ (i.e., Boolean formulas built from elements in $X$ using only $\wedge$ and $\vee$),
where we also allow the formulas $\text{\textit{true}}$ and $\text{\textit{false}}$. For a set $Y  \subseteq  X$ and a
formula $ \vartheta   \in  \mathcal{B}^+ (X)$, we say that $Y$ satisfies $ \vartheta$  iff assigning $\text{\textit{true}}$ to elements in
$Y$ and $\text{\textit{false}}$ to elements in $X \setminus Y$ makes  $\vartheta $ true.

The set $\text{\textit{types}}(\Gamma) = \{r \} \cup \Gamma$ 
describes the possible types of a node in the full infinite tree $\Gamma^*$. 
Here $r$ stands for the root, $\gamma \in \Gamma$ for a $\gamma$-child.

To navigate through the tree let $ext(\Gamma ) := \Gamma \biguplus \{ \epsilon , \uparrow \}$ be the extended set of directions: The symbol $\uparrow$ means “go to parent node”, 
and  $\epsilon$  means “stay on the present
node”. Formally we define $ \forall u  \in \$ \Gamma^* , d  \in  \Gamma, u.\epsilon  = u$ and $u.d. \! \uparrow= u$, while the node 
$ \epsilon . \! \uparrow $ is not defined.




\begin{definition}
An alternating two-way automaton over
 $\Gamma$-trees (ATWA) is a tuple
$\mathcal{A} = (Q,  \delta , q_I , 
\Omega)$ where
\begin{itemize}
\item $Q$ is a {\em finite set of states},
\item $\delta  : Q  \times  
%
\text{\textit{types}}(\Gamma)
%W \cup \$
  \to  \mathcal{B}^+ (Q \times ext(\Gamma))$ is a {\em transition function},
\item $q_I$ is an {\em initial control state}, and
\item $\Omega: Q \to \{ 0, 1, \ldots, m\}$ is a {\em priority function}.
\end{itemize}
\end{definition}



The idea of a transition $ \delta (q, a) = ( q',\uparrow ) \wedge (q'',d)$ with $a,d\in\Gamma$ is the following: 
if the automaton
$\mathcal{A}$ is in state $q$ on the node $x.a$ of the 
tree $T$, it will send a “copy” of $\mathcal{A}$  in state $q'$ to the node $x$ and
another “copy” in state $q''$ to $x.a.d$. 
After that the two copies are
running independently. They may come again to the same node with two different
states.


More precisely a run of an alternating two-way automaton $\mathcal{A}$ over a tree
is
a tree $T_r$ in which every node is labelled by an
element of $Q \times \Gamma^*$. The latter tree is like the unfolding of the run, its structure
is different from the tree $\Gamma^*$, and its set of directions range rather on the naturals smaller than $|Q|*|ext(\Gamma)|$; a node in $T_r$ labelled by $(q,x)$ describes a “copy”
of the automaton that is in state $q$ and is situated at the node $x$ of $\Gamma^*$. 
Note
that many nodes of $T_r$ can correspond to the same node of $\Gamma^*$, because the
automaton can come back to a previously visited node. 
The label of a node
and its successors have to satisfy the transition function,
that is, for a node $y$ of $T_r$ labelled by $(q,x)$, and $\delta(q,t)= \vartheta$ where $t$ is the type of $x$, there is a (possibly empty) set 
 $Y = \{ (q_0,d_0),(q_1,d_1), \ldots (q_\ell, d_\ell) \}  \subseteq  Q \times ext(\Gamma)$, such that $Y$ satisfies  $\vartheta $, and for all
$(q_i,d_i)  \in  Y$, the node $y$ has a corresponding son $y.i$ labelled by $(q_i,x.d_i)$. Finally, the root of $T_r$ is labelled by $(q_I,\epsilon)$.



Remember that $x.d$ can be $x. \epsilon$  or $x. \! \uparrow$, and the latter is defined only if 
%$x \neq \$ $.
$x  \neq   \epsilon $.
So the run cannot go up from the root of the input tree. Note that it cannot use a
transition  $\delta (q, a) = \text{\textit{false}}$ since the formula $\text{\textit{false}}$ cannot be satisfied.
A run $T_r$ is accepting if all its infinite paths satisfy the parity acceptance 
condition
given by the priority function. 
% The finite paths of a run that end with a transition $ \vartheta $ equivalent to  $true$, which is viewed as successful termination. 
An infinite path $ \beta   \in  T_r^{\omega}$  satisfies the acceptance condition iff the smallest priority appearing infinitely often in
%this path
the sequence $\gamma$ of corresponding labels
 is even, i.e.
%$\min ~ \text{\textit{Inf}}( \Omega (r( \beta ))) $
 $\min(\{ \Omega(q) \mid (q,x) \in \text{\textit{Inf}}(\gamma) \} $ is even. 
Such
a path in the run consists of following only one “copy” of the automaton. 


% The generalization of twa with nested pebbles came recently, by two independent motivations: On the one hand, with the advancement of XML theory, finite state recognizers (with name n-pebble tree automata) were used in [21] to show that the XML typechecking problem is decidable. On the other hand, the concept of n-pebble tree-walking automata (n-ptwa) were defined in [9] to recognize first-orderlogic on trees.  Later, in [10]n-ptwa were extended with a more general pebble handling.


\subsection{Alternating two-way tree automata with pebbles}

Informally an $n$-pebble alternating two-way tree automaton works much like an alternating two-way tree automaton, but extended with a fixed set of pebbles, numbered from $1$ to $n$ that it can place on the tree. 
% At each time, pebbles $1,\ldots, i$ are placed on some nodes of the tree, for some $i \in \{1, 2, \ldots , n \}$, or no pebble are placed on the tree. 
To navigate through the tree the automaton can stay at the current node, move to its parent, to its 
 children, it can lift a pebble or place a pebble on the current node.
 Which of these transitions can be applied depends on the current state, the set
 of pebbles at the current node, 
and the type of the current node (root, $\gamma$-child for $\gamma \in \Gamma$). 
Operations $\text{\textit{lift}}$ and $\text{\textit{drop}}$ manage the
pebbles much like in pebble pushdown systems.
We indicate the possible kinds of moves of a pebble automaton by elements of the set 
$ext_2(\Gamma) = ext(\Gamma) \biguplus \{ \text{\textit{lift}}, \text{\textit{drop}}\}$.
% Clearly $\text{\textit{drop}}$ refers to dropping a pebble and $\text{\textit{lift}}$ to lifting a pebble. Note that the notation is  unambiguous by virtue of the stack discipline. 
% Finally recall that if $S$ is a set then $\mathcal{P}(S)$ denotes the powerset of $S$.

\begin{samepage}
\begin{definition}
For $n\geq 0$, an $n$-pebble alternating two-way automaton ($n$-PATWA) over  
$\Gamma$-trees is a tuple
$\mathcal{A} = (Q,   
\delta , q_I , 
\Omega)$ where
\begin{itemize}
\item $Q$ is a {\em finite set of states},
\item $\delta  : Q  \times \{ 0, 1, \ldots, n\} \times \mathcal{P}(\{ 1, \ldots, n\}) \times 
\text{\textit{types}}(\Gamma)
  \to  \mathcal{B}^+ (Q \times ext_2(\Gamma))$ is a {\em transition function},
\item $q_I$ is an {\em initial control state}, and
\item $\Omega : Q \to \{ 0, 1, \ldots, m\}$ is a {\em priority function}.
\end{itemize}
\end{definition}
\end{samepage}
 
 
The idea of a transition in a $n$-pebble alternating two-way tree automaton works like a transition in an alternating two-way tree automaton,
except the transition has additional conditions on the pebble assignment, and additional moves for taking care of the pebble.
Consider for instance 
that the automaton
 $\mathcal{A}$ is on the node $x.a$, in state $q$, pebbles $1, 2$ are on the tree, and exactly pebble $1$ is placed on the node $x.a$.
Then a transition
$\delta(q, 2, \{ 1\}, a) =  (q',\uparrow) \wedge (q'',\text{\textit{drop}})$
 will send a “copy” of $\mathcal{A}$  in state $q'$ to the node $x$ and
another “copy” in state $q''$ to $x.a$ with both pebble $1$ and pebble $3$ placed on the node. 
After that the two copies are
running independently.






%A {\em pebble set} of $\mathcal{A}$ is a set $P \subseteq \{ 1, \ldots, n\}$. 
For a tree structure $T$, a 
{\em $P$-pebble assignment} for a pebble set $P$ is a function which maps each $j \in P$ to a node in $T$.
For $0 \leq i \leq n$, an {\em $i$-configuration} is a tuple $(q, x, f )$, where $x$ is a node, $q$ a state and
$f$ a $\{ 1, \ldots , i \}$-pebble assignment \---- $f$ is the $\emptyset$-pebble assignment in case $i=0$. 
We call $q$ the current state, $x$ the current node, and
$f$ the current pebble assignment. 
We also write $(q, x, x_{1} , \ldots , x_{i} )$ if 
$f(j) = x_j$, for each
$j \leq i$.






Then, formally, a run of an $n$-pebble alternating two-way automaton $\mathcal{A}$
over a tree is a tree $T_r$ in which every node is labelled by 
an $i$-configuration,
for some $i \in \{0, 1, \ldots, n\}$. A node in $T_r$ labelled by $(q, x, x_{1}, x_{2}, \ldots, x_{i})$ describes a “copy” of the automaton that is in state $q$, and is situated on the node $x$ of $\Gamma^*$, with the pebbles up to $i$ already dropped, with the $j$-th pebble for $j \leq i$ dropped on the node $x_{j}$.

The label of a
node and its successors have to satisfy the transition function. % However, recall there are restrictions on the drop-operation and the lift-operation : only
% the pebble with the number $i$ can be lifted and only the pebble with number $i - 1$ can be
% placed.
Formally, this works much the same way as for alternating two-way tree automata; 
that is, the root of $T_r$ is
labelled by $(q_I,\epsilon, f_I )$, and, for a node $y$ of $T_r$ labelled by an $i$-configuration $(q, x, f)$, and 
for
$\delta(q, i, S, t) = \vartheta$ where $t$ is
the type of $x$, 
and $f(j) = x$ for each $j \in S$, and $f(j) \neq x$ for each $j \in \{1,\ldots,i \} \setminus S$,
there is a (possibly empty) set 
$Y \subseteq Q \times ext_2(\Gamma )$, such that $Y$ satisfies $\vartheta$,
and for all $(q',d) \in Y$, the node $y$ has a corresponding son
labelled with %$(x.d, q', f')$, where
\begin{itemize}
\item if $d \in ext(\Gamma)$,
the $i$-configuration $(q ', x.d , f)$,
%
\item if $d = \text{\textit{drop}}$, the $(i+1)$-configuration  $ (q', x, f')$, where the domain of $f'$ is $\{ 1, \ldots, i, i+1\}$, $f'$ is equal to $f$ on the domain of $f$, and $f'(i+1) =x$, and
%
\item if $d = \text{\textit{lift}}$
and $f(i) = x$, i.e. if the pebble $i$ is on the current node%
, the $(i-1)$-configuration $(q', x, f')$, where $f'$ is the restriction of $f$ to $\{1, \ldots, i-1 \}$.
\end{itemize}








A run $T_r$ is {\em accepting} if all its infinite paths satisfy the parity acceptance 
condition 
given by the priority function $\Omega$.
% Again finite paths end with a transition with formula $ \vartheta $ equivalent to $\text{\textit{true}}$, which is viewed as successful termination. 
An infinite path $ \beta   \in  T_r^{\omega}$  satisfies the acceptance condition iff the smallest priority appearing infinitely often in
%this path
the sequence $\gamma$ of corresponding labels
 is even, i.e.
%$\min ~ \text{\textit{Inf}}( \Omega (r( \beta ))) $
 $\min(\{ \Omega(q) \mid (q,x,f) \in \text{\textit{Inf}}(\gamma) \} $ is even. 



By a pebble alternating tree-walking automaton (PATWA) we mean an $n$-PATWA for some $n$. 
%Note that a $n$-pebble automaton without pebbles is just an alternating tree-walking automaton.





\subsection{Reduction details}


Alternating two-way tree automata with parity winning conditions have already been used to solve parity games on pushdown graphs in \cite{cachat2002two}, with a technique first developed to solve the model-checking problem in \cite{kupferman2000automata}. The main
idea was to reduce a model-checking problem to an emptiness problem for the class of alternating two-way parity tree automata, and was then adapted to work on pushdown parity games.
%
%
This leads to the following reduction: given a 
parity game on pushdown graph $\mathcal{G}$ and a starting configuration $v_I$, we can
effectively construct an ATWA $\mathcal{A}$
such that
player $0$ has a winning strategy from $v_I$ in $\mathcal{G}$ iff $\mathcal{A}$ accepts the full
infinite tree $\Gamma^*$. The main idea of our reduction is that, when pushdown systems and ATWA are extended with a set of pebbles to manage, the technique used %is conserved.
still works.
We refer the reader to Appendix~\ref{appendix tree reduction} for more details on the proof.




\begin{theorem}~\label{theorem reduction PATWA}
For every $n$-PPDS $\mathcal{Z} = (Q_0 \biguplus Q_1, \Gamma,\Delta)$, 
priority mapping $\Omega$, and state $q_I \in Q_0 \biguplus Q_1$,
we can construct 
effectively
an $n$-PATWA $\mathcal{A}$ 
such that
player $0$ has a winning strategy from $(q_I, \$, f_I)$ in the parity game 
$\mathcal{G}_{\mathcal{Z},Q_0,Q_1,\Omega}$
iff
$\mathcal{A}$ accepts the full infinite tree $\Gamma^*$.
\end{theorem}


% Decidability of checking that the $n$-PATWA $\mathcal{A}$ accepts the full infinite tree $\Gamma^*$ is then dealt with later in the subsequent section.

\section{Decidability}\label{section decidability}








\section{Lower bound}

The aim of this section is to show a nonelementary lower bound for the 
problem of deciding whether player $0$ has a winning strategy from some starting configuration in
a pebble pushdown parity game.
% problem of deciding winning configuration for pebble pushdown parity games.

We reduce the satisfiability problem for first-order logic on words
to the problem of {\sc Solving pebble pushdown parity game}. 
We know from \cite{Sto74}
that satisfiability is nonelementary. 
%

% Here a function is in {\sf elementary} if it is  $O(h_k (n))$ for one of the $k$-fold exponential functions $h_k$ with $h_0 (n) = n$ and $h_{k+1} (n) = 2^{h_k (n)}$.

We define the tower function $T : \N \times \R \to \R$ by $T(0,r) =r$ and
$T(h+1,r)= 2^{T(h,r)}$ for all $h \in \N, r \in \R$. Thus $T(h,r)$ is a tower of $2$s of height $h$ with an $r$ sitting on top.
Observe that for all $n,h \in \N$ with $n\geq 1$, we have
$T(h, log^{(h)}(n)) = n$.
%
% 
Here a problem is in ${\sf ELEMENTARY}$ if it can be solved in time $O ( T(n, 0) )$.

For a finite alphabet $\Sigma$,  let $\tau (\Sigma )$ be the vocabulary consisting of a binary relational symbol $\leq$,  and a unary relational symbol $P_a$ for every $a \in \Sigma$. 
A word $w \in \Sigma^*$ can be seen as a $\tau (\Sigma )$-structure where the set of elements consists of a finite interval of $\N$. Elements are seen as positions in the word and for every element $i$, there exists precisely one $a \in \Sigma$ such that $i$ is in the interpretation of $P_a$. Lastly $\leq$ is interpreted as the linear order of $\N$.\\


It is well-known that if we are interested in the complexity of first-order or monadic
second-order model-checking on words, the alphabet can be assumed to be $\{0,1\}$ without loss of generality.
Thus the satisfiability problem is concerned with the following question.
\begin{samepage}
\problemx{$\FO$ satisfiability on words}
{A  
$\FO[\tau(\{0,1\})]$-sentence $\phi$.}
{Does there exists a word $w\in \{0, 1\}^*$, such that $w \models \phi$ ?\newline}
\end{samepage}



The result that is previously known and motivates the upcoming reduction consists in the following.


\begin{theorem}[\cite{Sto74}]\label{FOnonelementary}

%  Assume that $FPT \neq AW[*]$, and let $f$ be an elementary function and let $p$ be a polynomial.
%  There is no model-checking algorithm for first-order logic on the class of words 
%  whose running time is bounded by
%  $$ f(k) \cdot p(n)$$ 

% where $k$ denotes the size of the input sentence of the model-checking problem and $n$ the
% size of the input word.

The {\sc $\FO$ satisfiability on words} problem is not in {\sc ELEMENTARY}.
	% SAT(N,<) is not elementary-recursive
	% and in fact SAT(N,<) \not\in NSPACE(g(log_b(n),0)) for all b > 3.
\end{theorem}





\subsection{Reduction from FO SAT on words}

The following theorem states a polynomial time reduction from
{\sc $\FO$ satisfiability on words}
%We consider reduction 
towards the problem of %following problem.
{\sc Solving pebble pushdown parity game}.




\begin{samepage}
\begin{theorem}
{\sc FO satisfiability on words} is polynomial time reducible to {\sc Solving pebble pushdown parity game}.
\end{theorem}
\end{samepage}



\begin{proof}

\noindent
We start with
% a $\tau(\{0,1\})$-structure $w$ and
 a 
 $\FO[\tau(\{0,1\})]$%%
% $\FO$%
-formula $\phi$.
%
% Note that $w$ corresponds to a word $w \in \{0,1\}^*$ of size $n$.\\
%
% \noindent
We construct
a pebble pushdown system $\mathcal{Z}_\phi = (Q_\phi,W_\phi, \Delta_\phi )$, with a disjoint
union $Q_\phi = Q_0  \biguplus Q_1 $, a priority mapping $\Omega_\phi$, and an initial state $q_I \in Q_\phi$
such that
player $0$ has a winning strategy from $(q_I, \$, f_I)$ in the  parity game
$\mathcal{G}_{\mathcal{Z_\phi}, Q_0,Q_1, \Omega_\phi}$
if and only if
there exists a word $w \in \{0,1\}^*$ such that
$w  \models \phi $. \\

\noindent
We assume without loss of generality that $\phi$ is written in the prenex normal form
$\phi = \forall x_1 \exists x_2 \forall x_3 \ldots \exists x_k ~ \psi$, where 
an even number $k$ of quantifiers alternate between existential and universal ones, and 
$\psi$ is quantifier-free and in disjunctive normal form.






Player $0$ starts by pushing a word $w$ onto the stack, then places the first pebble at the current position. The goal of this first pebble is to remember the word $\$w$, which corresponds to the 
word “guessed” by player $0$. This pebble is meant to enforce that the stack remain a prefix of $\$w$, with each stack content of the play corresponding to a position in the word. To enforce this, once pebble $1$ is placed, we allow in $\mathcal{Z}_\phi$, after any decision, for a player to challenge the assumption that the last movement of the other player led to a position on the word. A player thus challenged only has the ability to add things onto the stack, and win if and only if able to find a way back to pebble $1$. 

Then players alternate placing pebbles on prefixes of $\$w$. Universal variables correspond to pebbles placed by player $1$ and existential variables to pebbles placed by player $0$. 
Once players have chosen values for their respective variables, the pebble assignment $f$ is fixed. It is then time
for player $0$ to ensure that
$w$ satisfies $\psi$ if the variables $x_1, \ldots, x_k$ are interpreted by 
$ |f(1)|-1, \ldots, |f(k)|-1$ respectively.



Player $0$, being the existential player, choses one of the conjunctive clause in $\psi$, essentially claiming to be able to prove it. Then player $1$, the universal player, chooses one of the atomic formula in the clause to test. 
%The process can be seen in Figure~\ref{DNF}. 
Testing the atomic formula is the purpose of a gadget 
%$ \C_{\psi_{i,j}}$ built such that
in which
 player $0$ has a winning strategy 
 %in $ \C_{\psi_{i,j}}$
  if and only if 
$w$ satisfies %$\psi_{i,j}$
the atomic formula
 if the variables $x_1, \ldots, x_k$ are interpreted by 
$ |f(1)|-1, \ldots, |f(k)|-1$ respectively.



The gadget depends on the type of the atomic formula. For a formula checking the letter at the position of a variable $x$, player $0$ goes to the position corresponding to $x$, then check that the the top of the stack
$x$ is the right letter. For one checking that $x = x'$ where $x$ and $x'$ are both variables,
player $0$ goes to the position corresponding to $x$ and check against the presence of the pebble corresponding to $x'$ as well. If the formula is a negation, players exchange their roles.
\end{proof}

Then, the nonelementary lower bound follows from Theorem~\ref{FOnonelementary}. 



\begin{theorem}

The problem of {\sc Solving pebble pushdown parity game} is not in {\sc ELEMENTARY}.

\end{theorem}





\section{Conclusion}

In this paper we have shown that
deciding
the winner of a parity game on a graph of a pebble pushdown system 
is nonelementary.

	For the nonelementary lower bound
we reduced the $\FO$ satisfiability problem on words \---- known to be nonelementary
from \cite{Sto74} \---- to the
problem of deciding whether player $0$ has a winning strategy from some starting configuration in 
a pebble pushdown parity game.

	For the decidability upper bound, we  used pebble alternating two-way tree automata to solve pebble pushdown parity games.
Alternating two-way tree automata (ATWA) with parity winning conditions have been used to solve parity games on pushdown graphs \cite{cachat2002two}. The technique was first developped to solve model-checking problem for pushdown graphs in \cite{kupferman2000automata}. 
We showed that, when pushdown graphs and ATWA are extended with a set of pebbles to manage, the technique
also works.
We then proved decidability 
% for checking that an $n$-PATWA $\mathcal{A}$ accepts the full infinite tree $\{0,1\}^*$ 
by %reducing it
a reduction
 to the problem of model-checking $\MSO$ on a well chosen structure. To ensure decidability of model-checking $\MSO$ on the chosen structure, we made use of 
Muchnik's Theorem: the structure is constructed by applying a successive number of times Muchnik's iteration to a structure $\mathfrak{A}_0$ for which model-checking $\MSO$ is decidable.
%  After providing this structure, we will show that  checking that an $n$-PATWA $\mathcal{A}$ accepts the full binary tree $\{0,1\}^*$ reduces itself to the problem of model-checking $\MSO$ on it.



\bibliography{bibliography/bib}









\appendix




\section{Reduction from PPDS to PATWA}\label{appendix tree reduction}
% \subsection{Construction}

We provide in this appendix more details on Theorem~\ref{theorem reduction PATWA}. Let us first recall the theorem.

\begin{theorem*}
For every $n$-PPDS $\mathcal{Z} = (Q_0 \biguplus Q_1, \Gamma,\Delta)$, 
a state $q_I \in Q = Q_0 \biguplus Q_1$,
and priority mapping $\Omega: Q \to \{1, \ldots, m\}$,
we can construct 
effectively
an $n$-PATWA $\mathcal{A}$ 
such that
player $0$ has a winning strategy from $(q_I, \$, f_I)$ in the parity game 
$\mathcal{G}_{\mathcal{Z},Q_0,Q_1,\Omega}$
iff
$\mathcal{A}$ accepts the full infinite tree $\Gamma^*$.
\end{theorem*}



We construct a pebble alternating two-way tree automaton $\mathcal{A}$ that determines if player
$0$ has a winning strategy from $(q_I, \$, f_I)$ in $\mathcal{G}_{\mathcal{Z},Q_0,Q_1,\Omega}$, i.e., wins every play, whatever player $1$ does. 
The
automaton $\mathcal{A}$ will simulate the transitions of the pebble pushdown system $\mathcal{Z}$ on the full
$\Gamma$-tree, guess nondeterministically the best moves of player $0$ and follow each
possible move of player $1$ using alternation.
This tree automaton can
go either up or down the tree: it goes down to simulate a rule that pushes
some new symbol, and it goes up to simulate a popping rule, and it drops or lifts a pebble to represent the dropping or lifting of a pebble. For each possible move, the control state has to
be updated in accordance with the pushdown transition rules. As we aim to
simulate a game, the tree automaton needs to be alternating: states
 associated to player $0$ states will have disjonctive booleans formulas
  while states associated to
player $1$ states will have conjonctive ones.
Finally, the priority mapping is inherited from $\Omega$.

As an example, the stack effect of a transition $(q, a, 0, \emptyset, q' , cb)  \in \Delta_{\mathcal{Z}} $ from a configuration $q(wa, \mu)$
of
the PPDS can be simulated by a $n$-PATWA over the full
$\Gamma$-tree from the node $ua$ by the following sequence of moves: 
$\uparrow, c, b$. Indeed
$wa \! \uparrow cb = wcb$. We have chosen suffix rewriting rather than prefix to conform with
the notation of the tree. 
The control states of $\mathcal{A}$ serve to store the control states of $\mathcal{Z}$.
In order to be able to simulate with many steps a transition of $\mathcal{Z}$, the control states of $\mathcal{A}$ are also used to remember the letters it has to write (see Figure~\ref{stack simulation} for an illustration).


\begin{center}
	\begin{figure}
		\hspace{2.1cm}
\includegraphics[width=0.66\textwidth]{figures/stack_simulation}
	\caption{Transition $(p, a, \emptyset, q, cb)$ of the RPDS
	%(long arrow)
	 simulated by the PATWA.}
	 \label{stack simulation}
	\end{figure}
\end{center}


Moving up or down the tree structure is enough in order to simulate the modifications of the stack
performed by transitions of $\mathcal{Z}$. Additionally, dropping or lifting pebble is enough to simulate the dropping or lifting of pebbles in the PPDS. As an example, we simulate
$(q, a, 0, \emptyset, q', \text{\textit{drop}})  \in   \Delta_{\mathcal{Z}} $ by the single move $\text{\textit{drop}}$, and
$(q, a, i, \{i\}, q', \text{\textit{lift}})  \in   \Delta_{\mathcal{Z}} $ by the single move 
$\text{\textit{lift}}$. Additionally, we simulate checks against pebble sets in $\mathcal{Z}$ by similar checks in the PATWA.


More formally now, the control states of $\mathcal{A}$ are tuples
from $( Q \times ext_2(\Gamma) )^*$  
composed of a control state of  $\mathcal{Z}$, and
a sequence of elements of $ ext_2(\Gamma)$.

The sequence of letters is used to remember the letters the automaton has to write. 
Each sequence of letters has at worst the length of a corresponding output of the transition relation of $\mathcal{Z}$, hence the sequences of letters of $\Gamma$ are more precisely elements of $\Gamma^{ \leq |\Delta_{\mathcal{Z}}|}$.  


From a state $(q, \epsilon)$, for a transition
$(q, a, i, S, q', x) \in \Delta_{\mathcal{Z}}$,
the automaton's first move simulating the transition, be it the dropping of a pebble, the lifting of a pebble, or going up the tree, is accompanied by 
checks that the pebbles $1, \ldots, i$ are placed on the tree \---- or that no pebble is on the tree if $i=0$ \---- and checks against the presence of the pebbles of $S$.

From a state $(q , bx)$ ($bx \neq \epsilon$ because $b  \in  \Gamma$) the automaton goes down
in direction $b$, that is to say writes $b$, whatever it reads, and remembers the
(sub)word $x$ it still has to write and the state $q$ of the pushdown system. 


If $q  \in  Q_1$ then $ (q,w)  \in  V_1$ for all $w \in \Gamma^{ \leq |\Delta_{\mathcal{Z}}|}$
and
$\mathcal{A}$
executes from $(q,\epsilon)$ all the possible moves of player $1$, to ensure that player $0$ can win after
each of these moves. But if $q  \in  Q_0$, $\mathcal{A}$ in $(q,\epsilon)$ chooses nondeterministically a move of
player $0$ and tries to make player $0$ win. \newline


The priority mapping of $\mathcal{A}$ is almost the same as the one of 
$\mathcal{G}_{\mathcal{Z},Q_0,Q_1,\Omega}$: 
$\Omega_\A(q, x) =
\Omega (q)$. 
The initial state of $\mathcal{A}$ causes it to go deterministically to the starting
configuration of the game.




\begin{lemma}~\label{lemma reduction PATWA}
Player $0$ has a winning strategy from $(q_I, \$, f_I)$ in $\mathcal{G}_{\mathcal{Z},Q_0,Q_1,\Omega}$ iff $\mathcal{A}$ accepts the full
infinite tree $\Gamma^* $.

\end{lemma}


This concludes our construction for the proof of Theorem~\ref{theorem reduction PATWA}.





%\subsection{A decidability result from reduction to parity games on automata with pebbles}\label{pebbles}
\section{%A decidability result via r
Reduction to pebble pushdown automata parity games
}\label{pebbles}



% \mh{Motivates the use of pebbles. See pebbles on two-way automata, two-way tree automata, weighted automata, etc.}



% 21. Globerman, N., Harel, D.: Complexity results for two-way and multi-pebble automata and their logics. Theor. Comput. Sci. 169, 161–184 (1996)
%		same as [13] bellow
%
% Engelfriet, J., Hoogeboom, H.J.: Tree-walking pebble automata. In: Karhumäki, J., Maurer, H., Pa ̆un, G., Rozenberg, G. (Eds.) Jewels are Forever, Contributions to Theoretical Computer Science in Honor of Arto Salomaa, pp. 72–83. Springer, Berlin (1999) for tree-walking automata.
%
% @incollection{engelfriet1999tree,
%  title={Tree-walking pebble automata},
%  author={Engelfriet, Joost and Hoogeboom, Hendrik Jan},
%  booktitle={Jewels are forever},
%  pages={72--83},
%  year={1999},
%  publisher={Springer}
%}
% 
%
% Two-way pebble transducers for partial functions and their composition, Joost Engelfriet
%
% Tree-Walking Pebble automata, Joost Engelfriet and Hendrik Jan Hoogeboom 
A pebble automaton is a two-way finite state automaton that uses a fixed, finite number of
pebbles that it can drop on, and lift from 
words, using them as markers.
Pebble automata recognize regular languages only, provided the life times of the pebbles, i.e. the times between dropping a pebble and lifting it again, are properly nested
\cite{globerman1996complexity, engelfriet1999tree}.
Automata with nested pebbles were also introduced
for tree-walking automata. 
It is known that tree-walking automata do not recognize all regular tree languages \cite{bojanczyk2008tree}%, the main trouble being that tree-walking automata get lost rather easily
.
Using pebbles is a remedy against getting lost along a tree, but 
the unrestricted use of pebbles leads to a class of tree languages much larger than the
regular tree languages, in fact to all tree languages in $\NSPACE(log ~ n)$.
Thus, in  both pebble word automata and pebble tree-walking automata, the placement of the pebble follows a strict stack discipline. It is traditional hence to represent syntactically the dropping and lifting of pebbles by operations \textit{lift} and \textit{drop}; a \textit{drop} simply records the current position with a fresh pebble (such a pebble should be available) % and returns to the beginning of the tape, 
and a \textit{lift} pops the last dropped pebble if the current position corresponds to the one recorded by it.




In this section, we extend these ideas to pushdown automata. Instead of using pebbles as markings on their input, pebble pushdown automata have the ability to lift or drop pebbles on their universe of stack contents; a \textit{drop} simply records the current stack content with a fresh pebble (such a pebble should be available) % and returns to the beginning of the tape, 
and a \textit{lift} pops the last dropped pebble 
while requiring that
% assuming
 it was placed on the current node. 
% Thus hierarchicality on the registers works slightly differently from hierarchicality on the pebbles, hierarchicality on the registers being a syntactic property, whereas hierarchicality on the pebbles is a semantical one.
One can think of a pebble as a register that can store a stack content for later comparisons.



We first define more formally our pebble pushdown automaton framework, and then provide a reduction from the problem of solving parametric pushdown parity games to the problem of solving pebble pushdown parity games.



%\subsubsection{Pebble pushdown automata}
\subsection{Pebble pushdown automata}

\newcommand{\ppda}{\mathcal{I}}
% \Denarius
% \vernal
% \mathscr{Z}
% \mathcal{I}
% \pluto
% \capricornus
% \Bumpeq
% \Pfund

An {\em $n$-pebble pushdown automaton} 
is a tuple 
$\ppda = (Q, \Gamma,  R, q_{init}, \gamma_{init}, F)$
where
\begin{itemize}
\item $Q$ is a {\em finite set of  states},
\item $\Gamma$ is a {\em finite stack alphabet},
\item  $R  \subseteq  Q  \times \Gamma \times
		 \{ 0, 1, \ldots n\} \times \mathcal{P}(\{1,\ldots,n\})
		 \times Q  \times (\Op(\Gamma) \cup \{\text{\textit{drop}}, \text{\textit{lift}}\})$ is a {\em finite set of rules},
		 where the fourth element $S$ of a rule $r$ is a subset of 
		$\mathcal{P}(\{1,\ldots,i\})$ where $i$ is the third element of $r$,
		and if the last element is
		$\text{\textit{lift}}$ then we additionally require $i \in S$,
\item $q_{init} \in Q$ is an {\em initial  state}, 
\item $\gamma_{init} \in \Gamma$ is an {\em initial stack symbol}, and
\item $F \subseteq Q$ is a {\em set of final  states}.
\end{itemize}

 
%%%%%%%%%%%%%%%%%%%%%%%%%%%%%%%%%%%%%%%%%%%%%%%%%%%%%%%%%%% 
% \mh{remind the reader what a partial function is}
 
\par\noindent\ignorespacesafterend 
Recall that for \iffalse a partial function $f$ from a set $S$ to a set $X$ 
%is a function defined on a subset $C$ of $S$ (possibly $S$ itself) with output values in $X$, and 
that is defined on a subset $C \subseteq S$
is denoted by
$ f : S \rightharpoonup X $.

For \fi a partial function $ f : S \rightharpoonup X $ (see page~\pageref{partial}), for notational purposes, we consider some element $\bot_X \not\in X$ and 
 associate $f$ with the function returning the 
bottom element $\bot_X$ when $f$ is undefined. Thus we write $(X \biguplus \{ \bot_X \})^S$ for the set of all partial functions from $S$ to $X$, which we here abbreviate as $(X \biguplus \{ \bot \})^S$. 


%%%%%%%%%%%%%%%%%%%%%%%%%%%%%%%%%%%%%%%%%%%%%%%%%%%%%%%%%%%
 
 
% An $i$-configuration of $\ppda$ is an element of
%$ Q\times W^* \times (W^*)^{\{1, \ldots i\}}$.
% By $\Conf_i(\ppda)$ we denote the set of $i$-configurations of $\ppda$.
By $\Conf(\ppda)=Q\times \Gamma^* \times (\Gamma^* \biguplus \{ \bot \})^{\{1, \ldots, n\}}$ we denote the set of
% By $\Conf(\ppda) = \bigcup_{0 \leq i \leq n} \Conf_i(\ppda)$ we denote the set of
{\em configurations} of $\ppda$. As expected, we rather write $q(w, \mu)$ instead of $(q, w, \mu)$.
An {\em$i$-configuration} for $i > 0$ of $\ppda$ is a configuration $q(z,\mu)$ where
$\text{Dom}(\mu) = \{1, \ldots, i\}$, while a $0$-configuration is a configuration 
 $q(z,\mu)$ where $\text{Dom}(\mu)=\emptyset$.



The idea of a transition $(q, a, i, S, q', m) \in R$
is that, if the automaton $\ppda$ is in state $q$ with pebbles $1,\ldots, i$ dropped \---- or without pebble dropped if $i = 0$ \---- with top stack symbol $a$, and stack content which corresponds to the stack contents of the pebbles from $S$ and only these pebbles, then
$\ppda$ goes to state 
$q'$ and makes modifications to the stack or the pebbles according to
$m$. 
% \mh{The set $S$ is used for testing the presence of certain pebbles.}
Note a pebble 
can be lifted only if the stack content which corresponds to the pebble
is the same as the current stack content. 
%$\Dzeta$ $\zeta$ $\Xi$
This is enforced by syntactically requiring
the last pebble dropped $i$ is in the set $S$ used for testing the presence of certain pebbles.

\iffalse
A {\em weak} $n$-pebble pushdown system is an $n$-pebble pushdown system $\ppda = (Q, \Gamma,  \Delta )$ in which $\text{\textit{lift}}$ operations also require the presence of the last pebble dropped, i.e.
for all $(q, a, i, S, q', \text{\textit{lift}}) \in \Delta$, we have $ i \in S $. 
%  is an additionnal restriction $f(i) = wa$ that needs to hold for the application of the $lift$ operation.
{\em Weak $n$-pebble pushdown games} are defined naturally as $n$-pebble pushdown games on 
weak $n$-pebble pushdown systems.

\fi



A {\em pebble set} of $\ppda$ is a set $U \subseteq \{ 1, \ldots, n\}$. For a stack alphabet $\Gamma$, a 
{\em $U \!$-pebble assignment} is a function which maps each $j \in U$ to a word in $ \Gamma^*$.
The $\emptyset$-pebble assignment is denoted by $\mu_{init} :  \{ 1, \ldots, n\} \rightharpoonup  \Gamma^* $ and is the 
%partial function from  $\{ 1, \ldots, n\}$ to $ \Gamma^* $ that
%does not map any $j \in \{ 1, \ldots, n \}$ to an output.
totally undefined function.


\iffalse
For $0 \leq i \leq n$, an {\em $i$-configuration} is a tuple $(q, w, \mu )$, where $w \in \Gamma^*$, $q$ is a state and
$\mu$ a $\{ 1, \ldots , i \}$-pebble assignment. 
We call $w$ the current stack, $q$ the current state and
$\mu$ the current pebble assignment. 
We also write $(q,w, w_{1} , \ldots , w_{i} )$ if 
$\mu(j) = w_j$, for each
$j \leq i$.

The set of configurations is the union of all $i$-configurations for $0 \leq i \leq n$. 
\fi


\begin{samepage}
%\begin{definition}
An $n$-pebble pushdown automaton $\ppda =  (Q, \Gamma,  R, q_{init}, {\gamma_{init}} , F)$ induces 
a transition system  $T_{\ppda} = (\Conf(\ppda), \rightarrow_{\ppda})$ 
 such that
for all $(q, a, i, S, q', m) \in R$, with $a \in \Gamma$,
 for all words $w \in  \Gamma^*$, and for all $\{ 1, \ldots , i \}$-pebble assignments $\mu$ such that $\mu(j) = wa$ 
 %(respectively $\mu(j)= {\gamma_{init}}  $) 
 for each
$j \in S$ 	and
 $\mu(j) \neq wa$ 
 %(resp. $\mu(j) \neq {\gamma_{init}} $) 
 for each $j \in \{ 1, \ldots , i \} \setminus S$,
the following holds
\begin{itemize}
\item if $m \in \Op(\Gamma)$, 
% $(q,wa,\mu) \rightarrow_{\ppda} (q',wm, \mu)$, 
either
\begin{itemize}
\item $ m = \text{\textit{push}}^\gamma$ and $(q,wa,\mu) \rightarrow_{\ppda} (q',wa\gamma, \mu)$,

\item $ m = \text{\textit{pop}}$ and $(q,wa,\mu) \rightarrow_{\ppda} (q',w, \mu)$, or

\item $ m = \text{\textit{skip}}$ and $(q,wa,\mu) \rightarrow_{\ppda} (q',wa, \mu)$,
\end{itemize}
\item if $ m = \text{\textit{drop}}$, 
$(q,wa,\mu) \rightarrow_{\ppda} (q',wa, \mu')$,  
where $\mu'$ is the $\{ 1, \ldots , i, i+1 \}$-pebble assignment such that
$\mu'(j) = \mu(j)$, for each
$j \leq i$, and $\mu'(i+1) = wa$, and
\item if $m = \text{\textit{lift}}$, and $i \in S$, i.e. the last pebble dropped belong of the set of pebble we test the presence of,
$(q,wa,\mu) \rightarrow_{\ppda} (q',wa, \mu')$, 
where $\mu'$ is the $\{ 1, \ldots , i - 1 \}$-pebble assignment such that
$\mu'(j) = \mu(j)$, for each
$j < i$.
\end{itemize}
%\end{definition}
\end{samepage}


\par\noindent\ignorespacesafterend
We are interested in games over pebble pushdown automata, % namely, 
%reachability, B\"uchi, and parity games.
mainly, parity games.

Again, given
a 
%labeled
 transition system, one needs only to provide 
%a mapping $\Omega$ and 
a partition of the set of
configurations 
%$S$ into two sets $S_0$ and $S_1$
 to obtain an arena.
 %
Given a partition of
$Q$ into $Q_0$ and $Q_1$, 
we partition the configurations of
$T_{\ppda}$
into
$\Conf_{\ppda,0}=Q_0\times \Gamma^* \times ( \Gamma^* \biguplus \{ \bot \})^{\{1, \ldots n\}}$
and
$\Conf_{\ppda,1}=Q_1\times  \Gamma^* \times ( \Gamma^* \biguplus \{ \bot \})^{\{1, \ldots n\}}$. 



With these notations in mind one can  define the arena
$$
A_{(\ppda, Q_0 , Q_1)}
=
(\Conf_{\ppda,0}, \Conf_{\ppda,1}, \rightarrow_{\ppda}).$$




As expected, given a 
priority function $\Omega: Q \to \{0, \ldots m\}$,
one naturally 
set the extension of $\Omega$ as
$\overline{\Omega}: \Conf(\ppda) \to \{0, \ldots m\}$
and
$\overline{\Omega}(q, w,\mu) = \Omega(q)$
for all
$w \in  \Gamma^*$
and
$\mu \in ( \Gamma^* \biguplus \{ \bot \})^{\{1, \ldots, n\}}$. \\









Concerning pebble pushdown automata, we are interested in the following decision problem.


\problemx{$n$-pebble pushdown parity game}
{An $n$-pebble pushdown automaton $\ppda= (Q, \Gamma,  R, q_{init}, {\gamma_{init}} , F)$, 
where $Q = Q_0  \biguplus Q_1 $, 
and
a priority mapping $\Omega: Q \to \{0, \ldots m\}$.}
{Does player $0$ have a winning strategy from $q_{init}(\gamma_{init}, \mu_{init})$ for the parity game 
$\mathcal{G}= (A_{(\ppda, Q_0 , Q_1)}, \win_{\overline{\Omega}})$ ?\newline}



% \subsubsection{Reduction from parameters to pebble pushdown automata}
\subsection{From parametric pushdown 
automata 
to pebble pushdown automata
}

We now provide a reduction from the problem of solving parametric pushdown parity games to the problem of solving pebble pushdown parity games. 


\begin{theorem}\label{PPDA reduction}
%The problem of 
{\sc $n$-parametric pushdown parity game} 
is polynomial time reducible to
%the problem of 
{\sc $n$-pebble pushdown parity game}.
\end{theorem}




\begin{proof}[Sketch]


Let us fix some
$n$-parametric pushdown automaton $\mathcal{Z}= (Q, \Gamma, 
P%\{p_1, \ldots, p_n \}
, q_{init}, { \gamma_{init}} , F)$,
 disjoint
union $Q = Q_0  \biguplus Q_1$
and $P = P_0 \biguplus P_1$,
and
some priority function 
$\Omega : Q \times \Gamma^* \times ( \Gamma^* \biguplus \{ \bot \})^R \to [0, m ]$.


We construct
% polynomial time
  an 
 $n$-pebble pushdown automaton $\ppda =(Q', \Gamma,  R', q_{init}', {\gamma_{init}} , F')$,
a 
 disjoint
union $Q' = Q_0'  \biguplus Q_1'$
and a mapping $\Omega' : Q \times  \Gamma'^* \times ( \Gamma'^* \biguplus \{ \bot \})^{\{1, \ldots, n\}} \to \{1, \ldots, m \}$,
such that
 %:
%
player $0$ has a winning strategy from 
$(q'_{init},\gamma_{init}, \mu_{init})$
in
		\iffalse
		for the parity condition on
		$\mathcal{G}_{(\ppda, Q'_0, Q'_1, \Omega')}$.
		\fi
the parity game
$\mathcal{G'}= (A_{(\ppda, Q'_0, Q'_1)}, \win_{\overline{\Omega'}})$
if and only if
player $0$ has a 
		\iffalse
		strategy $\sigma_0$ for the parameter valuation game  such that
		for all player $1$ strategies $\sigma_1$ for the parameter valuation game,
		player $0$ have a
		winning strategy from $q_{init}(\gamma_{init})$ for the parity condition on
		$\mathcal{G}_{(\mathcal{Z}, \mu_{\sigma_0, \sigma_1}, P_0, P_1, Q_0 , Q_1, \Omega)}$. 
		\fi
winning strategy from $\mu_\bot$ in 
the parity game
$\mathcal{G} =(A_{(\mathcal{Z}, P_0, P_1, Q_0 , Q_1)}, \win_{\widehat{\Omega}})$,
where		 
$A_{(\mathcal{Z}, P_0, P_1, Q_0 , Q_1)}$
and
$\widehat{\Omega}$
correspond to the definitions on page~\pageref{PPDA reachability game}.

%
%
%
%


The transition graph of the $n$-pebble pushdown automaton $\ppda$ will simulate the possible transitions graphs 
(one for every 
%total 
parameter assignment)
 of the pushdown automaton with $n$ parameters by using pebbles to represent the parameters. 
The $n$-pebble pushdown automaton will first simulate the 
parameter valuation arena.
Players take turns chosing particular stack contents on which to place pebbles,
using a gadget similar as the one from Figure~\ref{quantifiers}. 
Once every parameter has been assigned with the dropping of a corresponding pebble, 
the pebble assignment $\mu$ is fixed.
%%
%
%
%
%
%
%


Further configurations in the $n$-pebble pushdown automaton then consist of a word representing the status of the stack, a state, and a fixed pebble assignment. For a fixed pebble assignment there is then a one for one correspondance between configurations of the $n$-parametric pushdown 
automaton 
%transition system with assignment $\mu$
and these of the $n$-pebble pushdown automaton. The set of states of the $n$-pebble pushdown automaton apart from the initial gadget is the same as 
the set of states of the $n$-parametric pushdown automaton, and 
so are player $0$ states, player $1$ states and the priority mapping.
\end{proof}




\section{Reduction to higher-order pushdown automata}\label{section HPDA}




Higher-order pushdown automata (HPDA for short) were introduced as a 
generalization of pushdown automata \cite{aho1969nested,Gre70,maslov1976multilevel}. 
% Alfred V. Aho. Nested stack automata. J. ACM, 16(3):383–406, 1969.
% S. Greibach. Full AFL’s and nested iterated substitution. Inf. Control, 16(1):7–35, 1970.
% A.N. Maslov. Multilevel stack automata. Problemy Peredachi Informatsii, 12:55–62, 1976.
A stack of a pushdown automaton is seen as a {\em level $1$ stack}. 
A pushdown automaton of level $2$ (or $2$-HPDA) then works with a stack of level $1$ stacks. 
In addition to the ability to push
and to pop a symbol on the top-most level $1$ stack, an $2$-HPDA can copy or remove the entire topmost level $1$ stack. 
The definition generalizes to any $n \geq 2$, and $n$-HPDA are similarly defined for all level
$n$ as automata working with a stack of level $(n-1)$ stacks.
% HPDA have been extensively studied as language recognizers [Dam82,Eng83].



We recall the definition from \cite{cachat2007complexity} which itself is taken from \cite{knapik2002higher}. We then provide a
reduction from the problem of solving pebble pushdown parity games to the problem of solving
higher-order pushdown parity games.


\subsection{Higher-order pushdown automata}

A {\em level $1$ stack} (or {\em $1$-stack}) over an alphabet $\Gamma$ is 
simply a %word in $\Gamma^*$
	stack over $\Gamma$, i.e. a word  in $\Gamma^*$.
% an arbitrary sequence $w_0 w_1 \ldots w_n $ of elements of $\Gamma$ for some $n \in \N$. 
A  {\em level $n$ stack} (or {\em $n$-stack}) over an alphabet $\Gamma$, for $n \geq 2$, is a 
non-empty sequence
$ \langle s_0 \rangle \langle s_1 \rangle \ldots \langle s_m \rangle$
of
$(n-1)$-stacks over $\Gamma$,
for some $m \in \N$.
The set of $n$-stacks over $\Gamma$ is denoted by $\mathscr{S}_n(\Gamma)$,
or simply $\mathscr{S}_n$ in case the set $\Gamma$ is obvious from context.
The set of all stacks over $\Gamma$ is written $\mathscr{S}(\Gamma) = \bigcup_{n\in \N} \mathscr{S}_n(\Gamma)$.
We define $\epsilon_{1}$ as
$\epsilon \in \Gamma^*$ and we inductively define 
$\epsilon_{n} = \langle \epsilon_{n-1} \rangle$ in $\mathscr{S}_n$ for all $n > 1$.




A {\em higher-order stack operation} is a partial function from $\mathscr{S}(\Gamma)$ 
to $\mathscr{S}(\Gamma)$ which preserves the level of the input (i.e. the image of
an $n$-stack is an $n$-stack for all $n \in \N$). The {\em level } of an operation $op$
is the smallest $n \in \N$ such that $\text{Dom}(op) \cap \mathscr{S}_n \neq \emptyset$. 
The operations 
%we consider
additionally
 respect the hierarchicality of higher-order stacks, i.e. in
a level $n+1$ stack only the topmost level $n$ stack can be accessed. An operation
$op$ of level $n$, applied to a level $n+1$ stack 
$ \langle s_0 \rangle \langle s_1 \rangle \ldots \langle s_m \rangle$
of length $m \in \N$,
thus
returns the output
$ \langle s_0 \rangle \langle s_1 \rangle \ldots \langle op(s_m) \rangle$
if applicable.
The definition for  all levels of stacks greater than $n$ follows the same pattern.


% For all $m \in \N$, 
The following operations can be performed on a $1$-stacks of length $m \in \N$.
% $ w_0 w_1 \ldots w_m \in \Gamma^*$.
\begin{eqnarray*}
&\text{\textit{push}}_1^\gamma( w_0 w_1 \ldots w_m) &=\quad w_0 w_1 \ldots w_m \gamma   \text{ for all } \gamma \in  \Gamma, \\
&\text{\textit{pop}}_1(w_0 w_1 \ldots w_{m-1} w_m ) &=\quad w_0 w_1 \ldots w_{m-1}, \text{ if } m \geq 1\\
&\text{\textit{top}}( w_0 w_1 \ldots w_m ) &=\quad w_m.
\end{eqnarray*}
The operations added at level $n+1$ are the copy of the topmost $n$-stack
and the removal of the topmost $n$-stack.
More formally, if $ \langle s_0 \rangle \langle s_1 \rangle \ldots \langle s_m \rangle$ is a stack of 
level $n > 1$, 
the following operations are possible.
%
\begin{eqnarray*}
&\text{\textit{push}}_n( \langle s_0 \rangle \langle s_1 \rangle \ldots \langle s_m \rangle) &=\quad  \langle s_0 \rangle \langle s_1 \rangle \ldots \langle s_m \rangle \langle s_m \rangle , \\
&\text{\textit{push}}_j( \langle s_0 \rangle \langle s_1 \rangle \ldots \langle s_m \rangle) &=\quad \langle s_0 \rangle, \langle s_1 \rangle \ldots 
\langle \text{\textit{push}}_j(s_m) \rangle,
\text{ if } 2 \leq j < n\\
&\text{\textit{push}}_1^\gamma( \langle s_0 \rangle \langle s_1 \rangle \ldots \langle s_m \rangle) &= \quad
\langle s_0 \rangle \langle s_1 \rangle \ldots 
\langle \text{\textit{push}}_1^\gamma(s_m) \rangle,
\text{ for all } \gamma \in \Gamma,\\
&\text{\textit{pop}}_n( \langle s_0 \rangle \langle s_1 \rangle \ldots \langle s_{m-1} \rangle \langle s_m \rangle) &= \quad
\langle s_0 \rangle \langle s_1 \rangle \ldots \langle s_{m-1} \rangle
,\\
&\text{\textit{pop}}_j( \langle s_0 \rangle \langle s_1 \rangle \ldots \langle s_{m-1} \rangle \langle s_m \rangle) &= \quad
\langle s_0 \rangle \langle s_1 \rangle \ldots \langle s_{m-1} \rangle
\langle \text{\textit{pop}}_j(s_m) \rangle,
\text{ if } 1 \leq j < n\\
&\text{\textit{top}}( \langle s_0 \rangle \langle s_1 \rangle \ldots \langle s_m \rangle) &=\quad
\text{\textit{top}}(s_m).
\end{eqnarray*}
% We omit the stack alphabet $\Gamma$ from 
%The operation $pop_j$ is undefined on a stack whose top stack of level $j$ is empty.
The operations $pop_1$ and $top$ are undefined on a stack whose top $1$-stack is empty.  \\


Given a stack alphabet $\Gamma$ and $n \in \N$, 
we denote by  $\Op_n(\Gamma)$ the {\em base set of $n$-stack operations} as

\begin{itemize}
\item $\text{\textit{push}}_j$ for all $2 \leq j \leq n$,

\item $\text{\textit{push}}_1^\gamma$ for all $\gamma \in \Gamma$,

\item $\text{\textit{pop}}_j$ for all $1 \leq j \leq n$,

\item $\text{\textit{skip}}$, corresponding to the identity function of $\mathscr{S}(\Gamma)$.

\end{itemize}




% \begin{definition}
\begin{samepage}
\par\noindent\ignorespacesafterend
A {\em higher-order pushdown automata of level $n$} (or $n$-HPDA for short) 
is a tuple $\H=(Q, \Gamma, R, q_{init}, \gamma_{init}, F)$,
where 
\begin{itemize}
        \item $Q$ is a non-empty finite {\em set of  states},
        \item $\Gamma$ is a non-empty finite {\em  stack alphabet},
	\item $R\subseteq Q\times \Gamma \times Q \times \Op_n(\Gamma)$ is a finite {\em set of  rules},
        \item $q_{init}$ is the {\em initial  state},
        \item $\gamma_{init} \in \Gamma$ is the {\em %bottom-of-stack
initial stack symbol}, and
	\item $F\subseteq Q$ is a {\em set of final  states}.
		\end{itemize}
\end{samepage}
% \end{definition}

\par\noindent\ignorespacesafterend
By $\Conf(\mathcal{H}) = Q \times \mathscr{S}_n(\Gamma)$
we denote the set of {\em configurations} of an $n$-HPDA $\H$. 
%is a pair $(q, s)$ where $q \in Q$ and $s$ is an $n$-stack over $\Gamma$. 
As usual we abbreviate $(q,s) \in \Conf(\mathcal{H})$ as $q(s)$.



An $n$-HPDA $\H=(Q, \Gamma, R, q_{init}, F)$
induces the transition system $\T_\H = (Conf(\mathcal{H}), \rightarrow_{\mathcal{H}})$
where
for all $q,q'$ in $Q$,
for all $s,s'$ in $\mathscr{S}_n(\Gamma)$,
$q(s) \rightarrow_{\mathcal{H}} q'(s')$ if
there exists some rule
%$(q,\gamma,q',\theta) \in R$
$(q,\gamma,q',op) \in R$
such that
$\text{\textit{top}}(s) = \gamma$
and
%$s' = \theta(s)$.
$s' = op(s)$.




\iffalse
For our constructions it would be simpler to assume that k-HPDS can work
also on stacks of lower levels, in particular on 1-stacks. Of course we can always
simulate a j-stack, for j < k with an k-stack but in the notation it requires some
additional parenthesis that make it less readable.

To define a game on the graph of a HPDA, we assign a player to each control
state, and we consider an initial configuration: a game structure on a HPDS H
is a tuple G = (H, P0 , P1 , s0 ), where P = P0 ⊎ P1 is a partition of the control
states of H, and s0 ∈ Sk . This extends naturally to a partition of the set of
configurations: with the notations of Section 2.1, V0 = P0 × Sk , V1 = P1 × Sk ,
and E is defined above.
\fi

Again we are interested in parity games.
As expected, given an 
 $n$-HPDA $\mathcal{H}$
 and a partition of
$Q$ into $Q_0$ and $Q_1$,
we partition the configurations of
$T_{\mathcal{H}}$
into
$\Conf_{\mathcal{H},0}=Q_0\times \mathscr{S}_n(\Gamma)$
and
$\Conf_{\mathcal{H},1}=Q_1\times\mathscr{S}_n(\Gamma)$.
%
%
With these notations in mind one can define the arena
$$
A_{(\mathcal{H}, Q_0, Q_1)}
=
(\Conf_{\mathcal{H},0}, \Conf_{\mathcal{H},1}, \rightarrow_{\mathcal{H}})$$
\par\noindent\ignorespacesafterend
induced by an $n$-HPDA $\mathcal{H}$ and a partition of its set of states.


As expected, given a 
priority function $\Omega: Q \to [0, m]$,
we naturally extend the function
as follows,
by setting
$\Omega_{\mathscr{S}_n(\Gamma)}: \Conf(\mathcal{H}) \to [0, m]$
and
$\Omega_{\mathscr{S}_n(\Gamma)}(q, s) = \Omega(q)$
for all
$s \in \mathscr{S}_n(\Gamma)$.





Concerning higher-order pushdown automata, we are interested in the following decision problem. 
% \newpage

\begin{samepage}
\problemx{$n$-HPDA parity game}
{An $n$-HPDA $\H= (Q, \Gamma,  R, q_{init}, {\gamma_{init}} , F)$, 
where $Q = Q_0  \biguplus Q_1 $, 
and
a priority mapping $\Omega: Q \to \{0, \ldots m\}$.}
{Does player $0$ have a winning strategy 
from $q_{init}(\text{\textit{push}}_1^{\gamma_{init}}(\epsilon_{n}))$ 
for the parity game 
$\mathcal{G}= (A_{(\H, Q_0 , Q_1)}, \win_{\Omega_{\mathscr{S}_n(\Gamma)}})$ ?\newline}
\end{samepage}

It was shown in \cite{Cach03} that {\sc $n$-HPDA parity game} can
be solved in $n$-$\EXP$. This also gives an $n$-$\EXP$ algorithm for the $\mu$-calculus
model checking over transitions systems induced by  $n$-HPDA.
In \cite{cachat2007complexity} the matching lower bound was showed, even in the case of reachability games,
hence showing $n$-$\EXP$-completeness of {\sc $n$-HPDA parity game}.

\begin{theorem}{\cite{ Cach03, cachat2007complexity}}
{\sc $n$-HPDA parity game} is $n$-$\EXP$-complete.
\end{theorem}


% \subsection{From pebble pushdown automata to higher-order pushdown automata}


In a $2$-HPDA, the operation $\text{\textit{push}}_2$ % allowig to copy a part of the stack, are responsible for the fact the hierarchy of HPDS is strict.
allows to ``copy'' the top level $1$ stack. The current word is hereby stored away and left untouched
until the next operation $\text{\textit{pop}}_2$,
while $\text{\textit{push}}_1$ and $\text{\textit{pop}}_1$ can be performed on the additional ``copy'' in the meantime. This behavior is similar to that of dropping and lifting a pebble. 
The main difference is that there is no operation to test that the ``copy'', after many updates, is again equal to the ``original'', i.e. there is no operation to syntactically test that the two topmost level $1$ stacks are identical.

% But cannot test

\iffalse
In an $n$-HPDA, the operation $\text{\textit{push}}_j$ % allowig to copy a part of the stack, are responsible for the fact the hierarchy of HPDS is strict.
allows to ``copy'' the top stack of level $j$ for all $2 \leq j \leq k$. 
\fi


In \cite{CaWoe03, Woeh05, carayol2006automates} however 
%the authors
Carayol and W\"ohrle
 introduced a variant of $n$-HPDA by extending the 
$\text{\textit{pop}}_j$ operations for $2 \leq j \leq n$ with a built-in equality test. 
For $2 \leq j \leq n$ 
the new operation
$\text{\textit{pop}}_j^=$ has the same effect as $\text{\textit{pop}}_j$, but
can only be applied if the two top level $j$ stacks coincide.
In \cite{carayol2006automates} it is seen as a symmetrical operation in comparison to
$\text{\textit{push}}_j$.\\

A {\em higher-order pushdown automaton with equality pop of level $n$} ($n$-HPDA$^=$ for short) 
is a higher-order pushdown automaton of level $n$ where in
% Instrn
$\Op_n(\Gamma)$
 the operation 
 %popk
 $\text{\textit{pop}}_j$  
  is replaced by 
%  pop=
%k 
$\text{\textit{push}}_j^=$
%for 1 ≤ k ≤ n. 
for $2 \leq j \leq n$.
We denote
this new set of operations by 
% Instr= n
$\Op_n^=(\Gamma)$. 
More formally the new operations $\text{\textit{push}}_j^=$ are defined as
\begin{eqnarray*}
&\text{\textit{pop}}_k^=( \langle s_0 \rangle \langle s_1 \rangle \ldots  \langle s_m \rangle \langle s_m \rangle) &=\quad  \langle s_0 \rangle \langle s_1 \rangle \ldots \langle s_m \rangle , \quad \text{ and }\\
&\text{\textit{pop}}_j^=( \langle s_0 \rangle \langle s_1 \rangle \ldots \langle s_m \rangle) &=\quad \langle s_0 \rangle \langle s_1 \rangle \ldots 
\langle \text{\textit{pop}}_j^=(s_m) \rangle,
\text{ if } 2 \leq j < k. 
\end{eqnarray*}
%
%
%
%
\iffalse
Dans ce sous-paragraphe, nous présentons formellement les automates à pile
de niveau k sur le jeu d'opérations symétriques Opsk et sur le jeu d'opérations
classiques COpsk . Nous donnons à cette occasion un bref historique de la notion
d'automate à pile de piles. Enfin, nous établissons l'équivalence entre les deux
modèles d'automates à chaque niveau en tant qu'accepteurs de langages. Ce résultat a été obtenu dans [CW03] en collaboration ave Stefan Wöhrle. La preuve
complète apparait dans [Wöh05]. Indépendamment, ce résultat a été obtenu dans
[Fra05].
\fi
%
Transition systems induced by higher-order pushdown automata with equality pop of level $n$
are then defined as expected.  
Carayol and W\"ohrle proved \cite{Woeh05, carayol2006automates} that the two models,
namely
higher-order pushdown automata with equality pop of level $n$
and
higher-order pushdown automata of level $n$,
generate the same classes of transition systems.





\begin{theorem}{\cite{Woeh05, carayol2006automates}}
If $\H$ is an $n$-HPDA (resp. $n$-HPDA$^=$) then there
exists an $n$-HPDA$^=$ (resp. $n$-HPDA) $\H'$ such that
$\H$ and $\H'$ %generate the same graph.
induce isomorphic transition systems.
\end{theorem}

\iffalse
\begin{theorem}{\cite{Woeh05, carayol2006automates}}
If $\H$ is an $n$-HPDA  then there
exists an $n$-HPDA$^=$  $\H'$ such that
$\H$ and $\H'$ %generate the same graph.
induce isomorphic transition systems.
\end{theorem}
% Proposition 3.12

\begin{theorem}{\cite{Woeh05, carayol2006automates}}
If $\H$ is an $n$-HPDA$^=$ then there
exists an $n$-HPDA $\H'$ such that
$\H$ and $\H'$ %generate the same graph.
induce isomorphic transition systems.
\end{theorem}

% Proposition 3.18 If A is a higher-order pushdown system with equality pop
% of level n then there exists a higher-order pushdown system B of level n such
% that A and B generate the same graph.
\fi

From $n$-HPDA to $n$-HPDA$^=$ (Proposition 3.12 in \cite{Woeh05}) the proof relies on recreating a correct stack content to simulate higher-order 
$\text{\textit{pop}}$ operations by $\text{\textit{pop}}^=$, essentially ``guessing'' the correct stack content to be able to apply $\text{\textit{pop}}^=$. 
In the other direction (Proposition 3.18 in \cite{Woeh05}), 
the author
enrich the stack alphabet with new symbols stating which 
instructions have to be executed to recreate a
previous stack content. It is proven that such en encoding is possible since
there is 
for every stack $s$ of level $n$
% there is
 a unique shortest sequence of instructions which creates $s$ from
$\epsilon_n$. \\

Defined like {\sc $n$-HPDA parity game}, we write {\sc $n$-HPDA$^=$ parity game}  
in case the input is an $n$-HPDA$^=$ rather than an $n$-HPDA.



The algorithm from \cite{Cach03} actually provides an algorithmic solution to parity games on the graphs of the Caucal hierarchy. 
We skip a formal definition of a graph of level $n$ in the Caucal hierarchy, and refer the reader
to \cite{Cau02mfcs} for more details.
The $n$-$\EXP$ upper bound on $n$-HPDA parity games follows from the fact that every transition system induced by an $n$-HPDA $\H$ is a graph of the Caucal hierarchy \cite{Cach03, Woeh05},
whose vertices are almost in one-to-one correspondence with the configurations of $\H$.
In \cite{Woeh05} the converse direction is proven, i.e. that every graph of level $n$
of the Caucal hierarchy is generated by an $n$-HPDA.  
%
%
In \cite{carayol2006automates}
it is similarly shown that
every transition system induced by an $n$-HPDA$^=$ $\H$ is a graph of the Caucal hierarchy.
%
%
The algorithm from \cite{Cach03} hence lead to a solution for
$n$-HPDA$^=$ parity games as well.



\iffalse
\textcolor{red}{
Now then, because the class of $n$-HPDA$^=$ transitions systems
is the same as
the class of $n$-HPDA transitions systems, we have the following result.
}
\fi

\begin{theorem}\label{HPDA parity game}
{\sc $n$-HPDA$^=$ parity game} is in $n$-$\EXP$.
\end{theorem}





\iffalse
"To show that such an automaton exists we introduce the notion of a weak
popping higher-order pushdown automaton. A weak popping automaton is only
allowed to execute a pop instruction of level $j \geq 2$ if the two top level $j$ stacks
coincide. We skip a formal definition of a weak popping higher-order pushdown
automaton and just mention that even though this automaton model is equipped
with a built-in test on the equality of two stacks of the same level, it is equivalent
to the usual model. All proofs are given in the full version of this article [4]."

Pour pouvoir obtenir une notion pertinente d'ensemble rationnel de piles de
piles, il faut considérer un jeu d'opérations où la destruction inconditionnelle des
piles est remplacée par une version plus symétrique où la dernière pile de niveau ℓ
ne peut être détruite que si elle est égale à la précédente. Nous établissons dans le
paragraphe 4.1 l'équivalence (en temps qu'accepteurs de langages) les automates
à pile définis sur ces deux jeux d'opérations ( f. théorème 4.1.23). Nous montrons,
dans le paragraphe 4.6, le manque de pertinence de la notion de rationalité induite
par le jeu d'opérations classiques et par là même, nous établissons la nécessité de
considérer le jeu d'opérations symétriques.

L'opération de destruction symétrique a été introduite pour des raisons techniques dans 
[CW03] et a été utilisée de manière 
systématique dans [Car05] et
[Fra05]. 
Nous montrerons, dans le sous-paragraphe 4.1.5, que les automates à pile
de piles définis en utilisant les opérations destrk ou les opérations copyk acceptent
les mêmes langages.

Les automates à pile d'ordre supérieur ont été introduits dans les années 70.
Dans [Gre70℄, Greibach attribue l'idée de ces automates à Aho et Ullman. Ces
automates sont aussi définis dans [Mas76℄. Les automates sur COpsk diffèrent
légèrement des automates considérés par ces auteurs et de ceux onsidérés dans
[DG86, Eng91℄. La différence majeure réside dans la définition des piles de piles.
Les piles de niveau k sont des suites non vides de couples formés d'un symbole de
pile et d'une pile de niveau k − 1. Comme remarqué dans [KNU02℄, ces symboles
de pile supplémentaires peuvent être simulés dans le modèle des automates sur
COpsk .

Les langages a eptées par les automates à pile sur COpsk sont onnus sous
le nom de langages k -OI ou de langages indexés de niveau k . La première déno-
mination vient des travaux de Damm [Dam82]

Nous allons maintenant établir que les encodages des opérations simulent bien
les opérations sur les encodages des piles.

Lemme 4.1.21. Pour toute pile s ∈ Stacksk (Γ) et pour tout θ ∈ Opsk ,
[[ θ ]]k ([[ s ]]) =
{
[[ θ(s) ]]		si θ(s) est défini,	
non définie		sinon.


Proposition 4.1.22. Tout langage accepté par un automate à pile sur Opsk est
accepté par un automate à pile sur COpsk .

Démonstration. Soit A = (Γ,Σ,τ,Q,I,F,∆) un automate à pile sur Ops⋆k . Nous
définissons un automate B = (Γ,Σ,τ,Q,I,F,∆B ) où l'ensemble des transitions ∆B
est défini par:
∆B = {(p,x,η,q) | (p,x,θ,q) ∈ ∆ et η ∈ [[ θ ]]k }.
Par le lemme 4.1.21, il suit que pour tout w ∈ (Σ)∗ , p ∈ Q et s ∈ Stacksk (Γ):
w
(q0 , [ ]k ) −→ (p,s)
A
⇔
w
(q0 , [ ]k ) −→ (p,[[ s ]]).

Théorème 4.1.23. Les langages acceptés par les automates à pile sur Opsk et
sur COpsk coïncident.


\fi




\subsection{From pebble pushdown 
automata 
to higher-order pushdown 
automata
}



We use HPDA$^=$ parity games to simulate pebble pushdown automata
parity games.
A similar approach was used in \cite{carayol2006automates} to show that $(n + 2)$-level stack
automata could be used to simulate $n$-pebble alternating two-way word automata.


Let us start by discussing the simulation of the dropping and lifting of a pebble in the case of
a pebble pushdown automaton $\ppda$ with only one pebble.
As long as no pebble is dropped, a $2$-HPDA$^=$ $\H$ can simulate 
the behavior of a pebble pushdown automaton $\ppda$ with a $2$-stack containing a single $1$-stack
on which it performs the same operations $\ppda$ performs on its stack.
When the automaton $\ppda$ is in configuration $q(w)$ and drops a pebble, instead of making a pop or a push, we need
to %go to state $q'$ and word $w$ and 
	store the information that the pebble has been
dropped on $w$.
To simulate the next configuration in such a case, we use 
$\text{\textit{ push}}_2$
to store the information that the pebble has been dropped on the word $w$, leading to the new  $2$-stack
$\langle w \rangle \langle w \rangle$. 
Configurations afterwards have $2$-stacks of length two where the first component remains
$w$ until the pebble is lifted.
Lifting the pebble can be done only at the position the
pebble was dropped, by using $\text{\textit{pop}}^=_2$.
Checking that the node
corresponds to the one where the pebble has been dropped simply makes use of
the composition of $\text{\textit{pop}}^=_2$ and $\text{\textit{push}}_2$.
Checking the absence of the pebble can be done by challenging
the opponent to prove the presence of the pebble.

Now, let us detail the case when there is a second pebble, the construction
for each additional pebble being highly similar, where every additional pebble after
the first one would require the addition of another stack level.
Similarly as before, 
as long as no pebble is dropped, the stack of $\H$ 
is a $3$-stack containing a
single $2$-stack that contains a single $1$-stack.
To simulate configurations in the case one pebble have been dropped, the stack contains a single
 $2$-stack of the form $\langle w_1 \rangle \langle w \rangle$.
In the case the two pebbles have been dropped, 
the stack is of the form $ \langle \langle w_1 \rangle \langle w_2 \rangle \rangle
\langle \langle w_1 \rangle \langle w \rangle \rangle$. 



Intuitively speaking, $w_1$ is, as before, the position where the first pebble is
placed, and $w_2$ is where the second pebble is placed. Dropping pebbles like this
again involves the cloning operation, only, each pebble operates at a different 
stack level: thus, dropping the first pebble will use $\text{\textit{push}}_2$
and dropping the second pebble will use $\text{\textit{push}}_3$. Knowing the number of pebbles dropped can be done by 
keeping track using the states of $\H$.
Lifting or
checking for the presence 
(or absence) 
of the pebbles functions on a similar
basis as before. Checking the presence of the first pebble uses 
operation $\text{\textit{pop}}^=_2$ followed by $\text{\textit{push}}_2$,
while checking the presence of the second pebble uses 
operation $\text{\textit{pop}}^=_3$ followed by $\text{\textit{push}}_3$.
Lifting the first pebble uses $\text{\textit{pop}}^=_2$, but
check first that the second pebble has already been lifted.
Lifting the second pebble simply uses $\text{\textit{pop}}^=_3$.


By expanding this reasoning inductively, we conclude that given $n \in N$, and
given an $n$-pebble pushdown automaton 
$\ppda = (Q, \Gamma, R, q_{init}, \gamma_{init}, F)$, where $Q= Q_0 \biguplus Q_1$,
one can compute an $(n+1)$-HPDA 
$\H = (Q', \Gamma, R', q_{init}', \gamma_{init}', F')$  where $Q'= Q_0' \biguplus Q_1'$, 
such 
that 
player $0$ has a winning strategy from $q'_{init}(\text{\textit{push}}_1^{\gamma_{init}'}(\epsilon_{n}))$ in $A_{(\H,Q'_0,Q'_1)}$ 
if
and only if
player $0$ has a winning strategy from $q(\gamma_{init},\mu_{init})$ in $A_{(\ppda,Q_0,Q_1)}$. 
Hence the following reduction.

\begin{theorem} 
{\sc $n$-pebble pushdown parity game} is polynomial time reducible
to {\sc $(n+1)$-HPDA$^=$ parity game}.
\end{theorem}

The reduction implies decidability of pebble pushdown automata parity game. Furthermore,
by Theorem~\ref{HPDA parity game}, it implies the following complexity result.


\begin{theorem}
{\sc $n$-pebble pushdown automata parity game} is in $(n+1)$-$\EXP$.
\end{theorem}

Finally, the following complexity result is due the above and Theorem~\ref{PPDA reduction}.

\begin{corollary}
{\sc $n$-parametric pushdown automata parity game} is in $(n+1)$-$\EXP$.
\end{corollary}









































\end{document}

















