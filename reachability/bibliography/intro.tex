Counter automata, which comprise a finite-state controller together
with a number of counter variables, are a fundamental and widely-studied
computational model. One of the earliest results about counter
automata, which appeared in a seminal paper of Minsky's five decades
ago, is the fact that two counters suffice to achieve Turing
completeness~\cite{Min61}.

Following Minsky's work, much research has been directed towards
studying restricted classes of counter automata and related
formalisms. Among others, we note the use of restrictions to a single
counter, on the kinds of allowable tests on the counters, on the
underlying topology of the finite controller (such as
flatness~\cite{CJ98-cav,LS08-atva}), and on the types of computations
considered (such as reversal-boundedness~\cite{ID06-tcs}). Counter
automata are also closely related to Petri nets and pushdown automata.

In Minsky's original formulation, counters were represented as integer
variables that could be incremented, decremented, or tested for
equality with zero by the finite-state controller. More recently,
driven by complexity-theoretic considerations on the one hand, and
potential applications on the other, researchers have investigated
additional primitive operations on counters, such as additive updates
encoded in binary~\cite{BHIMV06,LS08-atva} or even
in \emph{parametric} form, i.e., whose precise values depend on
parameters~\cite{BIL06-icalp,IJTW93-icalp}. We refer to such counter automata as
\emph{succinct} and \emph{parametric} resp., the former being
viewed as a subclass of the latter. Natural applications of such
counter machines include the modelling of resource-bounded processes,
programs with lists, recursive or multi-threaded programs, and XML
query evaluation; see, e.g.,
\cite{BHIMV06,CR04,ID06-tcs}.

In most cases, investigations have centered around the decidability
and complexity of the \emph{reachability} problem, i.e., whether a
given control state can be reached starting from the initial
configuration of the counter automaton. Various instances of the
reachability problem for succinct and parametric counter automata are
examined, for example, in~\cite{DG-jlc09,HKOW10,IJTW93-icalp}.

The aim of the present paper is to study the decidability and
complexity of \emph{model checking} for succinct and parametric
one-counter automata. In view of Minsky's result, we restrict our
attention to \emph{succinct one-counter automata (SOCA)} and
\emph{parametric one-counter automata (POCA)}. On the specification
side, we focus on the three most prominent formalisms in the
literature, namely the temporal logics CTL and LTL, as well as the
modal $\mu$-calculus. For a counter automaton $\mathbb{A}$ and a
specification $\varphi$, we therefore consider the question of
deciding whether $\mathbb{A} \models \varphi$,
in case of POCA for all values of the parameters,
%\footnote{In the case of
%parametric automata, one asks whether the automaton satisfies the
%formula \emph{for all} legal values of the parameters.}
and investigate both the \emph{data} complexity (in which the formula
$\varphi$ is fixed) as well as the \emph{combined} complexity of this
problem. Our main results are summarized in Table \ref{tab:results}.
\begin{table}[t]
\begin{center}
\begin{tabular}{ >{\centering\arraybackslash}m{3cm}
  >{\centering\arraybackslash}m{2cm} |
  >{\centering\arraybackslash}m{3.3cm} |
> {\centering\arraybackslash}m{3.3cm} |}\\
  &  &{\bf SOCA}&{\bf POCA}\\ \hline
{\raisebox{-2ex}[0ex]{{\bf CTL, $\mu$-calculus}}}&   {\bf data} &   & \\ \cline{2-2}
{\raisebox{-2ex}[-5ex]{{$\phantom{\text{CTL}}$}}}&  {\bf combined}&
{\raisebox{2ex}[0ex]{$\EXPSPACE$-complete}} &
{\raisebox{2ex}[0ex]{ $\Pi^0_1$-complete}}\\ \cline{1-4}
{\raisebox{-2ex}[0ex]{{\bf LTL}}}&  {\bf data} &
\multicolumn{2}{c|} {\raisebox{0ex}[0ex]{$\coNP$-complete}}
\\ \cline{2-4}
{\raisebox{-2ex}[-5ex]{{$\phantom{\text{CTL}}$}}}&  {\bf combined}&
$\PSPACE$-complete &
$\coNEXP$-complete\\ \cline{1-4}
\end{tabular}
\end{center}
\label{tab:results}
\caption{The complexity of CTL, the modal $\mu$-calculus, and LTL
             on SOCA and POCA.}
\end{table}

One of the motivations for our work was the recent discovery that
reachability is decidable and in fact $\NP$-complete for both SOCA and
POCA~\cite{HKOW10}. We were also influenced by the work of Demri and
Gascon on model checking extensions of LTL over non-succinct,
non-parametric one-counter automata~\cite{DG-jlc09}, as well as the
recent result of G\"oller and Lohrey establishing that model checking
CTL on such counter automata is $\PSPACE$-complete \cite{GoLo10}.

%% We note some interesting differences between our results and
%% corresponding questions regarding finite automata. For the latter, the
%% (combined) model checking problems for CTL, the $\mu$-calculus, and
%% LTL are respectively known to be $\mathsf{P}$-complete, in
%% $\mathsf{NP} \cap \coNP$, and $\PSPACE$-complete. Somewhat
%% surprisingly, for SOCA and POCA, the complexity ordering is reversed
%% and LTL becomes easier to model check than either CTL or the
%% $\mu$-calculus.

On a technical level, the most intricate result is the
$\EXPSPACE$-hardness of CTL model checking for SOCA, which requires
several steps. We first show that $\mathsf{EXPSPACE}$ is `exponentially
$\mathsf{LOGSPACE}$-serializable', adapting the known proof that
$\PSPACE$ is $\mathsf{LOGSPACE}$-serializable. Unfortunately, and in
contrast to~\cite{GoLo10}, this does not immediately provide
an $\EXPSPACE$ lower bound. In a subsequent delicate stage of the
proof, we show how to partition the counter in order simultaneously to
perform $\PSPACE$ computations in the counter and manipulate numbers
of exponential size in a SOCA of polynomial size.

%For reasons of space, we have had to abbreviate or omit a number of
%proofs; full details can however be found in the technical
%report~\cite{techrep}.
