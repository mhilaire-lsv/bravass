\documentclass{llncs}
\usepackage{graphicx}
\usepackage{graphics}
\usepackage{epsfig}
\usepackage{latexsym}
\usepackage{amssymb}
\usepackage{amsmath}
\usepackage{stmaryrd}
\usepackage{ifthen}


%% \usepackage[thmmarks]{ntheorem}
%% %%% Theorem environments
%% \newtheorem{lemma}{Lemma}
%% \newtheorem{theorem}{Theorem}
%% \newtheorem{proposition}{Proposition}
%% \newtheorem{corollary}{Corollary}

%% \theoremstyle{definition}
%% \theoremsymbol{\ensuremath{\lozenge}}
%% \theorembodyfont{\normalfont}
%% \newtheorem{definition}{Definition}

%% \theoremstyle{nonumverplain}
%% \theoremheaderfont{\normalfont\bf}
%% \theorembodyfont{\normalfont}
%% \theoremseparator{.}
%% \theoremsymbol{\ensuremath{_\blacksquare}}
%% \newtheorem{example}{Example}

%% \theoremstyle{nonumberplain}
%% \theoremheaderfont{\normalfont\itshape}
%% \theorembodyfont{\normalfont}
%% \theoremseparator{.}
%% \theoremsymbol{\ensuremath{_\Box}}
%% \newtheorem{proof}{Proof}

% Symbols
\newcommand{\naturals}{\ensuremath{\mathbf{N}}}
\newcommand{\integers}{\ensuremath{\mathbf{Z}}}

% Machines and their ingredients
\newcommand{\automaton}{\ensuremath{\mathcal{A}}}
\newcommand{\clocations}{\ensuremath{Q}}
\newcommand{\iclocation}{\ensuremath{q_{\mathit{in}}}}
\newcommand{\fclocation}{\ensuremath{F}}
\newcommand{\transitionrel}{\ensuremath{\Delta}}
\newcommand{\transitionlabel}{\ensuremath{\lambda}}
\newcommand{\clocationlabel}{\ensuremath{\xi}}
\newcommand{\operations}{\ensuremath{Op}}
\newcommand{\operation}{\ensuremath{op}}
\newcommand{\add}[1]{\ensuremath{\mathit{add}(#1)}}
\newcommand{\zero}{\ensuremath{\mathit{zero}}}
\newcommand{\propositions}{\ensuremath{L}}
\newcommand{\configuration}{\ensuremath{c}}
\newcommand{\configurations}{\ensuremath{\mathcal{C}}}
\newcommand{\run}{\ensuremath{\pi}}
\newcommand{\trace}{\ensuremath{\tau}}
\newcommand{\pocatuple}{\ensuremath{\langle \clocations,
\iclocation, \fclocation, \transitionrel, \transitionlabel,
\clocationlabel \rangle}}

\newcommand{\circuit}{\ensuremath{\mathcal{C}}}

\newcommand{\tm}{\ensuremath{\mathcal{M}}}

%%% Names
\newcommand{\oca}{\textsc{OCA} }
\newcommand{\poca}{\textsc{POCA} }

%%% Problems
\newcommand{\stsat}{\textsc{Succinct 3-SAT} }
\newcommand{\tsat}{\textsc{3-SAT} }
\newcommand{\ltlpoca}{\textsc{LTL-POCA-MC} }

%%% Complexity classes
\newcommand{\nexptime}{\textsc{NExpTime} }
\newcommand{\conexptime}{\textsc{co-NExpTime} }
\newcommand{\pspace}{\textsc{PSpace} }

%%% LTL
\newcommand{\ltl}{\textsc{LTL} }
\newcommand{\until}{\ensuremath{\mathcal{~U~}}}
\newcommand{\propalphabet}{\ensuremath{\Sigma}}
\newcommand{\pastarred}{\ensuremath{(\mathcal{P}(\propalphabet))^{\omega}}}

%%% Logic
\newcommand{\true}{\ensuremath{\mathit{true}}}
\newcommand{\false}{\ensuremath{\mathit{false}}}

%%% Control commands
\newcommand{\defemph}[1]{\emph{#1}}

\begin{document}
\section{LTL Model Checking}
\begin{figure}
  \begin{center}
    \includegraphics{graphics/divisibility-1.pdf}
  \end{center}
  \begin{center}
    \includegraphics{graphics/divisibility-2.pdf}
  \end{center}
  \caption{Gadgets for testing the counter for divisibility and
    non-divisibility with $p$.}
  \label{fig:divisibility-gadgets}
\end{figure}
\begin{figure}
  \begin{center}
    \includegraphics{graphics/divisibility-0.pdf}
  \end{center}
  \caption{Gadget for adding a number between $0$ and $-2^{k+1}+1$ to
    the counter.}
  \label{fig:number-diamond}
\end{figure}

Let $\propalphabet$ be a set of \defemph{propsions}. In the following,
capital arabic letters such as $A$ will identify elements from
$\propalphabet$.  The set of \ltl formula is inductively defined
according to the following grammar:
\begin{equation*}
  \varphi ::= A \mid \neg \varphi \mid \varphi_1 \wedge \varphi_2 \mid
  \bigcirc \varphi \mid \varphi_1 \until \varphi_2
\end{equation*}
We define the Boolean abbreviations in a standard way, i.e.,
$\varphi_1 \vee \varphi_2 := \neg (\neg \varphi_1 \wedge \neg
\varphi_2)$, $\varphi_1\rightarrow \varphi_2:=\neg \varphi_1 \vee
\varphi_2$ and moreover $\top:=A\vee \neg A$ for some $A\in
\propalphabet$. The temporal abbreviations $\Diamond$
(\defemph{eventually}) and $\Box$ (\defemph{globally}) are defined as
$\Diamond \varphi:= \top \until \varphi$ and $\Box \varphi := \neg
\Diamond \neg \varphi$.  Let $\trace \in \pastarred$ be a trace, the
containment of $\trace$ in the trace language of $\varphi$ ($\trace
\models \varphi$) is defined by induction on the structure of
$\varphi$:
\begin{eqnarray*}
  \trace \models  A &  \iff & A\in\trace(0)\\
  \trace \models  \neg \varphi &\iff & \trace \not \models \varphi\\
  \trace \models \varphi_1 \wedge \varphi_2 &\iff& \trace
  \models \varphi_1 \text{ and } \trace \models \varphi_2\\
  \trace \models \bigcirc \varphi & \iff & \tau^1 \models \varphi\\
  \trace \models  \varphi_1 \until \varphi_2 & \iff &
  \text{there is } j\in\naturals \text{ s.t. } \tau^j\models \varphi_2
  \text{ and for all } i<j: \tau^j\models \varphi_1
\end{eqnarray*}
Thus, the language of a formula $\varphi$ is $L(\varphi):=\{ \tau \in
\pastarred : \tau\models \varphi\}$.  Given an \oca $\automaton$, we
write $\automaton\models \varphi$ iff $L(\automaton)\subseteq
L(\varphi)$. We can now define the LTL model checking problem for
parametric one-counter automata.
\begin{definition}[$\ltlpoca$]
  Let $\automaton$ be a $\poca$ and $\varphi$ an LTL formula. We write
  $\automaton \models \varphi$ iff for all assignments $\sigma:
  \rho(\automaton)\to \integers$ we have $\automaton(\sigma) \models
  \varphi$.  The \defemph{LTL model checking of parametric one-counter
    automata ($\ltlpoca$)} is to decide $\mathcal{A}\models \varphi$.
\end{definition}
A general automata theoretic approach to check a system given by a
finite state automaton $\automaton$ against an LTL specification
$\varphi$ was introduced by Vardi and Wolper in
\cite{VW86-lics}. Their idea is to translate the negation of
$\varphi$, which captures all bad behaviours, into a B\"uchi automaton
$\automaton_{\neg \varphi}$ and then to check for language emptiness
of the product automaton $\mathcal{B}=\automaton\times
\automaton_{\neg \varphi}$. Moreover, they shown that the size of the
automaton $\automaton_{\neg \varphi}$ is in
$\mathcal{O}(2^{|\varphi|})$, hence $|\mathcal{B}|\in \mathcal{O}
(|\automaton|\times 2^{|\varphi|})$.

We can employ this technique in a straight-forward way in our
setting. Given a \poca $\automaton$ and an LTL formula $\varphi$, in
order to check that $\automaton \not \models \varphi$, we need to find
an assignment $\sigma$ to $\rho(\automaton)$ such that
$\automaton(\sigma)\not\models \varphi$. Consequently, it is
sufficient to check whether $L(\automaton \times \automaton_{\neg
\varphi}) \neq \emptyset$, for which we have established the
complexity in the previous section.
\begin{proposition}
  \ltlpoca is in $\conexptime$.
\end{proposition}
Subsequently, we show that this upper bound cannot be improved.
Our prototype problem for showing $\conexptime$-hardness is $\stsat$
\cite{P94-book}. As in classical $\tsat$, the problem posed by an
instance of $\stsat$ is to find a satisfying assigment of a
propositional formula $\chi$ in 3-DNF. The difference however is that
$\chi$ is represented by a Boolean circuit. This representation allows
for an exponentially succinct representation of $\chi$, which renders
\stsat $\nexptime$-complete \cite{P94-book}.
\begin{definition}[\stsat]
  An instance of \defemph{$\stsat$} is a Boolean circuit $\circuit$
  with input $y;\ell$, where $y\in \{0, \ldots, 2^m-1\}$ is a number
  encoded in binary that identifies a clause for some arbitrary but
  fixed $m\in\naturals$, and $1\le \ell\le 3$ accesses the respective
  literal of the clause. The output of $\circuit$ is $z;b$, where
  $1\le z\le t$ is a number encoded in binary identifying a variable
  $X_z$ for some fixed $1\le t\le 3\cdot 2^m$, and $b\in\{0,1\}$
  indicates whether or not $X_z$ is negated. The propositional formula
  encoded by $\circuit$ is
  \begin{equation*}
    \chi(X_1,\ldots,X_t) = \bigwedge_{0\le y<2^m}\left(
    \bigvee_{\circuit(y,\ell)=z;0\atop 1\le \ell\le 3} \neg X_z
    \vee \bigvee_{C(y,\ell)=z;1\atop 1\le \ell \le 3} X_z \right).
  \end{equation*}
  \stsat is to decide whether there exists a valuation $v: \{X_1,
  \ldots X_{t}\}\to \{\true, \false\}$ such that $\chi$ evaluates to
  true under $v$.
\end{definition}
In order to establish $\conexptime$-hardness of $\ltlpoca$, given an
instance $\circuit$ of \stsat we construct a \poca $\automaton$ with
only one parameter and an LTL formula $\varphi$ such that the formula
$\chi$ induced by $\circuit$ is satisfiable iff $\automaton
\not\models \neg\varphi$. There are two key ideas for the reduction:
First, we choose a suitable encoding of arbirtray valuations in the
naturals so that we can use the parameter to guess and store
them. Second, we employ the expressive power of LTL in order to
perform arbitrary computations of \pspace Turing machines on traces of
runs of $\automaton$.

Regarding the encoding, let $X_1,\ldots, X_t$ be the propositional
variables ocurring in $\chi$, $p_1, \ldots, p_t$ the first $t$ prime
numbers and $a\in \naturals$. We define a valuation $v:\{X_1,\ldots,
X_t\}\to \{\true,\false\}$ such that $a\equiv 0 \mod p_i$ iff
$v(X_i)=\false, 1\le i\le t$. Thus, any number $a$ corresponds to a
valuation. Conversely, the Chinese Remainder Theorem guarantees that
for any valuation a suitable $a$ exists. Note that the Prime Number
Theorem ensures that the binary representation of the $p_i$ is
polynomial in $t$.

\begin{figure}
  \begin{center}
    \includegraphics[scale=0.9]{graphics/structure-0.pdf}
  \end{center}
  \caption{High-level description of the automaton $\automaton$ used for
  the reduction.}
  \label{fig:high-level}
\end{figure}

Let us now take a high-level look at the $\poca$ $\automaton$ that is
used for the reduction and sketched in Figure \ref{fig:high-level}.
$\automaton$ uses one parameter $a$ and employs several gadgets
(depicted as rectangles) that communicate via designated propsitional
variables. First, $\automaton$ loads the value of the parameter $a$
into the counter. It then passes through a gadget that initially
chooses a number $c$ identifying a clause of $\chi$ encoded as a
sequence of propositions. This gadget is constructed such that it
outputs the increment of $c$ modulo $2^m$ each time it is passed
thereafter. Now $\automaton$ branches non-deterministically into a
gadget $\automaton_\circuit$ in order to choose one literal of the
respective clause, i.e., evaluates the circuit $\circuit$ for input
$c;1, c;2$, respectively $c;3$. The identifier $z$ of the variable
$X_z$ of the respective literal, again encoded as a sequence of
propositions, is then used as input to a gadget $\automaton_p$ which
computes the $z$-th prime number $p_z$. Having computed $p_z$, if
$b=0$, i.e., $X_z$ is negative in $c$, it is checked that
$p_z|a$. Symmetrically, if $X_z$ is positive it is checked that
$p_z\not|a$. After the checks have been finished, it is ensured that
the counter has value $a$ and the process continues with clause $c+1
\mod 2^m$. Given an LTL formula $\varphi$ that ensures that all
gadgets work and communicate properly, it is clear that $\chi$ is
satisfiable iff $\automaton \not \models \neg \varphi$.

So it remains to show how the gadgets can be realized. Observe that
the computations performed by the gadgets $\automaton_\circuit$ and
$\automaton_p$ can be realized by $\pspace$ Turing machines:
evaluating a Boolean circuit clearly is in $\pspace$, as well is
computing the $i$-th prime number for $1\le i\le t$. Therefore, we
subsequently show how a generic $\pspace$ Turing machine can be
encoded as an \oca together with a family of LTL formulas that ensure
proper behaviour and communication.

\begin{figure}
\begin{center}
\includegraphics{graphics/tm-oca-0.pdf}
\end{center}
\caption{The \poca $\automaton_\tm$ used to generically perform
  computations of arbitrary \pspace Turing machines.}
\label{fig:tm}
\end{figure}
% assume TM goes into infinite loop if it does not accept
Let $\tm=\langle S, \{0, 1\}, s_0, \delta, F \rangle$ be a
deterministic Turing machine which does not use more than $n$ tape
cells during a computation, where $n$ is polynomial in
$|\tm|$. Following standard methods for showing $\pspace$-hardness of
LTL satisfiability, in order to encode computations of $\tm$ we
introduce propositions $T_{i,b}^\epsilon$ and $T^s_{i,b}$, $1\le i\le
n, b\in\{0,1\}, s\in S$. When simulating a computation,
$T_{i,b}^\epsilon$ indicates that the content of the $i$-th tape cell
is $b$ and $T_{i,b}^s$ additionally signals that the head currently
reads the $i$-th tape cell in state $s\in S$.  For technical
convenience, in the following $T_{n+1,b}^s$ abbreviates $\top$ for any
$s\in S\cup \{\epsilon\}, b\in\{0,1\}$.

Consider now the \poca $\automaton_\tm$ shown in Figure
\ref{fig:tm}. For each $1\le i\le n$ it contains a column of locations
such that each $T_{i,b}^\epsilon$ and $T_{i,b}^s$ occurs as a label of
exactly one location.  When simulating a computation of $\tm$, as long
as we have not reached an accepting state, we loop through the upper
component of $\automaton_\tm$. Once the simulated computation reaches
an accepting state, the final tape content is encoded into a trace of
a run in the lower component of $\automaton_\tm$, i.e., we pass
through a location labelled with $A_{i,b}$ iff the $i$-th tape cell
contains $b$ when the computation has finished for $1\le i\le n, b\in
\{0,1\}$.  The propositional variable $\$$ is used as a special marker
the allows for detecting when the upper component of $\automaton_\tm$
is traversed for the first time. We now give LTL formulas that
restrict the traces of runs of $\automaton_\tm$ to correspond to
computations of $\tm$.

At first, we need to make sure that the path we initially take in the
upper component corresponds to an initial configuration of $\tm$. As
previously said, we want to be able to use the output of one Turing
machine as the input to another.  Assume the output is encoded by
propositions $A_{1,b_1}',\ldots, A_{n,b_n}'$, the following formula
ensures that they are correctly reused as input to $\automaton_\tm$:
\begin{equation*}
  \Box \left( \bigwedge_{b\in \{0,1\}}(A_{1,b}' \rightarrow (( \neg
  T_{1,\bar{b}}^{s_i}\wedge \bigwedge_{s\in (S\setminus\{s_i\}\cup
    \{\epsilon\},\atop b'\in\{0,1\}} \neg T_{1,b'}^s) \until \$)
  \wedge \bigwedge_{2\le i\le n,\atop b\in\{0,1\}} A'_{i,b}
  \rightarrow ((\neg T_{i,\bar{b}}^\epsilon \wedge\bigwedge_{s\in
    S,\atop b'\in \{0,1\}} T_{i,b'}^s)\until \$)) \right)
\end{equation*}
%% by propositions Suppose the
%% initial tape content is given by $b_1,\ldots, b_n \in \{0,1\}$, this
%% can be achieved by the following formula:
%% \begin{equation}
%%   \left(\bigwedge_{s\in (S\setminus \{s_i\} \cup \{\epsilon\})\atop b\in\{0,1\}}
%%   \neg T_{1,b}^s\wedge \neg T_{1,\bar{b_1}}^{s_i}
%%   \wedge
%%   \bigwedge_{2\le i\le n} \bigwedge_{s\in S\atop
%%     b\in\{0,1\}}\neg T_{i,b}^s \wedge \neg
%%   T_{i,\bar{b_i}}^\epsilon\right)
%%   \until \$
%% \end{equation}
The reason for writing the above formula in this form is that it
allows us to transfer inputs from outputs of Turing machines even if
they are not directly connected to each other.

%todo: Add reason why we write formula like that
We can now give LTL formulas that constraint runs of $\automaton_\tm$
such that the propositions seen in the corresponding trace
successively encode valid configurations of $\tm$ with respect to
$\delta$.  First, whenever we see three consecutive propositions
indicating that the head is currently not reading any tape cell
corresponding to them, we require that the content of the second does
not change in the next encoded configuration. A special case is the
first tape cell, which needs to be treated seperately. The formula
enforcing this constraint is
\begin{equation*}
   \Box\Bigg(
     \bigwedge_{b_1,b_2\in \{0,1\}} \Big(
  (T_{1,b_1}^\epsilon \wedge \bigcirc^2 T_{2,b_2}^\epsilon) \rightarrow
  \bigcirc^{2n+1} (T_{1,b_1}^\epsilon \vee A_{1,b_1})
  \Big)\wedge
\end{equation*}
\begin{equation*}
  \wedge\bigwedge_{1\le i\le n-1\atop b_0,b_1,b_2\in \{0,1\}}
  \Big( (T_{i,b_0}^\epsilon \wedge \bigcirc^{2}T_{i+1,b_1}^\epsilon \wedge
  \bigcirc^{4}T_{i+2,b_2}^\epsilon) \rightarrow
  \bigcirc^{2(n+1)+1}(T_{i+1,b_1}^\epsilon\vee A_{i+1,b_1}) \Big)\Bigg)
\end{equation*}
%% \begin{equation}
%% \rightarrow
%%   (\bigcirc^{2n}(T_{i,b_0}\vee A_{i,b_0}) \wedge
%%    \wedge
%%   \bigcirc^{2(n+2)}(T_{i+2,b_2}\vee A_{i+2,b_2})\Big)\Bigg)
%%   \Box(\bigwedge_{1\le i\le n-1\atop b_0,b_1,b_2\in \{0,1\}}
%%   ( (T_{i,b_0} \wedge \bigcirc^{2}T_{i+1,b_1} \wedge
%%   \bigcirc^{4}T_{i+2,b_2}) \rightarrow\\ \rightarrow
%%   (\bigcirc^{2n}(T_{i,b_0}\vee A_{i,b_0}) \wedge
%%   \bigcirc^{2(n+1)}(T_{i+1,b_1}\vee A_{i+1,b_1}) \wedge
%%   \bigcirc^{2(n+2)}(T_{i+2,b_2}\vee A_{i+2,b_2})) \wedge\\
%% \end{equation}
Second, whenever we see a proposition indicating that in the
corresponding configuration the head is reading the respective
tape cell, we need to ensure that all the neighbour tape
cells are updated according to $\delta$:
\begin{equation*}
  \Box\Bigg(\bigwedge_{1\le i\le n-1\atop b_0,b_1,b_2\in \{0,1\}}
  \bigwedge_{s\in S} \Big( (T_{i,b_0}^\epsilon \wedge
  \bigcirc^{2}T_{i+1,b_1}^s \wedge \bigcirc^{4}T_{i+2,b_2}^\epsilon)
  \rightarrow
\end{equation*}
\begin{equation*}
  \rightarrow \left\{
    \begin{array}{ll}
      \bigcirc^{2n+1} T_{i,b_0}^{s'} \wedge \bigcirc^{2(n+1)+1}
      T_{i+1,b}^\epsilon \wedge \bigcirc^{2(n+2)+1} T_{i+2,b_2}^\epsilon))& \text{ if }
      \delta(s,b_1)=(s',b,-1), s'\not\in F\\
      \bigcirc^{2n+1} T_{i,b_1}^\epsilon \wedge \bigcirc^{2(n+1)+1}
      T_{i+1,b}^\epsilon \wedge \bigcirc^{2(n+2)+1} T_{i+2,b_2}^{s'}))& \text{ if }
      \delta(s,b_1)=(s',b,+1),s'\not\in F\\
      \bigcirc^{2n+1} A_{i,b_1} \wedge \bigcirc^{2(n+1)+1}
      A_{i+1,b} \wedge \bigcirc^{2(n+2)+1} A_{i+2,b_2}))& \text{ if }
      \delta(s,b_1)=(s',b,d), s'\in F
    \end{array}
    \right.
\end{equation*}
%% \begin{equation*}
%%     \wedge
%%     (T_{0,b_0}^q \wedge \bigcirc^{2} T_{1,b_1}) \rightarrow
%%     \left \{
%%     \begin{array}{ll}
%%       \bigcirc^{2n} T_{0,b} \wedge \bigcirc^{2(n+1)} T_{1,b_1}))
%%       & \text{ if } \delta(q,b_1)=(q',b,R) \text{ and } q'\neq q_f\\
%%       \bigcirc^{2n} A_{0,b} \wedge \bigcirc^{2(n+1)} A_{1,b_1}))
%%       & \text{ if } \delta(q,b_1)=(q',b,R) \text{ and } q'= q_f\\
%%     \end{array}
%%     \right.
%% \end{equation*}
Here, $A_{n+1,b}$ also abbreviates $\top$ for $b\in\{0,1\}$. In order
to improve readability, we have omitted the special case when the head
reads the first tape cell. The above formula can however
straight-forwardly be adopted to take this into account.

Coming back to Figure \ref{fig:high-level}, it now is easy to see how
we can implement a gadget that chooses a clause $c$ and subsequently
increments $c$ modulo $2^m$ each time it is passed. It is just a
$\pspace$ Turing machine that uses its own output as input and
increments it modulo $2^m$. A valid input for the first time it is
used can be provided by a suitable LTL formula.

We contiune with showing how the counter value can be tested for
{(non\-)} divisibility with a prime number that has previously been
computed. Suppose the prime number is encoded in propositions
$A_{1,b_1},\ldots,A_{k+1,b_{k+1}}$. As previously mentioned, Figure
\ref{fig:divisibility-gadgets} provides gadgets that can test the
counter for (non-)divisibility with some parametric value $p$.
Observe that both gadgets ensure that the counter value is $0$ when
they are left. We can obtain a generic version of those gadgets by
replacing every edge that is labelled with $\mathit{add}(-p)$ with the
gadget from Figure \ref{fig:number-diamond}. We just have constraint
the runs such that each time we pass through the gadget from Figure
\ref{fig:number-diamond} we take a path whose sum adds up to the
number encoded by the $A_{i,b_i}$. This is ensured by the following
formula:
\begin{equation*}
  \Box \left( \bigwedge_{1\le i\le k+1,\atop b\in\{0,1\}} A_{i,b}
  \rightarrow (\neg N_{i-1,\bar{b}}\until \$) \right)
\end{equation*}

Finally, it is not difficult to see that the techniques we used in the
previous paragraphs can also be used to correctly handle the branching
on $b$ performed in Figure \ref{fig:high-level}.

In summary, we can take the disjoint union of all the gadgets from
Figure \ref{fig:high-level} and their appendant LTL formula that we
described in this section, connect the gadgets correctly and take the
conjuction of the LTL formulas in order to obtain an automaton
$\automaton$ and a formula $\varphi$ such that $\automaton \not
\models \neg \varphi$ iff $\chi$ is satisfiable.

\begin{theorem}
  \ltlpoca is $\conexptime$-complete.
\end{theorem}

\bibliographystyle{plain}
\bibliography{bibliography}

\end{document}
