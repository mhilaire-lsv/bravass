\documentclass[times,envcountsame]{llncs}
\usepackage{a4wide}
\usepackage{times}
\usepackage{amsmath,amssymb,eufrak,theorem}%mnsymbol}
\usepackage{enumerate}
\usepackage{complexity}
\usepackage{gastex}

\usepackage{xspace,tabularx,multicol}


%\usepackage[thmmarks,standard,thref]{ntheorem}



\iffalse

\newtheorem{theorem}{{\bf Theorem}}[section]
\newtheorem{corollary}[theorem]{{\bf Corollary}}
\newtheorem{definition}[theorem]{{\bf Definition}}
\newtheorem{example}[theorem]{{\bf Example}}
\newtheorem{lemma}[theorem]{{\bf Lemma}}
\newtheorem{proposition}[theorem]{{\bf Proposition}}
\newtheorem{remark}[theorem]{{\bf Remark}}
\newtheorem{fact}[theorem]{Fact}

\fi

\newtheorem{fact}[theorem]{Fact}

\usepackage{stmaryrd,rotating,mathrsfs}


%\usepackage[active]{srcltx}

\iffalse
\newcommand{\problemx}[3]{
\par\noindent\underline{\sc#1}\par\nobreak\vskip.2\baselineskip
\begingroup\clubpenalty10000\widowpenalty10000
\setbox0\hbox{\bf INPUT: }\setbox1\hbox{\bf QUESTION: }
\dimen0=\wd0\ifnum\wd1>\dimen0\dimen0=\wd1\fi
\vskip-\parskip\noindent
\hbox to\dimen0{\box0\hfil}\hangindent\dimen0\hangafter1\ignorespaces#2\par
\vskip-\parskip\noindent
\hbox to\dimen0{\box1\hfil}\hangindent\dimen0\hangafter1\ignorespaces#3\par
\endgroup}

\newcommand{\problemy}[3]{
\par\noindent\underline{\sc#1}\par\nobreak\vskip.2\baselineskip
\begingroup\clubpenalty10000\widowpenalty10000
\setbox0\hbox{\bf INPUT: }\setbox1\hbox{\bf OUTPUT: }
\dimen0=\wd0\ifnum\wd1>\dimen0\dimen0=\wd1\fi
\vskip-\parskip\noindent
\hbox to\dimen0{\box0\hfil}\hangindent\dimen0\hangafter1\ignorespaces#2\par
\vskip-\parskip\noindent
\hbox to\dimen0{\box1\hfil}\hangindent\dimen0\hangafter1\ignorespaces#3\par
\endgroup}

\fi


\newcommand{\problemx}[3]{
\par\noindent\underline{\sc#1}\par\nobreak\vskip.2\baselineskip
\begingroup\clubpenalty10000\widowpenalty10000
\setbox0\hbox{\bf INPUT:\ }\setbox1\hbox{\bf QUESTION:\ }
\dimen0=\wd0\ifnum\wd1>\dimen0\dimen0=\wd1\fi
\vskip-\parskip\noindent
\hbox to\dimen0{\box0\hfil}\hangindent\dimen0\hangafter1\ignorespaces#2\par
\vskip-\parskip\noindent
\hbox to\dimen0{\box1\hfil}\hangindent\dimen0\hangafter1\ignorespaces#3\par
\endgroup}


\newcommand{\problemy}[3]{
\par\noindent\underline{\sc#1}\par\nobreak\vskip.2\baselineskip
\begingroup\clubpenalty10000\widowpenalty10000
\setbox0\hbox{\bf INPUT:\ }\setbox1\hbox{\bf OUTPUT:\ }
\dimen0=\wd0\ifnum\wd1>\dimen0\dimen0=\wd1\fi
\vskip-\parskip\noindent
\hbox to\dimen0{\box0\hfil}\hangindent\dimen0\hangafter1\ignorespaces#2\par
\vskip-\parskip\noindent
\hbox to\dimen0{\box1\hfil}\hangindent\dimen0\hangafter1\ignorespaces#3\par
\endgroup}





\renewcommand{\mod}{\text{mod }}
\newcommand{\modulo}{\mod}


%\renewcommand{\vec}{}

\def\AC{{\mathrm{AC}}}
\def\AG{{\mathsf{AG}}}
\def\AND{{\mathrm{AND}}}
\def\BIN{{\mathrm{BIN}}}
\def\CRR{{\mathrm{CRR}}}
\def\CTL{{\text{CTL}}}
\def\div{{\mathrm{div}}}
\def\dom{{\mathrm{dom}}}
\def\EF{{\mathsf{EF}}}
\def\AX{{\mathsf{AX}}}
\def\EXPTIME{{\mathsf{EXPTIME}}}
\def\inp{{\mathrm{in}}}
\def\leaf{{\mathrm{leaf}}}
\def\MAJ{{\mathrm{MAJ}}}
\def\NC{{\mathrm{NC}}}
\def\Omc{{\mathbb{O}}}
\def\OR{{\mathrm{OR}}}
\def\out{{\mathrm{out}}}
\def\Pmc{{\mathcal{P}}}
\def\PSPACE{{\mathsf{PSPACE}}}
\def\TC{{\mathrm{TC}}}
\def\U{{\mathsf{U}}}
\def\WU{{\mathsf{WU}}}
\def\X{{\mathsf{X}}}
\def\EX{{\mathsf{EX}}}
\def\E{{\mathsf{E}}}

\newcommand{\prop}{\rho}

\renewcommand{\L}{\mathsf{LOGSPACE}}

%% \usepackage[thmmarks]{ntheorem}
%% %%% Theorem environments
%% \newtheorem{lemma}{Lemma}
%% \newtheorem{theorem}{Theorem}
%% \newtheorem{proposition}{Proposition}
%% \newtheorem{corollary}{Corollary}

%% \theoremstyle{definition}
%% \theoremsymbol{\ensuremath{\lozenge}}
%% \theorembodyfont{\normalfont}
%% \newtheorem{definition}{Definition}

%% \theoremstyle{nonumverplain}
%% \theoremheaderfont{\normalfont\bf}
%% \theorembodyfont{\normalfont}
%% \theoremseparator{.}
%% \theoremsymbol{\ensuremath{_\blacksquare}}
%% \newtheorem{example}{Example}

%% \theoremstyle{nonumberplain}
%% \theoremheaderfont{\normalfont\itshape}
%% \theorembodyfont{\normalfont}
%% \theoremseparator{.}
%% \theoremsymbol{\ensuremath{_\Box}}
%% \newtheorem{proof}{Proof}

% Symbols
\newcommand{\naturals}{\ensuremath{\mathbf{N}}}
\newcommand{\integers}{\ensuremath{\mathbf{Z}}}

% Machines and their ingredients
\newcommand{\automaton}{\ensuremath{A}}
\newcommand{\clocations}{\ensuremath{Q}}
\newcommand{\iclocation}{\ensuremath{q_{\mathit{in}}}}
\newcommand{\fclocation}{\ensuremath{F}}
\newcommand{\transitionrel}{\ensuremath{\Delta}}
\newcommand{\transitionlabel}{\ensuremath{\lambda}}
\newcommand{\clocationlabel}{\ensuremath{\xi}}
\newcommand{\operations}{\ensuremath{Op}}
\newcommand{\operation}{\ensuremath{op}}
\newcommand{\add}[1]{\ensuremath{\mathsf{add}(#1)}}
\newcommand{\zero}{\ensuremath{\mathsf{zero}}}
\newcommand{\propositions}{\ensuremath{L}}
\newcommand{\configuration}{\ensuremath{c}}
\newcommand{\configurations}{\ensuremath{\mathcal{C}}}
\newcommand{\run}{\ensuremath{\pi}}
\newcommand{\trace}{\ensuremath{\tau}}
\newcommand{\pocatuple}{\ensuremath{\langle \clocations,
\iclocation, \fclocation, \transitionrel, \transitionlabel,
\clocationlabel \rangle}}

\newcommand{\circuit}{\ensuremath{\mathbb{C}}}

\newcommand{\tm}{\ensuremath{\mathcal{M}}}

%%% Names
\newcommand{\oca}{\textsc{OCA} }
\newcommand{\poca}{\textsc{POCA} }
\newcommand{\Poca}{\mathbb{P}}
\newcommand{\Soca}{\mathbb{S}}

%%% Problems
\newcommand{\stsat}{\textsc{Succinct 3-SAT} }
\newcommand{\tsat}{\textsc{3-SAT} }
\newcommand{\ltlpoca}{\textsc{LTL-POCA-MC} }

%%% Complexity classes
\newcommand{\nexptime}{\textsc{NExpTime} }
\newcommand{\conexptime}{\textsc{co-NExpTime} }
\newcommand{\pspace}{\textsc{PSpace} }

%%% LTL
\newcommand{\ltl}{\text{LTL}}
\newcommand{\ctl}{\text{CTL}}
%\newcommand{\until}{\ensuremath{\mathcal{~U~}}}
\newcommand{\until}{\ensuremath{\U}}
\newcommand{\eventually}{\ensuremath{\mathsf{F}}}
\newcommand{\globally}{\ensuremath{\mathsf{G}}}
\newcommand{\propalphabet}{\ensuremath{\Sigma}}
\newcommand{\pastarred}{\ensuremath{(\mathcal{P}(\propalphabet))^{\omega}}}

%%% Logic
\newcommand{\true}{\ensuremath{\mathtt{tt}}}
\newcommand{\false}{\ensuremath{\mathtt{ff}}}

%%% Control commands
\newcommand{\defemph}[1]{\emph{#1}}







%\newtheorem{fact}{Fact}

\newcommand{\LCM}{\text{LCM}}

\begin{document}


\title{Model Checking Succinct and Parametric One-Counter Automata}
\author{Stefan G\"oller\inst{1} \and
Christoph Haase\inst{2} \and
Jo\"el Ouaknine\inst{2} \and
James Worrell\inst{2}
}


\institute{
Universit\"at
Bremen, Institut f\"ur Informatik, Germany%\\
%\email{goeller@informatik.uni-bremen.de}
\and
Oxford University Computing Laboratory, UK
}

\maketitle

\begin{abstract}
  We investigate the decidability and complexity of various model
  checking problems over one-counter automata. More specifically, we
  consider \emph{succinct} one-counter automata, in which additive
  updates are encoded in binary, as well as \emph{parametric}
  one-counter automata, in which additive updates may be given as
  unspecified parameters. We fully determine the complexity of model
  checking these automata against CTL, LTL, and modal $\mu$-calculus
  specifications.
\end{abstract}


\section{Introduction}

Counter automata, which comprise a finite-state controller together
with a number of counter variables, are a fundamental and widely-studied
computational model. One of the earliest results about counter
automata, which appeared in a seminal paper of Minsky's five decades
ago, is the fact that two counters suffice to achieve Turing
completeness~\cite{Min61}.

Following Minsky's work, much research has been directed towards
studying restricted classes of counter automata and related
formalisms. Among others, we note the use of restrictions to a single
counter, on the kinds of allowable tests on the counters, on the
underlying topology of the finite controller (such as
flatness~\cite{CJ98-cav,LS08-atva}), and on the types of computations
considered (such as reversal-boundedness~\cite{ID06-tcs}). Counter
automata are also closely related to Petri nets and pushdown automata.

In Minsky's original formulation, counters were represented as integer
variables that could be incremented, decremented, or tested for
equality with zero by the finite-state controller. More recently,
driven by complexity-theoretic considerations on the one hand, and
potential applications on the other, researchers have investigated
additional primitive operations on counters, such as additive updates
encoded in binary~\cite{BHIMV06,LS08-atva} or even
in \emph{parametric} form, i.e., whose precise values depend on
parameters~\cite{BIL06-icalp,IJTW93-icalp}. We refer to such counter automata as
\emph{succinct} and \emph{parametric} respectively, the former being
viewed as a subclass of the latter. Natural applications of such
counter machines include the modeling of resource-bounded processes,
programs with lists, recursive or multi-threaded programs, and XML
query evaluation; see, e.g.,
\cite{ID06-tcs,BHIMV06,CR04}.

In most cases, investigations have centered around the decidability
and complexity of the \emph{reachability} problem, i.e., whether a
given control state can be reached starting from the initial
configuration of the counter automaton. Various instances of the
reachability problem for succinct and parametric counter automata are
examined, for example, in~\cite{DG-jlc09,HKOW10,IJTW93-icalp}.

The aim of the present paper is to study the decidability and
complexity of \emph{model checking} for succinct and parametric
one-counter automata. In view of Minsky's result, we restrict our
attention to \emph{succinct one-counter automata (SOCA)} and
\emph{parametric one-counter automata (POCA)}. On the specification
side, we focus on the three most prominent formalisms in the
literature, namely the temporal logics CTL and LTL, as well as the
modal $\mu$-calculus. For a counter automaton $\mathbb{A}$ and a
specification $\varphi$, we therefore consider the question of
deciding whether $\mathbb{A} \models \varphi$,
in case of POCA for all values of the parameters,
%\footnote{In the case of
%parametric automata, one asks whether the automaton satisfies the
%formula \emph{for all} legal values of the parameters.}
and investigate both the \emph{data} complexity (in which the formula
$\varphi$ is fixed) as well as the \emph{combined} complexity of this
problem. Our main results are summarized in Table \ref{T results}.
\begin{table}
\begin{center}
\begin{tabular}{ >{\centering\arraybackslash}m{3cm}
  >{\centering\arraybackslash}m{2cm} |
  >{\centering\arraybackslash}m{3.3cm} |
> {\centering\arraybackslash}m{3.3cm} |}\\
  &  &{\bf SOCA}&{\bf POCA}\\ \hline
{\raisebox{-2ex}[0ex]{{\bf CTL, $\mu$-calculus}}}&   {\bf data} &   & \\ \cline{2-2}
{\raisebox{-2ex}[-5ex]{{$\phantom{\text{CTL}}$}}}&  {\bf combined}&
{\raisebox{2ex}[0ex]{$\EXPSPACE$-complete}} &
{\raisebox{2ex}[0ex]{ $\Pi^0_1$-complete}}\\ \cline{1-4}
{\raisebox{-2ex}[0ex]{{\bf LTL}}}&  {\bf data} &
\multicolumn{2}{c|} {\raisebox{0ex}[0ex]{$\coNP$-complete}}
\\ \cline{2-4}
{\raisebox{-2ex}[-5ex]{{$\phantom{\text{CTL}}$}}}&  {\bf combined}&
$\PSPACE$-complete &
$\coNEXP$-complete\\ \cline{1-4}
\end{tabular}
\ \\[0.4cm]
\caption{The complexity of CTL, the modal $\mu$-calculus, and LTL on SOCA and
POCA.}{\label{T results}}
\end{center}
\end{table}

One of the motivations for our work was the recent discovery that
reachability is decidable and in fact $\NP$-complete for both SOCA and
POCA~\cite{HKOW10}. We were also influenced by the work of Demri and
Gascon on model checking extensions of LTL over non-succinct,
non-parametric one-counter automata~\cite{DG-jlc09}, as well as the
recent result of G\"oller and Lohrey establishing that model checking
CTL on such counter automata is $\PSPACE$-complete \cite{GoLo10}.

We note some interesting differences between our results and
corresponding questions regarding finite automata. For the latter, the
(combined) model checking problems for CTL, the $\mu$-calculus, and
LTL are respectively known to be $\mathsf{P}$-complete, in
$\mathsf{NP} \cap \coNP$, and $\PSPACE$-complete. Somewhat
surprisingly, for SOCA and POCA, the complexity ordering is reversed
and LTL becomes easier to model check than either CTL or the
$\mu$-calculus.

On a technical level, the most intricate result is the
$\EXPSPACE$-hardness of CTL model checking for SOCA, which requires
several steps. We first show that $\mathsf{EXPSPACE}$ is `exponentially
$\mathsf{LOGSPACE}$-serializable', adapting the known proof that
$\PSPACE$ is $\mathsf{LOGSPACE}$-serializable. Unfortunately, and in
contrast to~\cite{GoLo10}, this does not immediately provide
an $\EXPSPACE$ lower bound. In a subsequent delicate stage of the
proof, we show how to `split' the counter in order simultaneously to
perform $\PSPACE$ computations in the counter and manipulate numbers
of exponential size in a SOCA of polynomial size.



Our paper is organized as follows. In Section \ref{S Prelim} we
introduce general notations.
Section \ref{S CTL} deals with CTL and the modal $\mu$-calculus.
LTL model checking is content of Section \ref{S LTL}.
Finally, in Section \ref{S Concl} we give conclusions.

\section{Preliminaries}{\label{S Prelim}}

\newcommand{\N}{\mathbb{N}}
\newcommand{\Z}{\mathbb{Z}}
\renewcommand{\O}{\mathbb{O}}
%\renewcommand{\P}{\mathcal{P}}

\newcommand{\trans}[1]{\stackrel{#1}{\longrightarrow}}
\newcommand{\hook}[1]{\stackrel{#1}{\hookrightarrow}}
\renewcommand{\rule}[2]{\stackrel{#2}{\stackrel{#1}{\longrightarrow}}}

\renewcommand{\M}{\mathcal{M}}

\newcommand{\bit}{\text{bit}}
\renewcommand{\succ}{\text{succ}}

\newcommand{\Reach}{\text{Reach}}
\newcommand{\Moves}{\text{Moves}}
\newcommand{\bin}{\text{bin}}


By $\Z$ we denote the {\em integers} and by $\N=\{0,1,2,\ldots\}$ the
denote the {\em naturals}.  For each $i,j\in\Z$ we define
$[i,j]=\{k\in\Z\mid i\leq k\leq j\}$ and $[j]=[1,j]$.  For each
$i,n\in\N$, let $\bit_i(n)$ denote the $i^{\text{th}}$ least
significant bit of the binary representation of $n$. Hence
$n=\sum_{i\in\N}2^i\cdot\bit_i(n)$.  By $\bin_m(n)=\bit_0(n)\cdots
\bit_{m-1}(n)$ we denote the first $m$ least significant bits written
from {\em left to right}. When $m$ is not important we just write
$\bin(n)$.
%For any non-empty subset $M\subseteq\Z^+$, define $\gcd(M)$ to be the largest
%$k\geq 1$ that divides every element of $M$.
%We define $\gcd(\emptyset)=0$.
Let $p_i$ denote the $i^{\text{th}}$ prime number for each $i\geq 1$, i.e. $p_1=2$,
$p_2=3$ and so on. We define $\log(n)=\min\{i\geq 1\mid 2^i> n\}$, in other
words $\log(n)$ denotes the number of bits that are needed to represent $n$ in
binary.
All polynomials $p:\N\rightarrow\N$ that occur in this paper are assumed to satisfy
$p(n)\geq n$ for each $n\in\N$.
For each word $v=a_1\cdots a_n\in\Sigma^n$ over some finite alphabet $\Sigma$
and each $i,j\in[n]$ define $v[i,j]=a_i\cdots a_j$ and
$v(i)=v[i,i]$.

\bigskip

\renewcommand{\l}{\triangleright}
\renewcommand{\r}{\triangleleft}
\newcommand{\move}{\vdash}
\noindent
{\bf Turing machines and complexity theory:} In the following, we
introduce $f(n)$ space-bounded deterministic Turing machines (DTMs) in
a suitable way for proving lower bounds. These contain precisely one
input tape and one working tape.  In our setting, the working alphabet is
 assumed to be $\{0,1,\l,\r\}$,
where $\l$ is the {\em left marker}
and $\r$ is the {\em right marker}.  The working tape of the initial
configuration of such a DTM on an input $w\in\Sigma^n$ is
assumed to be $\l0^{f(n)}\r$, whereas its input tape is $\l w\r$.
Before we define DTMs, let $\Upsilon=\{-1,0,+1\}$ denote a set of
directions.

Formally, a {\em $f(n)$-space bounded deterministic Turing machine (DTM)} is a tuple
$\M=(S,\Sigma,s_0,F,\mu)$, where $S$ is a finite set of {\em states},
$\Sigma$ is a {\em finite input alphabet} with $\l,\r\not\in\Sigma$,
$s_0\in S$ is an {\em initial state}, $F\subseteq S$ is a set of {\em final
states}, and with $\Sigma_\M=\Sigma\uplus\{\l,\r\}$ we have that
$$\mu\quad:\quad S\times\Sigma_\M\times\{0,1,\l,\r\}\ \rightarrow\
S\times\Upsilon^2\times\{0,1,\l,\r\}$$
is the {\em transition function}, where $\mu(s,b_1,b_2)=(s',\delta_1,\delta_2,b)$ means
that $\M$ is currently in state $s$, its input head reads $b_1$, its
working tape reads $b_2$, and $\M$ changes to state $s'$, moves its
input head in direction $\delta_1$,
moves its working head in direction $\delta_2$, and
writes the bit $b$.
%We require
%moreover that whenever $\M$'s input (resp. working) head reads $\l$, then $\M$
%must move its input (resp. working) head to the right, i.e. $+1$, and whenever
%$\M$'s input (resp.
%working) head reads $\r$, %it must hold that
%$\M$ moves its
%input (resp. working) head to the left, i.e. $-1$. Formally,
%whenever $\delta(s,b_1,b_2)=(s',\delta_1,\delta_2,b)$, then
%for each $i\in\{1,2\}$ we have that $b_i=\l$ implies $\delta_i=+1$ and $b_i=\r$
%implies $\delta_i=-1$.
Moreover, we require that $\M$ writes a marker precisely when it reads a marker,
more formally: $b_2=m$ if and only if $b=m$ for each $m\in\{\l,\r\}$.
As expected, a {\em configuration of $\M$} is a tuple $(s,\l w\r,\l w'\r,i,j)$,
where $s\in S$ is the current state,
$w\in\Sigma^n$ is an input,
$w'\in\{0,1\}^{f(n)}$ is the content of the working tape of $\M$,
$i\in[0,n+1]$ is the current input head position, and
$j\in[0,f(n)+1]$ is the current working head position.
%For configurations $C_1$ and $C_2$, we write $C_1\move_\M C_2$ whenever $\M$ can
%move from $C_1$ to $C_2$ on one step.
We define the {\em language $L(\M)$ of $\M$} to consist of all words
$w\in\Sigma^*$ such that
the {\em initial configuration}
$(s_0,\l w\r,\l 0^{f(|w|)}\r,0,0)$ reaches a configuration
$(s_0,\l w\r,\l w'\r,i,j)$ satisfying $s\in F$.

By $\L$, $\PSPACE$, and $\EXPSPACE$ we denote the class of all
problems that can be decided by DTM that
is logarithmically, polynomially, exponentially space bounded,
respectively. Recall that $\Sigma^0_1$ (resp. $\Pi^0_1$) is the class of
all languages that are (resp. whose complements are) recursively enumerable.
\newcommand{\Prop}{\mathcal{P}}
\newcommand{\Op}{\text{Op}}
%\newcommand{\add}{\mathsf{add}}
\renewcommand{\A}{\mathbb{A}}
%\newcommand{\EF}{\mathsf{EF}}
%\newcommand{\CTL}{\mathsh{CTL}}
\bigskip


\noindent
\textbf{Transition systems: }
In the following, we fix a countable set of atomic propositions $\Prop$.
A {\em transition system} is a tuple $T=(S,\{S_\prop\mid \prop\in\Prop\},\rightarrow)$, where
$S$ is a set of {\em states}, $S_\prop\subseteq S$ for each $\prop\in\Prop$ and
$S_\prop$ is
non-empty for finitely many $\prop\in\Prop$, and finally
$\rightarrow \subseteq S\times S$ is a set of {\em transitions}.
We prefer to use the infix notation $s_1\rightarrow s_2$ to abbreviate
$(s_1,s_2)\in\rightarrow$.
%A {\em finite path} of length $n$ is sequence of transitions
%$\pi=s_0\rightarrow s_1\cdots \rightarrow s_n$, where $n\in\N $.
 An {\em infinite path} is an infinite sequence
$\pi=s_0\rightarrow s_1\cdots$.
% We denote
%by $|\pi|$ the length of the path $\pi$, which we define to be $\omega$
%if $\pi$ is infinite.
For each infinite path
$\pi= s_0\rightarrow s_1 \rightarrow \cdots$ and
$i\in\N$, we denote by $\pi^i$ the path $s_i\rightarrow s_{i+1}\cdots$
and by $\pi(i)$ the state $s_i$.
Define the {\em trace of $\pi$} as $\trace(\pi):\N\rightarrow2^\Prop$ where
$\trace(i)=\{\prop\in\Prop\mid s_i\in S_\prop\}$ for each $i\in\N$.
\bigskip

\noindent
\textbf{Succinct and parametric one-counter automata: }
A {\em succinct one-counter automaton (SOCA)} is a tuple
 $\Soca=(Q,\{Q_\prop\mid \prop\in\Prop\},E,\lambda)$, where $Q$ is a finite set
of {\em control states}, $Q_\prop\subseteq Q$ for each $\prop\in\Prop$ and
$Q_\rho$ is non-empty for finitely many $\prop\in\Prop$,
$E\subseteq Q\times Q$ is a finite set of {\em transitions}, and
$\lambda:E\rightarrow \Z\cup\{\zero\}$ labels the edges with decrements,
increments, and zero tests.
A {\em parametric one-counter automaton (POCA)} is a tuple
$\Poca(X)=(Q,\{Q_\prop\mid\prop\in\Prop\},E,\lambda)$, where the first
three components are same as for a SOCA,
where $X$ is a finite set of {\em parameters over the natural numbers},
and where $\lambda:E\rightarrow
(\Z\cup\{\zero\}\cup\{\circ x\mid\circ\in\{+,-\}, x\in X\})$.
For each assignment $\sigma:X\rightarrow\N$  the
{\em induced SOCA} is defined as $\Poca^\sigma=(Q,\{Q_\prop\mid\prop\in\Prop\},E,\lambda')$
where $\lambda'(e)=\circ\sigma(x)$ whenever $\lambda(e)=\circ x$
for some $\circ\in\{+,-\}$ and $\lambda'(e)=\lambda(e)$ otherwise.
If $X=\{x\}$ is a singleton, then we also write $\Poca(x)$ for $\Poca(X)$.
The {\em size} of a POCA is defined as
$|\Poca|=|Q|+|X|+|E|\cdot\max\{\log(|a|)\mid a\in\lambda(E)\cap\Z\}$.
Hence, we represent each appearing
integer in binary.
The size of a SOCA is defined analogously.
A SOCA $\Soca=(Q,\{Q_\prop\mid \prop\in\Prop\},E,\lambda)$ describes a transition system $T(\Soca)=(Q\times\N,\{Q_\prop\times\N\mid
\prop\in\Prop\}, \rightarrow)$, where for each $q_1,q_2\in Q$ and each
$n_1,n_2\in\N$ we have $q_1(n_1)\rightarrow q_2(n_2)$ if and only
if either $\lambda(q_1,q_2)=n_2-n_1$, or
($n_1=n_2=0$ and $\lambda(q_1,q_2)=\zero)$.


\section{CTL Model Checking}{\label{S CTL}}

In Section \ref{S CTL syn} we introduce the syntax and semantics of $\ctl$.
Since the upper bounds for the modal $\mu$-calculus follow from
known results and since the lower bounds follow from CTL, we do not introduce
the modal $\mu$-calculus formally.
In Section \ref{S Upper} we state an $\EXPSPACE$ upper bound for the combined
complexity of $\ctl$ and the modal $\mu$-calculus on SOCA and a $\Pi^0_1$ upper bound
on POCA. In Section \ref{S Serial} we recall basic notions from complexity theory
such as serializability and results on Chinese remainder representation.
In Section \ref{S CTL hardness} we prove the main result of this section, namely
that the data complexity of $\ctl$ on SOCA is $\EXPSPACE$-hard.
Finally, we prove a $\Pi^0_1$ lower bound for the data complexity
of $\ctl$ on POCA in Section \ref{S CTL und}.

\subsection{$\ctl$: Syntax and Semantics}{\label{S CTL syn}}
\noindent
Formulas $\varphi$ of $\ctl$ are given by the following
grammar, where $\prop$ ranges over $\Prop$:

$$
\varphi\quad ::=\quad \prop\ \mid \neg\varphi\ \mid\ \varphi\wedge\varphi\ \mid\
\EX\varphi\ \mid \E (\varphi\U\varphi)\ \mid\
\E(\varphi\WU\varphi)
$$

\noindent
We have the following abbreviations:
$\varphi_1\vee\varphi_2=\neg(\neg\varphi_1\wedge\neg\varphi_2)$,
$\varphi_1\rightarrow\varphi_2=\neg\varphi_1\vee\varphi_2$,
$\true=\prop\vee\neg\prop$ for some atomic proposition $\prop\in\Prop$,
and $\EF\varphi=\E(\true\U\varphi)$.
The {\em size} of a $\ctl$ formula is inductively defined as follows:
$|\prop|=1$ for each $\prop\in\Prop$, $|\neg\varphi|=|\EX\varphi|=|\varphi|+1$,
$|\varphi_1\wedge\varphi_2|=|\E(\varphi_1\U\varphi_2)|=|\E(\varphi_1\WU\varphi_2)|=
|\varphi_1|+|\varphi_2|+1$.
Given a transition system $T=(S,\{S_\prop\mid \prop\in\Prop\},\rightarrow)$, a state
$s\in S$, and some $\CTL$ formula $\varphi$, define $(T,s)\models\varphi$ by
induction on the structure of $\varphi$ as follows:

\begin{eqnarray*}
  (T,s)\models \prop &  \quad\iff\quad & s\in S_p\quad \text{ for each
}\prop\in\Prop\\
  (T,s)\models \varphi_1\wedge\varphi_2 &  \quad\iff\quad &
(T,s)\models\varphi_1\text{ and }
(T,s)\models \varphi_2 \\
  (T,s)\models \neg\varphi &  \quad\iff\quad & (T,s)\not\models\varphi\\
  (T,s)\models \EX\varphi &  \quad\iff\quad & (T,t)\models\varphi
\text{ for some } t\in S \text{ with }s\rightarrow t\\
  (T,s)\models \E(\varphi_1\U\varphi_2) &  \quad\iff\quad &
\exists s_0,\cdots,s_n\in S, n\geq 0: s_0=s, (T,s_n)\models\varphi_2 \text{ such
that}\\
&& \forall i\in[0,n-1]: (T,s_i)\models\varphi_1\text{ and } s_i\rightarrow
s_{i+1}\\
  (T,s)\models \E(\varphi_1\WU\varphi_2) &  \quad\iff\quad &
(T,s)\models\E(\varphi_1\U\varphi_2)\ \text{ or }\\
&& \exists s_0,s_1,\ldots\in S:\forall i\geq 0: (T,s_{i})\models\varphi_1\text{ and }
s_i\rightarrow s_{i+1}
\end{eqnarray*}

\noindent
Let us define the {\em $\ctl$ model checking problem} on SOCA and POCA
respectively.\\

\problemx{$\ctl$ model checking on SOCA}
{SOCA $\Soca=(Q,\{Q_\prop\mid\prop\in\Prop\},E,\lambda)$, $q\in Q$, $n\in\N$ in binary, and a $\ctl$ formula
$\varphi$.}
{$(T(\Soca),q(n))\models\varphi$?}
\

\problemx{$\ctl$ model checking on POCA}
{POCA $\Poca(X)=(Q,\{Q_\prop\mid\prop\in\Prop\},E,\lambda)$, $q\in Q$, $n\in\N$ in binary, and a $\ctl$ formula
$\varphi$.}
{$(T(\Poca^\sigma),q(n))\models\varphi$ for every $\sigma:X\rightarrow\N$?}


\subsection{Upper bounds}{\label{S Upper}}
Let us first state upper bounds on model
checking CTL and the modal $\mu$-calculus on SOCA.
 {\em One-counter automata (OCA)} are SOCA in which
 the numbers that occur in the transition labels are represented  in unary
 instead in binary.
%In this section we state the upper bounds on model checking $\ctl$ on SOCA and
%POCA. They follow from known results or are immediate.
The following theorem gives an upper bound on model checking the
{\em modal $\mu$-calculus} on OCA.
Consult \cite{ArnNiw01} for more details on the modal $\mu$-calculus.

\begin{theorem}[\cite{Serr06}]
The combined complexity of the modal $\mu$-calculus on OCA is in $\PSPACE$.
\end{theorem}

\noindent
Since every SOCA can be transformed into an exponentially larger OCA
 and since each $\ctl$ formula can be translated into an alternation-free
$\mu$-calculus formula with a linear blowup, the following corollary is
immediate.

\begin{corollary}{\label{C CTL up}}
The combined complexity of $\ctl$ and the modal $\mu$-calculus on SOCA is in $\EXPSPACE$.
\end{corollary}

\noindent
The following upper bound for $\ctl$ on POCA is straightforward.

\begin{corollary}
The combined complexity of $\ctl$ and the modal $\mu$-calculus on POCA is in $\Pi^0_1$.
\end{corollary}
\begin{proof}
Let $\Poca(X)=(Q,\{Q_\prop\mid\prop\in\Prop\},E,\lambda)$ be a POCA,
 $q\in Q$, and $n\in\N$, and $\varphi$ be
a formula.
From Corollary \ref{C CTL up} it follows that
for each given $\sigma:X\rightarrow\N$ the question
$
(T(\Poca^\sigma),q(n))\models\varphi
$
is a decidable predicate.
Hence, the question if
$
(T(\Poca^\sigma),q(n))\models\varphi
$
for each $\sigma:X\rightarrow\N$ is a $\Pi^0_1$-predicate since
we can encode the set of all such assignments into the naturals
via G\"odel encoding.
\qed
\end{proof}


\subsection{Serializability and Chinese remainder representation}{\label{S Serial}}
For a language $L\subseteq\Sigma^*$ let $\chi_L:\Sigma^*\rightarrow\{0,1\}$
denote the {\em characteristic function of $L$}, i.e.
$\chi_L(x)=1$ if $x\in L$ and $\chi_L(x)=0$ otherwise, for each $x\in\Sigma^*$.
We define the {\em lexicographic
order on $n$-bit strings} as $\preceq_n\subseteq\{0,1\}^n\times\{0,1\}^n$, where
$x\preceq y$ if and only if
$\bin_n(x)\leq\bin_n(y)$, e.g. $101\preceq_3 011$.

\renewcommand{\C}{\mathcal{C}}
\begin{definition}
Let $\C$ be a complexity class.
We say a language $L$ is {\em $\C$-serializable via some language
$R\subseteq\{0,1\}^*$} if there is some polynomial $p(n)$, and some language
$U\in\C$ such that for all $x\in\{0,1\}^n$
$$x\in L\qquad\!\!\iff\qquad\!\!
\chi_U\left(x,0^{p(n)}\right)\quad \cdots\quad \chi_U\left(x,1^{p(n)}\right)\in
R,$$ where with `$\cdots$'  we
refer to $\preceq_{p(n)}$ in each of the second components.
\end{definition}

\noindent
% it was shown that $\PSPACE$ is $\mathsf{P}$-serializable;
%in \cite{HLSVW93} this result was sharpened to $\AC^0$-serializability,
%see also \cite{Vol98}. It is not stated in \cite{HLSVW93,Vol98} but easy to
%see from the proofs that \emph{logspace-uniform} $\AC^0$ suffices for serializing
%$\PSPACE$, see the appendix for more details.
The following theorem was proven in \cite{GoLo09}.

\begin{theorem}[Theorem 22 in \cite{GoLo09}]{\label{T Serial}}
For every $L$ in $\PSPACE$ there is some regular language $R$ such that $L$ is
logspace-uniformly $\mathsf{AC}^0$-serializable via $R$.
\end{theorem}

\noindent
As an immediate consequence, we obtain the following corollary.

\begin{corollary}{\label{C Serial}}
For every $L$ in $\PSPACE$ there is some regular language $R$ such that $L$ is
$\L$-serializable via $R$.
\end{corollary}

\noindent
We remark that our definition of serializability is adopted from \cite{GoLo09}
and differs slightly from the standard
notion as used in \cite{CaFu91,HLSVW93,Vol98}.
We aim at lifting Corollary \ref{C Serial} via an appropriate notion of
serializability to $\EXPSPACE$.

\begin{definition}
Let $\C$ be some complexity class.
A language $L$ is {\em exponentially $\C$-serializable via some language
$R\subseteq\{0,1\}^*$} if there is some polynomial $p(n)$ and some language
$U\in\C$ such that for all $x\in\{0,1\}^n$
$$x\in L\qquad\!\!\iff\qquad\!\!
\chi_U\left(x,0^{2^{p(n)}}\right)\quad \cdots\quad \chi_U\left(x,1^{2^{p(n)}}\right)\in
R,$$ where with `$\cdots$' we
refer to $\preceq_{2^{p(n)}}$ in each of the second components.
\end{definition}


\noindent
The following proposition is folklore and immediate.
\begin{proposition}{\label{P Padding}}
Let $L\subseteq\Sigma^*$ be in $\EXPSPACE$. Then there is some polynomial $q$
such that the padded language
$\{x\$^{2^{q(|x|)}}\mid x\in L\}\subseteq(\Sigma\uplus\{\$\})^*$ is in $\PSPACE$.
\end{proposition}



\begin{theorem}\label{T succ seri}
For every language $L$ in $\EXPSPACE$ there is some regular language $R$ such
that $L$ is exponentially $\L$-serializable via $R$.
\end{theorem}
\begin{proof}
Let $L\subseteq\Sigma^*$ be some language in $\EXPSPACE$.
Then by Proposition \ref{P Padding} for some polynomial $q$ the language
$L'=\{x\$^{2^{q(|x|)}}\mid x\in L\}$ is in $\PSPACE$.
Due to $\L$-serializability of $\PSPACE$ by Corollary \ref{C Serial}, there exists
some polynomial $p'$, some regular language $R'$ and some $U'\in\L$
 such that for each $y\in(\Sigma\uplus\{\$\})^*$ we have
\begin{eqnarray}
y\in L'\quad\iff\quad \chi_{U'}\left(y,0^{p'(|y|)}\right)\ \ \cdots\ \
\chi_{U'}\left(y,1^{p'(|y|)}\right)\in R', \label{E Serial}
\end{eqnarray}
where `$\cdots$' refers to the lexicographic order $\preceq_{p'(|y|)}$.
To prove the lemma we have to find some regular language $R$, some
polynomial $p$ and some $U\in\L$
such that for all $x\in\Sigma^*$ we have
$$
x\in L\quad \iff\quad \chi_U\left(x,0^{2^{p(|x|)}}\right)\ \cdots\
\chi_U\left(x,1^{2^{p(|x|)}}\right)\in R,
$$
where here `$\cdots$' refers to the lexicographic order
$\preceq_{2^{p(|x|)}}$.
For now, we choose $p$ sufficiently fast growing,
let us postpone this to the end of the proof.
Let us describe the language $U$ and consider inputs $(x,w)$
 with $x\in\Sigma^n$ and $w\in\{0,1\}^{2^{p(n)}}$;
moreover let $y=x\$^{2^{q(|x|)}}$ be the padding of $x$.
Our language $U$ consists of all such pairs $(x,w)$
such that
$w$ can be factorized as $w=bz0^l$ for some $b\in\{0,1\}$
and some $l\geq 0$ such that
$b=0$ or $(y,z)\in U'$.
Roughly speaking, the goal of $U$ is to simulate $U'$
on the corresponding padded word,
however the length of the second component
$w$ has to be a power of two, in other words the last $l$ bits of $w$
 contain redundant information.
We will be able to filter out this redundant information via
the bit $b$.
Let us make this more precise.
Let
\begin{eqnarray*}
\gamma' &\quad=\quad&
\chi_{U'}\left(y,0^{p'(|y|)}\right)\ \ \cdots\ \
\chi_{U'}\left(y,1^{p'(|y|)}\right)\quad\text{and}\\
\gamma &\quad=\quad& \chi_U\left(x,0^{2^{p(|x|)}}\right)\ \ \cdots\ \
\chi_U\left(x,1^{2^{p(|x|)}}\right).
\end{eqnarray*}

\noindent
Then by definition of $U$, the first $2\cdot p'(|y|)$ bits of $\gamma$ are
$1\gamma'(1)1\gamma'(2)\cdots 1\gamma'(p'(|y|))$, where
the remaining bits of $\gamma$ are all $0$.
Hence, when reading $\gamma$ in {\em pairs of bits},
precisely when the first of the two bits
is set to $1$ we read some relevant information.

We have to give some regular language $R$ such that $\gamma'\in R'$ if and only
if $\gamma\in R$.
Recall that regular languages are closed under shuffle product $|\!|$ and
homomorphisms.
We define $R=\varphi(R'|\!|\{a\}^*)$, where
$\varphi:\{0,1,a\}\rightarrow\{0,1\}^*$ is the following homomorphism:
$\varphi(a)=00$, $\varphi(0)=10$, and $\varphi(1)=11$.

As expected, we choose $p$ to be an arbitrary polynomial satisfying
$2^{p(n)}\geq 1+p'(2^{q(n)}+n)$ for all $n\geq0$.
Finally, let us show that $U$ is in $\L$.
Let $(x,w)$ be an input to $U$ and let again $y=x\$^{2^{q(|x|)}}$ be the padding
of $x$.
Then it is straightforward to decide in logarithmic space if $w$ can be factorized as
$w=bzv$ such that $b\in\{0,1\}$, $|z|=p'(2^{q(|x|)}+|x|)$ and
$v\in\{0\}^*$.
The only thing that might remain to be verified in logarithmic space is whether
$(y,z)\in U'$, which in turn boils down to simulating some logarithmic space bounded TM
for $U$ on input $(y,z)$.
But this is possible since $|y|+|z|\leq 2\cdot(|x|+|w|)$ as
$$
|y|+|z|=|x|+2^{q(|x|)}+p'(2^{q(|x|)+|x|})\leq 2\cdot p'(2^{q(|x|)}+|x|)\leq
2\cdot 2^{p(|x|)}\leq 2\cdot (|x|+|w|).
$$
\qed
\end{proof}

\renewcommand{\M}{\mathcal{M}}

\noindent
\textbf{Chinese remainder representation: }
 For every $m,M\in\N$ we denote by $\CRR_m(M)$ {\em the Chinese remainder
representation} of $M$ as the Boolean tuple
$(b_{i,c})_{i\in[m],0\leq c<p_i}$, where $b_{i,c}=1$ if $M\modulo p_i=c$
and $b_{i,c}=0$ otherwise.
The following theorem tells us that in logarithmic space we can
compute the binary representation of a natural number that is given in Chinese
remainder representation.
It is a consequence of the result that division is in logspace-uniform
$\mathsf{NC}^1$, proven in \cite{ChDaLi01}.


\begin{theorem}[\cite{ChDaLi01} Theorem 3.3]{\label{T CRRBIN}}
The following problem is in $\L$:

INPUT: $\CRR_m(M),j\in[m]$, $b\in\{0,1\}$.

QUESTION: $\bit_j(M\ \modulo 2^m)=b$?
\end{theorem}


\subsection{$\EXPSPACE$-hardness of the data complexity of $\ctl$ on SOCA}{\label{S CTL hardness}}



In the rest of this section, we give the proof of $\EXPSPACE$-hardness
of the data complexity of $\ctl$ on SOCA.  Let $L\subseteq\{0,1\}^*$ be an arbitrary
language in $\EXPSPACE$. Then by Theorem \ref{T succ seri}, there is
some regular language $R\subseteq\{0,1\}^*$ such that $L$ is
exponentially $\L$-serializable via $R$. Hence there is some language $U\in\L$
and some polynomial $p$ such that for all $x\in\{0,1\}^n$ we have

\begin{eqnarray}{\label{E EXPSPACE char}}
x\in L\qquad\iff\qquad \chi_{U}\left(x,0^{2^{p(n)}}\right)
\quad \cdots\quad \chi_U\left(x,1^{2^{p(n)}}\right) \in R,
\end{eqnarray}
where '$\cdots$' refers to the lexicographic order $\preceq_{2^{p(n)}}$
on the bit strings on the right-hand side.

\noindent
For the rest of this section, let us fix an input $x_0\in\{0,1\}^n$.
Let $N=p(n)$ and let $A=(Q,\{0,1\},q_0,\delta,F)$ be some
deterministic finite automaton with $L(A)=R$.  Let us briefly recall
what $A$ consists of: $Q$ is a finite set of {\em states}, $\{0,1\}$
is the {\em input alphabet}, $q_0\in Q$ is the {\em initial state},
$\delta:Q\times\{0,1\}\rightarrow Q$ is the {\em transition function},
and $F\subseteq Q$ is the set of {\em final states}.  Let us describe
equivalence (\ref{E EXPSPACE char}) differently: We have $x_0\in L$ if
and only if the program in Figure \ref{F U} returns $\mathtt{true}$.

\begin{figure}
\normalsize

\ \hspace{5cm} $q\in Q$; $q:=q_0$;\\[-0.4cm]

\ \hspace{5cm} $d\in\N$; $d:=0$;

\ \hspace{5cm} $b\in\{0,1\}$;\\[-0.4cm]

\ \hspace{5cm} $\mathtt{while }\  d\not=2^{2^N}$\ $\mathtt{ loop}$\\[-0.4cm]

\ \hspace{6cm} $b:=\chi_U(x_0,\text{bin}_{2^N}(d))$;\\[-0.4cm]

\ \hspace{6cm} $q:=\delta(q,b)$;\\[-0.4cm]

\ \hspace{6cm} $d:=d+1$;\\[-0.4cm]

\ \hspace{5cm} $\mathtt{end loop}$\\[-0.4cm]

\ \hspace{5cm} $\mathtt{return}\ q\in F$;%\\[0.4cm]

\caption{A program that returns $\mathtt{true}$ if and only if $x_0\in L$}{\label{F U}}
\end{figure}

\noindent
Our goal is to mimic the execution of the program in Figure \ref{F U} by
a fixed $\ctl$ formula and by a SOCA that depends on
our input $x_0$.
Before we start with the reduction, let us discuss the obstacles
that arise:


\begin{enumerate}[{\bf(A)}]
\item We need some way of storing $d$ on the counter. Of course
there are a lot of ways to do this, but since we want to access
all bits of $d$ in the assignment $b:=\chi_U(x_0,\bin_{2^N}(d))$, the most
natural way is probably to represent $d$ in binary.
However, for this $2^N$ bits are required.
 More problematically, we need to be able to check if $d$ is equal to
$2^{2^N}$. This cannot be achieved by a transition in a SOCA
 that subtracts $2^{2^N}$, since the representation of this number
requires exponentially many bits in $n$.
\item As in \cite{GoLo10}, a solution to obstacle {\bf (A)} is to store $d$ in
Chinese remainder representation with the first $2^N$ prime numbers.
A polynomial number of bits (in $n$) suffice
to represent each of the occurring prime numbers, but
we need exponentially many of them.
Thus, we cannot equip a polynomial size SOCA with transitions for
each prime number, simply because there are too many of them.
\item
The assignment $b:=\chi_U(x_0,\bin_{2^N}(d))$ implies that we need to simulate on
the counter a logarithmically space bounded DTM for the language $U$
on an exponentially
large input (in $n$).
Speaking in terms of the input size $n$, this means that we need to provide
polynomially many bits on the counter that can be used to describe the
working tape for this DTM. However, we need to provide
some on-the-fly mechanism for reading the input.
\end{enumerate}


\iffalse

\begin{enumerate}[(1)]
\item For the representation of the variable $d$ we need $2^N+1$ bits
  on our counter. More challenging, we need to check if $d$ is
  equal to $2^{2^N}$. To check this, we cannot simply allow a
  transition in our counter automaton that subtracts $2^{2^N}$,
  since the representation of this number requires exponentially many
  bits in $n$.
\item As in \cite{GoLo10}, a solution to problem (1) is to store $d$
  in Chinese remainder representation with the first $2^N+1$ primes.
  Polynomially (in $n$) many bits suffice to represent each of the
  occurring prime numbers, but we need exponentially many different
  prime numbers, where each of the primes has up to exponentially many
  residue classes that we cannot simply write down. So we cannot equip
  the SOCA that we have to derive with transitions for each prime
  number, simply since there are too many of them.

\item The assignment $b:=\chi_U(x_0,\bin_{2^N}(d))$ means that we need to simulate on
our counter a logarithmically space bounded DTM for $U$ on an exponential
sized input (in $n$).
Speaking in terms of our input size $n$, this means that we need to provide
polynomially many bits on our counter that can be used to describe the
working tape for this DTM. However, we need to provide
some on-the-fly mechanism for reading the input.
\end{enumerate}

\fi
\noindent
To achieve this, let us give a high-level description of how we proceed.
In a first step, we carefully design a data structure on the counter
and describe the intuition behind it.
In a second step, we give five queries which we aim at implementing
via fixed CTL formulas and by SOCA that can be computed from $x_0$
in logarithmic space.

\bigskip

\noindent
{\bf The data structure and how to access it: }
Let $K=n+2^N+1$ denote the number of bits that are required to store an input
for $U$ plus one.
Let $\alpha=\log K$ denote the number of bits that we require for
storing a pointer to an input for $U$ and let $\beta$
be the number of bits that suffice for storing the
$K^{\text{th}}$ prime. Hence $\alpha=O(N)$ and by
the Prime Number Theorem, it follows that
$\beta\in O(\log(K\log(K)))=O(N)$.
The number $\alpha$ and such a sufficiently large number $\beta$ can be computed
 from $x_0$ in logarithmic space.


Let us describe how we will interpret the counter in our reduction.
Assume that the counter value is $v\in\N$.  Instead of
treating $v$ as a natural number, we are interested only in the $l$
least significant bits $V$ of the binary representation of $v$, where $l$
is some number that is exponentially bounded in $n$; the precise value
of $l$ will be made clear below.  Assume
$V=\bit_0(v)\cdots\bit_{l-1}(v)$.  We read $V$ to be factorized into
blocks of bits
$$\qquad\qquad\qquad\qquad\qquad\qquad\qquad\qquad V\quad=\quad I\ M\ C\ J\ X\ Y\ Z\ B,\qquad\qquad\qquad\qquad\qquad\qquad(\bigstar)$$ where
\begin{itemize}
\item $I\in\{0,1\}^{\alpha}$ represents a prime number index,
\item $M\in\{0,1\}^\beta$ represents $I^{\text{th}}$
prime number $p_I$,
\item $C\in\{0,1\}^\beta$ represents some residue class modulo $M$,
\item $J\in\{0,1\}^\alpha$ represents a pointer to some bit of $B$,
\item
each $X,Y$ and $Z$ consist of polynomially many bits (in $n$) and
represent the working tape of three space bounded
DTMs that we will comment on later in more detail, and
\item $B\in\{0,1\}^{n+2^N+1}$ with $B=xB'$ for some
$x\in\{0,1\}^n$ and some
$B'\in\{0,1\}^{2^N+1}$,
$B$ represents the current
input for $U$, where
$x$ is reserved to represent our input $x_0$
and
where $B'$ represents the counter $d$ from program in Figure \ref{F U} from above.
The block $B'$ consists of $2^N+1$ bits since we want to be able to test
 if $d=2^{2^{N}}$.
\end{itemize}


\renewcommand{\Re}{\Rightarrow}
\newcommand{\Li}{\Leftarrow}
\newcommand{\re}{\rightarrow}
\newcommand{\li}{\leftarrow}



\noindent
Let us introduce some more notation for addressing
leftmost (starting to count from $0$) and rightmost bit
positions in each of the above sequences of bits in $V$.
For each such sequence $\Theta$ let $\Theta_{\Li}$ and $\Theta_{\Re}$ denote the
respective positions of the leftmost and rightmost bits of $\Theta$ within the bit
string $V$, e.g. $I_\Re=\alpha-1$ and $C_\Li=\alpha+\beta$.


\medskip

\noindent
{\bf Important remark: } Throughout the rest of this section $v$ will
denote an arbitrary natural number, moreover $I,M,C,J,X,Y,Z$, and $B$
will implicitly be coupled with $v$.  Note that all of the bit strings
have polynomial length in $n$ except for $B$.  Moreover, we identify each
of the blocks with the natural number they represent.

\medskip

\noindent
A very simple but important gadget that we need is to decide, for each
$b\in\{0,1\}$ if the $i^{\text{th}}$ bit of $v$ is $b$.

\begin{lemma}{\label{L bit}}
For each bit $b\in\{0,1\}$ there exists a fixed $\ctl$ formula $\varphi_{\bit,b}$
such that the following is computable in logarithmic space:

INPUT: $i\in\N$ in unary.% and our input $x_0$.

OUTPUT: A SOCA $\Soca_{\bit,i}$ and some control state $q_{\bit,i}$ such that
$$
(T(\Soca_{\bit,i}),q_{\bit,i}(v))\models\varphi_{\bit,b}\quad\Longleftrightarrow\quad
\bit_i(v)=b.
$$
\end{lemma}
\begin{proof}
The SOCA $\Soca_{\bit,i}$ has the two atomic propositions $\tau_1$ and $\tau_2$ and is
depicted below:

\begin{center}
\begin{picture}(85,20)(-20,-5)
%   \put(-20,-28){\framebox(85,33){}}
  %\gasset{Nadjust=w,Nadjustdist=2,Nh=6,Nmr=1}
 % \gasset{Nadjust=w,Nadjustdist=2,Nh=6}%$,Nmr=1}
 % \node(A)(0,0){$q_0$}

\gasset{Nframe=n,loopdiam=9,ELdist=.7}
\gasset{Nadjust=wh,Nadjustdist=1}
\gasset{curvedepth=0}


\put(13,-2){\small $\tau_1$}
\put(41,-2){\small $\tau_2$}
  \node(A)(20,0){$q_{\bit,i}$}
  \node(B)(40,0){$\bullet$}
\drawloop(A){$-2^{i+1}$}
\drawedge(A,B){$-2^i$}
\end{picture}
\end{center}
The simple way to check if $\bit_i(v)=1$ (resp. $\bit_i(v)=0$) is to
repeatedly subtract $2^{i+1}$ from $v$ until no longer possible
and then to check if $2^i$ can (resp. cannot) be subtracted.
Hence we put
\begin{eqnarray*}
\varphi_{\bit,1}&\quad=\quad&
\tau_1\wedge
\EF(\tau_1\wedge\neg\EX \tau_1\wedge\EX
\tau_2)\qquad\text{ and}\\
\varphi_{\bit,0}&=&
\tau_1\wedge
\EF(\tau_1\wedge\neg\EX \tau_1\wedge\neg\EX
\tau_2).
\end{eqnarray*}
\qed
\end{proof}



\noindent
{\bf Queries that we need to implement: }
We will implement the following five queries Q1 to Q5 by instances
of the model checking problem, where each query builds on top
of its preceding queries.

\begin{enumerate}[(Q1)]
\item \fbox{When assuming $C<M$, does $B\equiv C\ \mod M$ hold?}
\item \fbox{Is $M$ the $I^{\text{th}}$ prime number, i.e. $M=p_I$?}
%\item Let $\CRR_{2^\alpha}(B)=(b_{i,r})_{i\in[2^\alpha],0\leq r<p_i}$.
%What is $b_{I,R}$? %resp. $b_{I,R}=1$ hold?
\item \fbox{What is $\bit_J(B)$?}
\item \fbox{Does $(B[1,n],B[n+1,n+2^N])\in U$ hold?}
%\item Does $B'=2^{2^N}$ hold?
\item \fbox{Does $x_0\in L$ hold?}
\end{enumerate}


\noindent
We implement each of the five queries by providing fixed CTL formulas
and SOCA that can be computed from $x_0$ in logarithmic space.
$\EXPSPACE$-hardness of the data complexity of CTL on SOCA will hence
follow from the implementation of Q5.
First, let us give an implementation of Q1.

\begin{lemma}{\label{L mod}} There exists some fixed $\ctl$ formula $\varphi_\mod$ such that we
can compute from $x_0$ in logarithmic space some SOCA $\Soca_\mod$ and some
control state $q_\mod$ such that
$(T(\Soca_{\mod}),q_\mod(v))\models\varphi_\mod$ if and only if $B\equiv C\ \mod M$.
\end{lemma}
\begin{proof}
The SOCA $\Soca_{\mod}$ contains the three atomic propositions $\prop_0,\prop_1$, and
$\prop_2$ and is depicted in Figure \ref{F Smod}.
The $\ctl$ formula $\varphi_\mod$ expresses that we
traverse the upper
sequence of diamonds and thereby repeatedly subtract $M$ from $B$.
The number of diamonds both in the first and the second row equals $\beta$,
the number of bits of $M$ and of $C$.
In the upper row, one diamond corresponds to one bit of $M$. In case bit the rightmost bit of $M$
(in other words bit $M_\Re$ of the counter) is $1$,
which we can verify by a transition to the initial control state of
the SOCA $\Soca_{\bit,M_\Re}$,
 we subtract $2^{B_\Li+\beta-1}$ from $B$, otherwise we do not modify the counter value.
After that, we move on to the second diamond, which represents the second
rightmost  bit of $M$, and so on.
Hence, traversing a cycle that starts and ends
in our initial control state $q_\mod $ will correspond to subtracting $M$ once.
 Then we traverse
the lower sequence of diamonds and similarly subtract $C$, but this
this time only once.
Finally, after having traversed the lower sequence of diamonds and seeing the atomic
proposition $\prop_2$, we check if $B=0$ by trying to subtract $2^{B_\Li}$.
\begin{figure}
\begin{center}
\begin{picture}(100,120)(5,-41)
%   \put(-20,-28){\framebox(85,33){}}
  %\gasset{Nadjust=w,Nadjustdist=2,Nh=6,Nmr=1}
 % \gasset{Nadjust=w,Nadjustdist=2,Nh=6}%$,Nmr=1}
 % \node(A)(0,0){$q_0$}
%\put(15,-5){$q(M_{\abs,\r})$}
%\put(43,-5){$p_2$}
\drawpolygon[Nframe=n,Nfill=y,fillgray=.9](5,20)(28,20)(28,40)(5,40)
\put(10,25){\Large$\Soca_{\bit,M_\Re}$}

\drawpolygon[Nframe=n,Nfill=y,fillgray=.9](35,20)(58,20)(58,40)(35,40)
\put(38,25){\Large$\Soca_{\bit,M_\Re-1}$}


\drawpolygon[Nframe=n,Nfill=y,fillgray=.9](85,20)(108,20)(108,40)(85,40)
\put(90,25){\Large$\Soca_{\bit,M_\Li}$}


\gasset{Nframe=n,loopdiam=9,ELdist=.7}
\gasset{Nadjust=wh,Nadjustdist=1}
\gasset{curvedepth=0}


  \node(1)(0,60){$q_{\mod}$}
  \node(2O)(15,70){$\bullet$}
  \node(2U)(15,50){$\bullet$}
  \put(17,72){\small$\prop_1$}
  \put(17,48){\small$\prop_0$}

  \node(1A)(15,35){$q_{\bit,M_\Re}$}
  \node(4A)(45,35){$q_{\bit,M_\Re-1}$}

  \node(4A)(45,35){$q_{\bit,M_\Re-1}$}

  \node(7A)(95,35){$q_{\bit,M_\Li}$}


  \node(3)(30,60){$\bullet$}
  \node(4O)(45,70){$\bullet$}

  \put(47,72){\small$\prop_1$}
  \put(47,48){\small$\prop_0$}
  \node(4U)(45,50){$\bullet$}
  \node(5)(60,60){$\bullet$}

  \node(6)(80,60){$\bullet$}
  \node(7O)(95,70){$\bullet$}
  \put(97,72){\small$\prop_1$}
  \put(97,48){\small$\prop_0$}

  \node(7U)(95,50){$\bullet$}
  \node(8)(110,60){$\bullet$}



  \put(65,58){\huge$\cdots$}


%\drawedge[curvedepth=-70](8,1){$0$}

\drawedge(1,2O){$0$}
\drawedge[ELside=r](1,2U){$0$}
\drawedge[ELside=l,ELpos=60](2O,3){$-2^{B_\Li+\beta-1}$}
\drawedge(2U,3){$0$}

\drawedge[curvedepth=-6,ELpos=70,ELside=r](2O,1A){$0$}
\drawedge(2U,1A){$0$}

\drawedge[curvedepth=-6,ELpos=70,ELside=r](4O,4A){$0$}
\drawedge(4U,4A){$0$}

\drawedge[curvedepth=-6,ELpos=70,ELside=r](7O,7A){$0$}
\drawedge(7U,7A){$0$}



\drawedge(3,4O){$0$}
\drawedge[ELside=r](3,4U){$0$}
\drawedge[ELside=l,ELpos=70](4O,5){$-2^{B_\Li+\beta-2}$}
\drawedge(4U,5){$0$}

\drawedge(6,7O){$0$}
\drawedge[ELside=r](6,7U){$0$}
\drawedge[ELside=l,ELpos=60](7O,8){$-2^{B_\Li}$}
\drawedge(7U,8){$0$}

\drawcurve(110,60)(112,70)(55,80)(0,70)(0,62)


%\drawedge[ELpos=70,curvedepth=1.5](8,9A){$0$}
%\drawedge[ELside=r,ELpos=60](8,9B){$0$}
  \put(60,81){$0$}
%\drawedge[ELpos=50](8,9C){$0$}
%\drawedge[ELside=r,ELpos=50](8,9D){$-2^{B_L}$}


%%%%COPY


\drawpolygon[Nframe=n,Nfill=y,fillgray=.9](5,-40)(28,-40)(28,-20)(5,-20)
\put(10,-35){\Large$\Soca_{\bit,C_\Re}$}

\drawpolygon[Nframe=n,Nfill=y,fillgray=.9](35,-40)(58,-40)(58,-20)(35,-20)
\put(38,-35){\Large$\Soca_{\bit,C_\Re-1}$}


\drawpolygon[Nframe=n,Nfill=y,fillgray=.9](85,-40)(108,-40)(108,-20)(85,-20)
\put(90,-35){\Large$\Soca_{\bit,C_\Li}$}






\gasset{Nframe=n,loopdiam=9,ELdist=.7}
\gasset{Nadjust=wh,Nadjustdist=1}
\gasset{curvedepth=0}


  \node(1C)(0,0){$\bullet$}
  \drawedge[ELside=r](1,1C){$0$}
  \node(2OC)(15,10){$\bullet$}
  \node(2UC)(15,-10){$\bullet$}
  \put(17,12){\small$\prop_1$}
  \put(17,-12){\small$\prop_0$}

  \node(1AC)(15,-25){$q_{\bit,C_\Re}$}
  \node(4AC)(45,-25){$q_{\bit,C_\Re-1}$}

  \node(4AC)(45,-25){$q_{\bit,C_\Re-1}$}

  \node(7AC)(95,-25){$q_{\bit,C_\Li}$}


  \node(3C)(30,0){$\bullet$}
  \node(4OC)(45,10){$\bullet$}

  \put(47,12){\small$\prop_1$}
  \put(47,-12){\small$\prop_0$}
  \node(4UC)(45,-10){$\bullet$}
  \node(5C)(60,0){$\bullet$}

  \node(6C)(80,0){$\bullet$}
  \node(7OC)(95,10){$\bullet$}
  \put(97,12){\small$\prop_1$}
  \put(97,-12){\small$\prop_0$}

  \node(7UC)(95,-10){$\bullet$}
  \node(8C)(110,0){$\bullet$}

  \node(9C)(110,-20){$\bullet$}

\drawedge(8C,9C){$-2^{B_\Li}$}

  \put(112,-2){$\prop_2$}


  \put(65,-2){\huge$\cdots$}


%\drawedge[curvedepth=-70](8,1){$0$}

\drawedge(1C,2OC){$0$}
\drawedge[ELside=r](1C,2UC){$0$}
\drawedge[ELside=l,ELpos=60](2OC,3C){$-2^{B_\Li+\beta-1}$}
\drawedge(2UC,3C){$0$}

\drawedge[curvedepth=-6,ELpos=70,ELside=r](2OC,1AC){$0$}
\drawedge(2UC,1AC){$0$}

\drawedge[curvedepth=-6,ELpos=70,ELside=r](4OC,4AC){$0$}
\drawedge(4UC,4AC){$0$}

\drawedge[curvedepth=-6,ELpos=70,ELside=r](7OC,7AC){$0$}
\drawedge(7UC,7AC){$0$}



\drawedge(3C,4OC){$0$}
\drawedge[ELside=r](3C,4UC){$0$}
\drawedge[ELside=l,ELpos=70](4OC,5C){$-2^{B_\Li+\beta-2}$}
\drawedge(4UC,5C){$0$}

\drawedge(6C,7OC){$0$}
\drawedge[ELside=r](6C,7UC){$0$}
\drawedge[ELside=l,ELpos=60](7OC,8C){$-2^{B_\Li}$}
\drawedge(7UC,8C){$0$}



\end{picture}
\caption{The SOCA $\Soca_\mod$}{\label{F Smod}
for checking if $B\equiv C\ \mod M$.}
\end{center}
\end{figure}
Finally, let us give the formula $\varphi_\mod$:
$$
\varphi_\mod\quad=\quad\E \left(\bigwedge_{i\in\{0,1\}} \prop_i
\rightarrow \EX \varphi_{\bit,i} \right) \U (\prop_2\wedge\neg\EX\true)
$$
\qed
\end{proof}

\noindent
We need the following
proposition for implementing Q2. Its proof is straightforward.

\begin{proposition}{\label{P prime}}
The following problem is in $\PSPACE$:

INPUT: $\bin(i)\$\bin(m)$

QUESTION: $m=p_i$?
\end{proposition}


%\renewcommand{\prime}{\text{prime}}
\newcommand{\prim}{\text{prime}}

\begin{lemma}{\label{L prim}} There is some fixed $\ctl$ formula $\varphi_\prim$ such that
from  $x_0$ we can compute in logarithmic space some SOCA
$\Soca_\prim$ and some control state $q_\prim$ such that
$(T(\Soca_\prim),q_\prim(v))\models\varphi_\prim$ %\quad\Leftrightarrow\quad
if and only if $M=p_I$.
\end{lemma}
\begin{proof}
By Proposition \ref{P prime} there is some $q(n)$ space-bounded DTM
$\M=(S,\Sigma,s_0,F,\mu)$
over the alphabet $\Sigma=\{0,1,\$\}$ that decides,
given $\bin(i)\$\bin(m)$, whether $m=p_i$, where $q$ is some polynomial.
The idea is to simulate $\M$ on input $z=I\$M$ using $X$ from $(\bigstar)$ as a working tape.
For this, we define our block $X$ to consist of
$l=q(|I|+|M|+1)=q(\alpha+\beta+1)$ many bits.
Moreover, we assume w.l.o.g. that $\M$'s behaviour on input $I\$M$ is
independent of the content of its initial working tape: This can be
achieved by adding extra states to $\M$ that first write $0^l$ onto
the working tape.
Hence, the initial configuration of $\M$ on input $z$ is
$(s_0,\l z\r,\l w \r,0,0)$ for some $w\in\{0,1\}^l$.
Before we give our SOCA $\Soca_\prim$, we describe the computation of
$\M$ on input $I\$M$ as a pseudo program that terminates if and only if $M=p_I$:\\


\ \hspace{2cm} $s\in S$; $s:=s_0$;\quad  (current state of $\M$)\\[-0.4cm]

\ \hspace{2cm} $i\in[0,\alpha+\beta+2]$; $i:=0$;\quad (current input head position)

\ \hspace{2cm} $h\in[0,l+1]$; $h:=0$;\quad (current working head position)

\ \hspace{2cm} $a\in\{0,1,\$,\l,\r\}$; $a:=\l$;\quad (current input symbol)

\ \hspace{2cm} $b\in\{0,1,\l,\r\}$; $b:=\l$;\quad (current working tape
symbol)\\[0.05cm]

\ \hspace{2cm} $\mathtt{while }\  s\not\in F$\ $\mathtt{ loop}$\\[-0.4cm]

\ \hspace{3cm} $a:=\begin{cases} \l& \text{ if }i=0\\
\r &\text{ if }i=\alpha+\beta+2\\
\$ &\text{ if }i=\alpha+1\\
\bit_i(I) &\text{ if }i\in[1,\alpha]\\
\bit_{i-\alpha-1}(M) &\text{ otherwise }
\end{cases}$

\ \hspace{3cm} $b:=\begin{cases} \l& \text{ if }h=0\\
\r &\text{ if }h=l+1\\
\bit_{h}(X) &\text{ otherwise }
\end{cases}$



%\\[-0.4cm]

\ \hspace{3cm} Let $\mu(s,a,b)=(s',\delta_1,\delta_2,b')$.\\[-0.4cm]

\ \hspace{3cm} $\mathtt{if}\ h\in[1,l]\ \ \mathtt{then}\  \ \bit_h(X):=b';\ \ \mathtt{fi}$\\[-0.4cm]

\ \hspace{3cm} $s:=s '$;\\[-0.4cm]


\ \hspace{3cm} $i:=i+\delta_1$;\\[-0.4cm]

\ \hspace{3cm} $h:=h+\delta_2$;\\[-0.4cm]

\ \hspace{2cm} $\mathtt{end loop}$\\[-0.2cm]

\newcommand{\inpu}{\text{input}}
\newcommand{\inz}{\text{input}_0}
\newcommand{\ino}{\text{input}_1}
\newcommand{\work}{\text{work}}
\newcommand{\workz}{\text{work}_0}
\newcommand{\worko}{\text{work}_1}


\noindent
Let us describe our SOCA $\Soca_\prim$.
The control states $Q$ of $\Soca_\prim$ will contain
$$
S\times[0,\alpha+\beta+2]\times[0,l+2]\times
\{0,1,\$,\l,\r\}\times\{0,1,\l,\r\}, \qquad\text{where}
$$
the first component corresponds to the variable $s$,
the second component corresponds to the variable $i$,
the third component corresponds to the variable $h$,
the fourth component corresponds to the variable $a$, and
finally the fifth component corresponds to the variable $b$
of the above program.
Moreover, our SOCA will contain
the SOCA $\Soca_{\bit,I_\Li},\Soca_{\bit,I_\Li+1}\ldots,\Soca_{\bit,I_\Re}$ and
$\Soca_{\bit,X_\Li},\Soca_{\bit,X_\Li+1}\ldots,\Soca_{\bit,X_\Re}$ in order to test if
certain bits
in $I$ and $X$ are set correctly.
Hence $Q$ contains the control states
$q_{\bit,I_\Li},q_{\bit,I_\Li+1}\ldots,q_{\bit,I_\Re}$ and
$q_{\bit,X_\Li},q_{\bit,X_\Li+1}\ldots,q_{\bit,X_\Re}$ as well.

\noindent
We will provide the atomic propositions
$\Prop=\{F,\inpu,\work,\inz,\ino,\workz,\worko\}$,
where
\begin{eqnarray*}
Q_F&\ =\ &\{(s,i,h,a,b)\in Q\mid s\in F\},\\
Q_\inpu&\ =\ &\{q_{\bit,k}\mid k\in[I_\Li,I_\Re]\},\\
Q_\work&\ =\ &\{q_{\bit,k}\mid k\in[X_\Li,X_\Re]\},\\
Q_{\inz}&\ =\ &\{(s,i,h,a,b)\in Q\mid a=0\},\\
Q_{\ino}&\ =\ &\{(s,i,h,a,b)\in Q\mid a=1\},\\
Q_{\workz}&\ =\ &\{(s,i,h,a,b)\in Q\mid b=0\},\\
Q_{\worko}&\ =\ &\{(s,i,h,a,b)\in Q\mid b=1\}.\\
\end{eqnarray*}
We do not give all transitions of $\Soca_\prim$, but illustrate
some cases by way of example.
Let us give the outgoing transitions of the control state
$(s,i,h,0,1)$, where $i\in[I_\Li,I_\Re]$ and $h\in[X_\Li,X_\Re]$,
i.e. we currently scan the $i^{\text{th}}$ bit
of $I$ and do not read a marker on our working tape.
Moreover, let us assume $\mu(s,0,1)=(s',+1,-1,0)$, i.e. we change to state $s'$,
we move the input head to the right, the working head to the left,
and modify the current content on the working tape from $1$ to $0$.
We realize the latter by subtracting $2^{X_\Li+h}$ from the counter,
 however we allow transitions
to states $(s',i+1,h-1,a,b)$, {\em for each} $a,b\in\{0,1\}$, hence we guess
the input tape symbol and the working tape symbol of the
successor configuration.
The $\ctl$ formula $\varphi_\prim$ will guarantee that our guessing was correct by
accessing the control states $q_{\bit,k}$ where
$k\in[I_\Li,I_\Re]\cup[M_\Li,M_\Re]$.

\begin{center}
\begin{picture}(85,73)(-20,4)
\gasset{Nframe=n,loopdiam=9,ELdist=.7}
\gasset{Nadjust=wh,Nadjustdist=1}
\gasset{curvedepth=0}


\put(-20,35){\drawpolygon[Nframe=n,Nfill=y,fillgray=.9](5,20)(28,20)(28,40)(5,40)}

\put(50,35){\drawpolygon[Nframe=n,Nfill=y,fillgray=.9](5,20)(28,20)(28,40)(5,40)}


\put(-12,68){\large$\Soca_{\bit,I_\Li+i}$}
\node(I)(-3,60){$q_{\bit,I_\Li+i}$}
  \put(-12,57){\tiny\bf$\inpu$}


\put(58,68){\large$\Soca_{\bit,X_\Li+h}$}
\node(X)(67,60){$q_{\bit,X_\Li+h}$}
  \put(72,57){\tiny\bf$\work$}



  \node(A)(30,40){$(s,i,h,0,1)$}
  \put(39,40){\tiny\bf$\inz,\worko$}
  \node(B1)(-35,10){$(s,i+1,h-1,0,0)$}
  \put(-40,6){\tiny\bf$\inz,\workz$}

  \put(0,6){\tiny\bf$\inz,\worko$}
  \put(40,6){\tiny\bf$\ino,\workz$}
  \put(80,6){\tiny\bf$\ino,\worko$}

  \node(B2)(5,10){$(s,i+1,h-1,0,1)$}
  \node(B3)(45,10){$(s,i+1,h-1,1,0)$}
  \node(B4)(85,10){$(s,i+1,h-1,1,1)$}
\drawedge[ELside=r](A,B1){$-2^{X_\Li+h}$}
\drawedge[ELside=r,ELpos=70](A,B2){$-2^{X_\Li+h}$}
\drawedge[ELside=r](A,B3){$-2^{X_\Li+h}$}
\drawedge[ELside=l](A,B4){$-2^{X_\Li+h}$}

\drawedge[ELside=l](A,I){$0$}
\drawedge[ELside=l](A,X){$0$}

\end{picture}
\end{center}
The other cases can be dealt with analogously.
We put $q_\prim=(s_0,0,0,\l,\l)$.
Our final formula is
$$
\varphi_\prim\quad=\quad\E
\left(
\bigwedge_{k\in\{0,1\}}
\left(\text{input}_k\rightarrow\EX(\inpu\wedge\varphi_{\bit,k})\right)
\wedge
\left(\text{work}_k\rightarrow\EX(\work\wedge\varphi_{\bit,k})\right)
\right)\U\ F.
$$
 Then
$(T(\Soca_\prim),q_\prim(v))\models\varphi_\prim$
if and only if
$I\$M\in L(\M)$ if and only if
$M=p_I$.
\qed
\end{proof}

\newcommand{\BIT}{\text{BIT}}

\noindent
Let us now give an implementation of query Q3.

\begin{lemma}{\label{L BIT}}
For each bit $b\in\{0,1\}$ there exists a fixed $\ctl$ formula $\varphi_{\BIT,b}$
such that from $x_0$
we can compute in logarithmic space some SOCA $\Soca_{\BIT}$ and some
control state $q_\BIT$ such that
$(T(\Soca_\BIT),q_\BIT(v))\models\varphi_{\BIT,b}$
if and only if $\bit_J(B)=b$.
\end{lemma}
\renewcommand{\C}{\mathcal{C}}
\renewcommand{\R}{\mathcal{R}}
\begin{proof}[Sketch]
By Lemma \ref{L prim}, there is a fixed $\ctl$
formula $\varphi_\prim$ such that we can compute from $x_0$ a
SOCA $\Soca_\prim$ that allows to check if $M=p_I$.
Moreover, by Lemma \ref{L mod} there is a fixed $\ctl$
formula $\varphi_\mod$ and a SOCA $\Soca_\mod$ that allows
to check whether $B\equiv C \mod\ M$.
In other words, we can deal with the Chinese remainder representation
of $B$, but our goal is to access the the $J^{\text{th}}$ bit in the
binary representation of $B$.
So let us assume that $\R=\CRR_{K}(B)=(b_{i,c})_{1\leq i\leq K,0\leq c<p_i}$
is the Chinese remainder representation of $B$.
Note that we do not have $\R$ stored anywhere on the counter.
However, the bit strings $I$ and $C$ serve as pointers to access the
bit $b_{I,C}$ of $\R$.
By Theorem \ref{T CRRBIN}, given $\R$ (on-the-fly by our pointers $I$ and $C$), our
bit string $J$, and the bit $b$,
we can decide if $\bit_J(B)=b$ in logarithmic space.
So let $\M=(S,\Sigma,s_0,F,\mu)$ be some $k\cdot\log(m)$ space-bounded
Turing machine machine for this, where $k\geq 1$ is some constant.
Hence, in order to decide if
$\bit_J(B)=b$ we need to simulate $\M$ on input $\langle \R, J, b\rangle$.
For this we store the space that $\M$ requires on our reserved
sequence of bits $Y$ from $(\bigstar)$, hence $Y$ consists of
$l=k\cdot\log(|\R Jb|)=O(\beta^2+\alpha+1)=O(N^2)$ many bits.
The definition of SOCA $\Soca_{\BIT,b}$ works analogously to the construction
of $\Soca_{\prim}$ in the proof of Lemma \ref{L prim}
i.e. we introduce in $\Soca_{\BIT,b}$ control states
that remember the current bit of the input head, the current bit of
the working head, and the position of the working head.
The only difference is that a pointer to the input head cannot be
completely stored
in the control states.
As mentioned above, for this we employ the bit blocks $I$ and $C$,
and also the sequence $M$ for storing the $I^{\text{th}}$ prime number.
In order to obtain bit $b_{I,C}$, we allow transitions to our SOCA $\Soca_\mod$
(which we can compute in logarithmic
space from $x_0$ by Lemma \ref{L mod}) and checking
(i) via the fixed formula $\varphi_\prim$ whether $M=p_I$, and
then (ii) via the fixed formula $\varphi_{\mod}$
whether $B\equiv C\ \mod M$.
Pointers to the remaining parts of the input for $\M$, namely $J$ and $b$,
can directly be handled by the control states of $\Soca_{\BIT,b}$,
in analogy to the proof of Lemma \ref{L prim}.
By way of example we explain the behavior of $\Soca_{\BIT}$ when the input head
of $\M$ currently scans bit $b_{I,C}$ of $\R$ and when getting
to the successor configuration requires moving the input head to the left.
Then we simply decrement $C$ by $1$ which corresponds to subtracting $2^{C_\Li}$
from the counter. If, however, $C$ currently equals $0$, we need to
decrement $I$ by $1$, overwrite $M$ with prime $p_{I-1}$, and finally
overwrite $C$ with $p_{I-1}-1$. Decrementing $I$ by $1$ can simply achieved by
subtracting $2^{I_\Li}$ from the counter. Overwriting $M$ with prime $p_{I-1}$
can be done by repeatedly subtracting $1$ from $M$ (i.e. subtracting $2^{M_\Li}$
from the counter) until $M$ equals $p_{I-1}$;
checking if $M=p_{I-1}$ can be done via the fixed $\ctl$ formula $\varphi_{\prim}$
and the SOCA $\Soca_{\prim}$ by Lemma \ref{L prim}.
The other cases work analogously.
\qed
\end{proof}

\noindent
The following lemma implements query Q4.

\begin{lemma}{\label{L U}}
For each $b\in\{0,1\}$ there is some fixed $\ctl$ formula $\varphi_{U,b}$ such that
from $x_0$
we can compute in logarithmic space some SOCA $\Soca_U$ and some
control state $q_U$ such that $(T(\Soca_U),q_U(v))\models\varphi_{U,b}$
if and only if $b=\chi_U(B[1,n],B[n+1,n+2^N])$.
\end{lemma}
\begin{proof}[sketch]
The proof is similar to the proof of Lemma \ref{L BIT}.
Since $U$ is in $\L$, there is some $k\cdot\log(m)$-space bounded
Turing machine that decides $U$, where $k\geq 1$ is some constant.
In order to decide if $b=\chi_U(B[1,n],B[n+1,n+2^N])$ we
need to simulate $\M$ on input $\langle B[1,n],B[n+1,n+2^N]\rangle$.
We store the space that $\M$ requires on its working tape
in the sequence of bits $Z$ from $(\bigstar)$, hence $Z$ consists of
$k\cdot\log(n+2^N+1)=O(N)$ many bits.
We use the bit sequence $J$ as a pointer for accessing bits of $B$.
For reading the $J^{\text{th}}$ bit of $B$ we make use of the $\ctl$ formula
$\varphi_{\BIT,b}$ and use the SOCA $\Soca_\BIT$ from Lemma \ref{L BIT}.
The rest of the proof follows along the same lines as the proof of
Lemma \ref{L prim}.
\qed
\end{proof}



\noindent
The following lemma implements query Q5 and concludes the
$\EXPSPACE$-hardness of the data complexity of $\ctl$ on SOCA.

\begin{lemma}
There is some fixed $\ctl$ formula $\varphi_L$ such that from $x_0$
we can compute in logarithmic space some SOCA $\Soca_L$ and some
control state $q_L$ such that $(T(\Soca_L),q_L(0))\models\varphi_L$
if and only if $x_0\in L$.
\end{lemma}
\begin{proof}
First note that our $\ctl$ formula $\varphi_L$ will be evaluated in state $q_L(0)$.
Recall that our bit sequence $B$ has length $n+2^N+1$ where $B$ is factorized
as $B=xB'$
for some $x\in\{0,1\}^n$ and some $B'\in\{0,1\}^{2^N+1}$.
The SOCA $\Soca_L$ and the $\ctl$ formula $\varphi_L$ will mimic the execution
of the program from Figure \ref{F U}.
In the bit string $x$ we store the value $x_0$.
The bit string $B'$ represents
the variable $d$ of the program, hence we will initialize
$B'$ with $0$.
Note that incrementing $B'$ by $1$ corresponds to adding $2^{B_\Li+n}$ to the counter.
Thus, checking when $d$ becomes $2^{2^N}$ for the first time boils down to checking
when the $n+2^N+1^{\text{st}}$ bit of $B$ becomes $1$ for the first time.
 By Lemma
\ref{L BIT} the $\ctl$ formula $\varphi_{\BIT,1}$ and the SOCA $\Soca_\BIT$ allow to test
if the $J^{\text{th}}$ bit of $B$ equals $1$.
Therefore we store in $J$ the number $n+2^N+1$.
The following claim tells us that we can test whether we have
initialized the counter correctly.
Its proof is simple and therefore omitted.
\newcommand{\init}{\text{init}}

\medskip

\noindent
{\em Claim:} There is a fixed $\ctl$ formula $\varphi_\init$ such that from
$x_0$ we can compute in logarithmic space some SOCA $\Soca_\init$ and
some control state $q_\init$ such that
$(T(\Soca_\init),q_\init(v))\models\varphi_\init$
if and only if $J=n+2^N+1$ and $B=x_00^{2^N+1}$.\\

\noindent
Recall that $A=(Q,\{0,1\},q_0,\delta,F)$ is the deterministic finite automaton
of the program in Figure \ref{F U} that needs to be simulated.
Let us, before giving $\Soca_L$, define the auxiliary SOCA $\Soca_A$
to be connected with the control state $q_U$ of $\Soca_U$ and with the control
state $q_\BIT$ of $\Soca_\BIT$ along with the additional
control states $S=Q\times\{0,1\}$ and
the following transitions:
$$
(q,b)\xrightarrow{+2^{B_\Li+n}}(\delta(q,b),b'),\qquad
(q,b)\xrightarrow{0}q_\BIT,\qquad\text{and}\qquad
(q,b)\xrightarrow{0}q_U\qquad\text{for each }
b,b'\in\{0,1\}.
$$

\noindent
Moreover, $\Soca_A$ contains the atomic propositions
$\{\bit_0,\bit_1,F\}$, where $S_{\bit_b}=Q\times\{b\}$ for each $b\in\{0,1\}$ and
$S_F=F\times\{0,1\}$.
Before giving $\varphi_L$, let us depict our SOCA $\Soca_L$ which has
two additional labels $\rho$ and $\init$:

\begin{center}
\begin{picture}(85,53)(-40,-12)
%   \put(-20,-28){\framebox(85,33){}}
  %\gasset{Nadjust=w,Nadjustdist=2,Nh=6,Nmr=1}
 % \gasset{Nadjust=w,Nadjustdist=2,Nh=6}%$,Nmr=1}
 % \node(A)(0,0){$q_0$}


\gasset{Nframe=n,loopdiam=9,ELdist=.7}
\gasset{Nadjust=wh,Nadjustdist=1}
\gasset{curvedepth=0}

\put(-13,0){
\drawpolygon[Nframe=n,Nfill=y,fillgray=.9](35,20)(58,20)(58,40)(35,40)
\put(48.5,28){\Large$\Soca_{\init}$}
}

\put(-51,-35){
\drawpolygon[Nframe=n,Nfill=y,fillgray=.9](35,20)(68,20)(68,40)(35,40)
\put(48.5,25){\Large$\Soca_A$}
}



\put(0,23){\small $\prop$}
\put(25,33){\small $\init$}
  \node(A)(-30,20){$q_L$}
  \node(B)(0,20){$\bullet$}
  \node(C)(30,30){$q_\init$}
  \node(D1)(-10,0){$(q_0,0)$}
  \node(D2)(10,0){$(q_0,1)$}
\drawloop(A){$+1$}
\drawedge(A,B){$0$}
\drawedge(B,C){$0$}
\drawedge[ELside=r](B,D1){$0$}
\drawedge(B,D2){$0$}
\end{picture}
\end{center}
Let us define the auxiliary formula $\chi=\EX\varphi_\BIT$ that allows
to test if $B'=2^{2^N}$.
Our final formula is
$$
\varphi_L\quad=\quad\EF\left(\rho\wedge \EX(\init\wedge\varphi_\init)\wedge
\E\left(\neg\chi\wedge\bigwedge_{b\in\{0,1\}}\bit_b\rightarrow
\EX\varphi_{U,b}\right)
 \U\ (\chi\wedge F)
\right)
$$
We have $x_0\in L$ if and only if
the program from Figure \ref{F U} returns $\mathtt{true}$ if and only if
$(T(\Soca_L),q_L(0))\models\varphi_L$.
\qed
\end{proof}



\begin{theorem}
The data and combined complexity of $\ctl$ and the modal $\mu$-calculus
 on SOCA is $\EXPSPACE$-complete.
\label{thm:ctl-poca}
\end{theorem}


\iffalse
\section{$\EF$ model checking is in $\PSPACE$}


For a finite and non-empty set of integers $S\subseteq \Z\setminus\{0\}$ not
containing $0$, we denote by {\em $\gcd(S)$} the greatest natural number that
divides all numbers $\{|a|: a\in S\}$.
A {\em linear equation} over $\Z$ is given by
$$
x\quad =\quad a_1\cdot x_1 + \cdots\ + a_k\cdot x_n,
$$
where $a_i\in\Z$ is a {\em coefficient} and $x_i$ is a {\em variable}
ranging over $\N$ for each $i\in[n]$.
We call $b\in\Z$ a {\em solution} to the above linear equation if there exists a
mapping $\sigma:\{x_1,\ldots,x_n\}\rightarrow\N$ such that
$b=\sum_{i\in[n]} a_i\cdot\sigma(x_i)$.
An {\em arithmetic progression} with {\em offset} $a\in\Z$ and {\em period
$b\in\Z$} is the set $a+b\N=\{a+b\cdot\N\mid n\in\N\}$.
The following lemma has been used in \cite{AnthonyIPL,Chrobak,Shallit}.

\begin{lemma}[\cite{lookup citation}]\label{L NT}
Let $N\in\N$ and let
\begin{eqnarray}
x\quad=\quad a_1\cdot x_1 + \cdots\ + a_n\cdot x_n,\label{L Eq}
\end{eqnarray}
be a linear equation such that $1<a_1<\cdots a_n\leq N$ and where
$b=\gcd(a_1,\ldots,a_n)$.
Then, the set of solutions to (\ref{L Eq}) is some set
$U\cup\ a+b\N$,
where $U\subseteq[1,N^2]$,
and
$a=\min\{y>N^2\mid y\equiv 0\textsl{ mod }g\}$.
\end{lemma}


\begin{lemma}
Let $N\in\N$ be some natural and let
\begin{eqnarray}
x\quad=\quad a_1\cdot x_1 + \cdots\ + a_n\cdot x_n,\label{L Eq neg}
\end{eqnarray}
be a linear equation
where $-N\leq a_1<a_2<\cdots <a_n\leq N$ and where $g=\gcd(a_1,\ldots,a_n)$.
Let $b=\gcd(a_1,\ldots,a_n)$ and let
$a=\min\{y>N^2\mid y\equiv 0\textsl{ mod }b\}$.
Then, the set of positive solutions to (\ref{L Eq neg}) is a set
$U\cup\  a+ b\N$, where $U\subseteq[1,N^2]$ and the
set of negative solutions to (\ref{L Eq neg}) is a set
$V\cup -a-b\N$, where $V\subseteq[-N^2,-1]$.
\end{lemma}
\begin{proof}
We only prove the lemma for the negative solutions.
Let $c<0$ be an arbitrary negative solution. Thus,
$c=d+a_1'\cdot m_1+\ldots +a_h'\cdot m_h$, where
$d\in[-N,0]$ and
$\{a_1',\ldots,a_h'\}\subseteq\{a_1,\ldots,a_n\}\cap\Z^-$ is a subset of
the negative coefficients.
Hence $c$ is the solution
to some linear equation $x=b_1\cdot x_1+\ldots +b_l\cdot x_l$, where
$\{b_1,\ldots,b_l\}\subseteq[-N,-1]$. By application of Lemma
\ref{L Eq} and switching from positive integers to negative integers, the lemma
follows.
\qed
\end{proof}


\begin{lemma}
  \label{lem:bounded-coefficients}
  Let $N,m$ be natural numbers and let $a_1,\ldots, a_n$ be integers
  such that $-N\le a_1 \le \ldots \le a_n\le N$, $m>N^3$ and
  \begin{equation*}
    m = a_1 \cdot x_1 + \cdots + a_n \cdot x_n
  \end{equation*}
  for some natural numbers $x_1, \ldots, x_n$. Then there exist
  natural numbers $y_1, \ldots, y_n$ such that
  \begin{equation*}
    m = a_1 \cdot y_1 + \cdots + a_n \cdot y_n
  \end{equation*}
  and $y_i\le N^2$ for $1\le i<n$.
\end{lemma}
\begin{proof}
  Let $a_\ell\in\{a_1,a_n\}$ be the $a_i$ with the largest
  \emph{absolute} value. Define $x_i' =x_i \text{ mod } a_\ell$ for
  $1\le i\le n, i\neq \ell$, and let $m'= \sum_{1\le i\le n, i\neq
    \ell} a_i\cdot m_i$. Clearly, we have $m'<N^3$ and $m'\equiv m
  \text{ mod } a_\ell$. Let $h=m-m'=a_\ell \cdot x_\ell'$ for some
  integer $x_\ell'$. In case $a_\ell=a_n$, the lemma follows by
  setting $y_i=x_i'$ for $1\le i\le n$. Otherwise, let $x_n^*= \lceil
  \frac{h}{|a_1a_n|} \rceil$. The lemma follows by setting $y_i=x_i'$
  for $1<i<n$, $y_1=\frac{x_n^*|a_1a_n|-h}{|a_1|}$ and $y_n=x_n'+
  x_n^*|a_1|$.
\end{proof}
\begin{remark}
  \label{rem:maximum-valley}
  It follows from the previous lemma that $\sum_{a_i<0} a_i\cdot
  y_i>-N^3$.
\end{remark}

\section{Periodicity analysis}

\newcommand{\cycles}{\textrm{cycles}}

Let us fix a SOCA $\O=(Q,\{Q_p\mid p\in\Prop\},\delta_0,\delta_{>0})$ for
 the rest of this section.

In the following, we will treat $\O$ as a weighted directed graph $G(\O)$, where
the weights are taken according to $\delta_{>0}$.
More precisely, we define $G(\O)=(Q,\delta_{>0})$
and write $q_1\stackrel{a}{\mapsto}q_2$ as an abbreviation for
$(q_1,a,q_2)\in\delta_{0}$ and simply $q_1\mapsto q_2$ if
$(q_1,a,q_2)\in\delta_{>0}$ for some $a\in\Z$.
Note that every non-zero path in the transition system $T(\O)$ corresponds to a unique path in
$G(\O)$.
A {\em strongly connected component} of $G(\O)$ is an
equivalence class of the equivalence relation $\mapsto^*$.
For every finite path $\pi$ in $G(\O)$, let $w(\pi)$ denote the sum of all
weights of edges of $\pi$.
For every strongly connected component $C$ in $G(\O)$, let $\cycles(C)$ denote
the set of chord-free cycles whose total weight does not summarize to $0$,
more precisely
the set of paths $\pi=q_0\stackrel{m_1}{\mapsto} q_2\cdots
\stackrel{m_l}{\mapsto} q_l$, where $l\geq1$,
$q_0=q_l$, such that $i\not= j$ implies $q_i\not=q_j$ for all
$i,j\in[l]$ and moreover $\sum_{i\in[l]}m_i\not=0$.


Let $M=\max\{|a|:\exists q_1,q_2\in Q. q_1\stackrel{a}{\mapsto}q_2\}$
denote the maximal absolute value of a weight that appears in $G(\O)$.
Moreover, let $g_C=\gcd(w(cycles(C)))$ denote the greatest common divisor
of all non-zero weights of chord-free cycles in $C$.
In the following define $$\Delta\ =\ \LCM\{g_C\mid C\text{ is a strongly
connected component of } G(\O)\}.$$
\noindent
Note that $\Delta\in\exp(|\O|)$ since firstly $g_C\leq (k-1)\cdot M$ for each strongly
connected component $C$ and secondly there are at most $|Q|$
strongly connected components in $G(\O)$.
Let $k=|Q|$.


\begin{lemma}{\label{L Cycles}}
Let $C$ be some strongly connected component of $G(\O)$, let $q\in Q$ be some
control location, and let $n\geq k^2\cdot M$ be some natural.
Then in $T(\O)$ there exists some non-zero path from $(q,n)$ of height $h$ with $h\equiv
0\text{ mod }g_C$ and $|h|\leq k\cdot(k-1)\cdot M$  that visits each
control location in $C$ at least once.
\end{lemma}
\begin{proof}
Since $C$ is a strongly connected component and $n\geq k^2\cdot M$, there exists some
non-zero path
$$
\pi\quad =\quad(q_1,n_1)\rightarrow(q_2,n_2)\ \cdots\ \rightarrow (q_l,n_l)
$$
in $T(\O)$ with $l\in[k(k-1)]$ that visits each control state in $C$ at least once
(i.e. for all $p\in C$ there is some $i\in[l]$ such that $p=q_i$) such that
moreover $q_1=q_l=q$ and $n_1=n$. Thus, it follows $|n_1-n_l|\leq k\cdot
(k-1)\cdot M$.
Observe that for every subpath $\pi[a,b]=(q_a,n_a)\cdots\rightarrow(q_b,n_b)$ of $\pi$ that is a chord-free
cycle, we have $n_a\equiv n_b\text{ mod } g_C$ by definition of $g_C$.
By successively cutting out subpaths of $\pi$ that correspond to chord-free
cycles, it is apparent that $n_1\equiv n_l\text{ mod }g_C$.
\qed
\end{proof}

Observe that Lemma \ref{L Cycles} implies that there is a non-zero path in
$T(\O)$ that visits every
control location in $C$ at least once from $(q,k^2\cdot M+\delta)$ with the
desired properties with respect to the height.

\begin{lemma}
  Let $q\in Q$ be a state of some SCC $C$ and let $m,m'$ be natural
  numbers such that $m-m'>k^2\cdot M+ k^3\cdot M^3$ and $m'>k\cdot
  M$. Then $(q,m)\rightarrow^*(q,m')$ iff
  $(q,m+\Delta)\rightarrow^*(q,m')$.
\end{lemma}
\begin{proof}
  $(\Rightarrow)$ Let $q_1,\ldots,q_n\in C, 1\le n\le k$ be states on
  loop-free cycles $\pi_{a_i}:q_i\rightarrow \ldots \rightarrow q_i$
  with weights $a_i=w(\pi_{a_i})$ such that
  $\text{gcd}(a_1,\ldots,a_n)=g_C$.  Let $\pi_i^1: q\rightarrow^* q_i$
  and $\pi_i^2: q_i\rightarrow^* q$ be disjoint loop-free paths from
  $q$ to $q_i$ respectively $q_i$ to $q$. Denote by
  $\pi_i=\pi_i^1\cdot \pi_i^2$ a loop-free cycle from $q$ to itself
  passing through $q_i, 1\le i\le n$.

  Let $h=m-m'-\sum_{1\le i\le n}w(\pi_i)+\Delta$. Similarly to the
  previous lemma, we find $g_C|m-m'$, $g_C|\sum_{1\le i\le n}w(\pi_i)$
  and hence $g_C|h$. Moreover, $h>k^3\cdot M^3$ and thus we can derive
  from Lemma \ref{lem:bounded-coefficients} that there are
  $y_1,\ldots, y_n$ such that $\sum_{1\le i\le n}a_i\cdot y_i=h$ and
  $y_i<k\cdot M$, $1\le i<n$.

  Now define $\pi_i'=\pi_i^1\cdot (\pi_{a_i})^{y_i} \cdot \pi_i^2$ and
  $\pi=\pi_1'\cdot\pi_2'\cdot\cdots\cdot\pi_n'$. Cleary,
  $w(\pi)=m+\Delta-m'$. Moreover, $m'> k\cdot M$ ensures that is a
  valid path. Hence $(q,m+\Delta)\rightarrow^*(q,m')$ is witnessed
  by $\pi$.

  $(\Leftarrow)$ The proof from the previous direction can directly
  be adopted to show the statement.
\end{proof}

The following lemma is important for our periodicity analysis later.

\begin{lemma}
Let $q,q'\in Q$ and $n,n'\in\N$ such that
$n-n'>4\cdot k^3\cdot M^2$ and
$n'>k\cdot M$.
If there is a path from $(q,n)$ to $(q',n')$ in $T(\O)$, then there
is a path from $(q,n+\Delta)$ to $(q',n')$ in $T(\O)$.
\end{lemma}
\begin{proof}
Let us fix some path
$$
\pi\quad=\quad(q_1,n_1)\rightarrow(q_2,n_2)\quad \cdots\quad \rightarrow (q_l,n_l),
$$
where $(q_1,n_1)=(q,n)$ and $(q_l,n_l)=(q',n')$.
Since $n_1-n_l>4\cdot k^3\cdot M^2$ and since there are at most $k$ strongly connected components in $G(\O)$,  there
exists a strongly connected component $C$ of $G(\O)$ and some subpath $\pi[s,t]$
of $\pi$ such that the following holds
\begin{enumerate}
\item  $n_s-n_t>4\cdot k^2\cdot M^2$,
\item $n_t\geq n'$, and
\item $q_i\in C$ for each $i\in[s,t]$
\end{enumerate}
by the pigeonhole principle.
Let $\delta=n_t-n_s<-4\cdot k^2\cdot M^2$ denote the height of the subpath $\pi[s,t]$.
Let us treat the path $\pi[s,t]$ as a path in $G(\O)$. Now think of consecutively cutting out
chord-free cycles in $\pi[s,t]$  until this is no longer possible, in a greedy
manner from beginning to the end, say. Let us denote the resulting
path (possibly empty) by $\rho$ and let  $h\in[-(k-1)\cdot M,(k-1)\cdot M]$ denote
its weight.
\begin{remark}{\label{R rho}}
The path $\rho$ can be seen as a non-zero path in $T(\O)$ of length at most
$k-1$ from $(q_s,n_s)$ to $(q_t,n_s+h)$.
\end{remark}

\noindent
The following claim states that the height $\delta$ of $\pi[s,t]$ lies in some
arithmetic progression of small offset and small period.


\smallskip

\noindent
{\em Claim A.} We have $\delta\in (c+h)-g_C\N$,
for some
$c\in[-3\cdot k^2\cdot M^2-M,-3\cdot k^2\cdot M^2]$ with $c\equiv 0\text{ mod }g_C$.

\medskip

\noindent
Let us postpone the proof of Claim A and finish the proof of the lemma.
It remains to show that there is some non-zero path
$(q,n+\Delta)\rightarrow^*(q',n')$ in $T(\O)$.
Before doing this, we claim that the following paths exist in $T(\O)$:
\begin{enumerate}[(1)]
\item $(q,n+\Delta)\rightarrow^*(q_s,n_s+\Delta)$.
\item $(q_t,n_t)\rightarrow^*(q',n')$.
\item $(q_s,n_t-h)\rightarrow^*(q_t,n_t)$.
\end{enumerate}
Point (1) is clear since $\pi[1,s]+\Delta$ witnesses this.
Point (2) holds by factorization of $\pi$.
For Point (3), observe that $\rho$ is a non-zero path from $(q_s,n_s)$ to $(q_t,n_s+h)$
of height $h$ and of length at most $k-1$
by Remark \ref{R rho}.
Since $n_t\geq n'>(k-1)\cdot M\geq h$, there is a non-zero path
from $(q_t,n_t-h)$ to $(q_t,n_t)$, namely
$\rho+(-n_s+n_t-h)=\rho+(\delta-h)$.
 By combining Points (3) and (2) we obtain some non-zero path $(q_t,n-h)\rightarrow^*
(q',n')$.
Due to Point (1), it suffices to show that there is some
non-zero path from $(q_s,n_s+\Delta)$ to $(q_s,n_t-h)$.
In other words, we have to prove that in $G(\O)$ there is some path from
$q_s$ to $q_s$ of weight
\begin{eqnarray*}
w\quad=\quad n_t-h-(n_s+\Delta)\quad=\quad n_t-n_s-h-\Delta\quad=\quad\delta-h-\Delta
\end{eqnarray*}
such that moreover the weight of any prefix of this path is strictly bigger
than $-(n_s+\Delta)$.
By Claim A we have $w=\delta-h-\Delta=(c+h)-r\cdot g_C-h-\Delta=c-r\cdot
g_C-\Delta$ for
some $r\in \N$ and some $c\in[-3\cdot k^2\cdot M^2-M,-3\cdot k^2\cdot M^2]$ with
$c\equiv 0\text{ mod } g_C$.
%Since too $\Delta\equiv 0\text{ mod } g_c$, there exists some  $d\in\N$ such
%that
Since moreover $\Delta\equiv 0\text{ mod }g_C$, we have $w\equiv 0\text{
mod }g_C$.
Furthermore, we have $w<-3\cdot k^2\cdot M^2$.
Since $n_s+\Delta>BLA>k^2\cdot M$, there is some non-zero path $\sigma$ of height
$h_\sigma$ with $|h_\sigma|\leq k\cdot(k-1)\cdot M$ and $h_\sigma\equiv 0\text{
mod } g_C$
from $(q_s,n_s+\Delta)$ that visits each control location in $C$ at
least one by Lemma \ref{L Cycles} .
When taking the path $\sigma\circ\sigma$, the remaining weight is congruent $0$
modulo $g_C$ and can be estimated
as $$w-2\cdot h_\sigma\ <\ -3\cdot k^2\cdot M^2-2\cdot h_\sigma\ <
-3\cdot k^2\cdot M^2+2\cdot k\cdot (k-1)\cdot M<
-k^2\cdot M^2.
$$
Hence by Lemma \ref{L Eq neg} there exist $m_1,\ldots, m_n\in\N$ such that the
remaining weight equals $\sum_{i=1}^n a_i\cdot m_i$.
To finally obtain a path from $q_s$ to $q_s$ of weight $w$ in $G(\O)$, we
modify the first traversal of $\sigma$ by choosing for each positive $a_i$ some
chord-free cycle of weight $a_i$ through which we loop precisely $m_i$ many times, followed by
the modification of the second traversal of $\sigma$, where we choose for each
negative $a_i$ some chord-free cycle of weight $a_i$ through which we loop
precisely $m_i$ many times.
This concludes the proof of the lemma.

\smallskip

\noindent
{\em Proof of Claim A.}\\
\noindent
Observe that $\delta=h+r$, where $r$ is some solution
of the linear equation
$$
x\ =\ a_1\cdot x_1+\ \ldots\ +a_m\cdot x_m,
$$
where $\{a_1,\ldots, a_m\}\subseteq w(cycles(C))$ are weights of the chord-free cycles that
were cut out from $\pi[s,t]$ in order to obtain $\rho$. %\subseteq[-k\cdot M,k\cdot M]$.
Let $d=\gcd(a_1,\ldots,a_m)$ which is a multiple of $g_C$ by definition.
Since we know that $\delta<-4\cdot k^2\cdot M^2$ and $h\in[-(k-1)\cdot M,(k-1)\cdot M]$, we have
$$
r=\delta-h\quad<\quad-4\cdot k^2\cdot M^2+(k-1)\cdot M\quad\leq\quad -3\cdot k^2\cdot M^2.
$$
Moreover, we have obviously $-\sqrt{3}\cdot k\cdot M<-k\cdot M\leq a_i\leq
k\cdot M<\sqrt{3}\cdot k\cdot M$ for each $i\in[m]$.
Now let us define $c=\max\{y<-3\cdot k^2\cdot M^2\mid y\equiv 0\text{ mod }
d\}$.
As $d\in[M]$, it follows $c\in[-3\cdot k^2\cdot M^2-M,-3\cdot k^2\cdot M^2]$ as
required.
By applying Lemma \ref{L Eq neg} we have that $r\in c-d\N$ and hence
$\delta\in (c+h)-d\N$.
Since $d$ is a multiple of $g_C$
it follows $\delta\in (c+h)-g_C\N$.
\qed
\end{proof}

\begin{theorem}
Model checking EF over SOCA is $\PSPACE$-complete.
\end{theorem}

\fi

\newcommand{\Two}{\mathbb{A}}
\renewcommand{\add}{\mathsf{add}}

\subsection{$\Pi^0_1$-Hardness of the Data Complexity of $\CTL$ on POCA}{\label{S CTL und}}
In this section we show that there already
exists a fixed $\CTL$ formula for which model checking of
$\poca$ is $\Pi^0_1$-hard.
We reduce from the complement of the emptiness problem for
\emph{two-counter automata}, which is $\Sigma^0_1$-complete \cite{Min61}.
Similar to a
SOCA, a {\em two-counter automaton} is a tuple $\Two=(Q, E, \lambda)$,
where $Q$ is a finite set of {\em control states},
$E\subseteq Q\times Q$ is a set of {\em edges},
 however
$\lambda: E \rightarrow \{\zero_1,\zero_2\}\cup \{\add_j(a)\mid j\in \{1,2\}, a\in
\{-1,1\} \}$. The {\em configuration graph induced by $\Two$} is defined
to be $G(\A)=(Q\times \N\times \N, \rightarrow)$, where
$(p,i_1,i_2)\rightarrow (q,i_1+d_1,i_2+d_2)$ if and only if
the following conditions are all satisfied:
\begin{itemize}
\item $(p,q)\in E$,
%\item  $i_1,i_2,i_1+d_1,i_2+d_2\geq 0$,
\item $\lambda(p,q)=\add_j(a)$ implies $d_j=a$ and $d_{3-j}=0$ for each
$j\in\{1,2\}$, and
\item $\lambda(p,q)=\zero_j$ implies $d_j=i_j=0$ and $d_{3-j}=0$ for each
$j\in\{1,2\}$.
\end{itemize}

\noindent
Let us define the following problem.
\smallskip

\problemx{Emptiness for two-counter automata}{Two-counter automaton $\Two=(Q,E,\lambda)$, control states
$q_0,q_1\in Q$.}{Are there $m,n\in\N$ such that $(q_0,0,0)\rightarrow^*(q_1,m,n)$ in
$G(\Two)$?}

\medskip

\noindent
The idea of our reduction is as follows: Given a two-counter automaton
$\Two$, we construct a $\poca$ $\Poca(x)$ with one parameter in such a way
that the two counters from $\A$ are encoded into the single counter
from $\Poca(x)$ as follows: for the counter value $n$ of $\Poca(x)$,
we have that $n\ \mod x$ encodes the value of the first and $n\ \div\ x$ encodes the
value of the second counter of $\Two$.
Hence, testing the first counter for zero corresponds to checking
whether $n\equiv 0\ \mod x$ while testing the second counter for
zero corresponds to checking  whether $n\geq x$.
Incrementation (resp. decrementation) on the first counter can be
mimicked by adding (resp. subtracting) $1$, whereas on counter two this
corresponds to adding (resp. subtracting) $x$.
Of course, we need to ensure that we do not overflow.
For example, if $n\equiv -1\ \mod x$ and we would simulate to increment the first counter
in the above manner, this
would correspond to setting the first counter to zero and simultaneously
 incrementing the second counter. However, if the emptiness problem is
solvable, then $x$ can be instantiated with a large enough
value such that such an overflow does not occur.
Our $\ctl$ formula will ensure that we do not overflow.
\newcommand{\test}{\mathsf{test}}
Before we give the reduction from the emptiness problem for two-counter
automata, let us introduce some gadgets that we need in our reduction.
\begin{lemma}{\label{L POCAs}}
  There exists some fixed $\ctl$ formula $\varphi_\test$
for which the following holds:
\begin{enumerate}[(1)]
\item
There exists some fixed $\poca$ $\Poca_{\not\equiv 0}(x)$ with some control state
$q_{\not\equiv 0}$ such that for each $\sigma:\{x\}\rightarrow\N$
 $$(\Poca_{\not\equiv 0}^\sigma,q_{\not\equiv0}(n)) \models
 \varphi_{\test} \quad\Longleftrightarrow\quad n\not
  \equiv 0 \textsl{ mod } \sigma(x).$$
\item There exists some fixed $\poca$ $\Poca_{\not\equiv-1}(x)$ with some control
state $q_{\not\equiv -1}$ such that for each $\sigma:\{x\}\rightarrow\N$
 $$(\Poca_{\not\equiv -1}^\sigma,q_{\not\equiv-1}(n)) \models
 \varphi_\test \quad\Longleftrightarrow\quad n\not
  \equiv -1 \textsl{ mod } \sigma(x).$$
\item There exists some fixed $\poca$ $\Poca_{<x}(x)$ with some control
state $q_{<x}$ such that for each $\sigma:\{x\}\rightarrow\N$
 $$(\Poca_{< x}^\sigma,q_{<x}(n)) \models
 \varphi_\test \quad\Longleftrightarrow\quad n< \sigma(x).$$

\end{enumerate}
\end{lemma}
\begin{proof}
We choose $\varphi_\test=\neg\EF\prop$, where $\prop$ is an atomic
proposition.
Below we depict $\Poca_{\not\equiv0}(x)$,
$\Poca_{\not\equiv-1}(x)$, and $\Poca_{<x}(x)$, respectively.
\begin{center}
\begin{picture}(85,20)(-20,-3.5)
%   \put(-20,-28){\framebox(85,33){}}
  %\gasset{Nadjust=w,Nadjustdist=2,Nh=6,Nmr=1}
 % \gasset{Nadjust=w,Nadjustdist=2,Nh=6}%$,Nmr=1}
 % \node(A)(0,0){$q_0$}

\gasset{Nframe=n,loopdiam=9,ELdist=.7}
\gasset{Nadjust=wh,Nadjustdist=1}
\gasset{curvedepth=0}

\put(-65,0){
\put(41,-2){\small $\prop$}
  \node(A)(20,0){$q_{\not\equiv0}$}
  \node(B)(40,0){$\bullet$}
\drawloop(A){$-x$}
\drawedge(A,B){$\zero$}
}
%%% AUTOMATON 2

\put(-27,0){
\put(61,-2){\small $\prop$}
  \node(A)(20,0){$q_{\not\equiv-1}$}
  \node(B)(40,0){$\bullet$}
  \node(C)(60,0){$\bullet$}
\drawloop(B){$-x$}
\drawedge(A,B){$+1$}
\drawedge(B,C){$\zero$}
}

%%% AUTOMATON 3


\put(32,0){
\put(61,-2){\small $\prop$}
  \node(A)(20,0){$q_{< x}$}
  \node(C)(40,0){$\bullet$}
  \node(D)(60,0){$\bullet$}
\drawloop(A){$-1$}
\drawedge(A,C){$-x$}
\drawedge(C,D){$\zero$}
%\drawedge[ELside=r](C,D){$\zero$}
}

\end{picture}
\end{center}
Correctness of the lemma is immediate.
\qed
\end{proof}

\newcommand{\up}{\mathsf{up}}


\noindent
Let us fix some instance of the emptiness problem for
 two-counter automata:
Let $\langle\Two,q_0,q_1\rangle$ be an instance to the
emptiness problem, where
$\Two=(Q,E,\lambda)$ is some two-counter automaton.
We will give a fixed $\ctl$ formula $\varphi$ such that from
$\langle\Two,q_0,q_1\rangle$ we can construct some $\poca$ $\Poca(x)$ such that
$(q_0,0,0)\rightarrow^*(q_1,m,n)$ for some $m,n\in\N$ if and only if
$(\Poca^\sigma,q_0(0))\models\varphi$ for some $\sigma:\{x\}\rightarrow\N$.

The control states and transitions will of $\Poca(x)$ can be described by the
following graph transformation rule that maps each pair of control states
$(p,q)\in E$ of $\Two$ with $\lambda(p,q)=l$
 into corresponding control states in $\Poca(x)$, by
possibly accessing the different $\poca$ of Lemma \ref{L POCAs}.
\begin{center}
\begin{picture}(85,30)(30,-10)
%   \put(-20,-28){\framebox(85,33){}}
  %\gasset{Nadjust=w,Nadjustdist=2,Nh=6,Nmr=1}
 % \gasset{Nadjust=w,Nadjustdist=2,Nh=6}%$,Nmr=1}
 % \node(A)(0,0){$q_0$}

\gasset{Nframe=n,loopdiam=9,ELdist=.7}
\gasset{Nadjust=wh,Nadjustdist=1}
\gasset{curvedepth=0}

\put(60,6){\Huge$\hookrightarrow$}


%\put(41,-2){\small $\prop$}
  \node(A)(20,10){$p$}
  \node(B)(40,10){$q$}

\drawedge(A,B){$l$}


\put(109,12){\small$l$}


  \node(C)(90,10){$p$}
  \node(D)(110,10){$\bullet$}
  \node(E)(130,10){$q$}
  \node(F)(110,-10){$\test(l)$}


%\drawloop(A){$-x$}
\drawedge(C,D){$0$}
\drawedge(D,E){$\up(l)$}
\drawedge(D,F){$0$}
\end{picture}
\end{center}

\noindent
where
$$
\test(l)\quad =\quad \begin{cases} (\Poca_{\not\equiv 0}(x),q_{\not\equiv 0}) & \text{ if }\quad
l\in\{ \add_1(-1),\zero_1\}\\
(\Poca_{\not\equiv -1}(x),q_{\not\equiv -1})& \text{ if }\quad l=\add_1(+1)\\
(\Poca_{<x}(x),q_{<x})& \text{ if }\quad l=\zero_2,
\end{cases}
$$

\noindent
and where
$$
\up(l)\quad =\quad \begin{cases}  a & \text{ if }\quad
l=\add_1(a)\\
\circ x& \text{ if }\quad l=\add_2(\circ 1)
\text{ for some }\circ\in\{+,-\}\\
0& \text{ otherwise.}
\end{cases}
$$

\noindent
Let us moreover ensure that every control state of the kind
$q_{\tau}$ with $\tau\in\{\not\equiv 0,\not\equiv\! -1,\geq x\}$ has
the label $\rho_\tau$.
Before we give our final formula, let us introduce the following constraint
formulas that guarantee that an overflow never occurs:
\begin{eqnarray*}
\psi_1&\quad=\quad& \add_1(+1)\longrightarrow \EX(\rho_{\not\equiv-1}
\wedge\varphi_\test)\
\\
\psi_2&\quad=\quad&
\add_1(-1)\longrightarrow\EX(\rho_{\not\equiv 0}\wedge\varphi_\test)\\
\psi_3&\quad=\quad&
\zero_2\longrightarrow\EX(\rho_{<x}\wedge\varphi_{\test})\\
\psi_4&\quad=\quad&
\zero_1\longrightarrow\EX(\rho_{\not\equiv 0}\wedge\neg\varphi_\test)
\end{eqnarray*}

\noindent
Note that we do not need any constraint when changing the second counter.
Let us introduce a label $f$ at control state $q_1$.
Our final $\ctl$ formula $\varphi$ is
$$
\varphi\quad=\quad\E\left(\bigwedge_{i\in[4]}\psi_i\ \U\  f\right).
$$
\iffalse
The correctness of the reduction follows from the following equivalences:
\begin{itemize}
\item $(T(\Poca),(q_0,0))\models \neg \varphi$
\item for all $\sigma:\{x\}\rightarrow \N$ we have $(\Poca^\sigma(x),q_0(0))\not\models\varphi$
\item for some $\sigma:\{x\}\rightarrow\N$ we have $(\Poca^\sigma(x),q_0(0))\models\varphi$
\item for some $\exists\sigma:\{x\}\rightarrow\N$ there is some path
$r_0(0)\rightarrow r_1(n_1)\cdots r_k(n_k)$ such that
$r_0=q_0$, $r_k=q_1$, $r_k\in F$ and
$(T(\Poca^\sigma(x)),r_i(n_i)



%$(q_0,0)(q_1,n_1\mod\sigma(x),n_1\div\sigma(x))\cdots (q_k,n_k\mod\sigma(x),
%n_k\div \sigma(x))$ in $T(\Poca^\sigma(x))$ such that


\end{itemize}
Then there is an assignment
$\sigma$ such that $(T(\Poca^\sigma),(q_0,0))\models \varphi$. It is
easily verified that the witnessing path $(q,0)(q_1,n_1)\ldots
(q',n')$ in $T(\A'(\sigma))$ corresponds to the path
$(q,0,0)(q_1,n_1\mod \sigma(x), n_1\div \sigma(x))\ldots (q',n'\mod
\sigma(x), n'\div \sigma(x))$ in $T(\A)$. Conversely, if
$(q,0,0)\rightarrow^* (q,n_1,n_2)$ in $T(\A)$, then the first counter
of $\A$ does not exceed some value $u \in \naturals$. Consequently, by
setting $\sigma(x):=u$, $(T(\A(\sigma)),(q,0))\models \varphi$.
\fi
We have that $(q_0,0,0)\rightarrow^*(q_1,m,n)$ for some $m,n\in\N$ in $G(\A)$  if and only
if
$(T(\Poca^\sigma(x)),q_0(0))\models\varphi$ for some
$\sigma:\{x\}\rightarrow\N$.
We obtain the following theorem.

\begin{theorem}
  The data and combined complexity of $\CTL$ and the modal $\mu$-calculus
 on POCA is $\Pi^0_1$-complete.
\end{theorem}




\section{$\ltl$ Model Checking}{\label{S LTL}}

%\subsection{Linear-time logic ($\ltl$)}
Formulas of $\ltl$ are inductively defined according to the
following grammar, where $\prop$ ranges over $\Prop$:
\begin{equation*}
  \varphi\quad ::=\quad \prop \mid \neg \varphi \mid \varphi \wedge \varphi \mid
  \X \varphi \mid \varphi \until \varphi
\end{equation*}
The Boolean abbreviations and the formula $\true$ are defined in the same way as
in CTL. The \emph{finally modality} $\eventually
\varphi$ is an abbreviation for $\true \until \varphi$ and the
\emph{globally modality} $\globally \varphi$ abbreviates $\neg
\eventually \neg \varphi$.

The semantics of LTL is given in terms of infinite paths in a
transition system. Let $T=(S,\{S_\prop \mid \prop \in
\Prop\},\rightarrow)$ be a transition system, let $\pi: s_0
\rightarrow s_1 \rightarrow \cdots$ be an infinite path in $T$, and
let $\varphi$ be an $\ltl$ formula, we define $(T,\pi)\models\varphi$
by induction of the structure of $\varphi$, where $\prop$ ranges over
$\Prop$:
\begin{align*}
  (T,\pi) \models  \prop &  \quad\iff\quad  \pi(0) \in S_\prop\\
  (T,\pi) \models  \neg \varphi &\quad\iff\quad (T,\pi) \not \models \varphi\\
  (T,\pi) \models \varphi_1 \wedge \varphi_2 &\quad\iff\quad (T,\pi)
  \models \varphi_1 \text{ and } (T,\pi) \models \varphi_2\\
  (T,\pi) \models \X \varphi & \quad\iff\quad (T,\pi^1) \models \varphi\\
  (T,\pi) \models  \varphi_1 \until \varphi_2 & \quad\iff\quad
  \exists j\ge 0: (T,\pi^j)\models \varphi_2
  \text{ and } \forall 0\le i<j: (T,\pi^i)\models \varphi_1
\end{align*}
Let us now define the $\ltl$ model checking problem on SOCA
and POCA.\\

\problemx{$\ltl$ model checking on SOCA}
{SOCA $\Soca=(Q,\{Q_\prop\mid\prop\in\Prop\},E,\lambda)$, $q\in Q$, $n\in\N$ in binary,
  and an $\ltl$ formula $\varphi$.}
{$(T(\Soca),\pi)\models\varphi$ for every infinite
path $\pi$ with $\pi(0)=q(n)$?}
\

\problemx{$\ltl$ model checking on POCA}
{POCA $\Poca(X)=(Q,\{Q_\prop\mid\prop\in\Prop\},E,\lambda)$, $q\in Q$, $n\in\N$ in binary, and an
$\ltl$ formula $\varphi$.}
{$(T(\Poca^\sigma),\pi)\models\varphi$ for every $\sigma:X\rightarrow\N$ and
every infinite path $\pi$ with $\pi(0)=q(n)$?}




\subsection{Upper bounds}

A standard approach to $\ltl$ model checking is the automata-based
approach, in which systems are modeled as non-deterministic B\"uchi
automata (NBA). Given an NBA $\automaton$ and an $\ltl$ formula
$\varphi$, the idea is to translate $\varphi$ into an NBA
$\automaton_{\neg \varphi}$ such that $\automaton_{\neg \varphi}$
accepts all words that violate $\varphi$.  Then, by checking for
emptiness of the product automaton
$\automaton'=\automaton\times\automaton_{\neg \varphi}$, it can be
decided whether or not $\automaton$ generates traces that violate
$\varphi$.  Emptiness can be decided by checking for recurrent
reachability of a control state in the transition system induced by
$\automaton'$, which can be performed non-deterministically in space logarithmically in the size of
$\automaton'$. Vardi and Wolper showed in \cite{VW86} that the size of
$\automaton_{\neg \varphi}$ is $2^{O(|\varphi|)}$, which yields that
checking for emptiness of $\automaton'$ is in $\PSPACE$. The $\PSPACE$
lower bound for finite state model checking was originally shown
in \cite{LP85-popl} and hence carries over to SOCA.

Constructing the product automaton of two NBA is a standard technique
in model checking, see e.g.\ \cite{BK08} for a detailed treatment of
this. Moreover, this construction can be adapted in a
straight-forward way into the setting of POCA by equipping a POCA with
a set of final states and defining emptiness with respect to B\"uchi
acceptance condition in the standard way. For brevity, we do not give
any further details. We then have that given a POCA $\Poca$, an $\ltl$
formula $\varphi$, a state $q$ and $n\in\N$ we can construct a
POCA $\Poca'$ as the product of $\Poca$ and $\automaton_{\neg\varphi}$
such that for all assignments $\sigma$ we have that $\Poca'$ is empty
with respect to B\"uchi acceptence condition if and only if
$(T(\Poca^\sigma),q(n))\models \varphi$. Moreover, the size of
$\Poca'$ is $2^{O(|\varphi|)}|\Poca|$.

It has been shown in \cite{HKOW10} that checking emptiness with
respect to B\"uchi acceptance condition is $\coNP$-complete for both
SOCA and POCA, and in \cite{DG-jlc09} that it is $\NL$-complete for
OCA. We use these results in order to establish upper bounds for our
$\ltl$ model checking problems. Let us first consider the case of a
fixed formula $\varphi$. In the terminology of the previous paragraph,
$|\Poca'|=O(|\Poca|)$ and hence the data complexity of $\ltl$ on SOCA
and POCA is in $\coNP$.  The matching $\coNP$ lower bound of the data
complexity of LTL on SOCA follows from the $\NP$-hardness of the
reachability problem for SOCA.
%We obtain $\coNP$-hardness from the fact that
%we can reduce deciding emptiness of some SOCA $\Soca$ with respect to
%B\"uchi acceptance condition to $\ltl$ model checking of a slightly
%altered SOCA $\Soca'$ against an easily constructed fixed $\ltl$
%formula.

\begin{proposition}
The data complexity of $\ltl$ model checking on SOCA and POCA is
  $\coNP$-complete.
\end{proposition}
As seen above, if both $\Poca$ and the formula $\varphi$ are part of
the input then $|\Poca'|=2^{O(|\varphi)}\cdot|\Poca|$, and hence $\coNEXP$
is an upper bound for the combined complexity of $\ltl$ model checking
on both SOCA and POCA. However, we can improve this upper bound in the
case of SOCA. Given a SOCA $\Soca$, let $m$ be the absolute value of
the maximal increment or decrement performed on the transitions in
$\Soca$ and let $\Soca'$ be the product automaton of $\Soca$ and
$\automaton_{\neg \varphi}$. Let $\Soca''$ be obtained from $\Soca'$
by replacing every transition labeled with $z$ with a sequence of
fresh transitions and control locations of length $z$, where each
transition is labeled with $+1$ respectively $-1$, depending on the
sign of $z$. We conclude that
$|\Soca''|=m\cdot|\Soca'|=m\cdot2^{O(|\varphi|)}\cdot|\Soca|$. Combining the result
from \cite{DG-jlc09} together with the fact that $\ltl$ model checking
of non-deterministic finite state automata is $\PSPACE$-hard
\cite{LP85-popl}, we obtain the following proposition.

\begin{proposition}
The combined complexity of $\ltl$ on SOCA is $\PSPACE$-complete.
\end{proposition}
It remains to prove that the combined complexity of $\ltl$ model
checking on POCA is $\coNEXP$-hard, which we will show in the next
section.



\subsection{Combined Complexity of $\ltl$ on POCA}

We are now going to show $\coNEXP$-hardness of $\ltl$ model checking
on POCA via a reduction from (the complement of) \textsf{Succinct
  3-SAT}, which is a $\NEXP$-complete problem \cite{Papa94}. An input
of \textsf{Succinct 3-SAT} is given by a Boolean circuit $\mathbb{C}$
that encodes a Boolean formula $\psi_\circuit$ in 3-CNF,
i.e.\ $\psi_\circuit=\bigwedge_{j\in [0,M]}(\ell^j_1\vee \ell^j_2\vee
\ell^j_3)$ for some $M=O(2^{|\circuit|})$, and $\psi_\circuit$ is free
over Boolean variables $y_1,\ldots,y_{N}$ for some
$N=O(2^{|\circuit|})$. Let $j\in [0,M]$ be the index of a clause
encoded in \emph{binary} and $k\in\{1,2,3\}$. On input $(j\cdot k)$,
the output of $\circuit$ is $(i\cdot b)$, where $i\in [N]$ is the
index of the Boolean variable that appears in literal $\ell_k^j$, and
where $b=0$ when $\ell_k^j$ is negative and $b=1$ when $\ell_k^j$ is
positive.
\vspace*{0.25cm}
\problemx{\textsf{Succinct 3-SAT}}
{Boolean circuit $\circuit$}
{Is $\psi_\circuit$ satisfiable?}
\vspace*{0.25cm}

%\medskip

\noindent
In order to establish $\coNEXP$-hardness for the combined complexity
of $\ltl$ model checking, given an input $\circuit$ of
$\textsf{Succinct 3-SAT}$, we construct a POCA $\Poca(x)$ and an
$\ltl$ formula $\varphi$ such that $\psi_\circuit$ is satisfiable if
and only if there is an assignment $\sigma$ such that
$(T(\Poca^\sigma),q_{\mathsf{start}}(0))\models\varphi$ for some
distinguished state $q_{\mathsf{start}}$ of $\Poca(x)$.
%There are two key
%ideas for the reduction.

First, let us provide a suitable encoding of truth assignments by
natural numbers. The encoding we use has also been employed for
establishing lower bounds for model checking OCA~\cite{JKMZ04-ic}.
Recall that $p_i$ denotes the $i^{\text{th}}$ prime number. Every
natural number $x$ defines a truth assignment
$\nu_x:\{y_1,\ldots,y_N\} \to \{ 0,1\}$ such that $\nu_x(y_i)=1$ iff
$p_i$ divides $x$.  By the Prime Number Theorem, $p_N=O(N \log N)$ and
hence $O(|\circuit|)$ bits are sufficient to represent $p_N$. Of
course, since we need exponentially many prime numbers they cannot be
hard-wired into $\Poca(x)$.

%% Second, we employ $\ltl$ in order to perform arbitrary computations of
%% space-bounded Turing machines on traces of runs of SOCA. However, we
%% cannot make use of the approach we employed in $\ctl$. Intuitively
%% speaking, $\ctl$ allows for performing tests on the counter value
%% without destroying its value and this property has no correspondant in
%% the $\ltl$ setting. We can however use atomic propositions to model
%% computations along traces.

\begin{figure}
  \begin{center}
    \begin{picture}(142,60)(0,0)
      %   \put(-20,-28){\framebox(85,33){}}
      %\gasset{Nadjust=w,Nadjustdist=2,Nh=6,Nmr=1}
      % \gasset{Nadjust=w,Nadjustdist=2,Nh=6}%$,Nmr=1}
      % \node(A)(0,0){$q_0$}

      \gasset{Nframe=n,loopdiam=9,ELdist=.7}
      \gasset{Nadjust=wh,Nadjustdist=1}
      \gasset{curvedepth=0}

      \put(0,25){
        \drawpolygon[Nframe=n,Nfill=y,fillgray=.9](0,0)(25,0)(25,15)(0,15)
        \put(6,7){$\Soca_{\mathsf{inc}}(j)$}
      }

      \put(35,25){
        \drawpolygon[Nframe=n,Nfill=y,fillgray=.9](0,0)(25,0)(25,15)(0,15)
        \put(3,8){$\Soca_{\circuit}(j\cdot 2)$}
        \put(10,4){$=(i\cdot b)$ }
      }
      \put(35,45){
        \drawpolygon[Nframe=n,Nfill=y,fillgray=.9](0,0)(25,0)(25,15)(0,15)
        \put(3,8){$\Soca_{\circuit}(j\cdot 1)$}
        \put(10,4){$=(i\cdot b)$ }
      }
      \put(35,5){
        \drawpolygon[Nframe=n,Nfill=y,fillgray=.9](0,0)(25,0)(25,15)(0,15)
        \put(3,8){$\Soca_{\circuit}(j\cdot 3)$}
        \put(10,4){$=(i\cdot b)$ }
      }

      \put(70,25){
        \drawpolygon[Nframe=n,Nfill=y,fillgray=.9](0,0)(25,0)(25,15)(0,15)
        \put(3,6){$\Soca_{\mathsf{prime}}(i)=p_i$ }
      }

      \put(110,35){
        \drawpolygon[Nframe=n,Nfill=y,fillgray=.9](0,0)(25,0)(25,15)(0,15)
        \put(3,6){$\Soca_{\mathsf{divides}}(p_i,x)$ }
        \put(8,11){$b=0?$}
      }

      \put(110,10){
        \drawpolygon[Nframe=n,Nfill=y,fillgray=.9](0,0)(25,0)(25,15)(0,15)
        \put(3,6){$\Soca_{\mathsf{not\_divides}}(p_i,x)$ }
        \put(8,11){$b=1?$}
      }

      \node(A1)(12,29){}
      \node(A2)(20,32){}

      \node(B11)(40,8){}
      \node(B12)(55,15){}

      \node(B21)(40,28){}
      \node(B22)(55,35){}

      \node(B31)(40,48){}
      \node(B32)(55,55){}

      \node(C1)(73,36){}
      \node(C2)(90,28){}

      \node(D11)(115,39){}
      \node(D12)(132,39){}

      \node(D21)(113,13){}
      \node(D22)(132,13){}

      \node(E)(140,0){$\bullet$}
      \node(F)(12,0){$q_{\mathsf{start}}$}

      \drawedge(F,A1){$+x$}

      \drawedge(A2,B11){$0$}
      \drawedge(A2,B21){$0$}
      \drawedge(A2,B31){$0$}

      \drawedge(B12,C1){$0$}
      \drawedge(B22,C1){$0$}
      \drawedge(B32,C1){$0$}

      \drawedge(C2,D11){$0$}
      \drawedge(C2,D21){$0$}

      \drawqbedge(D12,142,25,E){$0$}
      \drawedge(D22,E){$0$}

      \drawedge(E,F){$\mathsf{zero}$}
    \end{picture}

  \end{center}
  \caption{High-level description of the automaton $\Poca(x)$ used for
    the reduction.}
  \label{fig:high-level}
\end{figure}

Let us now take a look at $\Poca(x)$, which is sketched in Figure
\ref{fig:high-level}. It uses one parameter $x$ and employs several
gadgets. The only gadgets manipulating the counter are
$\Soca_{\mathsf{divides}}$ and $\Soca_{\mathsf{not\_divides}}$. The
remaining gadgets are designed such that they communicate via
designated propositional variables, and not as in Section \ref{S CTL
  hardness} with the help of the counter. First, $\Poca(x)$ loads the
value of the parameter $x$ on the counter.  Think of $x$ encoding a
truth assignment of $\psi_\circuit$.  Next, $\Poca(x)$ traverses
through $\Soca_{\mathsf{inc}}$, which initially chooses an arbitrary
index $j$ identifying a clause of $\psi_\circuit$. Every time
$\Soca_{\mathsf{inc}}$ is traversed afterwards, it increments $j$
modulo $M+1$ and hereby moves on to the next clause.  Now $\Poca(x)$
branches non-deterministically into a gadget $\Soca_\circuit$ in order
to compute $(i\cdot b)$ from $\circuit$ on input $(j\cdot 1), (j\cdot
2)$, resp.\ $(j\cdot 3)$. The index $i$ is then used as input to a
gadget $\Soca_{\mathsf{prime}}$, which computes $p_i$. Then if $b=0$,
it is checked in $\Soca_{\mathsf{not\_divides}}$ that $p_i$ does not
divide the value of $x$, and likewise in $\Soca_{\mathsf{divides}}$
that $p_i$ divides the value of $x$ if $b=1$. For checking the latter,
the counter needs to be modified. After the checks have been finished,
we restore the value $x$ on the counter and the process continues with
clause $j+1~ \mod M+1$.

It remains to show how the gadgets and the communication between them
can be realized. The first observation is that the computations of
$\Soca_{\mathsf{inc}}$, $\Soca_{\circuit}$ and
$\Soca_{\mathsf{prime}}$ can be realized by space bounded DTM using no
more than $p(|\circuit|)$ tape cells for their in- and output tape for
some polynomial $p$ that is fixed once $\circuit$ is provided. Indeed,
it is easily seen that incrementing modulo $M+1$, evaluating
$\circuit$ and computing the $i^{\text{th}}$ prime number can be done
by such a DTM. Thus, we now show how from a generic DTM $\M$, we can
construct in logarithmic space a SOCA $\Soca_\M$ and some LTL formulas
that mimic computations of $\M$ on traces of $\Soca_\M$.

\begin{figure}
  \begin{center}
    \begin{picture}(140,100)(0,-52)
      \gasset{Nframe=n,loopdiam=9,ELdist=.7}
      \gasset{Nadjust=wh,Nadjustdist=1}
      \gasset{curvedepth=0}

      \node(DASH)(20,40){}
%      \node(STATEIN)(-5,25){}
      \node(START)(0,25){}
      \node(A)(10,25){$\bullet$}
      \node(B)(20,25){$q$}
      \node(C1)(30,45){$\bullet$}
      \node(C2)(30,35){$\bullet$}
      \node(C3)(30,25){$\vdots$}
      \node(C4)(30,15){$\bullet$}
      \node(C5)(30,5){$\bullet$}
      \node(D)(40,25){$\bullet$}
%      \node(STATEOUT)(70,25){}

%      \drawedge[dash={1.5}0](DASH,B){}
%      \drawedge(STATEIN,A){}
      \drawedge(START,A){}
      \drawedge(A,B){}
      \drawedge(B,C1){}
      \drawedge[ELside=r](B,C2){}
%     \drawedge(B,C3){$0$}
      \drawedge(B,C4){}
      \drawedge[ELside=r](B,C5){}
      \drawedge(C1,D){}
      \drawedge[ELside=r](C2,D){}
%     \drawedge(C3,D){$0$}
      \drawedge(C4,D){}
      \drawedge[ELside=r](C5,D){}
%      \drawedge(D,STATEOUT){}

      \put(9,21){$\zeta$}
      \put(18,21){$\alpha$}
      \put(29,47){$\mathsf{s}_0$}
      \put(29,37){$\mathsf{s}_1$}
      \put(27,12){$\mathsf{s}_{k-1}$}
      \put(29,2){$\mathsf{s}_k$}
%      \put(60,21){$\mathsf{s}_\r$}


%%%%%%%%%%%%%%%%%% ihead

%      \node(DASH)(20,40){}
%      \node(STATEIN)(-5,25){}
%      \node(A)(5,25){$\bullet$}
      \node(E)(50,25){$\bullet$}
      \node(F1)(60,45){$\bullet$}
      \node(F2)(60,35){$\bullet$}
      \node(F3)(60,25){$\vdots$}
      \node(F4)(60,15){$\bullet$}
      \node(F5)(60,5){$\bullet$}
      \node(G)(70,25){$\bullet$}
%      \node(STATEOUT)(70,25){}

      \drawedge(D,E){}

%     \drawedge[dash={1.5}0](DASH,B){}
%      \drawedge(STATEIN,A){}
%      \drawedge(A,B){$0$}
      \drawedge(E,F1){}
      \drawedge[ELside=r](E,F2){}
%     \drawedge(B,C3){$0$}
      \drawedge(E,F4){}
      \drawedge[ELside=r](E,F5){}
      \drawedge(F1,G){}
      \drawedge[ELside=r](F2,G){}
%     \drawedge(C3,D){$0$}
      \drawedge(F4,G){}
      \drawedge[ELside=r](F5,G){}
%      \drawedge(D,STATEOUT){}

%      \put(18,21){$\alpha$}
      \put(58,47){$\mathsf{ih}_0$}
      \put(58,37){$\mathsf{ih}_1$}
      \put(57,12){$\mathsf{ih}_{m}$}
      \put(56,2){$\mathsf{ih}_{m+1}$}
%      \put(60,21){$\mathsf{s}_\r$}


%%%%%%%%%%%%%%%%%% whead

%      \node(DASH)(20,40){}
%      \node(STATEIN)(-5,25){}
%      \node(A)(5,25){$\bullet$}
      \node(E)(80,25){$\bullet$}
      \drawedge(G,E){}
      \node(F1)(90,45){$\bullet$}
      \node(F2)(90,35){$\bullet$}
      \node(F3)(90,25){$\vdots$}
      \node(F4)(90,15){$\bullet$}
      \node(F5)(90,5){$\bullet$}
      \node(G)(100,25){$\bullet$}
      \node(H)(100,0){$\bullet$}


%     \drawedge[dash={1.5}0](DASH,B){}
%      \drawedge(STATEIN,A){}
%      \drawedge(A,B){$0$}
      \drawedge(E,F1){}
      \drawedge[ELside=r](E,F2){}
%     \drawedge(B,C3){$0$}
      \drawedge(E,F4){}
      \drawedge[ELside=r](E,F5){}
      \drawedge(F1,G){}
      \drawedge[ELside=r](F2,G){}
%     \drawedge(C3,D){$0$}
      \drawedge(F4,G){}
      \drawedge[ELside=r](F5,G){}
      \drawedge(G,H){}

%      \put(18,21){$\alpha$}
      \put(88,47){$\mathsf{wh}_0$}
      \put(88,37){$\mathsf{wh}_1$}
      \put(87,12){$\mathsf{wh}_{n}$}
      \put(86,2){$\mathsf{wh}_{n+1}$}
%      \put(60,21){$\mathsf{s}_\r$}

      \node(I)(0,0){$\bullet$}
      \drawedge(H,I){}

%%%%%%%%%%%%%%%% wtape

%      \node(WORKIN)(0,45){}
      \node(A)(0,-15){$\bullet$}
      \drawedge(I,A){}
      \node(B1)(15,-25){$\bullet$}
      \node(B2)(15,-5){$\bullet$}
      \node(C)(30,-15){$\bullet$}
      \node(D1)(45,-25){$\bullet$}
      \node(D2)(45,-5){$\bullet$}
      \node(E)(60,-15){$\bullet$}
      \node(F)(70,-15){$\ldots$}
      \node(G)(80,-15){$\bullet$}
      \node(H1)( 95,-25){$\bullet$}
      \node(H2)(95,-5){$\bullet$}
      \node(I)(110,-15){$\bullet$}

      \node(J)(110,-30){$\bullet$}
      \node(K)(0,-30){$\bullet$}
      \drawedge(I,J){}
      \drawedge(J,K){}
%      \node(J)(135,-15){$\bullet$}
%      \node(DASH)(120,-25){}
%      \node(WORKOUT)(145,-15){}

%      \drawedge(WORKIN,A){}
      \drawedge(A,B1){}
      \drawedge(A,B2){}
      \drawedge(B1,C){}
      \drawedge(B2,C){}
      \drawedge(C,D1){}
      \drawedge(C,D2){}
      \drawedge(D1,E){}
      \drawedge(D2,E){}
      \drawedge(G,H1){}
      \drawedge(G,H2){}
      \drawedge(H1,I){}
      \drawedge(H2,I){}
%      \drawedge(I,J){}
%      \drawedge[dash={1.5}0](I,DASH){}

%      \put(4,42){$\mathsf{w}_\l$}
      \put(13,-9){$\mathsf{i}_{1,1}$}
      \put(13,-29){$\mathsf{i}_{1,0}$}
      \put(43,-9){$\mathsf{i}_{2,1}$}
      \put(43,-29){$\mathsf{i}_{2,0}$}
      \put(93,-9){$\mathsf{i}_{m,1}$}
      \put(93,-29){$\mathsf{i}_{m,0}$}
%      \put(120,42){$\mathsf{w}_\r$}
%      \put(133,42){$\mathsf{yield}$}

%%%%%%%%%%%%%%% wtape
      \node(A)(0,-45){$\bullet$}

      \drawedge(K,A){}

      \node(B1)(15,-55){$\bullet$}
      \node(B2)(15,-35){$\bullet$}
      \node(C)(30,-45){$\bullet$}
      \node(D1)(45,-55){$\bullet$}
      \node(D2)(45,-35){$\bullet$}
      \node(E)(60,-45){$\bullet$}
      \node(F)(70,-45){$\ldots$}
      \node(G)(80,-45){$\bullet$}
      \node(H1)( 95,-55){$\bullet$}
      \node(H2)(95,-35){$\bullet$}
      \node(I)(110,-45){$\bullet$}
      \node(J)(120,-45){$\bullet$}
      \node(K)(120,50){$\bullet$}
      \node(L)(20,50){$\bullet$}
      \node(M)(20,25){}
      \node(N)(130,-45){$\bullet$}
      \node(P)(140,-45){}

      \drawedge(I,J){}
      \drawedge(J,K){}
      \drawedge(K,L){}
      \drawedge(L,M){}

      \drawedge(J,N){}
      \drawedge(N,P){}


%      \node(DASH)(120,-25){}
%      \node(WORKOUT)(145,-15){}

%      \drawedge(WORKIN,A){}
      \drawedge(A,B1){}
      \drawedge(A,B2){}
      \drawedge(B1,C){}
      \drawedge(B2,C){}
      \drawedge(C,D1){}
      \drawedge(C,D2){}
      \drawedge(D1,E){}
      \drawedge(D2,E){}
      \drawedge(G,H1){}
      \drawedge(G,H2){}
      \drawedge(H1,I){}
      \drawedge(H2,I){}
%      \drawedge(I,J){}
%      \drawedge[dash={1.5}0](I,DASH){}

%      \put(4,42){$\mathsf{w}_\l$}
      \put(12,-39){$\mathsf{w}_{1,1}$}
      \put(12,-58){$\mathsf{w}_{1,0}$}
      \put(42,-39){$\mathsf{w}_{2,1}$}
      \put(42,-58){$\mathsf{w}_{2,0}$}
      \put(92,-39){$\mathsf{w}_{n,1}$}
      \put(93,-58){$\mathsf{w}_{n,0}$}

      \put(119,-49){$\gamma$}
      \put(129,-49){$\beta$}

%      \put(120,42){$\mathsf{w}_\r$}
%      \put(133,42){$\mathsf{yield}$}

    \end{picture}
  \end{center}
  \caption{SOCA $\Soca_\M$ for the simulation of space bounded Turing
    machines.}
  \label{fig:tm}
\end{figure}

Let $\M=(S, \Sigma, s_0, F, \mu)$ be a DTM with a fixed input tape
with $m$ tape cells, and $n$ working tape cells, and let
$S=\{s_0,\ldots,s_k\}$. Figure \ref{fig:tm} shows the SOCA $\Soca_\M$
that we use for the simulation of $\M$. All transitions of $\Soca_\M$
are implicitly labelled with $0$.  A simulation starts when $\Soca_\M$
is entered at the location labeled with $\zeta$ and is finished when
the location labeled with $\beta$ is reached.

The sequence of propositions occurring on a trace starting from and
ending in $q$ encodes a configuration of $\M$.  In detail,
$\mathsf{s}_i$ indicates that $\M$ is in state $s_i$; $\mathsf{ih}_i$
that the input tape head scans cell $i$; $\mathsf{wh}_i$ that the
working tape head scans cell $i$; $\mathsf{i}_{i,b}$ that the
$i^{\text{th}}$ bit of the input tape is set to $b$; and
$\mathsf{w}_{i,b}$ that the $i^{\text{th}}$ bit of the working tape is
set to $b$, where $i$ is in the respective range and $b\in \{0,1\}$.

\newcommand{\state}{\mathsf{state}}
\newcommand{\work}{\mathsf{work}}
\newcommand{\iNp}{\mathsf{input}}
\newcommand{\inh}{\mathsf{inhead}}
\newcommand{\woh}{\mathsf{wohead}}
\renewcommand{\succ}{\mathsf{succ}}

Let us introduce some LTL formulas that allow for testing properties
of the current configuration. Think of all of them as being evaluated
in $q$. The formula $\state_i=\X \mathsf{s}_i$ for each $i\in[0,k]$
expresses that the current state is $s_i$.  By the formula
$\inh_{i}=\X\X\X\X \mathsf{ih}_i$ we express that the input head is at
position $i$, where $i\in[m]$. Similarly, define the formulas
$\woh_i$, $\work_{j,b}$, and $\iNp_{i,b}$ for expressing that the
working head is at position $i$, that the $i^{\text{th}}$ bit of the
input tape is $b$, and that the $j^{\text{th}}$ bit of the working
tape is $b$, respectively, where $i\in[m]$,$j\in[n]$, and
$b\in\{0,1\}$.

The LTL formula below, assumed to be evaluated in $\alpha$, ensures
that the transition function is correctly encoded into traces of
$\Soca_\M$ for states $s\in S\setminus F$ whenever the input
respectively working tape head does not scan a start ($\l$)
respectively end marker ($\r$):

\begin{multline*}
\bigwedge_{s_l\in S\setminus F}
\bigwedge_{i\in[m]}
\bigwedge_{j\in[n]}
\bigwedge_{b_1,b_2\in\{0,1\}}
\state_l\wedge\inh_i\wedge\iNp_{i,b_1}\wedge\woh_j\wedge
\work_{j,b_2} \longrightarrow\\
\longrightarrow
\left(
\left(\X(\neg\alpha \wedge \neg \beta)
\U(\alpha\wedge \succ(s_l,i,j,b_1,b_2)\right)
\wedge
\bigwedge_{j'\not= j}\bigwedge_{b\in\{0,1\}}
\left(\work_{j',b}\leftrightarrow
(\X\neg\alpha\U(\alpha\wedge\work_{j',b}))\right)
\right).
\end{multline*}
Here, whenever $\mu(s_l,b_1,b_2)=(s_h,d_1,d_2,b)$, the formula
\begin{align*}
\succ(s_l,i,j,b_1,b_2)\quad =\quad\state_h\wedge\inh_{i+d_1}\wedge\woh_{j+d_2}\wedge
\work_{j,b}
\end{align*}
guarantees that the correct bit is ``written'' to the working tape and
that the state, the input head position, and the working tape position
of the next configuration seen indeed match the successor
configuration. A similar formula can be constructed for the case when
one or both of the input or working heads point to a start
respectively end marker. Once we have reached a final state $s_i\in
F$, we require that $\Soca_\M$ is left which is expressed by the
following formula when evaluated in $q$:
\begin{align*}
   \bigwedge_{s_i\in F}\state_i \longrightarrow (\neg \alpha \until
   \beta).
\end{align*}
It is now easily seen that we can construct a formula $\Phi$ that is
derived from a conjunction of the formulas from above such that the
formula $\globally (\alpha \rightarrow \Phi)$ constraints paths
through $\Soca_\M$ in such a way that their traces yield the encoding
of a valid computation of $\M$.

Let us now address towards ensuring that once we enter $\Soca_\M$ we
initially traverse it in such a way that the trace corresponds to an
initial configuration of $\M$. The formula
\begin{align*}
  \globally\left(\zeta \longrightarrow \X(\state_0 \wedge \inh_1 \wedge
  \woh_1\wedge \bigwedge_{1\le j\le n}\mathsf{work}_{j,0} ) \right)
\end{align*}
makes sure that the heads of the input and working tape point to the
first tape cell, that the working tape is filled with $0$s and that we
are in the initial state. In case the input tape can be initialized
with an arbitrary content, we are done. Otherwise, suppose that we
want to transfer the first $j$ bits of the output of a TM $\M'$ from
its corresponding SOCA $\Soca_{\M'}$ to the input of $\Soca_\M$.  For
$b\in \{0,1\}$, let $\bar{b}=0$ if $b=1$ and $\bar{b}=1$ otherwise,
and suppose that all atomic propositions are primed in $\Soca_{\M'}$.
The formula
\begin{align*}
  \bigwedge_{1\le i\le j\atop{b\in \{0,1\}}}
  \globally
   ((\mathsf{w}'_{i,b} \wedge (\neg \alpha' \until \beta'))
  \longrightarrow (\neg \mathsf{i}_{i,\bar{b}}\until \gamma)
  %\right)
\end{align*}
guarantees that we traverse through the first $j$ bits of the
component of $\Soca_\M$ representing the input tape of $\M$ in the
same way as we traverse the first $j$ bits of the working tape
component of $\M'$ in $\Soca_{\M'}$ when a computation has finished.
Coming back to Figure \ref{fig:high-level}, we have thus seen how the
SOCA $\Soca_{\mathsf{inc}}, \Soca_{\circuit}$ and
$\Soca_{\mathsf{prime}}$ and the communication between them can be
realized. The only major question left open is how we can perform a
divisibility respectively non-divisibility test of the counter value
with a prime number computed in $\Soca_{\mathsf{prime}}$. For this,
let us consider the SOCA $\Soca_{\mathsf{divides}}$ from Figure
\ref{fig:soca-divisibility}.

\begin{figure}[t]
  \begin{center}
    \begin{picture}(140,30)(0,0)
      \gasset{Nframe=n,loopdiam=9,ELdist=.7}
      \gasset{Nadjust=wh,Nadjustdist=1}
      \gasset{curvedepth=0}

      %% \node(WORKIN)(0,45){}
      %% \node(A)(10,45){$\bullet$}
      %% \node(B1)(25,55){$\bullet$}
      %% \node(B2)(25,35){$\bullet$}
      %% \node(C)(40,45){$\bullet$}
      %% \node(D1)(55,55){$\bullet$}
      %% \node(D2)(55,35){$\bullet$}
      %% \node(E)(70,45){$\bullet$}
      %% \node(F)(80,45){$\ldots$}
      %% \node(G)(90,45){$\bullet$}
      %% \node(H1)(105,55){$\bullet$}
      %% \node(H2)(105,35){$\bullet$}
      %% \node(I)(120,45){$\bullet$}
      %% \node(J)(135,45){$\bullet$}
      %% \node(DASH)(120,55){}
      %% \node(WORKOUT)(145,45){}

      %% \drawedge(WORKIN,A){}
      %% \drawedge(A,B1){$0$}
      %% \drawedge(A,B2){$0$}
      %% \drawedge(B1,C){$0$}
      %% \drawedge(B2,C){$0$}
      %% \drawedge(C,D1){$0$}
      %% \drawedge(C,D2){$0$}
      %% \drawedge(D1,E){$0$}
      %% \drawedge(D2,E){$0$}
      %% \drawedge(G,H1){$0$}
      %% \drawedge(G,H2){$0$}
      %% \drawedge(H1,I){$0$}
      %% \drawedge(H2,I){$0$}
      %% \drawedge(I,J){$0$}
      %% \drawedge[dash={1.5}0](I,DASH){}

      %% \put(4,42){$\mathsf{work}_\l$}
      %% \put(21,57){$\mathsf{work_{1,1}}$}
      %% \put(21,32){$\mathsf{work_{1,0}}$}
      %% \put(51,57){$\mathsf{work_{2,1}}$}
      %% \put(51,32){$\mathsf{work_{2,0}}$}
      %% \put(101,57){$\mathsf{work}_{N,1}$}
      %% \put(101,32){$\mathsf{work}_{N,0}$}
      %% \put(120,42){$\mathsf{work}_\r$}
      %% \put(133,42){$\mathsf{yield}$}


      \node(DIVIDESIN)(0,15){}
      \node(A)(10,15){$\bullet$}
      \node(B1)(25,25){$\bullet$}
      \node(B2)(25,5){$\bullet$}
      \node(C)(40,15){$\bullet$}
      \node(D1)(55,25){$\bullet$}
      \node(D2)(55,5){$\bullet$}
      \node(E)(70,15){$\bullet$}
      \node(F)(80,15){$\ldots$}
      \node(G)(90,15){$\bullet$}
      \node(H1)(105,25){$\bullet$}
      \node(H2)(105,5){$\bullet$}
      \node(I)(120,15){$\bullet$}
      \node(J)(135,15){$\bullet$}
      \node(DIVIDESOUT)(145,15){}

      \drawedge(DIVIDESIN,A){}
      \drawedge(A,B1){$0$}
      \drawedge(A,B2){$0$}
      \drawedge(B1,C){$-2^0$}
      \drawedge(B2,C){$0$}
      \drawedge(C,D1){$0$}
      \drawedge(C,D2){$0$}
      \drawedge(D1,E){$-2^1$}
      \drawedge(D2,E){$0$}
      \drawedge(G,H1){$0$}
      \drawedge(G,H2){$0$}
      \drawedge(H1,I){$-2^l$}
      \drawedge(H2,I){$0$}
      \drawedge(I,J){$\mathsf{\zero}$}
      \drawbpedge[ELside=r](I,290,28,A,250,28){$0$}

%      \put(4,12){$\mathsf{work}_\l$}
      \put(22,27){$\prop_{0,1}$}
      \put(22,2){$\prop_{0,0}$}
      \put(52,27){$\prop_{1,1}$}
      \put(52,2){$\prop_{1,0}$}
      \put(102,27){$\prop_{l,1}$}
      \put(102,2){$\prop_{l,0}$}
%      \put(120,12){$\mathsf{work}_\r$}
      \put(134,12){$\xi$}


    \end{picture}
  \end{center}
  \caption{The SOCA $\Soca_{\mathsf{divides}}$  for testing the
    counter for divisibility with some natural number of
    bit length $l+1$.}
  \label{fig:soca-divisibility}
\end{figure}

One cycle through $\Soca_{\mathsf{divides}}$ subtracts some natural
number of bit length $l+1$ from the counter. However, in order to test
for divisibility we need to make sure that we remain on the same path
in every cycle. In the $\CTL$ setting, this problem was resolved by
branching into the additional SOCA $\Soca_{\mathsf{bit}}$. In
contrast, in $\ltl$ we cannot branch, but use the propositions
$\prop_{j,b_j}, j\in [0,l], b\in \{0,1\}$ in order to stay on
precisely one path in every cycle. Assuming that the number $p$ for
which we want to test for divisibility with the current counter value
is encoded as a sequence of propositions $\mathsf{w}_{j,b_j}$ of some
SOCA $\Soca_\M$, the subsequent formula enforces that we always
subtract $p$ in cycles of $\Soca_{\mathsf{divides}}$:
\begin{align*}
  \globally \left( \bigwedge_{0\le j\le l \atop{b_{j}\in \{0,1\}}}
  ((\mathsf{w}_{j,b_j} \wedge (\neg \alpha \until \beta) ) \longrightarrow
  ( \neg \prop_{j,\bar{b_j}} \until \xi)) \right).
\end{align*}
It is straight forward to derive a similar SOCA
$\Soca_{\mathsf{not\_divides}}$ and an appropriate LTL formula for
testing non-divisibility of the counter value with a previously
computed prime number. Finally, we can also adopt these techniques in
order to correctly handle the branching on $b$ performed in Figure
\ref{fig:high-level}.

In summary, by taking the disjoint union of all the gadgets from
Figure \ref{fig:high-level}, their appendent $\ltl$ formula that we
described in this section, connecting the gadgets correctly and taking
the conjunction of the relevant $\ltl$ formulas, we can construct
$\Poca(x)$ and an $\ltl$ formula $\varphi$ in logarithmic space such
that there is an assignment $\sigma$ assigning a natural number to $x$
such that $(T(\Poca^\sigma), q_{\mathsf{start}}(0))\not\models
\varphi$ if and only if $\psi_\circuit$ given by an input $\circuit$
of $\textsf{Succinct 3-SAT}$ is satisfiable.

\begin{theorem}
  The combined complexity of $\ltl$ model checking on POCA is
  $\coNEXP$-complete.
\end{theorem}

\section{Conclusion}{\label{S Concl}}

In this paper, we have settled the computational complexity of model
checking CTL and LTL on SOCA and POCA with respect to data and
combined complexity. Our proofs for providing lower bounds have
introduced some involved concepts and techniques, which we believe may
be of independent interest for providing lower bounds for decision
problems in the verification of infinite state systems.

An interesting aspect of future work could be the consideration of
\emph{synthesis problems} for POCA. Given a POCA $\Poca(X)$ and an LTL
formula $\varphi$, a natural question to pose is whether there
\emph{exists an assignment} $\sigma$ such that $(T(\Poca^\sigma),
\pi)\models \varphi$ for all infinite paths $\pi$ starting in some
state of $T(\Poca^\sigma)$.  For CTL, such a problem is undecidable by
Theorem \ref{thm:ctl-poca}, but we claim that it might be decidable
for LTL. Moreover, the CTL fragment EF seems to be a good candidate of
a branching-time logic for which model checking on POCA could be
decidable, but this remains subject to further investigations.



\bibliography{bib}
\bibliographystyle{plain}


%\thebibliography

\iffalse

\begin{thebibliography}{1}


\bibitem{GoLo10}
S.~G\"oller, M.~Lohrey.
\newblock Branchning-time model checking of one-counter processes.
\newblock In Proceedings of STACS 2010, to appear
\newblock {\em Arxiv:0909.1102v1}

\end{thebibliography}
\fi

\end{document}

