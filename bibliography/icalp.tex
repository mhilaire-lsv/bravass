\documentclass[times,envcountsame]{llncs}

\usepackage{times}
\usepackage{amsmath,amssymb,eufrak,theorem}%mnsymbol}
\usepackage{enumerate}
\usepackage{complexity}
\usepackage{gastex}
\usepackage{url}

\usepackage{xspace,tabularx,multicol}


\usepackage{wrapfig}
%\usepackage{floatfig}


\newtheorem{fact}[theorem]{Fact}

\usepackage{stmaryrd,rotating,mathrsfs}

\newcommand{\prim}{\text{prime}}


\newcommand{\problemx}[3]{
\par\noindent\underline{\sc#1}\par\nobreak\vskip.2\baselineskip
\begingroup\clubpenalty10000\widowpenalty10000
\setbox0\hbox{\bf INPUT:\ }\setbox1\hbox{\bf QUESTION:\ }
\dimen0=\wd0\ifnum\wd1>\dimen0\dimen0=\wd1\fi
\vskip-\parskip\noindent
\hbox to\dimen0{\box0\hfil}\hangindent\dimen0\hangafter1\ignorespaces#2\par
\vskip-\parskip\noindent
\hbox to\dimen0{\box1\hfil}\hangindent\dimen0\hangafter1\ignorespaces#3\par
\endgroup}


\newcommand{\problemy}[3]{
\par\noindent\underline{\sc#1}\par\nobreak\vskip.2\baselineskip
\begingroup\clubpenalty10000\widowpenalty10000
\setbox0\hbox{\bf INPUT:\ }\setbox1\hbox{\bf OUTPUT:\ }
\dimen0=\wd0\ifnum\wd1>\dimen0\dimen0=\wd1\fi
\vskip-\parskip\noindent
\hbox to\dimen0{\box0\hfil}\hangindent\dimen0\hangafter1\ignorespaces#2\par
\vskip-\parskip\noindent
\hbox to\dimen0{\box1\hfil}\hangindent\dimen0\hangafter1\ignorespaces#3\par
\endgroup}


\renewcommand{\mod}{\text{mod }}
\newcommand{\modulo}{\mod}


%\renewcommand{\vec}{}

\def\AC{{\mathsf{AC}}}
\def\AG{{\mathsf{AG}}}
\def\AND{{\mathrm{AND}}}
\def\BIN{{\mathrm{BIN}}}
\def\CRR{{\mathrm{CRR}}}
\def\CTL{{\text{CTL}}}
\def\div{{\mathrm{div}}}
\def\dom{{\mathrm{dom}}}
\def\EF{{\mathsf{EF}}}
\def\AX{{\mathsf{AX}}}
\def\EXPTIME{{\mathsf{EXPTIME}}}
\def\inp{{\mathrm{in}}}
\def\leaf{{\mathrm{leaf}}}
\def\MAJ{{\mathrm{MAJ}}}
\def\NC{{\mathrm{NC}}}
\def\Omc{{\mathbb{O}}}
\def\OR{{\mathrm{OR}}}
\def\out{{\mathrm{out}}}
\def\Pmc{{\mathcal{P}}}
\def\PSPACE{{\mathsf{PSPACE}}}
\def\TC{{\mathrm{TC}}}
\def\U{{\mathsf{U}}}
\def\WU{{\mathsf{WU}}}
\def\X{{\mathsf{X}}}
\def\EX{{\mathsf{EX}}}
\def\E{{\mathsf{E}}}

\newcommand{\prop}{\rho}

\renewcommand{\L}{\mathsf{LOGSPACE}}

%% \usepackage[thmmarks]{ntheorem}
%% %%% Theorem environments
%% \newtheorem{lemma}{Lemma}
%% \newtheorem{theorem}{Theorem}
%% \newtheorem{proposition}{Proposition}
%% \newtheorem{corollary}{Corollary}

%% \theoremstyle{definition}
%% \theoremsymbol{\ensuremath{\lozenge}}
%% \theorembodyfont{\normalfont}
%% \newtheorem{definition}{Definition}

%% \theoremstyle{nonumverplain}
%% \theoremheaderfont{\normalfont\bf}
%% \theorembodyfont{\normalfont}
%% \theoremseparator{.}
%% \theoremsymbol{\ensuremath{_\blacksquare}}
%% \newtheorem{example}{Example}

%% \theoremstyle{nonumberplain}
%% \theoremheaderfont{\normalfont\itshape}
%% \theorembodyfont{\normalfont}
%% \theoremseparator{.}
%% \theoremsymbol{\ensuremath{_\Box}}
%% \newtheorem{proof}{Proof}

% Symbols
\newcommand{\naturals}{\ensuremath{\mathbf{N}}}
\newcommand{\integers}{\ensuremath{\mathbf{Z}}}

% Machines and their ingredients
\newcommand{\automaton}{\ensuremath{A}}
\newcommand{\clocations}{\ensuremath{Q}}
\newcommand{\iclocation}{\ensuremath{q_{\mathit{in}}}}
\newcommand{\fclocation}{\ensuremath{F}}
\newcommand{\transitionrel}{\ensuremath{\Delta}}
\newcommand{\transitionlabel}{\ensuremath{\lambda}}
\newcommand{\clocationlabel}{\ensuremath{\xi}}
\newcommand{\operations}{\ensuremath{Op}}
\newcommand{\operation}{\ensuremath{op}}
\newcommand{\add}[1]{\ensuremath{\mathsf{add}(#1)}}
\newcommand{\zero}{\ensuremath{\mathsf{zero}}}
\newcommand{\propositions}{\ensuremath{L}}
\newcommand{\configuration}{\ensuremath{c}}
\newcommand{\configurations}{\ensuremath{\mathcal{C}}}
\newcommand{\run}{\ensuremath{\pi}}
\newcommand{\trace}{\ensuremath{\tau}}
\newcommand{\pocatuple}{\ensuremath{\langle \clocations,
\iclocation, \fclocation, \transitionrel, \transitionlabel,
\clocationlabel \rangle}}

\newcommand{\circuit}{\ensuremath{\mathbb{C}}}

\newcommand{\tm}{\ensuremath{\mathcal{M}}}

%%% Names
\newcommand{\oca}{\textsc{OCA} }
\newcommand{\poca}{\textsc{POCA} }
\newcommand{\Poca}{\mathbb{P}}
\newcommand{\Soca}{\mathbb{S}}

%%% Problems
\newcommand{\stsat}{\textsc{Succinct 3-SAT} }
\newcommand{\tsat}{\textsc{3-SAT} }
\newcommand{\ltlpoca}{\textsc{LTL-POCA-MC} }

%%% Complexity classes
\newcommand{\nexptime}{\textsc{NExpTime} }
\newcommand{\conexptime}{\textsc{co-NExpTime} }
\newcommand{\pspace}{\textsc{PSpace} }

%%% LTL
\newcommand{\ltl}{\text{LTL}}
\newcommand{\ctl}{\text{CTL}}
%\newcommand{\until}{\ensuremath{\mathcal{~U~}}}
\newcommand{\until}{\ensuremath{\U}}
\newcommand{\eventually}{\ensuremath{\mathsf{F}}}
\newcommand{\globally}{\ensuremath{\mathsf{G}}}
\newcommand{\propalphabet}{\ensuremath{\Sigma}}
\newcommand{\pastarred}{\ensuremath{(\mathcal{P}(\propalphabet))^{\omega}}}

%%% Logic
\newcommand{\true}{\ensuremath{\mathtt{tt}}}
\newcommand{\false}{\ensuremath{\mathtt{ff}}}

%%% Control commands
\newcommand{\defemph}[1]{\emph{#1}}







%\newtheorem{fact}{Fact}

\newcommand{\LCM}{\text{LCM}}

\begin{document}


\title{Model Checking Succinct and Parametric\\ One-Counter Automata}
\author{Stefan G\"oller\inst{1} \and
Christoph Haase\inst{2} \and
Jo\"el Ouaknine\inst{2} \and
James Worrell\inst{2}
}


\institute{
Universit\"at
Bremen, Institut f\"ur Informatik, Germany%\\
%\email{goeller@informatik.uni-bremen.de}
\and
Oxford University Computing Laboratory, UK
}

\maketitle

\begin{abstract}
  We investigate the decidability and complexity of various model
  checking problems over one-counter automata. More specifically, we
  consider \emph{succinct} one-counter automata, in which additive
  updates are encoded in binary, as well as \emph{parametric}
  one-counter automata, in which additive updates may be given as
  unspecified parameters. We fully determine the complexity of model
  checking these automata against CTL, LTL, and modal $\mu$-calculus
  specifications.
\end{abstract}


\section{Introduction}
Counter automata, which comprise a finite-state controller together
with a number of counter variables, are a fundamental and widely-studied
computational model. One of the earliest results about counter
automata, which appeared in a seminal paper of Minsky's five decades
ago, is the fact that two counters suffice to achieve Turing
completeness~\cite{Min61}.

Following Minsky's work, much research has been directed towards
studying restricted classes of counter automata and related
formalisms. Among others, we note the use of restrictions to a single
counter, on the kinds of allowable tests on the counters, on the
underlying topology of the finite controller (such as
flatness~\cite{CJ98-cav,LS08-atva}), and on the types of computations
considered (such as reversal-boundedness~\cite{ID06-tcs}). Counter
automata are also closely related to Petri nets and pushdown automata.

In Minsky's original formulation, counters were represented as integer
variables that could be incremented, decremented, or tested for
equality with zero by the finite-state controller. More recently,
driven by complexity-theoretic considerations on the one hand, and
potential applications on the other, researchers have investigated
additional primitive operations on counters, such as additive updates
encoded in binary~\cite{BHIMV06,LS08-atva} or even
in \emph{parametric} form, i.e., whose precise values depend on
parameters~\cite{BIL06-icalp,IJTW93-icalp}. We refer to such counter automata as
\emph{succinct} and \emph{parametric} resp., the former being
viewed as a subclass of the latter. Natural applications of such
counter machines include the modelling of resource-bounded processes,
programs with lists, recursive or multi-threaded programs, and XML
query evaluation; see, e.g.,
\cite{BHIMV06,CR04,ID06-tcs}.

In most cases, investigations have centered around the decidability
and complexity of the \emph{reachability} problem, i.e., whether a
given control state can be reached starting from the initial
configuration of the counter automaton. Various instances of the
reachability problem for succinct and parametric counter automata are
examined, for example, in~\cite{DG-jlc09,HKOW10,IJTW93-icalp}.

The aim of the present paper is to study the decidability and
complexity of \emph{model checking} for succinct and parametric
one-counter automata. In view of Minsky's result, we restrict our
attention to \emph{succinct one-counter automata (SOCA)} and
\emph{parametric one-counter automata (POCA)}. On the specification
side, we focus on the three most prominent formalisms in the
literature, namely the temporal logics CTL and LTL, as well as the
modal $\mu$-calculus. For a counter automaton $\mathbb{A}$ and a
specification $\varphi$, we therefore consider the question of
deciding whether $\mathbb{A} \models \varphi$,
in case of POCA for all values of the parameters,
%\footnote{In the case of
%parametric automata, one asks whether the automaton satisfies the
%formula \emph{for all} legal values of the parameters.}
and investigate both the \emph{data} complexity (in which the formula
$\varphi$ is fixed) as well as the \emph{combined} complexity of this
problem. Our main results are summarized in Table \ref{tab:results}.
\begin{table}[t]
\begin{center}
\begin{tabular}{ >{\centering\arraybackslash}m{3cm}
  >{\centering\arraybackslash}m{2cm} |
  >{\centering\arraybackslash}m{3.3cm} |
> {\centering\arraybackslash}m{3.3cm} |}\\
  &  &{\bf SOCA}&{\bf POCA}\\ \hline
{\raisebox{-2ex}[0ex]{{\bf CTL, $\mu$-calculus}}}&   {\bf data} &   & \\ \cline{2-2}
{\raisebox{-2ex}[-5ex]{{$\phantom{\text{CTL}}$}}}&  {\bf combined}&
{\raisebox{2ex}[0ex]{$\EXPSPACE$-complete}} &
{\raisebox{2ex}[0ex]{ $\Pi^0_1$-complete}}\\ \cline{1-4}
{\raisebox{-2ex}[0ex]{{\bf LTL}}}&  {\bf data} &
\multicolumn{2}{c|} {\raisebox{0ex}[0ex]{$\coNP$-complete}}
\\ \cline{2-4}
{\raisebox{-2ex}[-5ex]{{$\phantom{\text{CTL}}$}}}&  {\bf combined}&
$\PSPACE$-complete &
$\coNEXP$-complete\\ \cline{1-4}
\end{tabular}
\end{center}
\label{tab:results}
\caption{The complexity of CTL, the modal $\mu$-calculus, and LTL
             on SOCA and POCA.}
\end{table}

One of the motivations for our work was the recent discovery that
reachability is decidable and in fact $\NP$-complete for both SOCA and
POCA~\cite{HKOW10}. We were also influenced by the work of Demri and
Gascon on model checking extensions of LTL over non-succinct,
non-parametric one-counter automata~\cite{DG-jlc09}, as well as the
recent result of G\"oller and Lohrey establishing that model checking
CTL on such counter automata is $\PSPACE$-complete \cite{GoLo10}.

%% We note some interesting differences between our results and
%% corresponding questions regarding finite automata. For the latter, the
%% (combined) model checking problems for CTL, the $\mu$-calculus, and
%% LTL are respectively known to be $\mathsf{P}$-complete, in
%% $\mathsf{NP} \cap \coNP$, and $\PSPACE$-complete. Somewhat
%% surprisingly, for SOCA and POCA, the complexity ordering is reversed
%% and LTL becomes easier to model check than either CTL or the
%% $\mu$-calculus.

On a technical level, the most intricate result is the
$\EXPSPACE$-hardness of CTL model checking for SOCA, which requires
several steps. We first show that $\mathsf{EXPSPACE}$ is `exponentially
$\mathsf{LOGSPACE}$-serializable', adapting the known proof that
$\PSPACE$ is $\mathsf{LOGSPACE}$-serializable. Unfortunately, and in
contrast to~\cite{GoLo10}, this does not immediately provide
an $\EXPSPACE$ lower bound. In a subsequent delicate stage of the
proof, we show how to partition the counter in order simultaneously to
perform $\PSPACE$ computations in the counter and manipulate numbers
of exponential size in a SOCA of polynomial size.

%For reasons of space, we have had to abbreviate or omit a number of
%proofs; full details can however be found in the technical
%report~\cite{techrep}.


For reasons of space, we have had to abbreviate or omit a number of
proofs; full details can however be found in the technical
report~\cite{techrep}.


\section{Preliminaries}

\newcommand{\N}{\mathbb{N}}
\newcommand{\Z}{\mathbb{Z}}
\renewcommand{\O}{\mathbb{O}}
%\renewcommand{\P}{\mathcal{P}}

\newcommand{\trans}[1]{\stackrel{#1}{\longrightarrow}}
\newcommand{\hook}[1]{\stackrel{#1}{\hookrightarrow}}
\renewcommand{\rule}[2]{\stackrel{#2}{\stackrel{#1}{\longrightarrow}}}

\renewcommand{\M}{\mathcal{M}}

\newcommand{\bit}{\text{bit}}
\renewcommand{\succ}{\text{succ}}

\newcommand{\Reach}{\text{Reach}}
\newcommand{\Moves}{\text{Moves}}
\newcommand{\bin}{\text{bin}}
\renewcommand{\l}{\triangleright}
\renewcommand{\r}{\triangleleft}
\newcommand{\move}{\vdash}


%By $\L$, $\PSPACE$, and $\EXPSPACE$
%we denote the class of all problems that can be decided by some deterministic
%Turing machine that
%logarithmically, polynomially, and exponentially space bounded, respectively.
\newcommand{\Prop}{\mathcal{P}}
\newcommand{\Op}{\text{Op}}
%\newcommand{\add}{\mathsf{add}}
\renewcommand{\A}{\mathbb{A}}


By $\Z$ we denote the {\em integers} and by $\N=\{0,1,2,\ldots\}$ the {\em naturals}.
For each $i,j\in\Z$ we define $[i,j]=\{k\in\Z\mid i\leq k\leq j\}$ and $[j]=[1,j]$.
For each $i,n\in\N$, let $\bit_i(n)$ denote the $i^{\text{th}}$ least significant bit of
the binary representation of $n$. Hence $n=\sum_{i\in\N}2^i\cdot\bit_i(n)$.
By $\bin_m(n)=\bit_0(n)\cdots \bit_{m-1}(n)$ we denote the first $m$ least significant
bits written from {\em left to right}.
%When $m$ is not important we just write $\bin(n)$.
Let $p_i$ denote the $i^{\text{th}}$ prime number for each $i\geq 1$.
We define $\log(n)=\min\{i\ge 1 \mid 2^i> n\}$, i.e.
$\log(n)$ denotes the number of bits that are needed to represent $n$ in
binary.
%All polynomials $p:\N\rightarrow\N$ that occur in this paper, are assumed to satisfy
%$p(n)\geq n$ for each $n\in\N$.
For each word $v=a_1\cdots a_n\in\Sigma^n$ over some finite alphabet $\Sigma$
and each $i,j\in[n]$ define $v[i,j]=a_i\cdots a_j$ and
$v(i)=v[i,i]$.
For the rest of the paper, we fix a countable set of {\em atomic propositions} $\Prop$.
A {\em transition system} is a tuple $T=(S,\{S_\prop\mid \prop\in\Prop\},\rightarrow)$, where
$S$ is a set of {\em states}, $S_\prop\subseteq S$ for each $\prop\in\Prop$ and
$S_\prop$ is
non-empty for finitely many $\prop\in\Prop$, and finally
$\rightarrow \subseteq S\times S$ is a set of {\em transitions}.
We prefer to use the infix notation $s_1\rightarrow s_2$ instead of
$(s_1,s_2)\in\rightarrow$.
 An {\em infinite path} is an infinite sequence
$\pi=s_0\rightarrow s_1\rightarrow \cdots$.
% We denote
%by $|\pi|$ the length of the path $\pi$, which we define to be $\omega$
%if $\pi$ is infinite.
For each infinite path
$\pi= s_0\rightarrow s_1 \rightarrow \cdots$ and each
$i\in\N$, we denote by $\pi^i$ the suffix $s_i\rightarrow
s_{i+1}\rightarrow\cdots$
and by $\pi(i)$ the state $s_i$.
% Comment: Formal definition not needed
%Define the {\em trace of $\pi$} as $\trace(\pi):\N\rightarrow2^\Prop$ where
%$\trace(i)=\{\prop\in\Prop\mid s_i\in S_\prop\}$  for each $i\in\N$.
%\noindent
%\textbf{Succinct and parametric one-counter automata: }
A {\em SOCA} is a tuple
 $\Soca=(Q,\{Q_\prop\mid \prop\in\Prop\},E,\lambda)$, where $Q$ is a finite set
of {\em control states}, $Q_\prop\subseteq Q$ for each $\prop\in\Prop$ and
$Q_\rho$ is non-empty for finitely many $\prop\in\Prop$,
$E\subseteq Q\times Q$ is a finite set of {\em transitions}, and
$\lambda:E\rightarrow \Z\cup\{\zero\}$.
We call a SOCA $\Soca$ with $\lambda:E\rightarrow
\{-1,0,1\}\cup\{\zero\}$ a {\em unary one-counter automaton (OCA)}.
A {\em POCA} is a tuple
$\Poca(X)=(Q,\{Q_\prop\mid\prop\in\Prop\},E,\lambda)$, where the first
three components are same as for a SOCA,
$X$ is a finite set of {\em parameters over the naturals},
and $\lambda:E\rightarrow
(\Z\cup\{\zero\}\cup\{-x, +x\mid x\in X\})$.
For each assignment $\sigma:X\rightarrow\N$  the
induced SOCA is defined as $\Poca^\sigma=(Q,E,\lambda')$
where $\lambda'(e)=\sigma(x)$ if $\lambda(e)=x$,
$\lambda'(e)=-\sigma(x)$ if $\lambda(e)=-x$, and
$\lambda'(e)=\lambda(e)$ otherwise.
%% $\lambda'(e)=\circ\sigma(x)$ whenever $\lambda(e)=\circ x$
%% for some $\circ\in\{+,-\}$, $x\in X$, and $\lambda'(e)=\lambda(e)$ otherwise.
If $X=\{x\}$ we also write $\Poca(x)$ instead of $\Poca(X)$.
The {\em size} of a POCA is defined as
$|\Poca|=|Q|+|X|+|E|\cdot\max\{\log(|a|)\mid a\in\lambda(E)\cap\Z\}$.
Hence, we represent each appearing
integer in binary.
The size of a SOCA is defined analogously.
A SOCA $\Soca=(Q,\{Q_\prop\mid \prop\in\Prop\},E,\lambda)$ describes a transition system $T(\Soca)=(Q\times\N,\{Q_\prop\times\N\mid
\prop\in\Prop\}, \rightarrow)$, where for each $q_1,q_2\in Q$ and each
$n_1,n_2\in\N$ we have $q_1(n_1)\rightarrow q_2(n_2)$ iff
either $\lambda(q_1,q_2)=n_2-n_1$, or both
$n_1=n_2=0$ and $\lambda(q_1,q_2)=\zero$.


\section{$\CTL$ Model Checking}
Formulas $\varphi$ of CTL are given by the following
grammar, where $\prop$ ranges over $\Prop$:
\begin{equation*}
\varphi\quad ::=\quad \prop\ \mid \neg\varphi\ \mid\ \varphi\wedge\varphi\ \mid\
\EX\varphi\ \mid \E (\varphi\U\varphi)\ \mid\
\E(\varphi\WU\varphi)
\end{equation*}
\iffalse

\noindent
We have the following abbreviations:
$\varphi_1\vee\varphi_2=\neg(\neg\varphi_1\wedge\neg\varphi_2)$,
$\varphi_1\rightarrow\varphi_2=\neg\varphi_1\vee\varphi_2$,
$\true=\prop\vee\neg\prop$ for some atomic proposition $\prop\in\Prop$, $\false=\neg\true$,
$\EF\varphi=\E(\true\U\varphi)$, and $\AX\varphi=\neg\EX\neg\varphi$.
%The {\em size} of a CTL formula is inductively defined as follows:
%$|\prop|=1$ for each $\prop\in\Prop$, $|\neg\varphi|=|\EX\varphi|=|\varphi|+1$,
%$|\varphi_1\wedge\varphi_2|=|\E\varphi_1\U\varphi_2|=|\E\varphi_1\WU\varphi_2|=
%|\varphi_1|+|\varphi_2|+1$.
\fi
Given a transition system $T=(S,\{S_\prop\mid \prop\in\Prop\},\rightarrow)$, a state
$s\in S$, and some $\CTL$ formula $\varphi$, define $(T,s)\models\varphi$ by
induction on the structure of $\varphi$ as follows:
\begin{eqnarray*}
  (T,s)\models \prop &  \quad\iff\quad & s\in S_\prop\quad \text{ for each
}\prop\in\Prop\\
  (T,s)\models \varphi_1\wedge\varphi_2 &  \quad\iff\quad &
(T,s)\models\varphi_1\text{ and }
(T,s)\models \varphi_2 \\
  (T,s)\models \neg\varphi &  \quad\iff\quad & (T,s)\not\models\varphi\\
  (T,s)\models \EX\varphi &  \quad\iff\quad & (T,t)\models\varphi
\text{ for some } t\in S \text{ with }s\rightarrow t\\
  (T,s)\models \E(\varphi_1\U\varphi_2) &  \quad\iff\quad &
\exists s_0,\ldots,s_n\in S, n\geq 0: s_0=s, (T,s_n)\models\varphi_2, \\
&& \forall i\in[0,n-1]: (T,s_i)\models\varphi_1\text{ and } s_i\rightarrow
s_{i+1}\\
  (T,s)\models \E(\varphi_1\WU\varphi_2) &  \quad\iff\quad &
(T,s)\models\E(\varphi_1\U\varphi_2)\ \text{ or } \exists s_0,s_1,\ldots\in
S.\forall i\geq 0:\\
&& s=s_0,(T,s_{i})\models\varphi_1\text{ and }
s_i\rightarrow s_{i+1}
\end{eqnarray*}

\noindent
Subsequently we use the standard abbreviations for disjunction, and
implication.  Moreover, we define $\true=\prop\vee\neg\prop$ for some
$\prop\in\Prop$ and $\EF\varphi=\E\ (\true\ \U\varphi)$.  Let us
define the {\em CTL model checking problem} on SOCA and POCA
resp.\\[-0.3cm]

\problemx{CTL model checking on SOCA}
{SOCA $\Soca=(Q,E,\lambda)$, $q\in Q$, $n\in\N$ in binary, a CTL formula
$\varphi$.}
{Does $(T(\Soca),q(n))\models\varphi$?}
\ \\[-0.4cm]

\problemx{CTL model checking on POCA}
{POCA $\Poca(X)=(Q,E,\lambda)$, $q\in Q$, $n\in\N$ in binary,  a CTL formula
$\varphi$.}
{Does $(T(\Poca^\sigma),q(n))\models\varphi$ for every $\sigma:X\rightarrow\N$?}


\subsection{Upper Bounds}{\label{S Upper}}
\noindent
%\textbf{Upper bounds:} First, let us state the upper bounds on model
%checking CTL on SOCA and
% Comment: Moved to preliminaries
%% {\em One-counter automata (OCA)} are SOCA in which
%% the numbers that occur in the transition labels are represented  in unary
%% instead in binary.
%% The following theorem gives an upper bound on model checking the
%% {\em modal $\mu$-calculus} on OCA.
Due to space restrictions we do not formally introduce the modal
$\mu$-calculus and refer to \cite{ArnNiw01} for more details instead.
In \cite{Serr06} Serre showed that the combined complexity of the
modal $\mu$-calculus on OCA is in $\PSPACE$.  Since every SOCA can be
transformed into an OCA of exponential size, and since each CTL
formula can be translated into an alternation-free $\mu$-calculus
formula with a linear blowup, the following proposition is immediate
by adjusting the resulting $\mu$-calculus formula appropriately.
Moreover, this immediately implies the containment in the arithmetic
hierarchy of the combined complexity of the modal $\mu$-calculus and
CTL on POCA.

%a $\Pi^0_1$ upper bound for the combined complexity of
%the modal $\mu$-calculus and CTL on POCA.
\begin{proposition}{\label{P CTL up}}
For the modal $\mu$-calculus and CTL the combined complexity on SOCA is in
$\EXPSPACE$, whereas it is in $\Pi^0_1$ on POCA.
\end{proposition}


\subsection{Hardness of the Data Complexity of CTL on SOCA}{\label{S CTL hardness}}
Before we prove $\EXPSPACE$-hardness of the data complexity of CTL on SOCA, we
introduce some notions and results from complexity theory.
Given a language $L\subseteq\Sigma^*$, let $\chi_L:\Sigma^*\rightarrow\{0,1\}$
denote the {\em characteristic function of $L$}.
We define the {\em lexicographic
order on $n$-bit strings}
% Comment: obsolete
%as $\preceq_n\subseteq\{0,1\}^n\times\{0,1\}^n$, where
by $x\preceq_n y$ if and only if
$\bin_n(x)\leq\bin_n(y)$, e.g. $011 \preceq_3 101$.
\renewcommand{\C}{\mathcal{C}}
We say a language $L$ is {\em exponentially $\C$-serializable via some language
$R\subseteq\{0,1\}^*$} if there is some polynomial $p(n)$ and some language
$U\in\C$ s.t.\ for all $x\in\{0,1\}^n$\\[-0.2cm]
$$x\in L\qquad\!\!\iff\qquad\!\!
\chi_U\left(x \cdot 0^{2^{p(n)}}\right)\quad \cdots\quad \chi_U\left(x
\cdot 1^{2^{p(n)}}\right)\in
R,\\[-0.1cm]
$$ where the bit strings on the right-hand side of the
concatenation symbol are enumerated in
lexicographic order.
%% by `$\cdots$' we
%% refer to $\preceq_{2^{p(n)}}$ in each of the second components.
This definition is a padded variant of the serializability notion used
in \cite{GoLo09}, which in turn is a variant of the serializability notion
from \cite{CaFu91,HLSVW93,Vol98}.
Some subtle technical padding arguments are required
to lift $\AC^0$-serializability of $\PSPACE$, proven in
Theorem 22 in \cite{GoLo09}, to exponential
$\L$-serializability of $\EXPSPACE$.%, for proving the following theorem.

\begin{theorem}\label{T succ seri}
For every language $L$ in $\EXPSPACE$ there is some regular language $R$ such
that $L$ is exponentially $\L$-serializable via $R$.
\end{theorem}
\renewcommand{\M}{\mathcal{M}}
%\noindent
%\textbf{Chinese remainder representation: }
A further concept we use is the Chinese remainder
representation of a natural number. For every $m,M\in\N$ we denote by $\CRR_m(M)$ {\em the Chinese remainder
representation} of $M$ as the Boolean tuple
$(b_{i,c})_{i\in[m],0\leq c<p_i}$, where $b_{i,c}=1$ if $M\modulo p_i=c$
and $b_{i,c}=0$ otherwise.
The following theorem tells us that in logarithmic space we can
compute the binary representation of a natural number from its Chinese
remainder representation.
This result is a consequence of \cite{ChDaLi01}, where it is shown that division is in logspace-uniform
$\mathsf{NC}^1$.


\begin{theorem}[\cite{ChDaLi01} Theorem 3.3]{\label{T CRRBIN}}
The following problem is in $\L$:

INPUT: $\CRR_m(M),j\in[m]$, $b\in\{0,1\}$.

QUESTION: Is $\bit_j(M\ \modulo 2^m)=b$?
\end{theorem}
In the rest of this section, we sketch the proof of $\EXPSPACE$-hardness of
the data complexity of CTL on SOCA.
Let $L\subseteq\{0,1\}^*$ be an arbitrary language in $\EXPSPACE$.
Then by Theorem \ref{T succ seri}, there is some regular
language $R\subseteq\{0,1\}^*$ s.t.\ $L$ is exponentially
$\L$-serializable via $R$. Hence there is some
language $U\in\L$ s.t.\ for all $x\in\{0,1\}^n$ we have\\[-0.2cm]
\begin{eqnarray}{\label{E EXPSPACE char}}
x\in L\qquad\iff\qquad \chi_{U}\left(x \cdot 0^{2^{p(n)}}\right)
\quad \cdots\quad \chi_U\left(x \cdot 1^{2^{p(n)}}\right) \in R,
\end{eqnarray}
\noindent
where the bit strings on the right-hand side of the concatenation
symbol are enumerated in lexicographic order.  For the rest of this
section, let us fix an input $x_0\in\{0,1\}^n$.  Let $N=p(n)$ and
$A=(Q,\{0,1\},q_0,\delta,F)$ be some deterministic finite automaton
with $L(A)=R$.  Let us describe equivalence (\ref{E EXPSPACE char})
differently: We have $x_0\in L$ iff the program in Fig.\ \ref{F prog}
returns $\mathtt{true}$.  We are going to mimic the execution of the
program by a fixed CTL formula and a SOCA that can be computed from
$x_0$ in logarithmic space.
\begin{wrapfigure}{r}{0.3\textwidth}
%  \begin{center}
\normalsize
\vspace{-0.5cm}
 $q\in Q$; $q:=q_0$;\\[-0.4cm]

 $d\in\N$; $d:=0$;

$b\in\{0,1\}$;\\[-0.4cm]

 $\mathtt{while }\  d\not=2^{2^N}$\ $\mathtt{ loop}$\\[-0.4cm]

$\ \ \ \ b:=\chi_U(x_0\cdot\text{bin}_{2^N}(d))$;\\[-0.4cm]

 $\ \ \ \ q:=\delta(q,b)$;\\[-0.4cm]

$\ \ \ \ d:=d+1$;\\[-0.4cm]

$\mathtt{end loop}$\\[-0.4cm]

 $\mathtt{return}\ q\in F$;%\\[0.4cm]

 % \end{center}
\caption{A program that returns $\mathtt{true}$ iff $x_0\in L$.}{\label{F
U}}{\label{F prog}}
\vspace{-0.3cm}
\end{wrapfigure}
Before we start with the reduction, let us discuss the obstacles
that arise:

\textbf{(A)} We need some way of storing $d$ on the counter. Of course
there are a lot of ways to do this, but since we want to access
all bits of $d$ in the assignment $b:=\chi_U(x_0\cdot \bin_{2^N}(d))$, the most
natural way is probably to represent $d$ in binary.
However, for this $2^N$ bits are required.
 More problematically, we need to be able to check if $d$ is equal to
$2^{2^N}$. This cannot be achieved by a transition in a SOCA
 that subtracts $2^{2^N}$, since the representation of this number
requires exponentially many bits in $n$.
\textbf{(B)} As in \cite{GoLo10}, a solution to obstacle (A) is to store $d$ in
Chinese remainder representation with the first $2^N$ prime numbers.
A polynomial number of bits (in $n$) suffice
to represent each of the occurring prime numbers, but
we need exponentially many of them.
Thus, we cannot equip a polynomial size SOCA with transitions for
each prime number, simply because there are too many of them.
\textbf{(C)}
The assignment $b:=\chi_U(x_0\cdot \bin_{2^N}(d))$ implies that we need to simulate on
the counter a logarithmically space bounded DTM for the language $U$
on an exponentially
large input (in $n$).
Speaking in terms of the input size $n$, this means that we need to provide
polynomially many bits on the counter that can be used to describe the
working tape for this DTM. However, we need to provide
some on-the-fly mechanism for reading the input.

\noindent
Let us give a high-level description of the $\EXPSPACE$-hardness proof.
In the first step, we carefully design a data structure on the counter
and explain the intuition behind it.
In the second step, we list five queries which we aim at implementing via
fixed CTL formulas and by SOCA that can be computed from the input $x_0$
in logarithmic space.

\smallskip

\noindent
{\bf The data structure and how to access it: }
 Let $K=n+2^N+1$ denote the number of bits that are required to store an input
for $U$.
Let $\alpha=\log K$ denote the number of bits that we require for
storing a pointer to an input for $U$ and let $\beta$
be the number of bits that suffice for storing the
$K^{\text{th}}$ prime. Hence $\alpha=O(N)$ and by
the Prime Number Theorem, it follows that
$\beta= O(\log(K\log(K)))=O(N)$.
The number $\alpha$ and such a sufficiently large number $\beta$ can be computed
 from $x_0$ in logarithmic space.

\begin{wrapfigure}{r}{0.4\textwidth}
\begin{center}
\begin{picture}(85,13)(5,-10)

\gasset{Nframe=n,loopdiam=9,ELdist=.7}
\gasset{Nadjust=wh,Nadjustdist=1}
\gasset{curvedepth=0}


\put(13,-2){\small $\tau_1$}
\put(41,-2){\small $\tau_2$}
  \node(A)(20,0){$q_{\bit,i}$}
  \node(B)(40,0){$\bullet$}
\drawloop(A){$-2^{i+1}$}
\drawedge(A,B){$-2^i$}

\put(4,-10){
\scriptsize
$%\tau_1\wedge
\varphi_{\bit,1}=\tau_1\wedge\EF(\tau_1\wedge\neg\EX \tau_1\wedge\EX
\tau_2)$
}
\put(4,-14){
\scriptsize
$%\tau_1\wedge
\varphi_{\bit,0}=\tau_1\wedge\EF(\tau_1\wedge\neg\EX \tau_1\wedge\neg\EX
\tau_2)$
}


\end{picture}
\end{center}

\caption{SOCA $\Soca_{\bit,i}$ and CTL formulas
$\varphi_{\bit,b}$ for checking if $\bit_i(v)=b$.}  {\label{F Bit}}
\end{wrapfigure}
Let us describe the data structure on the counter in our reduction.
Assume that the counter value is $v\in\N$.
We are interested only in the $l$ least
significant bits of the binary representation of $v$, where $l$ is some number
that is exponentially bounded in $n$; the
value of $l$ will be made clear below.
Assume $V=\bit_0(v)\cdots\bit_{l-1}(v)$.
We imagine $V$ to be factorized into blocks of bits\\[-0.4cm]
 $$V\ \ =\ \ I\ M\ C\ J\ X\ Y\ Z\ B\quad\qquad(*)$$
where
$I\in\{0,1\}^{\alpha}$ is a prime number index;
$M\in\{0,1\}^\beta$ is intended to represent the $I^{\text{th}}$
prime number $p_I$;
$C\in\{0,1\}^\beta$ is some residue class
modulo $M$;
 $J\in\{0,1\}^\alpha$ represents a pointer to some bit of $B$;
$X,Y$ and $Z$ consist of polynomially many bits (in $n$) and are
 intended to represent the working tape of three space-bounded
DTMs that we will comment on later in more detail; and
$B\in\{0,1\}^{n+2^N}$ with $B=x_0B'$ for some
$B'\in\{0,1\}^{2^N}$.
Our intention is that $B$ represents the current input for $U$, and
in particular
$B'$ represents the counting variable $d$ from Fig. \ref{F prog}.
%The block $B'$ consists of $2^N+1$ bits since we want to be able to test
% if $d=2^{2^{N}}$.
\noindent
Throughout the rest of this section, $v$
will denote an arbitrary natural number.
Moreover
$I,M,C,J,X,Y,Z$ and $B$ will implicitly be coupled with $v$ via the
factorization $(*)$.
Note that all of the bit strings have polynomial length in $n$ except for
$B$.
Subsequently, we identify each of the blocks with the natural number they represent. A simple but powerful gadget, which will subsequently be used
to check for each $b\in\{0,1\}$ if the $i^{\text{th}}$ bit
of the counter is $b$, is
shown in Fig.\ \ref{F Bit}. We have that
$q_{\bit,i}(v)$ satisfies $\varphi_{\bit,b}$ iff $\bit_i(v)=b$,
for each $b\in\{0,1\}$.

%
\renewcommand{\Re}{\Rightarrow}
\newcommand{\Li}{\Leftarrow}
\newcommand{\re}{\rightarrow}
\newcommand{\li}{\leftarrow}
%
%\noindent
\iffalse
Let us introduce some more notation for addressing
leftmost (starting to count from $0$) and rightmost bit
positions in each of the above sequences from $(*)$.
For each such sequence $\Theta$ let $\Theta_\Li$ denote
the {\em absolute position of the leftmost bit} of $\Theta$ in $V$
and let $\Theta_\re$ denote
the {\em position of the rightmost bit of $\Theta$ relative
to $\Theta_\Li$}, i.e. $\Theta_\re=|\Theta|-1$.
For example $C_\Li=\alpha+\beta$ and $J_\re=\alpha-1$.
\fi
%

\smallskip
\noindent
{\bf Queries that we need to implement: } Next, we list five queries that we aim
at answering by instances of the model checking problem.
Each query is based on its preceding queries.
\begin{enumerate}[(Q1)]
\item  Assuming $C<M$, does $C\equiv B\ \mod M$ hold?
\item  Is $M$ the $I^{\text{th}}$ prime number, i.e. $M=p_I$?
\item  What is $\bit_J(B)$?
\item  Does $(B[1,n] \cdot B[n+1,n+2^N])\in U$ hold?
\item  Does $x_0\in L$ hold?
\end{enumerate}
\noindent
$\EXPSPACE$-hardness of data complexity of CTL on SOCA will hence follow from the
implementation of query Q5.
%Before we give the implementation of Q1,
Let $\gamma=O(N)$ denote the absolute
value of the leftmost bit position of $B$ in $V$.
Hence, when the $\gamma^{\text{th}}$ bit of $V$ is set to $1$ and we subtract
$2^\gamma$ from the counter, then the leftmost bit of $B$ is set to $0$.
Similarly, let $\mu$ denote the leftmost bit position of $M$ in $V$.
Q1 can now be realized as follows.

\begin{lemma}{\label{L mod}} There is a  CTL formula $\varphi_\mod$ s.t.
we can compute from $x_0$ in logarithmic space a SOCA $\Soca_\mod$ and a
control state $q_\mod$ s.t.
$(T(\Soca_{\mod}),q_\mod(v))\models\varphi_\mod$ iff $C\equiv B\ \mod M$.
\end{lemma}
\begin{proof}
For brevity, we illustrate the special case $C=0$, i.e.
$(T(\Soca_\mod),q_\mod(v))\models\varphi_\mod$ iff
$B\equiv 0\ \mod M$.
The SOCA $\Soca_{\mod}$ contains four atomic propositions $\prop_0,\prop_1,y,z$
 and is depicted below. The $\CTL$ formula $\varphi_\mod$ expresses that we
traverse the sequence of diamonds and thereby repeatedly subtract $M$ from $B$.
The number of diamonds equals $\beta$, the number of bits of $M$.
One diamond corresponds to one bit of $M$. In case bit $\beta$ of $M$ is $1$,
which we can verify by a transition to the initial control state of
the SOCA $\Soca_{\bit,\mu+\beta}$ (see Fig. \ref{F Bit}),
 we subtract $2^{\gamma+\beta}$ from $B$, otherwise we do not modify the counter value.
This process is repeated until we reach the last diamond in which we consider
the first bit of $M$.
Finally,  the transition from $q_{\mod}$ to the control state satisfying $z$
serves for checking if $B=0$ by trying to subtract $2^\gamma$.
%\begin{figure}
\begin{center}
\begin{picture}(100,44)(5,33)
%   \put(-20,-28){\framebox(85,33){}}
  %\gasset{Nadjust=w,Nadjustdist=2,Nh=6,Nmr=1}
 % \gasset{Nadjust=w,Nadjustdist=2,Nh=6}%$,Nmr=1}
 % \node(A)(0,0){$q_0$}
%\put(15,-5){$q(M_{\abs,\r})$}
%\put(43,-5){$p_2$}
\drawpolygon[Nframe=n,Nfill=y,fillgray=.9](5,33)(28,33)(28,47)(5,47)
\put(10,35){\large$\Soca_{\bit,\mu+\beta}$}

\drawpolygon[Nframe=n,Nfill=y,fillgray=.9](35,33)(58,33)(58,47)(35,47)
\put(38,35){\large$\Soca_{\bit,\mu+\beta-1}$}


\drawpolygon[Nframe=n,Nfill=y,fillgray=.9](85,33)(108,33)(108,47)(85,47)
\put(90,35){\large$\Soca_{\bit,\mu}$}


\gasset{Nframe=n,loopdiam=9,ELdist=.7}
\gasset{Nadjust=wh,Nadjustdist=1}
\gasset{curvedepth=0}


  \node(1)(0,60){$q_{\mod}$}
  \put(-6,62){$y$}
  \node(z)(0,40){$\bullet$}
  \put(-3,40){$z$}
\drawedge[ELside=r](1,z){$-2^{\gamma}$}
  \node(2O)(15,65){$\bullet$}
  \node(2U)(15,55){$\bullet$}
  \put(17,67){\small$\prop_1$}
  \put(17,53){\small$\prop_0$}

  \node(1A)(15,42){$q_{\bit,\mu+\beta}$}
  \node(4A)(45,42){$q_{\bit,\mu+\beta-1}$}

  \node(4A)(45,42){$q_{\bit,\mu+\beta-1}$}

  \node(7A)(95,42){$q_{\bit,\mu}$}


  \node(3)(30,60){$\bullet$}
  \node(4O)(45,65){$\bullet$}

  \put(47,67){\small$\prop_1$}
  \put(47,53){\small$\prop_0$}
  \node(4U)(45,55){$\bullet$}
  \node(5)(60,60){$\bullet$}

  \node(6)(80,60){$\bullet$}
  \node(7O)(95,65){$\bullet$}
  \put(97,67){\small$\prop_1$}
  \put(97,53){\small$\prop_0$}

  \node(7U)(95,55){$\bullet$}
  \node(8)(110,60){$\bullet$}



  \put(65,58){\huge$\cdots$}


%\drawedge[curvedepth=-70](8,1){$0$}

\drawedge(1,2O){$0$}
\drawedge[ELside=r](1,2U){$0$}
\drawedge[ELside=l,ELpos=60](2O,3){$-2^{\gamma+\beta}$}
\drawedge(2U,3){$0$}

\drawedge[curvedepth=-6,ELpos=70,ELside=r](2O,1A){$0$}
\drawedge(2U,1A){$0$}

\drawedge[curvedepth=-6,ELpos=70,ELside=r](4O,4A){$0$}
\drawedge(4U,4A){$0$}

\drawedge[curvedepth=-6,ELpos=70,ELside=r](7O,7A){$0$}
\drawedge(7U,7A){$0$}



\drawedge(3,4O){$0$}
\drawedge[ELside=r](3,4U){$0$}
\drawedge[ELside=l,ELpos=70](4O,5){$-2^{\gamma+\beta-1}$}
\drawedge(4U,5){$0$}

\drawedge(6,7O){$0$}
\drawedge[ELside=r](6,7U){$0$}
\drawedge[ELside=l,ELpos=60](7O,8){$-2^{\gamma}$}
\drawedge(7U,8){$0$}

\drawcurve(110,60)(112,62)(55,75)(0,63)(0,62)


%\drawedge[ELpos=70,curvedepth=1.5](8,9A){$0$}
%\drawedge[ELside=r,ELpos=60](8,9B){$0$}
  \put(60,76){$0$}
%\drawedge[ELpos=50](8,9C){$0$}
%\drawedge[ELside=r,ELpos=50](8,9D){$-2^{B_L}$}


%%%%COPY


\end{picture}
%\caption{The diamond SOCA $\Soca_\mod$}{\label{F Smod}}
\end{center}
%\end{figure}

\noindent
We put
$
\varphi_\mod=\E \left(\bigwedge_{b\in\{0,1\}} \prop_b
\rightarrow \EX \varphi_{\bit,b} \right) \U (y\wedge\neg\EX z)
$.
\qed
\end{proof}

\newcommand{\BIT}{\text{BIT}}
\renewcommand{\R}{\mathcal{R}}

\noindent
Let us give some informal ideas on how to implement the queries
Q2 to Q5. We strongly recommend
the reader to consult the technical report \cite{techrep}
to understand the technical subtleties.

For implementing Q2 we simulate with a SOCA $\Soca_\prim$ some
polynomially space-bounded DTM
that decides, on the input $\langle I,M\rangle$, whether
$M=p_I$. We use the bit string $X$ from $(*)$ for storing the working tape
of this DTM on the counter.
The current input and working tape symbol and the position of the
input and working tape in the counter can directly be hard-wired
into the control states of $\Soca_\prim$.
We can construct a fixed CTL formula that
simulates the computation of this DTM.

Implementing Q3, i.e. deciding the $J^{\text{th}}$ bit of
$B$, is more involved.
Recall that $B$ consists of $n+2^N$ bits and $J$ consists of
$\alpha=O(N)$ bits.
Hence checking if $\bit_J(B)=1$ cannot be done in a similar fashion as in
Fig. \ref{F Bit}, since $J$ is too big.
The solution is the following: By making use of $\Soca_\prim$, one can initialize
 $M$ with $p_I$ and after that
decide if $C\equiv B\ \mod p_I$
by making use of $\Soca_{\mod}$ and $\varphi_\mod$
from Lemma \ref{L mod}.
Hence, we can access bits of the Chinese remainder representation of $B$.
Let us assume that $\R=\CRR_{K}(B)=(b_{i,c})_{1\leq i\leq K,0\leq c<p_i}$
is the Chinese remainder representation of $B$.
Observe that $|\R|$ is exponential in $n$ and
$\R$ is not stored anywhere on the counter.
However we
can use the bit strings $I$ and $C$ as pointers to access
the bit $b_{I,C}$ of $\R$.
By Theorem \ref{T CRRBIN}, given $\R$ (in our case on-the-fly by the pointers
$I$ and $C$), the bit string $J$ and $b\in\{0,1\}$,
we can decide if $\bit_J(B)=b$
by simulating a logarithmically space-bounded DTM
on the input $\langle\R,J,b\rangle$ of exponential size.
In the block $Y$ of $(*)$ we reserve the space that this DTM requires.
Q4 can be implemented similarly as Q3 by simulating
a logarithmically space-bounded DTM that decides $U$ on input
$B[1,n]\cdot B[n+1,n+2^N]$ of exponential size.
We use $Z$ for simulating the working tape and the bit sequence $J$ as a
pointer to access the bits of $B$.

For implementing Q5 we
simulate the program from in Fig.\ \ref{F prog}.
Recall that our bit sequence $B$ is of length $n+2^N$.
We initialize the first $n$ bits of $B$ with $x_0$.
The remaining bit sequence $B'$ stores the variable $d$ of the program, initialized with $0$
and being repeatedly incremented by adding $2^{\gamma+n}$.
%Note that incrementing $B'$ by $1$ corresponds to adding $2^{B_\Li+n}$ to the one-counter.
Thus, checking when $d$ becomes $2^{2^N}$ for the first time boils down to checking
when $B'$ overflows for the first time.
This can be checked by initializing $J$ appropriately and
being able to access the $J^{\text{th}}$ bit via query Q3.
The states of the automaton $A$ can directly be handled by the control states of the
SOCA. To obtain $\chi_U(x_0,\bin_{2^N}(d))$, we invoke the query Q4 and store
this bit in the control state of the SOCA. This concludes our $\EXPSPACE$-hardness proof.

\begin{theorem}
The data complexity and the combined complexity of CTL
and the modal $\mu$-calculus on SOCA is $\EXPSPACE$-complete.
\end{theorem}



\iffalse
\begin{lemma}{\label{L prim}} There is some fixed $\CTL$ formula $\varphi_\prim$ s.t.\
from our input $x_0$ we can compute in logarithmic space some SOCA
$\Soca_\prim$ and some control state $q_\prim$ s.t.\
$(T(\Soca_\prim),q_\prim(v))\models\varphi_\prim$ %\quad\Leftrightarrow\quad
iff $M=p_I$.
\end{lemma}
\begin{proof}
By Proposition \ref{P prime} there is some $q(n)$ space-bounded Turing machine
$\M=(S,\Sigma,s_0,F,\mu)$
over the alphabet $\Sigma=\{0,1,\$\}$ that decides,
given $\bin(i)\$\bin(m)$, whether $m=p_i$, where $q$ is some polynomial.
The idea is to simulate $\M$ on input $z=I\$M$ and using $X$ as a working tape.
For this, we define our block $X$ to consist of precisely
$l=q(|I|+|M|+1)=q(\alpha+\beta+1)$ many bits.
Moreover, we assume w.l.o.g.\ that $\M$'s behaviour on input $I\$M$ is
independent on the content of its initial working tape: This can be
achieved by adding extra states to $\M$ that first write $0^l$ onto
the working tape.
Hence, the initial configuration of $\M$ on input $z$ is
$(s_0,\l z\r,\l w \r,0,0)$ for some $w\in\{0,1\}^l$.
\iffalse
Before we give our SOCA $\Soca_\prim$, we describe the computation of
$\M$ on input $I\$M$ as a pseudo program that terminates iff $M=p_I$:\\


\ \hspace{2cm} $s\in S$; $s:=s_0$;\quad  (current state of $\M$)\\[-0.4cm]

\ \hspace{2cm} $i\in[0,\alpha+\beta+2]$; $i:=0$;\quad (current input head position)

\ \hspace{2cm} $h\in[0,l+1]$; $h:=0$;\quad (current working head position)

\ \hspace{2cm} $a\in\{0,1,\$,\l,\r\}$; $a:=\l$;\quad (current input symbol)

\ \hspace{2cm} $b\in\{0,1,\l,\r\}$; $b:=\l$;\quad (current working tape
symbol)\\[0.05cm]

\ \hspace{2cm} $\mathtt{while }\  s\not\in F$\ $\mathtt{ loop}$\\[-0.4cm]

\ \hspace{3cm} $a:=\begin{cases} \l& \text{ if }i=0\\
\r &\text{ if }i=\alpha+\beta+2\\
\$ &\text{ if }i=\alpha+1\\
\bit_i(I) &\text{ if }i\in[1,\alpha]\\
\bit_{i-\alpha-1}(M) &\text{ otherwise }
\end{cases}$

\ \hspace{3cm} $b:=\begin{cases} \l& \text{ if }h=0\\
\r &\text{ if }h=l+1\\
\bit_{h}(X) &\text{ otherwise }
\end{cases}$



%\\[-0.4cm]

\ \hspace{3cm} Let $\mu(s,a,b)=(s',\delta_1,\delta_2,b')$.\\[-0.4cm]

\ \hspace{3cm} $s:=s '$;\\[-0.4cm]

\ \hspace{3cm} $\bit_h(X):=b'$;\\[-0.4cm]

\ \hspace{3cm} $i:=i+\delta_1$;\\[-0.4cm]

\ \hspace{3cm} $h:=h+\delta_2$;\\[-0.4cm]

\ \hspace{2cm} $\mathtt{end loop}$\\[-0.2cm]

\fi
\newcommand{\inpu}{\text{input}}
\newcommand{\inz}{\text{input}_0}
\newcommand{\ino}{\text{input}_1}
\newcommand{\work}{\text{work}}
\newcommand{\workz}{\text{work}_0}
\newcommand{\worko}{\text{work}_1}


\noindent
Let us describe our SOCA $\Soca_\prim$.
The control states $Q$ of $\Soca_\prim$ will contain
$$
S\times[0,\alpha+\beta+2]\times[0,l+2]\times
\{0,1,\$,\l,\r\}\times\{0,1,\l,\r\}, \qquad\text{where}
$$
the first component corresponds to the variable $s$,
the second component corresponds to the variable $i$,
the third component corresponds to the variable $h$,
the fourth component corresponds to the variable $a$, and
finally the fifth component corresponds to the variable $b$
of the above program.
Moreover, our SOCA will contain
the SOCA $\Soca_{\bit,I_\Li},\ldots,\Soca_{\bit,I_\Re}$ and
$\Soca_{\bit,X_\Li},\ldots,\Soca_{\bit,X_\Re}$ in order to test if
certain bits
in $I$ and $X$ are set correctly.
Hence $Q$ contains the control states
$q_{\bit,I_\Li},\ldots,q_{\bit,I_\Re}$ and
$q_{\bit,X_\Li},\ldots,q_{\bit,X_\Re}$ as well.

\noindent
We will provide the atomic propositions
$\Prop=\{F,\inpu,\work,\inz,\ino,\workz,\worko\}$,
where
\begin{eqnarray*}
Q_F&\ =\ &\{(s,i,h,a,b)\in Q\mid s\in F\},\\
Q_\inpu&\ =\ &\{q_{\bit,k}\mid k\in[I_\Li,I_\Re]\},\\
Q_\work&\ =\ &\{q_{\bit,k}\mid k\in[X_\Li,X_\Re]\},\\
Q_{\inz}&\ =\ &\{(s,i,h,a,b)\in Q\mid a=0\},\\
Q_{\ino}&\ =\ &\{(s,i,h,a,b)\in Q\mid a=1\},\\
Q_{\workz}&\ =\ &\{(s,i,h,a,b)\in Q\mid b=0\},\\
Q_{\worko}&\ =\ &\{(s,i,h,a,b)\in Q\mid b=1\}.\\
\end{eqnarray*}
We do not give all transitions of $\Soca_\prim$, but illustrate
them at an example.
Let us give the outgoing transitions of the control state
$(s,i,h,0,1)$, where $i\in[I_\Li,I_\Re]$ and $h\in[X_\Li,X_\Re]$, i.e. we currently scan
a bit of $I$ and do not read a marker on our working tape.
Moreover, let us assume $\mu(s,0,1)=(s',+1,-1,0)$, i.e. we change to state $s'$,
we move the input head to the right, the working head to the left,
and modify the current content on the working tape from $1$ to $0$.
We realize the latter by subtracting $2^{X_\Li+h}$, however we allow transitions
to states $(s',i+1,h-1,a,b)$, for each $a,b\in\{0,1\}$, hence we guess
the input tape symbol and the working tape symbol of the
successor configuration.
Our CTL formula $\varphi_\prim$ will guarantee that our guessing was correct by
accessing the control states $q_{\bit,k}$ where
$k\in[I_\Li,I_\Re]\cup[M_\Li,M_\Re]$.

\begin{center}
\begin{picture}(85,73)(-20,4)
\gasset{Nframe=n,loopdiam=9,ELdist=.7}
\gasset{Nadjust=wh,Nadjustdist=1}
\gasset{curvedepth=0}


\put(-20,35){\drawpolygon[Nframe=n,Nfill=y,fillgray=.9](5,20)(28,20)(28,40)(5,40)}

\put(50,35){\drawpolygon[Nframe=n,Nfill=y,fillgray=.9](5,20)(28,20)(28,40)(5,40)}


\put(-12,68){\large$\Soca_{\bit,I_\Li+i}$}
\node(I)(-3,60){$q_{\bit,I_\Li+i}$}
  \put(-12,57){\tiny\bf$\inpu$}


\put(58,68){\large$\Soca_{\bit,X_\Li+h}$}
\node(X)(67,60){$q_{\bit,X_\Li+h}$}
  \put(72,57){\tiny\bf$\work$}



  \node(A)(30,40){$(s,i,h,0,1)$}
  \put(39,40){\tiny\bf$\inz,\worko$}
  \node(B1)(-35,10){$(s,i+1,h-1,0,0)$}
  \put(-40,6){\tiny\bf$\inz,\workz$}

  \put(0,6){\tiny\bf$\inz,\worko$}
  \put(40,6){\tiny\bf$\ino,\workz$}
  \put(80,6){\tiny\bf$\ino,\worko$}

  \node(B2)(5,10){$(s,i+1,h-1,0,1)$}
  \node(B3)(45,10){$(s,i+1,h-1,1,0)$}
  \node(B4)(85,10){$(s,i+1,h-1,1,1)$}
\drawedge[ELside=r](A,B1){$-2^{X_\Li+h}$}
\drawedge[ELside=r,ELpos=70](A,B2){$-2^{X_\Li+h}$}
\drawedge[ELside=r](A,B3){$-2^{X_\Li+h}$}
\drawedge[ELside=l](A,B4){$-2^{X_\Li+h}$}

\drawedge[ELside=l](A,I){$0$}
\drawedge[ELside=l](A,X){$0$}

\end{picture}
\end{center}
The other cases can be dealt with analogously.
Our final formula is
$$
\varphi_\prim\quad=\quad\E
\left(
\bigwedge_{k\in\{0,1\}}
\left(\text{input}_k\rightarrow\EX(\inpu\wedge\varphi_{\bit,k})\right)
\wedge
\left(\text{work}_k\rightarrow\EX(\work\wedge\varphi_{\bit,k})\right)
\right)\U F.
$$
We put $q_\prim=(s_0,0,0,\l,\l)$. Then
$(T(\Soca_\prim),q_\prim(v))\models\varphi_\prim$
iff
$I\$M\in L(\M)$ iff
$M=p_I$.
\qed
\end{proof}

\newcommand{\BIT}{\text{BIT}}

\noindent
Let us give an answer to question (iii).

\begin{lemma}{\label{L BIT}}
For each bit $b\in\{0,1\}$ there exists a fixed CTL formula
$\varphi_{\BIT,b}$ s.t.\ from our input $x_0$ we can compute in
logarithmic space some SOCA $\Soca_{\BIT}$ and some control state
$q_\BIT$ s.t. $(T(\Soca_\BIT),q_\BIT(v))\models\varphi_{\BIT,b}$ iff
$\bit_J(B)=b$.
\end{lemma}
\renewcommand{\C}{\mathcal{C}}
\renewcommand{\R}{\mathcal{R}}
\begin{proof}[Sketch]
By Lemma \ref{L prim}, there is a fixed CTL
formula $\varphi_\prim$ s.t.\ we can compute from $x_0$ a
SOCA $\Soca_\prim$ that allows to check if $M=p_I$.
Analogously, by Lemma \ref{L mod} there is a fixed CTL
formula $\varphi_\mod$ and a SOCA $\Soca_\mod$ that allows
to check whether $C=B\ \mod M$.
In other words, we can deal with the Chinese remainder representation
of $B$, but we aim at accessing the $J^{\text{th}}$ bit of $B$.
So let us assume that $\R=\CRR_{K}(B)=(b_{i,c})_{1\leq i\leq K,0\leq c<p_i}$
is the Chinese remainder representation of $B$.
%The idea is that the bit strings $I$ and $C$ serve as pointers for accessing
%bits of $\R$.
%To prove the lemma, in a nutshell, we translate our input from Chinese remainder
%representation to binary representation.
By Corollary \ref{C CRRBIN}, given $\R$, our
bit string $J$, and the bit $b$,
we can decide if $\bit_J(B)=b$ in logarithmic space.
So let $\M=(S,\Sigma,s_0,F,\mu)$ be some $k\cdot\log(m)$ space-bounded
Turing machine machine for this, where $k\geq 1$ is some constant.
Hence, in order to decide
$\bit_J(B)=b$ we need to simulate $\M$ on input $\langle \R, J, b\rangle$.
For this we store the space that $\M$ requires on our reserved
sequence of bits $Y$, hence $Y$ consists of
$l=k\cdot\log(|\R Jb|)=O(\beta^2+\alpha+1)=O(N^2)$ many bits.
The crucial difficulty is that we do not have $\R$ stored anywhere on our
counter, since $\R$ consists of has exponentially many bits in $n$.
The solution to this problem is to access bit strings $I$ and $C$
and to treat them as pointers to access bits of $\R$.
The definition of SOCA $\Soca_{\BIT,b}$ works analogously as the construction
of $\Soca_{\prim}$ as in the proof of Lemma \ref{L prim}
i.e. we introduce in $\Soca_{\BIT,b}$ control states
that remember the current bit of the input head, the current bit of
the working head, and the position of the working head.
The only difference is that a pointer to the input head cannot completely stored
in the control states.
As mentioned above, for this we employ the bit blocks $I$ and $C$,
and also the sequence $M$ for storing the $I^{\text{th}}$ prime number.
In order to obtain bit $b_{I,C}$, we allow transitions to our SOCA $\Soca_\mod$
(which we can compute in logarithmic
space from $x_0$ by Lemma \ref{L mod}) and checking
(i) via the fixed formula $\varphi_\prim$ whether $M=p_I$, and
then (ii) via the fixed formula $\varphi_{\mod,1}$ (resp. $\varphi_{\mod,0}$)
 whether $B\ \mod M=C$ (resp. $B\ \mod M\not=C$).
Pointers to the remaining parts of the input for $\M$, namely $J$ and $b$,
can directly be handled by the control states of $\Soca_{\BIT,b}$,
in analogy to the proof of Lemma \ref{L prim}.
Let us exemplarily explain the behavior of $\Soca_{\BIT}$ when the input head
of $\M$ currently scans bit $b_{I,C}$ of $\R$ and let us assume that getting
to the successor configuration requires moving the input head to the left.
Then we simply decrement $C$ by $1$ which corresponds to subtracting $2^{C_\Li}$
from our counter. If, however, $C$ currently equals $0$, we need to
decrement $I$ by $1$, overwrite $M$ with prime $p_{I-1}$, and finally
overwrite $C$ with $p_{I-1}-1$. Decrementing $I$ by $1$ can simply achieved by
subtracting $2^{I_\Li}$ from our counter. Overwriting $M$ with prime $p_{I-1}$
can be done by repeatedly subtracting $1$ from $M$ (i.e. subtracting $2^{M_\Li}$
from the counter) until $M$ equals $p_{I-1}$;
checking if $M=p_{I-1}$ can be done via the fixed CTL formula $\varphi_{\prim}$
and the SOCA $\Soca_{\prim}$ by Lemma \ref{L prim}.
The other cases work analogously.
\qed
\end{proof}

\noindent
The following lemma answers question (iv).

\begin{lemma}{\label{L U}}
For each $b\in\{0,1\}$ there is some fixed CTL formula $\varphi_{U,b}$ s.t.\
from our input $x_0$
we can compute in logarithmic space some SOCA $\Soca_U$ and some
control state $q_U$ s.t.\ $(T(\Soca_U),q_U)\models\varphi_{U,b}$
iff $b=\chi_U(B[1,n],B[n+1,n+2^N])$.
\end{lemma}


\noindent
The following Lemma answers question (v) and hence the proof of
$\EXPSPACE$-hardness of the data complexity of CTL on SOCA.

\begin{lemma}
There is some fixed CTL formula $\varphi_L$ s.t.\ from our input $x_0$
we can compute in logarithmic space some SOCA $\Soca_L$ and some
control state $q_L$ s.t.\ $(T(\Soca_L),q_L)\models\varphi_L$
iff $x_0\in L$.
\end{lemma}
\begin{proof}
First note that our CTL formula $\varphi_L$ will be evaluated in state $q_L(0)$.
Recall that our bit sequence $B$ had length $n+2^N+1$ where $B$ is factorized
as $B=xB'$
for some $x\in\{0,1\}^n$ and some $B'\in\{0,1\}^{2^N+1}$.
The SOCA $\Soca_L$ and the CTL formula $\varphi_L$ will mimic the execution
of the program from Fig.\ \ref{F U}.
In the bit string $x$ we want to store the value $x_0$.
The bit string $B'$ is meant to represent
the variable $d$ of the program, hence we will initialize
$B'$ with $0$.
Note that incrementing $B'$ by $1$ corresponds to adding $2^{B_\Li+n}$ to the counter.
Thus, checking when $d$ becomes $2^{2^N}$ for the first time boils down to checking
when the $n+2^N+1^{\text{st}}$ bit of $B$ becomes $1$ for the first time.
 By Lemma
\ref{L BIT} the CTL formula $\varphi_{\BIT,1}$ and the SOCA $\Soca_\BIT$ allow to test
if the $J^{\text{th}}$ bit of $B$ equals $1$.
Therefore we store in $J$ the number $n+2^N+1$.
The following claim tells us that we can test whether we have
initialized our counter correctly.
Its proof is simple and therefore omitted.
\newcommand{\init}{\text{init}}

\medskip

\noindent
{\em Claim:} There is a fixed CTL formula $\varphi_\init$ s.t.\ from
$x_0$ we can compute in logarithmic space some SOCA $\Soca_\init$ and
some control state $q_\init$ s.t.\
$(T(\Soca_\init),q_\init(v))\models\varphi_\init$
iff $J=n+2^N+1$ and $B=x_00^{2^N+1}$.\\

\noindent
Recall that $A=(Q,\{0,1\},q_0,\delta,F)$ is the deterministic finite automaton
that needs to be simulated.
Let us, in a first step, define the auxiliary SOCA $\Soca_A$
to be connected with the control state $q_U$ of $\Soca_U$ and with the control
state $q_\BIT$ of $\Soca_\BIT$ along with the additional
control states $S=Q\times\{0,1\}$ and
the following transitions:
$$
(q,b)\xrightarrow{+2^{B_\Li+n}}(\delta(q,b),b'),\qquad
(q,b)\xrightarrow{0}q_\BIT,\qquad\text{and}\qquad
(q,b)\xrightarrow{0}q_U\qquad\text{for each }
b,b'\in\{0,1\}.
$$

\noindent
Moreover, $\Soca_A$ contains the atomic propositions
$\{\bit_0,\bit_1,F\}$, where with $S_{\bit_b}=Q\times\{b\}$ for each $b\in\{0,1\}$ and
$S_F=F\times\{0,1\}$.
Before giving $\varphi_L$, let us depict our SOCA $\Soca_L$ which has
two additional labels $\rho$ and $\init$:

\begin{center}
\begin{picture}(85,53)(-40,-12)
%   \put(-20,-28){\framebox(85,33){}}
  %\gasset{Nadjust=w,Nadjustdist=2,Nh=6,Nmr=1}
 % \gasset{Nadjust=w,Nadjustdist=2,Nh=6}%$,Nmr=1}
 % \node(A)(0,0){$q_0$}


\gasset{Nframe=n,loopdiam=9,ELdist=.7}
\gasset{Nadjust=wh,Nadjustdist=1}
\gasset{curvedepth=0}

\put(-13,0){
\drawpolygon[Nframe=n,Nfill=y,fillgray=.9](35,20)(58,20)(58,40)(35,40)
\put(48.5,28){\Large$\Soca_{\init}$}
}

\put(-51,-35){
\drawpolygon[Nframe=n,Nfill=y,fillgray=.9](35,20)(68,20)(68,40)(35,40)
\put(48.5,25){\Large$\Soca_A$}
}



\put(0,23){\small $\prop$}
\put(25,33){\small $\init$}
  \node(A)(-30,20){$q_L$}
  \node(B)(0,20){$\bullet$}
  \node(C)(30,30){$q_\init$}
  \node(D1)(-10,0){$(q_0,0)$}
  \node(D2)(10,0){$(q_0,1)$}
\drawloop(A){$+1$}
\drawedge(A,B){$0$}
\drawedge(B,C){$0$}
\drawedge[ELside=r](B,D1){$0$}
\drawedge(B,D2){$0$}
\end{picture}
\end{center}
Let us define the auxiliary formula $\chi=\EX\varphi_\BIT$.
Our final formula is
$$
\varphi_L=\EF\left(\rho\wedge \EX(\init\wedge\varphi_\init)\wedge
\E\left(\neg\chi\wedge\bigwedge_{b\in\{0,1\}}\bit_b\rightarrow
\EX\varphi_{U,b}\right)
 \U (\chi\wedge F)
\right)
$$
We have $x_0\in L$ iff
$(T(\Soca_L),q_L(0))\models\varphi_L$.
\qed
\end{proof}



\begin{theorem}
Data complexity of CTL on SOCA is $\EXPSPACE$-hard.
\end{theorem}


\iffalse
\section{$\EF$ model checking is in $\PSPACE$}


For a finite and non-empty set of integers $S\subseteq \Z\setminus\{0\}$ not
containing $0$, we denote by {\em $\gcd(S)$} the greatest natural number that
divides all numbers $\{|a|: a\in S\}$.
A {\em linear equation} over $\Z$ is given by
$$
x\quad =\quad a_1\cdot x_1 + \cdots\ + a_k\cdot x_n,
$$
where $a_i\in\Z$ is a {\em coefficient} and $x_i$ is a {\em variable}
ranging over $\N$ for each $i\in[n]$.
We call $b\in\Z$ a {\em solution} to the above linear equation if there exists a
mapping $\sigma:\{x_1,\ldots,x_n\}\rightarrow\N$ s.t.\
$b=\sum_{i\in[n]} a_i\cdot\sigma(x_i)$.
An {\em arithmetic progression} with {\em offset} $a\in\Z$ and {\em period
$b\in\Z$} is the set $a+b\N=\{a+b\cdot\N\mid n\in\N\}$.
The following lemma has been used in \cite{AnthonyIPL,Chrobak,Shallit}.

\begin{lemma}[\cite{lookup citation}]\label{L NT}
Let $N\in\N$ and let
\begin{eqnarray}
x\quad=\quad a_1\cdot x_1 + \cdots\ + a_n\cdot x_n,\label{L Eq}
\end{eqnarray}
be a linear equation s.t.\ $1<a_1<\cdots a_n\leq N$ and where
$b=\gcd(a_1,\ldots,a_n)$.
Then, the set of solutions to (\ref{L Eq}) is some set
$U\cup\ a+b\N$,
where $U\subseteq[1,N^2]$,
and
$a=\min\{y>N^2\mid y\equiv 0\textsl{ mod }g\}$.
\end{lemma}


\begin{lemma}
Let $N\in\N$ be some natural and let
\begin{eqnarray}
x\quad=\quad a_1\cdot x_1 + \cdots\ + a_n\cdot x_n,\label{L Eq neg}
\end{eqnarray}
be a linear equation
where $-N\leq a_1<a_2<\cdots <a_n\leq N$ and where $g=\gcd(a_1,\ldots,a_n)$.
Let $b=\gcd(a_1,\ldots,a_n)$ and let
$a=\min\{y>N^2\mid y\equiv 0\textsl{ mod }b\}$.
Then, the set of positive solutions to (\ref{L Eq neg}) is a set
$U\cup\  a+ b\N$, where $U\subseteq[1,N^2]$ and the
set of negative solutions to (\ref{L Eq neg}) is a set
$V\cup -a-b\N$, where $V\subseteq[-N^2,-1]$.
\end{lemma}
\begin{proof}
We only prove the lemma for the negative solutions.
Let $c<0$ be an arbitrary negative solution. Thus,
$c=d+a_1'\cdot m_1+\ldots +a_h'\cdot m_h$, where
$d\in[-N,0]$ and
$\{a_1',\ldots,a_h'\}\subseteq\{a_1,\ldots,a_n\}\cap\Z^-$ is a subset of
the negative coefficients.
Hence $c$ is the solution
to some linear equation $x=b_1\cdot x_1+\ldots +b_l\cdot x_l$, where
$\{b_1,\ldots,b_l\}\subseteq[-N,-1]$. By application of Lemma
\ref{L Eq} and switching from positive integers to negative integers, the lemma
follows.
\qed
\end{proof}


\begin{lemma}
  \label{lem:bounded-coefficients}
  Let $N,m$ be natural numbers and let $a_1,\ldots, a_n$ be integers
  such that $-N\le a_1 \le \ldots \le a_n\le N$, $m>N^3$ and
  \begin{equation*}
    m = a_1 \cdot x_1 + \cdots + a_n \cdot x_n
  \end{equation*}
  for some natural numbers $x_1, \ldots, x_n$. Then there exist
  natural numbers $y_1, \ldots, y_n$ such that
  \begin{equation*}
    m = a_1 \cdot y_1 + \cdots + a_n \cdot y_n
  \end{equation*}
  and $y_i\le N^2$ for $1\le i<n$.
\end{lemma}
\begin{proof}
  Let $a_\ell\in\{a_1,a_n\}$ be the $a_i$ with the largest
  \emph{absolute} value. Define $x_i' =x_i \text{ mod } a_\ell$ for
  $1\le i\le n, i\neq \ell$, and let $m'= \sum_{1\le i\le n, i\neq
    \ell} a_i\cdot m_i$. Clearly, we have $m'<N^3$ and $m'\equiv m
  \text{ mod } a_\ell$. Let $h=m-m'=a_\ell \cdot x_\ell'$ for some
  integer $x_\ell'$. In case $a_\ell=a_n$, the lemma follows by
  setting $y_i=x_i'$ for $1\le i\le n$. Otherwise, let $x_n^*= \lceil
  \frac{h}{|a_1a_n|} \rceil$. The lemma follows by setting $y_i=x_i'$
  for $1<i<n$, $y_1=\frac{x_n^*|a_1a_n|-h}{|a_1|}$ and $y_n=x_n'+
  x_n^*|a_1|$.
\end{proof}
\begin{remark}
  \label{rem:maximum-valley}
  It follows from the previous lemma that $\sum_{a_i<0} a_i\cdot
  y_i>-N^3$.
\end{remark}

\section{Periodicity analysis}

\newcommand{\cycles}{\textrm{cycles}}

Let us fix a SOCA $\O=(Q,\{Q_p\mid p\in\Prop\},\delta_0,\delta_{>0})$ for
 the rest of this section.

In the following, we will treat $\O$ as a weighted directed graph $G(\O)$, where
the weights are taken according to $\delta_{>0}$.
More precisely, we define $G(\O)=(Q,\delta_{>0})$
and write $q_1\stackrel{a}{\mapsto}q_2$ as an abbreviation for
$(q_1,a,q_2)\in\delta_{0}$ and simply $q_1\mapsto q_2$ if
$(q_1,a,q_2)\in\delta_{>0}$ for some $a\in\Z$.
Note that every non-zero path in the transition system $T(\O)$ corresponds to a unique path in
$G(\O)$.
A {\em strongly connected component} of $G(\O)$ is an
equivalence class of the equivalence relation $\mapsto^*$.
For every finite path $\pi$ in $G(\O)$, let $w(\pi)$ denote the sum of all
weights of edges of $\pi$.
For every strongly connected component $C$ in $G(\O)$, let $\cycles(C)$ denote
the set of chord-free cycles whose total weight does not summarize to $0$,
more precisely
the set of paths $\pi=q_0\stackrel{m_1}{\mapsto} q_2\cdots
\stackrel{m_l}{\mapsto} q_l$, where $l\geq1$,
$q_0=q_l$, such that $i\not= j$ implies $q_i\not=q_j$ for all
$i,j\in[l]$ and moreover $\sum_{i\in[l]}m_i\not=0$.


Let $M=\max\{|a|:\exists q_1,q_2\in Q. q_1\stackrel{a}{\mapsto}q_2\}$
denote the maximal absolute value of a weight that appears in $G(\O)$.
Moreover, let $g_C=\gcd(w(cycles(C)))$ denote the greatest common divisor
of all non-zero weights of chord-free cycles in $C$.
In the following define $$\Delta\ =\ \LCM\{g_C\mid C\text{ is a strongly
connected component of } G(\O)\}.$$
\noindent
Note that $\Delta\in\exp(|\O|)$ since firstly $g_C\leq (k-1)\cdot M$ for each strongly
connected component $C$ and secondly there are at most $|Q|$
strongly connected components in $G(\O)$.
Let $k=|Q|$.


\begin{lemma}{\label{L Cycles}}
Let $C$ be some strongly connected component of $G(\O)$, let $q\in Q$ be some
control location, and let $n\geq k^2\cdot M$ be some natural.
Then in $T(\O)$ there exists some non-zero path from $(q,n)$ of height $h$ with $h\equiv
0\text{ mod }g_C$ and $|h|\leq k\cdot(k-1)\cdot M$  that visits each
control location in $C$ at least once.
\end{lemma}
\begin{proof}
Since $C$ is a strongly connected component and $n\geq k^2\cdot M$, there exists some
non-zero path
$$
\pi\quad =\quad(q_1,n_1)\rightarrow(q_2,n_2)\ \cdots\ \rightarrow (q_l,n_l)
$$
in $T(\O)$ with $l\in[k(k-1)]$ that visits each control state in $C$ at least once
(i.e. for all $p\in C$ there is some $i\in[l]$ such that $p=q_i$) such that
moreover $q_1=q_l=q$ and $n_1=n$. Thus, it follows $|n_1-n_l|\leq k\cdot
(k-1)\cdot M$.
Observe that for every subpath $\pi[a,b]=(q_a,n_a)\cdots\rightarrow(q_b,n_b)$ of $\pi$ that is a chord-free
cycle, we have $n_a\equiv n_b\text{ mod } g_C$ by definition of $g_C$.
By successively cutting out subpaths of $\pi$ that correspond to chord-free
cycles, it is apparent that $n_1\equiv n_l\text{ mod }g_C$.
\qed
\end{proof}

Observe that Lemma \ref{L Cycles} implies that there is a non-zero path in
$T(\O)$ that visits every
control location in $C$ at least once from $(q,k^2\cdot M+\delta)$ with the
desired properties with respect to the height.

\begin{lemma}
  Let $q\in Q$ be a state of some SCC $C$ and let $m,m'$ be natural
  numbers such that $m-m'>k^2\cdot M+ k^3\cdot M^3$ and $m'>k\cdot
  M$. Then $(q,m)\rightarrow^*(q,m')$ iff
  $(q,m+\Delta)\rightarrow^*(q,m')$.
\end{lemma}
\begin{proof}
  $(\Rightarrow)$ Let $q_1,\ldots,q_n\in C, 1\le n\le k$ be states on
  loop-free cycles $\pi_{a_i}:q_i\rightarrow \ldots \rightarrow q_i$
  with weights $a_i=w(\pi_{a_i})$ such that
  $\text{gcd}(a_1,\ldots,a_n)=g_C$.  Let $\pi_i^1: q\rightarrow^* q_i$
  and $\pi_i^2: q_i\rightarrow^* q$ be disjoint loop-free paths from
  $q$ to $q_i$ respectively $q_i$ to $q$. Denote by
  $\pi_i=\pi_i^1\cdot \pi_i^2$ a loop-free cycle from $q$ to itself
  passing through $q_i, 1\le i\le n$.

  Let $h=m-m'-\sum_{1\le i\le n}w(\pi_i)+\Delta$. Similarly to the
  previous lemma, we find $g_C|m-m'$, $g_C|\sum_{1\le i\le n}w(\pi_i)$
  and hence $g_C|h$. Moreover, $h>k^3\cdot M^3$ and thus we can derive
  from Lemma \ref{lem:bounded-coefficients} that there are
  $y_1,\ldots, y_n$ such that $\sum_{1\le i\le n}a_i\cdot y_i=h$ and
  $y_i<k\cdot M$, $1\le i<n$.

  Now define $\pi_i'=\pi_i^1\cdot (\pi_{a_i})^{y_i} \cdot \pi_i^2$ and
  $\pi=\pi_1'\cdot\pi_2'\cdot\cdots\cdot\pi_n'$. Cleary,
  $w(\pi)=m+\Delta-m'$. Moreover, $m'> k\cdot M$ ensures that is a
  valid path. Hence $(q,m+\Delta)\rightarrow^*(q,m')$ is witnessed
  by $\pi$.

  $(\Leftarrow)$ The proof from the previous direction can directly
  be adopted to show the statement.
\end{proof}

The following lemma is important for our periodicity analysis later.

\begin{lemma}
Let $q,q'\in Q$ and $n,n'\in\N$ such that
$n-n'>4\cdot k^3\cdot M^2$ and
$n'>k\cdot M$.
If there is a path from $(q,n)$ to $(q',n')$ in $T(\O)$, then there
is a path from $(q,n+\Delta)$ to $(q',n')$ in $T(\O)$.
\end{lemma}
\begin{proof}
Let us fix some path
$$
\pi\quad=\quad(q_1,n_1)\rightarrow(q_2,n_2)\quad \cdots\quad \rightarrow (q_l,n_l),
$$
where $(q_1,n_1)=(q,n)$ and $(q_l,n_l)=(q',n')$.
Since $n_1-n_l>4\cdot k^3\cdot M^2$ and since there are at most $k$ strongly connected components in $G(\O)$,  there
exists a strongly connected component $C$ of $G(\O)$ and some subpath $\pi[s,t]$
of $\pi$ such that the following holds
\begin{enumerate}
\item  $n_s-n_t>4\cdot k^2\cdot M^2$,
\item $n_t\geq n'$, and
\item $q_i\in C$ for each $i\in[s,t]$
\end{enumerate}
by the pigeonhole principle.
Let $\delta=n_t-n_s<-4\cdot k^2\cdot M^2$ denote the height of the subpath $\pi[s,t]$.
Let us treat the path $\pi[s,t]$ as a path in $G(\O)$. Now think of consecutively cutting out
chord-free cycles in $\pi[s,t]$  until this is no longer possible, in a greedy
manner from beginning to the end, say. Let us denote the resulting
path (possibly empty) by $\rho$ and let  $h\in[-(k-1)\cdot M,(k-1)\cdot M]$ denote
its weight.
\begin{remark}{\label{R rho}}
The path $\rho$ can be seen as a non-zero path in $T(\O)$ of length at most
$k-1$ from $(q_s,n_s)$ to $(q_t,n_s+h)$.
\end{remark}

\noindent
The following claim states that the height $\delta$ of $\pi[s,t]$ lies in some
arithmetic progression of small offset and small period.


\smallskip

\noindent
{\em Claim A.} We have $\delta\in (c+h)-g_C\N$,
for some
$c\in[-3\cdot k^2\cdot M^2-M,-3\cdot k^2\cdot M^2]$ with $c\equiv 0\text{ mod }g_C$.

\medskip

\noindent
Let us postpone the proof of Claim A and finish the proof of the lemma.
It remains to show that there is some non-zero path
$(q,n+\Delta)\rightarrow^*(q',n')$ in $T(\O)$.
Before doing this, we claim that the following paths exist in $T(\O)$:
\begin{enumerate}[(1)]
\item $(q,n+\Delta)\rightarrow^*(q_s,n_s+\Delta)$.
\item $(q_t,n_t)\rightarrow^*(q',n')$.
\item $(q_s,n_t-h)\rightarrow^*(q_t,n_t)$.
\end{enumerate}
Point (1) is clear since $\pi[1,s]+\Delta$ witnesses this.
Point (2) holds by factorization of $\pi$.
For Point (3), observe that $\rho$ is a non-zero path from $(q_s,n_s)$ to $(q_t,n_s+h)$
of height $h$ and of length at most $k-1$
by Remark \ref{R rho}.
Since $n_t\geq n'>(k-1)\cdot M\geq h$, there is a non-zero path
from $(q_t,n_t-h)$ to $(q_t,n_t)$, namely
$\rho+(-n_s+n_t-h)=\rho+(\delta-h)$.
 By combining Points (3) and (2) we obtain some non-zero path $(q_t,n-h)\rightarrow^*
(q',n')$.
Due to Point (1), it suffices to show that there is some
non-zero path from $(q_s,n_s+\Delta)$ to $(q_s,n_t-h)$.
In other words, we have to prove that in $G(\O)$ there is some path from
$q_s$ to $q_s$ of weight
\begin{eqnarray*}
w\quad=\quad n_t-h-(n_s+\Delta)\quad=\quad n_t-n_s-h-\Delta\quad=\quad\delta-h-\Delta
\end{eqnarray*}
such that moreover the weight of any prefix of this path is strictly bigger
than $-(n_s+\Delta)$.
By Claim A we have $w=\delta-h-\Delta=(c+h)-r\cdot g_C-h-\Delta=c-r\cdot
g_C-\Delta$ for
some $r\in \N$ and some $c\in[-3\cdot k^2\cdot M^2-M,-3\cdot k^2\cdot M^2]$ with
$c\equiv 0\text{ mod } g_C$.
%Since too $\Delta\equiv 0\text{ mod } g_c$, there exists some  $d\in\N$ such
%that
Since moreover $\Delta\equiv 0\text{ mod }g_C$, we have $w\equiv 0\text{
mod }g_C$.
Furthermore, we have $w<-3\cdot k^2\cdot M^2$.
Since $n_s+\Delta>BLA>k^2\cdot M$, there is some non-zero path $\sigma$ of height
$h_\sigma$ with $|h_\sigma|\leq k\cdot(k-1)\cdot M$ and $h_\sigma\equiv 0\text{
mod } g_C$
from $(q_s,n_s+\Delta)$ that visits each control location in $C$ at
least one by Lemma \ref{L Cycles} .
When taking the path $\sigma\circ\sigma$, the remaining weight is congruent $0$
modulo $g_C$ and can be estimated
as $$w-2\cdot h_\sigma\ <\ -3\cdot k^2\cdot M^2-2\cdot h_\sigma\ <
-3\cdot k^2\cdot M^2+2\cdot k\cdot (k-1)\cdot M<
-k^2\cdot M^2.
$$
Hence by Lemma \ref{L Eq neg} there exist $m_1,\ldots, m_n\in\N$ such that the
remaining weight equals $\sum_{i=1}^n a_i\cdot m_i$.
To finally obtain a path from $q_s$ to $q_s$ of weight $w$ in $G(\O)$, we
modify the first traversal of $\sigma$ by choosing for each positive $a_i$ some
chord-free cycle of weight $a_i$ through which we loop precisely $m_i$ many times, followed by
the modification of the second traversal of $\sigma$, where we choose for each
negative $a_i$ some chord-free cycle of weight $a_i$ through which we loop
precisely $m_i$ many times.
This concludes the proof of the lemma.

\smallskip

\noindent
{\em Proof of Claim A.}\\
\noindent
Observe that $\delta=h+r$, where $r$ is some solution
of the linear equation
$$
x\ =\ a_1\cdot x_1+\ \ldots\ +a_m\cdot x_m,
$$
where $\{a_1,\ldots, a_m\}\subseteq w(cycles(C))$ are weights of the chord-free cycles that
were cut out from $\pi[s,t]$ in order to obtain $\rho$. %\subseteq[-k\cdot M,k\cdot M]$.
Let $d=\gcd(a_1,\ldots,a_m)$ which is a multiple of $g_C$ by definition.
Since we know that $\delta<-4\cdot k^2\cdot M^2$ and $h\in[-(k-1)\cdot M,(k-1)\cdot M]$, we have
$$
r=\delta-h\quad<\quad-4\cdot k^2\cdot M^2+(k-1)\cdot M\quad\leq\quad -3\cdot k^2\cdot M^2.
$$
Moreover, we have obviously $-\sqrt{3}\cdot k\cdot M<-k\cdot M\leq a_i\leq
k\cdot M<\sqrt{3}\cdot k\cdot M$ for each $i\in[m]$.
Now let us define $c=\max\{y<-3\cdot k^2\cdot M^2\mid y\equiv 0\text{ mod }
d\}$.
As $d\in[M]$, it follows $c\in[-3\cdot k^2\cdot M^2-M,-3\cdot k^2\cdot M^2]$ as
required.
By applying Lemma \ref{L Eq neg} we have that $r\in c-d\N$ and hence
$\delta\in (c+h)-d\N$.
Since $d$ is a multiple of $g_C$
it follows $\delta\in (c+h)-g_C\N$.
\qed
\end{proof}

\begin{theorem}
Model checking EF over SOCA is $\PSPACE$-complete.
\end{theorem}

\fi


\fi
\newcommand{\Two}{\mathbb{A}}
\renewcommand{\add}{\mathsf{add}}



\subsection{Hardness of the data complexity of $\CTL$ on POCA}{\label{S CTL und}}
We now show that there exists a fixed $\CTL$ formula for which model
checking of $\poca$ is $\Pi^0_1$-hard by a reduction from the emptiness
problem for \emph{two-counter automata}, which is
$\Pi^0_1$-complete~\cite{Min61}. Similar to a SOCA, a two-counter
automaton $\A$ consists of a finite set of control states and transitions
between them. However, each transition of $\Two$ acts on two counters,
which it can in- and decrement and test for $\zero$.

%% is a tuple $\Two=(Q, E, \lambda)$ with a finite set of
%% contro states $Q$, an edge relation $E\subseteq Q\times Q$, however
%% $\lambda: E \rightarrow \{\zero_1,\zero_2\}\cup \{\add_j(a)\mid j\in
%% \{1,2\}, a\in \{-1,1\} \}$.

%% The {\em directed graph induced by $\Two$}
%% is defined to be $G(\A)=(Q\times \N\times \N, \rightarrow)$, where
%% $(p,i_1,i_2)\rightarrow (q,i_1+d_1,i_2+d_2)$ iff $(p,q)\in E$,
%% $\lambda(p,q)=\add_j(a)$ implies $d_j=a$ and $d_{3-j}=0$, and
%% $\lambda(p,q)=\zero_j$ implies $d_j=i_j=0$ and $d_{3-j}=0$.  The {\em
%%   reachability problem on two-counter automata} asks, given a
%% two-counter automaton $\Two=(Q,E,\lambda)$, and control states
%% $q_0,q_1\in Q$, whether there are $m,n\in\N$ such that
%% $(q_0,0,0)\rightarrow^*(q_1,m,n)$ in $G(\Two)$?

%\noindent

The idea of our reduction is as follows: Given a two-counter automaton
$\Two$, we construct a $\poca$ $\Poca(x)$ with one parameter in such a
way that the two counters from $\A$ are encoded into the single
counter from $\Poca(x)$. Given a counter value $n$ of $\Poca(x)$,
$n\ \mod x$ encodes the value of the first, and $n\ \div\ x$ encodes
the value of the second counter of $\Two$.  Hence, testing whether the
first equals $0$ corresponds to checking whether $n\equiv 0\ \mod
x$, while testing whether the second counter equals $0$ corresponds to
checking whether $n\geq x$.  Incrementing (resp.\ decrementing) the
first counter of $\A$ can be mimicked by adding (resp.\ subtracting) $1$,
whereas on counter two this corresponds to adding (resp. subtracting)
$x$. Of course, we need $\ctl$ formulas to ensure that we do not
overflow when simulating an increment of the first counter of $\A$.
For instance, if $n\equiv -1\ \mod x$ and we want to
simulate an increment of the first counter of $\Two$ in that way, we
would actually set the first counter to $0$ and simultaneously
increment the second counter. However, if $\Two$ is not empty, then
$x$ can be instantiated with a large enough value such that such an
overflow does not occur. Conversely, if $\Two$ is empty then there is
no such instantiation.  \iffalse

\newcommand{\test}{\mathsf{test}}
Before we give the reduction from the reachability problem in two-counter
automata, let us introduce some gadgets that we need in our reduction.
\begin{lemma}{\label{L POCAs}}
  There exists some fixed CTL formula $\varphi_\test$
for which the following holds:
\begin{enumerate}[(1)]
\item
There exists some fixed $\poca$ $\Poca_{\not\equiv 0}(x)$ with some control state
$q_{\not\equiv 0}$ such that
 $$(\Poca_{\not\equiv 0}^\sigma,q_{\not\equiv0}(n)) \models
 \varphi_{\test} \quad\Leftrightarrow\quad n\not
  \equiv 0 \textsl{ mod } \sigma(x).$$
\item There exists some fixed $\poca$ $\Poca_{\not\equiv-1}(x)$ with some control
state $q_{\not\equiv -1}$ such that
 $$(\Poca_{\not\equiv -1}^\sigma,q_{\not\equiv-1}(n)) \models
 \varphi_\test \quad\Leftrightarrow\quad n\not
  \equiv -1 \textsl{ mod } \sigma(x).$$
\item There exists some fixed $\poca$ $\Poca_{\geq x}(x)$ with some control
state $q_{\geq x}$ such that
 $$(\Poca_{\geq x}^\sigma,q_{\geq x}(n)) \models
 \varphi_\test \quad\Leftrightarrow\quad n\geq \sigma(x).$$

\end{enumerate}
\end{lemma}
\begin{proof}
We choose $\varphi_\test=\neg\EF\prop$, where $\prop$ is an atomic
proposition.
The following figure depicts $\Poca_{\not\equiv0}(x)$,
$\Poca_{\not\equiv-1}(x)$, and $\Poca_{\geq x}(x)$, resp.
\begin{center}
\begin{picture}(85,30)(-20,-12)
%   \put(-20,-28){\framebox(85,33){}}
  %\gasset{Nadjust=w,Nadjustdist=2,Nh=6,Nmr=1}
 % \gasset{Nadjust=w,Nadjustdist=2,Nh=6}%$,Nmr=1}
 % \node(A)(0,0){$q_0$}

\gasset{Nframe=n,loopdiam=9,ELdist=.7}
\gasset{Nadjust=wh,Nadjustdist=1}
\gasset{curvedepth=0}

\put(-60,0){
\put(41,-2){\small $\prop$}
  \node(A)(20,0){$q_{\not\equiv0}$}
  \node(B)(40,0){$\bullet$}
\drawloop(A){$-x$}
\drawedge(A,B){$\zero$}
}
%%% AUTOMATON 2

\put(-15,0){
\put(61,-2){\small $\prop$}
  \node(A)(20,0){$q_{\not\equiv-1}$}
  \node(B)(40,0){$\bullet$}
  \node(C)(60,0){$\bullet$}
\drawloop(B){$-x$}
\drawedge(A,B){$+1$}
\drawedge(B,C){$\zero$}
}

%%% AUTOMATON 3


\put(50,7){
\put(41,-2){\small $\prop$}
  \node(A)(20,0){$q_{\geq x}$}
  \node(B)(20,-20){$\bullet$}
  \node(C)(40,-20){$\bullet$}
  \node(D)(40,0){$\bullet$}
\drawloop(A){$+1$}
\drawedge(A,B){$+1$}
\drawedge(B,C){$-x$}
\drawedge[ELside=r](C,D){$\zero$}
}

\end{picture}
\end{center}
Correctness of the lemma is immediate.
\qed
\end{proof}

\newcommand{\up}{\mathsf{up}}


\noindent
Let us fix some instance of the reachability problem for
 two-counter automata:
Let $\langle\Two,q_0,q_1\rangle$ be an input to the
reachability problem, where
$\Two=(Q,E,\lambda)$ is some two-counter automaton.
We will give a fixed CTL formula $\varphi$ such that from
$\langle\Two,q_0,q_1\rangle$ we can construct some $\poca$ $\Poca(x)$ such that
$(q_0,0,0)\rightarrow^*(q_1,m,n)$ for some $m,n\in\N$ iff
$(\Poca^\sigma,q_0(0))\models\varphi$ for some $\sigma:\{x\}\rightarrow\N$.


The control states and transitions will of $\Poca(x)$ can be described by the
following graph transformation rule that maps each pair of control states
$(p,q)\in E$ of $\Two$ with $\lambda(p,q)=l$
 into corresponding control states in $\Poca(x)$, by
possibly accessing the different $\poca$ of Lemma \ref{L POCAs}.
\begin{center}
\begin{picture}(85,30)(30,-10)
%   \put(-20,-28){\framebox(85,33){}}
  %\gasset{Nadjust=w,Nadjustdist=2,Nh=6,Nmr=1}
 % \gasset{Nadjust=w,Nadjustdist=2,Nh=6}%$,Nmr=1}
 % \node(A)(0,0){$q_0$}

\gasset{Nframe=n,loopdiam=9,ELdist=.7}
\gasset{Nadjust=wh,Nadjustdist=1}
\gasset{curvedepth=0}

\put(60,6){\Huge$\hookrightarrow$}


%\put(41,-2){\small $\prop$}
  \node(A)(20,10){$p$}
  \node(B)(40,10){$q$}

\drawedge(A,B){$l$}


\put(109,12){\small$l$}


  \node(C)(90,10){$p$}
  \node(D)(110,10){$\bullet$}
  \node(E)(130,10){$q$}
  \node(F)(110,-10){$\test(l)$}


%\drawloop(A){$-x$}
\drawedge(C,D){$0$}
\drawedge(D,E){$\up(l)$}
\drawedge(D,F){$0$}
\end{picture}
\end{center}

\noindent
where
$$
\test(l)\quad =\quad \begin{cases} (\Poca_{\not\equiv 0}(x),q_{\not\equiv 0}) & \text{ if }\quad
l\in\{ \add_1(-1),\zero_1\}\\
(\Poca_{\not\equiv -1}(x),q_{\not\equiv -1})& \text{ if }\quad l=\add_1(+1)\\
(\Poca_{\geq x}(x),q_{\geq x})& \text{ if }\quad l=\zero_2,
\end{cases}
$$

\noindent
and where
$$
\up(l)\quad =\quad \begin{cases}  a & \text{ if }\quad
l=\add_1(a)\\
\circ x& \text{ if }\quad l=\add_2(\circ 1)
\text{ for some }\circ\in\{+,-\}\\
0& \text{ otherwise}
\end{cases}
$$

\noindent
Let us moreover ensure that every control state of the kind
$q_{\tau}$ with $\tau\in\{\not\equiv 0,\not\equiv -1,\geq x\}$ has
the label $\rho_\tau$.
Before we give our final formula, let us introduce the following constraint
formulas that guarantee that an overflow never occurs:

\begin{eqnarray*}
\psi_1&\quad=\quad& \add_1(+1)\longrightarrow \EX(\rho_{\not\equiv-1}
\wedge\varphi_\test)\
\\
\psi_2&\quad=\quad&
\add_1(-1)\longrightarrow\EX(\rho_{\not\equiv 0}\wedge\varphi_\test)\\
\psi_3&\quad=\quad&
\zero_1\longrightarrow\EX(\rho_{\not\equiv 0}\wedge\neg\varphi_\test)\\
\psi_4&\quad=\quad&
\zero_2\longrightarrow\EX(\rho_{\geq x}\wedge\neg\varphi_{\test})\\
\end{eqnarray*}

\noindent
Let us introduce a label $f$ at control state $q_1$.
Our final CTL formula $\varphi$ is
$$
\varphi\quad=\quad\E\left(\bigwedge_{i\in[4]}\psi_i\ \U\  f\right).
$$
\iffalse
The correctness of the reduction follows from the following equivalences:
\begin{itemize}
\item $(T(\Poca),(q_0,0))\models \neg \varphi$
\item for all $\sigma:\{x\}\rightarrow \N$ we have $(\Poca^\sigma(x),q_0(0))\not\models\varphi$
\item for some $\sigma:\{x\}\rightarrow\N$ we have $(\Poca^\sigma(x),q_0(0))\models\varphi$
\item for some $\exists\sigma:\{x\}\rightarrow\N$ there is some path
$r_0(0)\rightarrow r_1(n_1)\cdots r_k(n_k)$ such that
$r_0=q_0$, $r_k=q_1$, $r_k\in F$ and
$(T(\Poca^\sigma(x)),r_i(n_i)



%$(q_0,0)(q_1,n_1\mod\sigma(x),n_1\div\sigma(x))\cdots (q_k,n_k\mod\sigma(x),
%n_k\div \sigma(x))$ in $T(\Poca^\sigma(x))$ such that


\end{itemize}
Then there is an assignment
$\sigma$ such that $(T(\Poca^\sigma),(q_0,0))\models \varphi$. It is
easily verified that the witnessing path $(q,0)(q_1,n_1)\ldots
(q',n')$ in $T(\A'(\sigma))$ corresponds to the path
$(q,0,0)(q_1,n_1\mod \sigma(x), n_1\div \sigma(x))\ldots (q',n'\mod
\sigma(x), n'\div \sigma(x))$ in $T(\A)$. Conversely, if
$(q,0,0)\rightarrow^* (q,n_1,n_2)$ in $T(\A)$, then the first counter
of $\A$ does not exceed some value $u \in \naturals$. Consequently, by
setting $\sigma(x):=u$, $(T(\A(\sigma)),(q,0))\models \varphi$.
\fi
We have that $(q_0,0,0)\rightarrow^*(q_1,m,n)$ for some $m,n\in\N$ in $G(\A)$
iff
$(T(\Poca^\sigma(x)),q_0(0))\models\varphi$ for some
$\sigma:\{x\}\rightarrow\N$.
We obtain the following theorem.
\fi
\begin{theorem}
  The data and combined complexity of $\CTL$ on POCA is
  $\Pi^0_1$-complete.
  \label{thm:ctl-poca}
\end{theorem}





\section{$\ltl$ Model Checking}
Formulas of $\ltl$ are given by the following grammar, where $\prop$
ranges over $\Prop$:
\begin{equation*}
  \varphi \quad ::= \quad \prop \mid \neg \varphi \mid \varphi
  \wedge \varphi \mid
  \X \varphi \mid \varphi \until \varphi
\end{equation*}
%% The Boolean abbreviations and $\true$ and $\false$ are defined in the
%% same way as in $\CTL$. The \emph{eventually modality} $\eventually
%% \varphi$ is an abbreviation for $\true \until \varphi$ and the
%% \emph{globally modality} $\globally \varphi$ abbreviates $\neg
%% \eventually \neg \varphi$.

The semantics of $\ltl$ is given in terms of infinite paths in a
transition system. Let $T=(S,\{S_\prop \mid \prop \in
\Prop\},\rightarrow)$ be a transition system, $\pi= s_0 \rightarrow
s_1 \rightarrow \cdots$ an infinite path in $T$ and $\varphi$ an
$\ltl$ formula, we define $(T,\pi)\models \varphi$ by induction on the
structure $\varphi$.
\begin{align*}
  (T,\pi) \models  \prop &  \iff  \pi(0) \in S_\prop, \prop \in \Prop &
  (T,\pi) \models  \neg \varphi &\iff \pi \not \models \varphi\\
  (T,\pi) \models \varphi_1 \wedge \varphi_2 &\iff
  \forall i\in \{1,2\}:(T,\pi) \models \varphi_i &
  (T,\pi) \models \X \varphi & \iff (T,\pi^1) \models \varphi
\end{align*}
\vspace*{-1.0cm}
\begin{align*}
  (T,\pi) \models  \varphi_1 \until \varphi_2  \iff
  \exists j\ge 0: (T,\pi^j) \models \varphi_2
  \text{ and } \forall 0\le i<j: (T,\pi^i) \models \varphi_1
\end{align*}
%% Let us now define the $\ltl$ model checking problem on SOCA
%% respectively POCA.\\

\problemx{$\ltl$ model checking on SOCA}
{SOCA $\Soca=(Q,E,\lambda)$, $q\in Q$, $n\in\N$ in binary,
  an $\ltl$ formula $\varphi$.}
{Does $(T(\Soca),\pi)\models\varphi$ for all infinite
paths $\pi$ with $\pi(0)=q(n)$?}
\

\problemx{$\ltl$ model checking on POCA}
{POCA $\Poca(X)=(Q,E,\lambda)$, $q\in Q$, $n\in\N$ in binary, an
$\ltl$ formula $\varphi$.}
{Does $(T(\Poca^\sigma),\pi)\models\varphi$ for all $\sigma:X\rightarrow\N$
and for all infinite paths $\pi$ with $\pi(0)=q(n)$?}




\subsection{Upper Bounds}

A standard approach to $\ltl$ model checking is the automata-based
approach, in which behaviours of \emph{systems} are modelled as
non-deterministic B\"uchi automata (NBA). Given an NBA $\automaton$
modelling a system and an $\ltl$ formula $\varphi$, the idea is to
translate $\varphi$ into an NBA $\automaton_{\neg \varphi}$ of size
$2^{O(|\varphi|)}$ such that the language of $\automaton\times
\automaton_{\neg \varphi}$ is empty iff $\varphi$ holds on all
infinite traces of $\automaton$. The concept of B\"uchi automata can
easily be adopted to the setting of counter automata. Then a
\emph{B\"uchi-SOCA} is not empty if there is an infinite path on which
some designated control states occurs infinitely often. The latter
boils down to just checking for recurrent reachability. Moreover, the
B\"uchi-SOCA obtained from the product of a SOCA and an NBA can be
defined and constructed in a straightforward way. We omit details for brevity.

It was shown in \cite{HKOW10} that checking emptiness is
$\coNP$-complete for both B\"uchi-SOCA and B\"uchi-POCA, and in
\cite{DG-jlc09} that it is $\NL$-complete for B\"uchi-OCA. We use
these results for establishing upper bounds for the $\ltl$ model
checking problems.

For every fixed $\ltl$ formula $\varphi$, and every POCA $\Poca$, the
size of $\Poca\times \automaton_{\neg \varphi}$ is $O(|\Poca|)$, hence
the data complexity of $\ltl$ on SOCA and POCA is in $\coNP$. Hardness
for $\coNP$ follows from $\NP$-hardness of reachability using a fixed
formula $\true \until \rho$ for some $\rho\in \Prop$.

If both $\Poca$ and $\varphi$ are part of the input then $|\Poca\times
\automaton_{\neg \varphi}|=|\Poca|\cdot 2^{O(|\varphi|)}$, and hence
\cite{HKOW10} gives a $\coNEXP$ upper bound for the combined
complexity of $\ltl$ model checking on both SOCA and POCA. This upper
bound can however be improved for SOCA. Given a SOCA $\Soca$, let $m$
be the absolute value of the maximum increment or decrement on the
transitions in $\Soca$.  Let $\Soca'$ be the B\"uchi-SOCA obtained
from the product $\Soca\times\automaton_{\neg \varphi}$ by replacing
every transition labeled with $z$ with a sequence of fresh transitions
and control states of length $z$, where, depending on the sign of
$z$, each transition is labeled with $+1$ resp.\ $-1$. We have
$|\Soca'|= m\cdot|\Soca|\cdot 2^{O(|\varphi|)}$, and hence the $\NL$
upper bound for emptiness of B\"uchi-OCA from \cite{DG-jlc09} yields a
$\PSPACE$ upper bound.

\begin{proposition}
  The data complexity of $\ltl$ model checking on SOCA and POCA is
  $\coNP$-complete, the combined complexity of $\ltl$ model
  checking on SOCA is $\PSPACE$-complete, and the combined complexity
  of $\ltl$ model checking on POCA is in $\coNEXP$.
\end{proposition}


\subsection{Hardness of the Combined Complexity of $\ltl$ on POCA}

We are now going to sketch a proof of $\coNEXP$-hardness of $\ltl$
model checking on POCA via a reduction from the complement of
the $\NEXP$-complete \textsf{Succinct 3-SAT} problem \cite{Papa94}.
An input of \textsf{Succinct 3-SAT} is given by a
Boolean circuit $\mathbb{C}$ that encodes a Boolean formula $\psi$ in
3-CNF, i.e.\ $\psi=\bigwedge_{0\le j< M}(\ell^j_1\vee \ell^j_2\vee
\ell^j_3)$. Let $j\in [M]$ be the index of a clause encoded in
\emph{binary} and $k\in\{1,2,3\}$. Assume that $\psi$ uses
$N$ different variables $y_1,\ldots,y_N$.
 On input $(j\cdot k)$, the output
of $\circuit$ is $(i\cdot b)$, where $i\in [N]$
 is the index of the Boolean variable that appears in
literal $\ell_k^j$, and where $b=0$ when $\ell_k^j$ is negative
and $b=1$ when $\ell_k^j$ is positive.
 \textsf{Succinct 3-SAT} is to decide whether $\psi$ is satisfiable. Fig.\
\ref{fig:high-level} depicts on a high-level the POCA $\Poca(x)$ derived from
$\circuit$ that we are using in our reduction.

As a first step, let us provide a suitable encoding of truth assignments
by natural numbers. The encoding we use has also been employed for establishing
lower bounds for model checking OCA~\cite{JKMZ04-ic}.
%%%%%% HIER KOENNTE MAN ZITIEREN
Recall that $p_i$ denotes the $i^{\text{th}}$ prime
number.
 Every natural number $n$ defines a truth assignment
 $\nu_n:\{y_1,\ldots,y_N\} \to \{ 0,1\}$ such that $\nu_n(y_i)=1$
iff $p_i$ divides $n$.
By the Prime
Number Theorem, $p_N=O(N \log N)$ and hence $O(|\circuit|)$ bits are
sufficient to represent $p_N$. Of course, since we need exponentially
many prime numbers they cannot be hard-wired into $\Poca(x)$.

\begin{figure}
  \begin{center}
    \begin{picture}(120,52)(0,0)
      %   \put(-20,-28){\framebox(85,33){}}
      %\gasset{Nadjust=w,Nadjustdist=2,Nh=6,Nmr=1}
      % \gasset{Nadjust=w,Nadjustdist=2,Nh=6}%$,Nmr=1}
      % \node(A)(0,0){$q_0$}

      \gasset{Nframe=n,loopdiam=9,ELdist=.7}
      \gasset{Nadjust=wh,Nadjustdist=1}
      \gasset{curvedepth=0}

      \put(0,25){
        \drawpolygon[Nframe=n,Nfill=y,fillgray=.9](0,0)(18,0)(18,15)(0,15)
        \put(3,7){$\Soca_{\mathsf{inc}}(j)$}
      }

      \put(30,25){
        \drawpolygon[Nframe=n,Nfill=y,fillgray=.9](0,0)(18,0)(18,15)(0,15)
        \put(3,8){$\Soca_{\circuit}(j\cdot 2)$}
        \put(4,4){$=(i\cdot b)$ }
      }
      \put(30,45){
        \drawpolygon[Nframe=n,Nfill=y,fillgray=.9](0,0)(18,0)(18,12)(0,12)
        \put(3,8){$\Soca_{\circuit}(j\cdot 1)$}
        \put(4,4){$=(i\cdot b)$ }
      }
      \put(30,6){
        \drawpolygon[Nframe=n,Nfill=y,fillgray=.9](0,0)(18,0)(18,12)(0,12)
        \put(3,8){$\Soca_{\circuit}(j\cdot 3)$}
        \put(4,4){$=(i\cdot b)$ }
      }

      \put(62,28){
        \drawpolygon[Nframe=n,Nfill=y,fillgray=.9](0,0)(18,0)(18,12)(0,12)
        \put(3,8){$\Soca_{\mathsf{prime}}(i)$}
        \put(5,4){$=p_i$}
      }

      \put(87,35){
        \drawpolygon[Nframe=n,Nfill=y,fillgray=.9](0,0)(25,0)(25,15)(0,15)
        \put(3,6){$\Soca_{\mathsf{divides}}(p_i,x)$ }
        \put(8,11){$b=0?$}
      }

      \put(87,10){
        \drawpolygon[Nframe=n,Nfill=y,fillgray=.9](0,0)(25,0)(25,15)(0,15)
        \put(3,6){$\Soca_{\mathsf{not\_divides}}(p_i,x)$ }
        \put(8,11){$b=1?$}
      }

      \node(A1)(5,29){}
      \node(A2)(13,32){}

      \node(B11)(33,16){}
      \node(B12)(44,16){}

      \node(B21)(33,27){}
      \node(B22)(44,27){}

      \node(B31)(33,48){}
      \node(B32)(44,48){}

      \node(C1)(65,36){}
      \node(C2)(77,36){}

      \node(D11)(90,39){}
      \node(D12)(109,39){}

      \node(D21)(90,13){}
      \node(D22)(109,13){}

      \node(E)(115,0){$\bullet$}
      \node(F)(5,0){$q_{\mathsf{start}}$}

      \drawedge(F,A1){$+x$}

      \drawedge(A2,B11){$0$}
      \drawedge(A2,B21){$0$}
      \drawedge(A2,B31){$0$}

      \drawedge(B12,C1){$0$}
      \drawedge(B22,C1){$0$}
      \drawedge(B32,C1){$0$}

      \drawedge(C2,D11){$0$}
      \drawedge(C2,D21){$0$}

      \drawqbedge(D12,115,25,E){$0$}
      \drawedge(D22,E){$0$}

      \drawedge(E,F){$\mathsf{zero}$}
    \end{picture}

  \end{center}
  \caption{High-level description of the POCA $\Poca(x)$ used for the
    reduction from $\textsf{Succinct 3-SAT}$.}
  \label{fig:high-level}
\end{figure}

Let us now take a look at $\Poca(x)$. It uses one parameter $x$ and
employs several gadgets. Only the gadgets $\Soca_{\mathsf{divides}}$
and $\Soca_{\mathsf{not\_divides}}$ manipulate the counter.
All gadgets are designed so that they communicate via
designated propositional variables, and not as in Section \ref{S CTL
  hardness} with the help of the counter. Starting
in $q_{\mathsf{start}}$, $\Poca(x)$ first loads the
value of the parameter $x$ on the counter.  Think of $x$ encoding a
truth assignment of $\psi$.  Next, $\Poca(x)$ traverses through
$\Soca_{\mathsf{inc}}$, which initially chooses an arbitrary index $j$
identifying a clause of $\psi$. Every time $\Soca_{\mathsf{inc}}$ is
traversed afterwards, it increments $j$ modulo $N$ and hereby moves on
to the next clause.  Now $\Poca(x)$ branches non-deterministically
into a gadget $\Soca_\circuit$ in order to compute $(i\cdot b)$ from
$\circuit$ on input $(j\cdot 1), (j\cdot 2)$, resp.\ $(j\cdot 3)$. The
index $i$ is then used as input to a gadget $\Soca_{\mathsf{prime}}$,
which computes $p_i$. Then if $b=0$, it is checked that $p_i$ does not
divide $x$, and likewise that $p_i$ divides $x$ if $b=1$. Those
checks need to modify the counter. After they
have finished, we restore the value $x$ on the counter and the
process continues with clause $j+1~ \mod N$. We can construct an
$\ltl$ formula $\varphi$ that ensures that all gadgets work and
communicate correctly, and prove that $\psi$ is satisfiable iff there
is an assignment $\sigma$ and an infinite path $\pi=
q_{\mathsf{start}}(0)\rightarrow \cdots$ such that $(T(\Poca^\sigma),
\pi) \not \models \varphi$. The gadgets $\Soca_{\mathsf{inc}}$,
$\Soca_{\mathsf{circuit}}$ and $\Soca_{\mathsf{inc}}$ can be realized
by simulating space-bounded Turing machines with SOCA and some
appropriate $\ltl$ formulas. Here it is important that our LTL formula
$\varphi$ is not fixed.  Divisibility resp. non-divisibility is
checked similar as in the $\CTL$ case, cf.\ Lemma \ref{L mod}. We
refer the reader to the technical report for further details
\cite{techrep}.

\begin{theorem}
  The combined complexity of $\ltl$ model checking on POCA is
  $\coNEXP$-complete.
\end{theorem}

\section{Conclusion}
In this paper, we have settled the computational complexity of model
checking CTL, the modal $\mu$-calculus and LTL on SOCA and POCA with
respect to data and combined complexity. Our proofs for providing
lower bounds have introduced some non-trivial concepts and techniques,
which we believe may be of independent interest for providing lower
bounds for decision problems in the verification of infinite state
systems.

An interesting aspect of future work could be to consider
\emph{synthesis problems} for POCA. Given a POCA $\Poca(X)$ and a
formula $\varphi$, a natural question to ask is whether there
\emph{exists an assignment} $\sigma$ such that $(T(\Poca^\sigma), \pi)
\models \varphi$ on all infinite paths $\pi$ starting in some state of
$T(\Poca^\sigma)$. For CTL resp. the modal $\mu$-calculus, such
a problem is undecidable by Theorem \ref{thm:ctl-poca}. However for
LTL it seems conceivable that this problem can be translated into a
sentence of a decidable fragment of Presburger arithmetic with
divisibility, similar to those studied in \cite{BozgaI05}.

\footnotesize

%\bibliography{bib}
%\bibliographystyle{plain}

\def\cprime{$'$}
\begin{thebibliography}{10}

\bibitem{ArnNiw01}
A.~Arnold and D.~Niwi{\'n}ski.
\newblock {\em Rudiments of $\mu$-calculus}, volume 146 of {\em Studies in
  Logic and the Foundations of Mathematics}.
\newblock {N}orth-{H}olland, 2001.

\bibitem{BHIMV06}
A.~Bouajjani, M.~Bozga, P.~Habermehl, R.~Iosif, P.~Moro, and T.~Vojnar.
\newblock Programs with lists are counter automata.
\newblock In {\em Proc. of CAV}, volume 4144 of {\em LNCS}, pages 517--531. Springer, 2006.

\bibitem{BIL06-icalp}
M.~Bozga, R.~Iosif, and Y.~Lakhnech.
\newblock Flat parametric counter automata.
\newblock In {\em Proc. of ICALP}, volume 4052 of {\em LNCS}, pages 577--588. Springer, 2006.

\bibitem{BozgaI05}
Marius Bozga and Radu Iosif.
\newblock On decidability within the arithmetic of addition and divisibility.
\newblock In {\em Proc. of FOSSACS}, volume 3441 of {\em LNCS}, pages 425--439. Springer,
  2005.

\bibitem{CaFu91}
Jin-Yi Cai and Merrick Furst.
\newblock P{SPACE} survives constant-width bottlenecks.
\newblock {\em International Journal of Foundations of Computer Science}, 2(1):67--76, 1991.

\bibitem{CR04}
Cristiana Chitic and Daniela Rosu.
\newblock On validation of xml streams using finite state machines.
\newblock In {\em Proc. of WebDB}, pages 85--90. ACM, 2004.

\bibitem{ChDaLi01}
Andrew Chiu, George Davida, and Bruce Litow.
\newblock Division in logspace-uniform {$\rm NC\sp 1$}.
\newblock {\em Theoretical Informatics and Applications. Informatique
  Th\'eorique et Applications}, 35(3):259--275, 2001.

\bibitem{CJ98-cav}
H.~Comon and Y.~Jurski.
\newblock Multiple counters automata, safety analysis and presburger
  arithmetic.
\newblock In {\em Proc. of {CAV}}, volume 1427 of {\em LNCS}, pages 268--279. Springer, 1998.

\bibitem{DG-jlc09}
St{\'e}phane Demri and R{\'e}gis Gascon.
\newblock The effects of bounding syntactic resources on {P}resburger {LTL}.
\newblock {\em Journal of Logic and Computation}, 19(6):1541--1575, December 2009.

\bibitem{techrep}
Stefan G{\"o}ller, Christoph Haase, Jo{\"e}l Ouaknine, and James Worrell.
\newblock Model checking succinct and parametric one-counter automata.
\newblock Technical report, University of Bremen, 2010.
\newblock Available via\\
  \url{http://www.informatik.uni-bremen.de/tdki/research/papers/succ.pdf}.

\bibitem{GoLo09}
Stefan G{\"o}ller and Markus Lohrey.
\newblock Branching-time model checking of one-counter processes.
\newblock Technical report, arXiv.org, 2009.
\newblock {\tt http://arxiv.org/abs/0909.1102}.

\bibitem{GoLo10}
Stefan G{\"o}ller and Markus Lohrey.
\newblock Branching-time model checking of one-counter processes.
\newblock In {\em Proc. of STACS}, pages 405--416. IFIB Schloss Dagstuhl, 2010.

\bibitem{HKOW10}
Christoph Haase, Stephan Kreutzer, Joel Ouaknine, and James Worrell.
\newblock Reachability in parametric one-counter automata.
\newblock 2010.
\newblock Submitted. Available via:\\
  \url{http://www.comlab.ox.ac.uk/files/2833/iandc.pdf}.

\bibitem{HLSVW93}
Ulrich Hertrampf, Clemens Lautemann, Thomas Schwentick, Heribert Vollmer, and
  Klaus~W. Wagner.
\newblock On the power of polynomial time bit-reductions.
\newblock In {\em Proc. of CoCo}, pages 200--207. IEEE Computer Society Press, 1993.

\bibitem{IJTW93-icalp}
O.~H. Ibarra, T.~Jiang, N.~Tr{\^a}n, and H.~Wang.
\newblock New decidability results concerning two-way counter machines and
  applications.
\newblock In {\em Proc. of ICALP}, volume 700 of {\em LNCS}, pages 313--324. Springer, 1993.

\bibitem{ID06-tcs}
Oscar~H. Ibarra and Zhe Dang.
\newblock On the solvability of a class of diophantine equations and
  applications.
\newblock {\em Theor. Comput. Sci.}, 352(1):342--346, 2006.

\bibitem{JKMZ04-ic}
P.~Jan\v{c}ar, A.~Ku\v{c}era, F.~Moller, and Z.~Sawa.
\newblock {DP} lower bounds for equivalence-checking and model-checking of
  one-counter automata.
\newblock {\em Information Computation}, 188(1):1--19, 2004.

\bibitem{LS08-atva}
J.~Leroux and G.~Sutre.
\newblock Flat counter automata almost everywhere!
\newblock In {\em Proc. of {ATVA}'05}, volume 3707 of {\em LNCS}, pages 489--503. Springer, 2005.

\bibitem{Min61}
Marvin~L. Minsky.
\newblock Recursive unsolvability of {P}ost's problem of ``tag'' and other
  topics in theory of {T}uring machines.
\newblock {\em Annals of Mathematics. Second Series}, 74:437--455, 1961.

\bibitem{Papa94}
C.~H. Papadimitriou.
\newblock {\em Computational Complexity}.
\newblock Addison Wesley, 1994.

\bibitem{Serr06}
Olivier Serre.
\newblock Parity games played on transition graphs of one-counter processes.
\newblock In  {\em Proc. of FOSSACS}, volume 3921 of {\em LNCS}, pages 337--351. Springer, 2006.

\bibitem{Vol98}
Heribert Vollmer.
\newblock A generalized quantifier concept in computational complexity theory.
\newblock Technical report, arXiv.org, 1998.
\newblock \url{http://arxiv.org/abs/cs.CC/9809115}.

\end{thebibliography}




%\thebibliography

\iffalse

\begin{thebibliography}{1}


\bibitem{GoLo10}
S.~G\"oller, M.~Lohrey.
\newblock Branchning-time model checking of one-counter processes.
\newblock In Proceedings of STACS 2010, to appear
\newblock {\em Arxiv:0909.1102v1}

\end{thebibliography}
\fi

\end{document}

