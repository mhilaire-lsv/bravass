



\section{State resilience}


Resilience is a strong property since it implies that from every element of $\Bad$ there must exist a path to $\Safe$. But, when one considers a system with an initial state $s_0$, it could be sufficient to ask that only from $\Bad \cap post^*(s_0)$, there must exist a path to $\Safe$. 
%
%			However, this condition seems more difficult since one also must decide whether a state $s \in \Bad$ is reachable from $s_0$.
%
%\problemx{(I,J)-$k$-resilience problem for WSTS}
%{A state $s$ of a WSTS $(S,\rightarrow, \leq)$, an effective set $I$ (with a given basis), a set $J$.}
%{$\forall s' \in J ~ (s \rightarrow^* s') \implies \exists s'' \in I ~ s' \rightarrow^{\leq k} s''$ ?\newline}
%
%\problemx{general $k$-resilience problem for WSTS}
%{A state $s$ of a WSTS $(S,\rightarrow, \leq)$, an effective set $I$ (with a given basis).}
%{$\forall s'  ~ (s \rightarrow^* s') \implies \exists s'' \in I ~ s' \rightarrow^{\leq k} s''$ ?\newline}

%		Again, we first consider the case where the recovery time is bounded by a $k \in \N$.

The three previous problems become:


\problemx{State-resilience problem (SRP)}
{A transition system $\mathscr{S}=(S,\rightarrow)$, $s \in S$ and two sets $\Safe, \Bad \subseteq S$.}
% 		an upward-closed set $\Safe$ with a given basis, a decidable downward-closed set $\Bad$.}
% 		{$ ~ \forall s' \in \Bad, ~ s \rightarrow^* s' \implies \exists s'' \in \Safe, ~ s' \rightarrow^{*} s''$ ?\newline}
{$\Bad \cap \post^*(s_0)  \rightarrow^{*} \Safe $ ? \newline}


\problemx{$k$-state-resilience problem (kSRP)}
{A transition system $\mathscr{S}=(S,\rightarrow)$, $s \in S$ and two sets $\Safe, \Bad \subseteq S$.}
{ $\Bad \cap \post^*(s) \longrightarrow^{\leq k} \Safe$ ?  \newline}
%

\problemx{bounded-state-resilience problem (BSRP)}
{A transition system $\mathscr{S}=(S,\rightarrow)$, $s \in S$ and two sets $\Safe, \Bad \subseteq S$.}
%{$\exists k \geq 0 ~ \forall s' \in D ~ s \rightarrow^* s' \implies \exists s'' \in U ~ s' \rightarrow^{\leq k} s''$ ?\newline}
{$\exists k \geq 0$ such that $\mathscr{S}$ is $k$-state-resilient. ?\newline}


%\textcolor{blue}{
%justification des definitions: si on enlève la condition  an upward-closed set $U$ (extended resilience $U$ qq) reachability reduces to resilience (avec $U=\{x\}$ donc indecidable pour les modèles à reachability indécidable.
%}



Since these problems are undecidable for general infinite-state transition systems, we restrict our study to WSTS.

As in the Section~3, we study decidability results for different pairs of sets $\Safe$ and $\Bad$ downard closed and upward closed. We will consider here the cases where 
$\Safe = \uparrow \Safe$ and $\Bad = \downarrow \Bad$ 
and where 
$\Safe = \downarrow \Safe$ and $\Bad = \uparrow \Bad$.

% \mathieu{Ici on s'intéresse seulement aux cas $\uparrow \downarrow$ et $\downarrow \uparrow$.}

\subsection{Case: $\Safe = \uparrow \Safe$ and $\Bad = \downarrow \Bad$}



Let us recall a result about state-resilience (called resilience in \cite{DBLP:conf/gg/Ozkan22,DBLP:journals/corr/abs-2108-00889}).

\begin{theorem}\cite{DBLP:conf/gg/Ozkan22,DBLP:journals/corr/abs-2108-00889}\label{ref ozkan}
{\sc bounded-state-resilience} and {\sc $k$-state-resilience} are decidable for WSTS $S$ with strong upward compatibility and such that $\uparrow \post^*(s)$ is computable for $s \in S$
when
$\Safe=\uparrow \Safe$
and $\Bad=\downarrow \Bad$.
\end{theorem}

We may immediately generalyse this last result by strengthening to \emph{unbounded} state-resilience. The proof is essentially the same than the previous one.

\begin{corollary}\label{postcomputable}
{\sc state-resilience} is decidable for WSTS with strong upward compatibility and such that $\uparrow \post^*(s)$ is computable for $s \in S$
when
$\Safe=\uparrow \Safe$
and $\Bad=\downarrow \Bad$.
\end{corollary}

\begin{proof}
%From Fact~\ref{stop condition},
Since $\mathscr{S}$ is a WSTS there exists $n_0 \in \N$ such that
$\pred^*(\Safe) =  \pred^{\leq n_0}(\Safe)$. We can compute this $n_0$ by iteratively computing 
%\alain{mauvaises notations: $k, k_m, Safe^k,...$ n'a aucun sens...}
$\pred^{\leq n+1}(\Safe)$ from $\pred^{\leq n}(\Safe)$, checking 
$\pred^{\leq n+1}(\Safe) = \pred^{\leq n}(\Safe)$, 
returning $n$ if that is the case.
Then, because $n_0$-state-resilience is decidable, 
checking $\uparrow post^*(s) \cap \Bad \subseteq \pred^{\leq n}(\Safe) = \pred^*(\Safe)$ is,
and state-resilience is decidable.
\end{proof}

\iffalse
\alain{definir downward reachability problem.....
downward-closed problem given a state $s$ of a WSTS
% in the regarded class 
with strong upward compatibility 
and a decidable downward-closed set $D$, it can be decided whether $\exists s' \in D ~ s \to^* s'$. }
\fi


The computability of $\uparrow \post^*(s)$ seems a strong hypothesis. What are the WSTS for which $\uparrow \post^*(s)$ is computable for $s \in S$ ?
Ozkan \cite{DBLP:conf/gg/Ozkan22} argues that it is precisely the WSTS for which the following problem is decidable.

\problemx{Downward reachability problem}
{A transition system $\mathscr{S}=(S,\rightarrow)$, $s \in S$ and a downward-closed set $D
\subseteq S$.}
{$\exists s' \in D ~ s \to^* s'$? \newline}



%\"Ozkan
\begin{proposition}[Proposition 1 in \cite{DBLP:conf/gg/Ozkan22}]\label{post*}
For finite-branching WSTS%with strong upward compatibility
, a basis of $\uparrow \post^*(s)$ is computable for every state $s$ iff the downward-reachability problem is decidable.
%i 			.e. given a state $s$ of a WSTS
% in the regarded class 
%with strong upward compatibility 
%and a decidable downward-closed set $D$, it can be decided whether $\exists s' \in D ~ s \to^* s'$. 
\end{proposition}

% \alain{rappeler l'idée de la preuve}
The idea behind the proof is the following. For deciding whether a downward-closed set $D$ is reachable from $s$, we check whether
$B_{\uparrow \post^*(s)} \cap D = \emptyset$, equivalent to $\post^*(s)\cap D = \emptyset$ by
Lemma~\ref{Lemma intersection}. For the converse direction, we compute the sequence of upward-closed
$U_n = \uparrow post^{\leq n}(s)$ until it becomes stationnary. 
Decidability of downward reachability leads to the decidability of the following stop condition:
asking whether $S \setminus U_n$ is reachable from $s$.


The characterization is used to show that Petri nets are $\post^*$-effective. Since the downward reachability problem is decidable for Petri-nets, we deduce the following:

\begin{proposition}
Let $P = (N,M,W)$ be a Petri net and $s \in M$ a configuration of $P$, then one can compute a basis of $\uparrow \post^*(s)$.
\end{proposition}

\iffalse
\begin{proof}
By Proposition~\ref{post*}, a basis of $\uparrow \post^*(s)$ is computable for every state $s$ iff the downward-reachability problem is decidable.
Let us now show that the downward-reachability problem is decidable for Petri nets.
Let us consider a downward closed set $D$ in a Petri net.
Let us consider $B$ a basis of the upward closed set $S \setminus D$.

$D$ is  defined by ... states where your number of counter is bounded by a constant (deduce this from ... having to be smaller than elements of the basis $B$), and others where it is unbounded. You have $m$ counters where it is bounded by $n$
then you have $m \times n$ possibilities for the values in these counters
you have unlimited possibilities for the values of the other counters,
but that is where the submarking problem comes in handy,
i.e. you ask the submarking problem for all of the $m \times n$ possibilities.

So downward-set reachability is decidable for PN.
\end{proof}
\fi

It is well-known that Petri nets are WSTS with strong upward compatibility (Thm.~6.1 in \cite{DBLP:journals/tcs/FinkelS01}). 
Since VASS are WSTS with strong upward compatibility and since there is an algorithm that computes a finite basis of  $\uparrow \post^*(s)$, \cite{DBLP:conf/gg/Ozkan22} deduced that bounded state-resilience is decidable for VASS.
Hence state-resilience is decidable for both Petri nets and VASS.
However, the hypothesis that $\uparrow \post^*$ is computable cannot be tested in the general WSTS framework. Moreover, we may show:

\begin{proposition}
There exist classes of WSTSs with strong upward compatibility for which there don't exist an algorithm computing a basis of $\uparrow \post^*$.
\end{proposition}


\begin{proof}
Let us show that Reset VASS, that are WSTSs with strong upward compatibility, don't enjoy the property that $\uparrow \post^*$ is computable.
Suppose that one is able to compute a finite basis of $\uparrow \post^*$ for Reset VASS. Then, one would be able to decide whether $0$ is reachable 
%in counter $i$ 
by examining if there is %some vector $v$ in the basis such that $v(i)=0$
$0$ in the basis%
. But reachability of $0$  %in a counter $i$ 
is undecidable for Reset VASS. 
% sont WSTS with strong compatibility. et l'accessibilité est undecidable.
%	$\uparrow post^*$ n'est pas calculable pour les LCS non plus car $\uparrow post^*= \uparrow \downarrow post^*$ n'est pas calculable ???
\end{proof}

%il existe des modèles où on peut calculer $\uparrow post^*$: inserted channel systems: on sait calculer $post^*$ qui est rationnel donc on sait calculer $\uparrow post^*$.


%
%si on garde strong mais on enlève effective basis of $post^*$:
%
%\begin{proposition}(à prouver)
%{\sc $k$-resilience}, hence {\sc bounded resilience}, is undecidable for effective WSTS with strong upward compatibility.
%\end{proposition}
%
%\begin{proof}
%pas de effective basis of $post^*$. leur algorithme n'est plus un algo. ex reset PN ?  \textcolor{red}{pas certain ! à faire Mathieu}
%\end{proof}






Unfortunately, state-resilience is undecidable for (general) WSTS even with strong upward-compatibility.
This stems from the fact that $t$-liveness is undecidable for reset Petri-Nets.


\begin{theorem}[Adjusting Theorem~5.5 from \cite{peterson1981petri}]\label{liveness reset}
The  Reset Petri Net zero reachability problem can be reduced to the liveness problem.
\end{theorem}

\iffalse
\begin{proof}
If we wish to determine if $0 \in R(C_1, \mu_1)$ in any
reset Petri net $C_1 = (P_1, T_1, I_1, O_1, R_1 )$ (where $R_1(t) =$ places that are reset when t fires off), with initial marking $\mu_1$, we construct a
reset petri net $C_2 = (P_2, T_2, I_2, O_2, R_2 )$ with initial marking
$\mu_2$ which is live if and only if the zero marking is not reachable from $\mu_1$.

The reset PN $C_2$ is constructed from $C_1$ by the addition of two places $r_1$ and $r_2$
and $|P_1| +2$ transitions $s$, $\{ s_p | p \in P_1 \}$ and $s'$.

We first modify all transitions of $T_1$ to include $r_1$ as both an input and an output.
The initial marking $\mu_2$ will include a token in $r_1$. 


The place $r_1$ is a 'run' place; as long as the token remains in $r_1$ the transitions of $T_1$ can fire normally.
Thus any marking which is reachable in the places of $P_1$ is also reachable in $C_2$.



Transition $s$ is defined to have $r_1$ as an input and a null output.
This allows the token in $r_1$ to be removed, disabling all transitions in $T_1$
and "freezing" the marking of $P_1$.

Note that all transitions of $T_1$ are in conflict and, by construction if not by definition, no more than one transition can fire at a time.

The place $r_1$ and transition $s$ allow the net $C_1$ to reach any reachable marking and then for $s$ to fire and freeze the net at that marking.

Now we need to see if that marking is zero.

We introduce a new place $r_2$ and new transitions $s_p$ which have $p$ as input and $r_2$ as output.

If $p$ can ever become zero, $s_p$ is not live. The entire net is dead if
$s$ fires in the marking zero.

If we can always have some $p_i$ not be zero, then we can always fire some $s_{p_i}$, putting a token in $r_2$.

In this case we must put a token back in $r_1$ and assure that all transitions in $C_2$ are live.
We must be sure that $C_2$ is live even if $C_1$ is {\em not} live.
This is accomplished by a transition $s'$ which  "floods" the net $C_2$ with tokens, assuring that every transition is live if a token is ever put in $r_2$.
Transition $s'$ has $r_2$ as its input and every place of $C_2$ as output

Now if $0$ is reachable
in $R(C_1, \mu_1)$ 
then the net $C_2$ can also reach this marking
in the place of $P_1$ by executing the same sequence of transition firings.
Then $s$ can fire, freezing the $C_1$ subset.
Since $mu(p_i) = 0$ for all $p_i \in P_1$, no transition $s_{p_i}$ with $p_i \in P_1$ can fire, and $C_2$ is dead.
Thus if $0$ is reachable then $C_2$ is not live.

Conversey, if $C_2$ is not live, then a marking $\mu$ must be reachable in which 
$\mu(r_2) = 0$ and there is no reachable state in which $r_2$ has a token
(in particular, since we do not allow token removal from $r_2$, the marking $\mu$ must be reached in a sequence of transitions that do not place any token in $r_2$.)
% (Rappel: tant que il y a un token quelque part, on peut mettre un token en r_2
% et r_2 commence sans token)
If $r_2$ has no token and cannot get any, then the transitions $s_p$ are not live, and the markings of $p$ for $p \in P_1$ must be zero. 
Thus if $C_2$ is not live then a marking is reachable in which the marking of each $p$ in $P_1$ is zero. 

\end{proof}
\fi

% \mathieu{I moved the proof to the appendix since it's a slight adjustment but takes a lot of room.}
\begin{proof}
See Appendix~\ref{appendix} for the complete proof.
\end{proof}


Since the Reset Petri Net zero reachability problem is undecidable in Reset Petri-Nets, the reduction implies the following:

\begin{corollary}
Reset Petri Net liveness is undecidable.
\end{corollary}

Since moreover liveness can be reduced to 
$t$-liveness, we deduce:

\begin{corollary}
Reset Petri Net $t$-liveness is undecidable.
\end{corollary}

We now reduce $t$-liveness to the state resilience problem.

\begin{proposition}\label{reductions}
In WSTS, $t$-liveness is reducible to the state resilience problem.
%	home state and reachability
\end{proposition}


\begin{proof}
We reformulate $t$-liveness as the following formula
% $$ ~ \forall s' \in S, ~ s \rightarrow^* s' \implies \exists s'' \in U_t, ~ s' \rightarrow^{*} s''$$
$$ ~ \forall s \in S, ~ s_0 \rightarrow^* s \implies \exists s' \in U_t, ~ s \rightarrow^{*} s'$$  
where
$U_t=\uparrow \pred(t)$
is the upward closure of the preconditions to use transition $t$.  
The problem reduces itself to the state resilience problem
where $\Safe = \uparrow \pred(t)$ and $\Bad = S \setminus \Safe$.

\end{proof}



\iffalse

\begin{theorem}
The downward reachability problem is reducible to the state resilience problem.
\end{theorem}

\begin{proof}
Let $s \in S$, and a downward-closed set $D$. \alain{je ne comprends pas cette "preuve"}
Take $\Safe = \emptyset$ and $\Bad = D$.
The state resilience problem on $\Safe,\Bad$ asks
whether $\Bad \cap \post^*(s) \rightarrow^* \Safe$  that is equivalent to $\Bad \cap \post^*(s) \subseteq \emptyset$.
If the answer is negative then $s \to^* \Bad$,
else $s \not\to^* \Bad$.
\end{proof}

\fi


Since $t$-liveness is undecidable for reset Petri-Nets that are WSTS with strong upward compatibility, from Proposition \ref{reductions},  we deduce that state-resilience is undecidable for WSTS with strong upward-compatibility.

\begin{theorem}\label{srp up down}
{\sc State-resilience} is undecidable for strongly compatible WSTS with effective pred-basis
when
$\Safe=\uparrow \Safe$
and $\Bad=\downarrow \Bad$.
\end{theorem}


This undecidability result furthermore implies the undecidability of the other two state resilience problems by straightforward reductions.


\begin{proposition}
In WSTS with strong compatibility and effective pred-basis,  $\Safe=\uparrow \Safe$, the bounded-state-resilience problem is equivalent to the state-resilience problem.
\end{proposition}

\begin{proof}
Since $\Safe=\uparrow \Safe$ and
$\mathscr{S}=(S,\rightarrow,\leq)$ is a WTSTS with strong %upward-
compatibility, then $\pred^n(\Safe)= \uparrow~\pred^n(\Safe)$ for all $n \in \N$,
and there exists $n_0 \in \N$ such that 
$\pred^{n_0}(\Safe) = \uparrow \pred^{n_0}(\Safe) = \uparrow \pred^*(\Safe) = \pred^*(\Safe)$.
Hence the equivalence.
\end{proof}

\begin{proposition}
In WSTS with strong compatibility and effective pred-basis, the bounded state-resilience problem is
reducible to the $k$-state-resilience problem.
\end{proposition}

\begin{proof}
% strong WSTS
% $\pred^*(I) = \pred^{k}(I)$
% Check all $k$ until $\pred^*(I) = \pred^{k}(I)$.
Since $\Safe=\uparrow \Safe$ and
$\mathscr{S}=(S,\rightarrow,\leq)$ is a WTSTS with strong %upward-
compatibility, then $\pred^n(\Safe)= \uparrow~\pred^n(\Safe)$ for all $n \in \N$,
and there exists $n_0 \in \N$ such that 
$\pred^{n_0}(\Safe) = \uparrow \pred^{n_0}(\Safe) = \uparrow \pred^*(\Safe) = \pred^*(\Safe)$.
We compute 
$n_0$, then iteratively check whether $k$-state resilience hold for $k$ from $0$ to $n_0$.  
\end{proof}

\begin{corollary}\label{bsrp up down}
Bounded state-resilience and 
$k$ state-resilience are undecidable for strongly compatible WSTS with effective pred-basis
when
$\Safe=\uparrow \Safe$
and $\Bad=\downarrow \Bad$.
\end{corollary}

\begin{proof} 
  From the two reductions above. 
\end{proof}











Keeping the $\uparrow post^*$ effectiveness hypothesis but loosening the strong compatibility one still yields some decidability result for the general state-resilience problem.


\begin{lemma}\label{Lemma intersection}
Let $A \subseteq S$, $D \subseteq S$ be a downward-closed set and $U \subseteq S$ be an upward-closed set. 
Then $A \cap D \subseteq U$  iff $ (\uparrow  A) \cap D \subseteq U$.
\end{lemma}


\begin{proof}
Let us suppose that $A \cap D \subseteq U$. Then ${\uparrow (A \cap D)} \subseteq {\uparrow U} = U$.
Let us show that $({\uparrow A}) \cap D \subseteq {\uparrow (A \cap D)}$.
Let $x \in ({\uparrow A}) \cap D$, then there exists $a \in A$ such that $x \geq a$.
Since $x \in D$ and $D$ is downward-closed, we also have $a \in D$.
Hence $a \in A \cap D$ and then $x \in { \uparrow (A \cap D)}$.
In the other direction,
since $A \subseteq {\uparrow A}$, the inclusion
$({\uparrow  A}) \cap D \subseteq U$ implies
$A \cap D \subseteq ({\uparrow  A}) \cap D \subseteq U$.
\end{proof}



\begin{theorem}{(Adjusting Theorem 1 from \cite{DBLP:journals/corr/abs-2108-00889})}\label{post srp}
{\sc state-resilience} is decidable for (not necessarily strong) WSTS with effective 
$\uparrow$ $\post^*$ basis
when
$\Safe=\uparrow \Safe$
and $\Bad=\downarrow \Bad$.
\end{theorem}


\begin{proof}
Let $B_p$ be a basis of $\uparrow \post^*(s)$, $B_\Safe$ a basis of $\Safe$
and $\Bad$ decidable downward-closed.
By applying Lemma~\ref{Lemma intersection} twice, we obtain

$$ post^*(s) \cap \Bad \subseteq \pred^*(\Safe) \equiv B_p \cap \Bad \subseteq \pred^*(\Safe)$$


Since $B_p$ is finite and $\Bad$ is decidable, we can directly compute $ B_p \cap \Bad$.
% For every element of $ B_p \cap J$, checking that it is in $\pred^*(I)$ is the same as checking coverability of I, and thus decidable.
We can compute a basis of $\pred^*(\Safe)$ from $B_\Safe$, and hence check that $B_p \cap \Bad \setminus \pred^*(\Safe) = \emptyset$. 
\end{proof}

% In the proof of Thm. 1, it was crucial that we have strong compatibility. This approach does not work
% for WSTSs in general. We loose precision when we only demand compatibility. Thus, we conjecture
% that both resilience problems are undecidable for WSTSs in general, but this question remains still open.
%
% Well, maybe their version of resilience but resilience with * seems fine.
%



However when removing strong compatibility, precision is lost and the algorithm do not work anymore.

\begin{proposition}
The algorithm from \cite{DBLP:conf/gg/Ozkan22} for {\sc $k$-state-resilience} terminate but is incorrect for WSTS with upward compatibility (not strong) and an effective basis of $\uparrow \post^*$.
\end{proposition}

\begin{proof}[Sketch]
\mathieu{ (Formalism should be made more explicit so that the proof can be really pertinent)}
\textcolor{red}{
Let us consider a LCS where erasing a message take one step. Then the upward compatibility is not strong. 
For instance with a rule $\delta(q,b) = \text{ add some }b $,
we have $(q,b) \to (q,bb)$, $(q, baaa) \geq (q,b)$ and
$(q, baaa) \to^* (q,bb)$ however it takes more than a single step. 
In this case then $\pred(\uparrow \{(q,b)\})$ is not upward-closed. Hence 
$post^* (s) \cap D \subseteq \pred(\uparrow \{(q,b)\})$
do not imply
$\uparrow post^* (s) \cap D \subseteq \pred(\uparrow \{(q,b)\})$ for a downward-closed set $D$.
In particular by lossiness $(q, \epsilon)$ would belong to
$post^* (s)$,
hence $(q,a)$, $(q,aa)$, etc. would belong to $\uparrow post^* (s)$ although not necessarily to
$post^* (s)$. $(q,aaa)$ not being in $pred(\uparrow \{ (q,b)\})$ means the algorithm 
from \cite{DBLP:conf/gg/Ozkan22} would deduce that $1$-rechable-resilience does not hold in this example when $\Safe=\uparrow \{(q,b)\}$ and $\Bad =S$,
because $\uparrow post^* (s) \cap S \not\subseteq \pred(\uparrow \{(q,b)\})$,
despite having 
$post^* (s) \cap S \subseteq \pred(\uparrow \{(q,b)\})$.
}

\iffalse
\mathieu{

On considère un LCS, où on autorise à effacer des messages (lossyness) mais effacer un message prends une étape (contrairement à [ref], où ils font une sorte de cloture de effacer-transition-effacer). On n'a alors pas la strong compatibility: 

On prends: $(q, b) 	\to (q,bb)$ par exemple et $(q, baaaaaa) \geq (q,b) $, alors
$(s, baaaaaa) \to^* (s,b) \to (s,bb)$
MAIS

C'est pas "en une étape".

(On considère -il faut le mentionner/vérifier- que on ne peut effaçer que d'un coté ? et pas au milieu?)

Dans ce cas là, du coup, $pred(\uparrow \{(q,b)\}$ n'est pas clos par le haut. \\

Revenons au Corollaire

On voudrais avoir

$post^* (s) \cap J \subseteq I^k$

MAIS pas

$ (\uparrow post^* (s)) \cap J \subseteq I^k$ \\


Ça serait un cas où leurs algorithme vas dire 
"On n'a pas $(\uparrow post^* (s)) \cap J \subseteq I^k$ donc on n'a pas la $k$-résilience "

Alors qu'on aurait la $k$-résilience (i.e. $post^* (s) \cap J \subseteq I^k$)

Pour $I^k$, le contre exemple prendrait du coup $pred(\uparrow \{ (q,b)\})$ qui n'est pas clos par le haut.

On voudrait

$post^*(s) \cap J \subseteq pred(\uparrow \{ (q,b)\})$


Je propose de prendre $J$ = ensemble des configurations je pense le truc ici doit venir de
$\uparrow post^* (s)$ being too big
donc autant pas se restreindre du tout sur $J$


$post^*(s) \subseteq pred(\uparrow \{ (q,b)\})$

Si $s=(q,b)$ par exemple, et que on peut aller de $q$ vers $q$ en écrivant un 
$b$ ? et évidement en perdant le dernier message (parce qu'on est lossy). Et c'est tout, disons.

Déjà par lossiness $(q,\epsilon)$ serait dans $post^*((q,b))$.

Et aussi dans $pred(\uparrow \{ (q,b)\})$ puisque $(q,\epsilon) \to (q,b)$.

Donc on aurait:

$post^*((q,b))  \subseteq pred(\uparrow \{ (q,b)\})$

Et on aurait $(q,aaaaaa)$ dans 
$\uparrow post^* ((q,b))$ puisque $(q,\epsilon)$ est dans $post^* ((q,b))$.

Hors $(q,aaaaa)$ est pas dans $pred(\uparrow \{ (q,b)\})$
(même si il est dans $pred^*(\uparrow \{ (q,b)\})$).

}
\fi 
 
\end{proof}

\mathieu{In case we have  effective basis of $post^*$ and upward compatibility (not strong) then we don't know the decidability status of k-state-resilience and bounded-state-resilience. Not necessarily a big deal, but should write a comment or two about this.}

\iffalse
\begin{proposition}
\textcolor{red}{
{\sc bounded resilience} is undecidable for WSTS with upward compatibility with an effective basis of $post^*$.
CONJECTURE
}
\end{proposition}

\begin{proof}
 \textcolor{red}{pas certain ! à faire Mathieu}
\end{proof}
\fi


\subsection{Case: $\Safe = \downarrow \Safe$ and $\Bad = \uparrow \Bad$}


The authors from \cite{DBLP:conf/gg/Ozkan22,DBLP:journals/corr/abs-2108-00889}
have only been concerned with the case $\Safe = \uparrow \Safe$
and $\Bad =\downarrow \Bad$, 
\mathieu{pas tout à fait vrai: dans le technical report \cite{DBLP:conf/gg/Ozkan22} il considère aussi le cas où Bad et Safe sont tous les deux clos par le haut mais aussi le cas Safe clos par le bas Bad clos par le haut.}
 however as mentioned in Section~3
the property for instance of mutual exclusion is often modelised with $\Safe = \downarrow \Safe$.
Hence it makes sense to consider this symmetrical case as well. \\


\mathieu{rappeler résultats de \cite{DBLP:conf/gg/Ozkan22}}


\begin{definition}{marked WSTS} 
A {\em marked} WSTS is a tuple $(S, \leq, \to, INIT)$
where $(S, \leq, \to)$ is a WSTS and $INIT \subseteq S$. 
If $INIT$ is finite, we call it {\em fin-marked}.
\end{definition}

\begin{definition}
A marked WSTS is
\begin{itemize}
\item lossy if $\downarrow \post^* (INIT) = \post^* (INIT)$,
\item $\bot$-bounded (bottom-bounded) if there is a natural number 
$\ell \geq 0$ s.t. $s_B \in \post^{\leq \ell} (s) $ for every $ s \in S$ and every element 
$s_B$ of a basis of $S$ with $s \geq s_B$.
\end{itemize}
\end{definition}

\mathieu{}

\begin{theorem}{\cite{DBLP:conf/gg/Ozkan22}}
The bounded-state-resilience problem is decidable for \textcolor{red}{fin-marked} 
strongly compatible WSTS
 which are 
 {\em \textcolor{red}{lossy} } 
 and 
 {\em \textcolor{red}{$\bot$-bounded}  }
 if $\Bad = \uparrow \Bad$ and $\Safe = \downarrow \Safe$.
\end{theorem}



Unfortunately however the problem is undecidable in general.

\begin{theorem}
The state resilience problem is undecidable for effective WSTS with  strong  compatibility 
when
$\Safe=\downarrow \Safe$
and $\Bad=\uparrow \Bad$.
\end{theorem}

\begin{proof}
Consider Reset VASS
in the case $\Safe$ contains only the minimal element %$0$ 
and $\Bad = S \setminus \Safe$, 
it follows the state resilience problem is equivalent 
to the problem of whether $0$ can be visited infinitely often. 
%
% Since reachability of $0$ is undecidable for Reset VASS, we conclude.
Hence the undecidability.
\end{proof}

\mathieu{But what if we are in the case of $\uparrow \post^*$ effectiveness?}

\mathieu{Écrire que c'est globalement symmétrique au cas d'avant - je pense pas nécessairement qu'on arrive à faire beaucoup mieux que de juste dire ça.}

\begin{lemma}(Symmetrical from Lemma~\ref{Lemma intersection})\label{Lemma intersection 2}
Let $A \subseteq S$, $D \subseteq S$ be a downward-closed set and $U \subseteq S$ be an upward-closed set. 
Then $A \cap U \subseteq D$  iff $ (\downarrow  A) \cap U \subseteq D$.
\end{lemma}


\begin{proof}
Let us suppose that $A \cap U \subseteq D$. Then ${\downarrow (A \cap U)} \subseteq {\downarrow D} = D$.
Let us show that $({\downarrow A}) \cap U \subseteq {\downarrow (A \cap U)}$.
Let $x \in ({\downarrow A}) \cap U$, then there exists $a \in A$ such that $x \leq a$.
Since $x \in U$ and $U$ is downward-closed, we also have $a \in U$.
Hence $a \in A \cap U$ and then $x \in { \downarrow (A \cap U)}$.
In the other direction,
since $A \subseteq {\downarrow A}$, the inclusion
$({\downarrow  A}) \cap U \subseteq D$ implies
$A \cap U \subseteq ({\uparrow  A}) \cap U \subseteq D$.
\end{proof}

Lemma~\ref{Lemma intersection 2} can be used to show that
$\post^*(s) \cap \Bad \subseteq D$  iff $ (\downarrow  \post^*(s)) \cap \Bad \subseteq D$.

If, as in Section~XX, we have the \textcolor{red}{added hypothesis}
that
$\pred^*(\Safe)$ is downward closed this means that
$\post^*(s) \cap \Bad \subseteq \pred^*(\Safe)$  iff $ (\downarrow  \post^*(s)) \cap \Bad \subseteq \pred^*(\Safe)$.

And hence decidability follows.




\subsection{Synthesis}


\iffalse
\mathieu{Should try to put all results in a single table}


With only strong compatibility hypothesis:

\begin{center}
\begin{tabular}{ | l | c | c | r |}
\hline   \Safe~\Bad & $\uparrow$~ $\downarrow$~ & $\downarrow$~ $\uparrow$~  \\ \hline
   SRP & Undecidable (Thm~\ref{srp up down}) & Undecidable  \\ \hline
   BSRP & Undecidable (Corollary~\ref{bsrp up down}) &  ??  \\ \hline
      kSRP & Undecidable (Corollary~\ref{bsrp up down}) & ?? \\ \hline
 \end{tabular}
\end{center}

With $\uparrow \post^*$ effectiveness hypothesis:

\begin{center}
\begin{tabular}{ | l | c | c | r |}
\hline   \Safe~\Bad & $\uparrow$~ $\downarrow$~ & $\downarrow$~ $\uparrow$~  \\ \hline
   SRP & Decidable (Thm~\ref{post srp}) & ??  \\ \hline
   BSRP & ?? &  ??  \\ \hline
      kSRP & ?? & ?? \\ \hline
 \end{tabular}
\end{center}

With $\uparrow \post^*$ effectiveness and strong compatibility hypothesis:

\begin{center}
\begin{tabular}{ | l | c | c | r |}
\hline   \Safe~\Bad & $\uparrow$~ $\downarrow$~ & $\downarrow$~ $\uparrow$~  \\ \hline
   SRP & Decidable (Thm~\ref{postcomputable})& ??  \\ \hline
   BSRP & Decidable (Thm~\ref{ref ozkan})&  ??  \\ \hline
      kSRP & Decidable (Thm~\ref{ref ozkan}) & ?? \\ \hline
 \end{tabular}
\end{center}

\fi

Results in the case $\Safe = \uparrow \Safe$ and $\Bad = \downarrow \Bad$:


\begin{center}
\begin{tabular}{ | l | c | c | c | c |}
\hline  Hypothesis & strong compatibility ~ & $\uparrow post^*$ effective & both  \\ \hline
   SRP & Undecidable (Thm~\ref{srp up down}) & Decidable (Thm~\ref{post srp})  & Decidable (Thm~\ref{postcomputable})\\ \hline
   BSRP & Undecidable (Corollary~\ref{bsrp up down}) &  ??  & Decidable (Thm~\ref{ref ozkan}) \\ \hline
      kSRP & Undecidable (Corollary~\ref{bsrp up down}) & ?? & Decidable (Thm~\ref{ref ozkan}) \\ \hline
 \end{tabular}
\end{center}


Results in the case $\Safe = \downarrow \Safe$ and $\Bad = \uparrow \Bad$:



