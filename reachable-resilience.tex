



\section{Resilience from a single-state for WSTS}


%Uniform r
Resilience is a strong property since it implies that from bad there must exist a path to Safe. But, when one considers a system with an initial state $s_0$, it could be sufficient to ask that only from $post^*(s_0) \cap Bad$, there must exist a path to Safe. However, this condition seems more difficult since one also must decide whether a state $s \in Bad$ is reachable from $s_0$.
%
%\problemx{(I,J)-$k$-resilience problem for WSTS}
%{A state $s$ of a WSTS $(S,\rightarrow, \leq)$, an effective set $I$ (with a given basis), a set $J$.}
%{$\forall s' \in J ~ (s \rightarrow^* s') \implies \exists s'' \in I ~ s' \rightarrow^{\leq k} s''$ ?\newline}
%
%\problemx{general $k$-resilience problem for WSTS}
%{A state $s$ of a WSTS $(S,\rightarrow, \leq)$, an effective set $I$ (with a given basis).}
%{$\forall s'  ~ (s \rightarrow^* s') \implies \exists s'' \in I ~ s' \rightarrow^{\leq k} s''$ ?\newline}

Again, we first consider the case where the recovery time is bounded by a $k \in \N$.

\problemx{$k$-resilience problem for WSTS}
{A state $s$ of a WSTS $(S,\rightarrow, \leq)$, an upward-closed set $U$ with a given basis, a decidable downward-closed set $D$.}
{$D \longrightarrow^{\leq k} U$ ?  ou $D \cap post^*(s) \longrightarrow^{\leq k} U$ ?  \newline}
%
%  		{$\forall s' ~ (s \rightarrow^* s') \wedge s' \in D  \implies \exists s'' \in U ~ s' \rightarrow^{\leq k} s''$ ?\newline}
%
$D \longrightarrow^{\leq k} U$: $\forall s' \in D \exists s'' \in U ~ s' \rightarrow^{\leq k} s''$ ?\\
%
$D \cap post^*(s) \longrightarrow^{\leq k} U$ ?
%
$\forall s' ~ (s \rightarrow^* s') \wedge s' \in D  \implies \exists s'' \in U ~ s' \rightarrow^{\leq k} s''$ ?
 \alain{ça signifie pour tout $d \in D$ il existe un chemin de longueur $\leq k$ qui arrive dans un $u \in U$. c'est plus simple et plus général et ça ne demande moins de décider l'accessibilité....comment faire pour vérifier pour tous les (D,U) ? sinon ça dépend tjs de l'état intial $s$ et s'il faut vérifier pour tous les $s$....}


%ou plus simple : $D \rightarrow^{\leq k} U$ ? 


%Analogously, we formulate the resilience problem for WSTSs.

\problemx{bounded Resilience problem for WSTS}
{A state $s$ of a WSTS $W=(S,\rightarrow, \leq)$, an upward-closed set $U$ with a given basis, a decidable downward-closed set $D$.}
%{$\exists k \geq 0 ~ \forall s' \in D ~ s \rightarrow^* s' \implies \exists s'' \in U ~ s' \rightarrow^{\leq k} s''$ ?\newline}
{$\exists k \geq 0$ such that $W$ is $k$-resilient. ?\newline}

\problemx{Resilience problem for WSTS}
{A state $s$ of a WSTS $(S,\rightarrow, \leq)$, an upward-closed set $U$ with a given basis, a decidable downward-closed set $D$.}
{$ ~ \forall s' \in D ~ (s \rightarrow^* s') \implies \exists s'' \in U ~ s' \rightarrow^{*} s''$ ?\newline}

%\textcolor{blue}{
%justification des definitions: si on enlève la condition  an upward-closed set $U$ (extended resilience $U$ qq) reachability reduces to resilience (avec $U=\{x\}$ donc indecidable pour les modèles à reachability indécidable.
%}

\begin{remark} 
We require $U$ to be upward-closed, as, in the general case if we lift restrictions on $U$ and $D$, then reachability reduces itself to resilience. Indeed, the particular resilience problem where we take $D = \{s\}$ and $U = \{x\}$ is reachability of $x$ from $s$, and thus generally undecidable for WSTS. 
\end{remark}

\textcolor{blue}{
et encore: renforcer sur l'hypothèses sur I, J tous les deux clos par le haut+bas.
}

\textcolor{blue}{
compute the sets of $(I,J,k)$ st S is resilient.\\
max des I,J possibles \\
}






\begin{proposition}\label{reductions}
coverability and $t$-liveness are reduceable to resilience.
%	home state and reachability
\end{proposition}


\begin{proof}

\begin{itemize}
\item Let us choose $J=S= \downarrow S$ and $I=\uparrow x$ then resilience is equivalent to coverability of $x$ from $s$.
%\item $J=\{s\}$ and $I=\{x\}$ this particular resilience is reachability of $x$ from $s$.

\item $t$-liveness  can be expressed as the following formula
$ ~ \forall s' \in S ~ (s \rightarrow^* s') \implies \exists s'' \in U_t ~ s' \rightarrow^{*} s''$ 
where
$U_t=\uparrow \pred(t)$
is the upward closure of the preconditions to use transition $t$.  
The problem reduces itself to the resilience problem
where $J=S$ and $I = \uparrow \pred(t)$.

%\item the home state problem can be expressed as $\ldots$
\end{itemize}

%	\mathieu{Ne marche pas (un des deux sens est invalide imho)}
%  \textcolor{red}{  Let $J=\downarrow S = S$ and $I = \uparrow t$ where $t \in S$. Now resilience remains to the following property: $\forall s' \in S,  \exists s'' \in I, ~ s' \rightarrow^{*} s''$. Let us apply this property to LCS with $I = \uparrow (q,\epsilon) = (q, \Sigma^*)$. This property, $\forall s' \in S,  \exists w \in \Sigma^* $ such that $ ~ s' \rightarrow^{*} (q,w)$ expresses that there is an infinite run passing infinitely often through the control-state $q$; this LTL-property has been proved undecidable for Lossy Channel Systems (LCS) \cite{DBLP:conf/icalp/AbdullaJ94}.
% Since LCS are effective WSTS with strong monotony (with an adequate semantics), we conclude.}
%	\mathieu{Ne marche pas (un des deux sens est invalide imho)}
\end{proof}




% Since Reset Petri nets are also effective WSTS with strong monotony, we could think to use the undecidability of LTL for Reset Petri nets.

% and with effective basis of $post^*$ ???

% expliquer toutes les variantes de la résilience

% Let $(X,\leq)$ be a wqo and $U$ be an upward closed subset of $X$. We say that $U$ is \emph{computable} if there is an algorithm that computes a finite basis of $U$.


\begin{theorem}\cite{DBLP:conf/gg/Ozkan22,DBLP:journals/corr/abs-2108-00889}
{\sc bounded resilience} is decidable for WSTS $S$ with strong upward compatibility and such that $\uparrow post^*(S)$ is computable.
\end{theorem}

We may immediately generalyse this last result by strenghthenin to \emph{unbounded} resilience. The proof is essentially the same than the previous one.

\begin{corollary}\label{postcomputable}
{\sc resilience} is decidable for WSTS with strong upward compatibility and such that $\uparrow post^*(S)$ is computable.
\end{corollary}


\begin{proof}
From Fact~\ref{stop condition}, there exists $k \in \N$ such that
$\post^*(I) = I^{k_m}$. We can compute this $k_m$ by iteratively computing
$I^{k+1}$ from $I^k$, checking $I^{k+1}=I^k$, 
returning $k_m$ if that is the case.
Then, because $k_m$-resilience is decidable, 
checking $\uparrow post^*(s) \cap J \subseteq I^k = \pred^*(I)$ is,
and resilience is decidable.
% \textcolor{red}{à faire Mathieu}
\end{proof}

%\"Ozkan
\begin{proposition}[Proposition 1 in \cite{DBLP:conf/gg/Ozkan22}]
For finite-branching WSTS with strong upward compatibility, a basis of $\uparrow \post^*(s)$ is computable for every given state $s$ iff the downward-closed problem is decidable, i.e. given a state $s$ of a WSTS
% in the regarded class 
with strong upward compatibility 
and a decidable downward-closed set $J$, it can be decided whether $\exists s' \in J ~ s \to^* s'$. 
\end{proposition}

The characterization is used to show that Petri nets are $\post^*$-effective. It is well-known that Petri nets constitute WSTS with strong upward compatibility (Thm.~6.1 in \cite{DBLP:journals/tcs/FinkelS01}). Hence Resilience is decidable for Petri nets and VASS.
However, the hypothesis that $\uparrow \post^*$ is computable cannot be tested in the general WSTS framework.

\begin{proposition}
There exist classes of WSTSs with strong upward compatibility for which there don't exist an algorithm computing a basis of $\uparrow \post^*$.
\end{proposition}


\begin{proof}
Let us show that Reset VASS don't enjoy the property that $\uparrow \post^*$ is computable.
Suppose that one are able to compute a finite basis of $\uparrow post^*$ for Reset VASS. Then, one would be able to decide whether $0$ is reachable 
%in counter $i$ 
by examining if there is %some vector $v$ in the basis such that $v(i)=0$
$0$ in the basis%
. But reachability of $0$  %in a counter $i$ 
is undecidable for Reset VASS. 
% sont WSTS with strong compatibility. et l'accessibilité est undecidable.
%	$\uparrow post^*$ n'est pas calculable pour les LCS non plus car $\uparrow post^*= \uparrow \downarrow post^*$ n'est pas calculable ???
\end{proof}

%il existe des modèles où on peut calculer $\uparrow post^*$: inserted channel systems: on sait calculer $post^*$ qui est rationnel donc on sait calculer $\uparrow post^*$.

remarque si on enlève la strong monotony mais en gardant with effective basis of $\uparrow post^*$:


\begin{proposition}(à prouver)
 leur algorithme for {\sc bounded resilience} termine mais pas correct for WSTS with upward compatibility, not necessarly strong, and an effective basis of $\uparrow post^*$ ?
\end{proposition}

\begin{proof}
leur algorithme termine mais pas correct car l'intersection de post et J car alors le lemmme xxxx est faux (contre-exemple)
 \textcolor{red}{à faire Mathieu}
 
\mathieu{En train de chercher un contre exemple pour le corollaire du Lemme~\ref{Lemma intersection}.} 
 
\mathieu{

On considère un LCS, où on autorise à effacer des messages (lossyness) mais effacer un message prends une étape (contrairement à [ref], où ils font une sorte de cloture de effacer-transition-effacer). On n'a alors pas la strong compatibility: 

On prends: $(q, b) 	\to (q,bb)$ par exemple et $(q, baaaaaa) \geq (q,b) $, alors
$(s, baaaaaa) \to^* (s,b) \to (s,bb)$
MAIS

C'est pas "en une étape".

(On considère -il faut le mentionner/vérifier- que on ne peut effaçer que d'un coté ? et pas au milieu?)

Dans ce cas là, du coup, $pred(\uparrow \{(q,b)\}$ n'est pas clos par le haut. \\

Revenons au Corollaire

On voudrais avoir

$post^* (s) \cap J \subseteq I^k$

MAIS pas

$ (\uparrow post^* (s)) \cap J \subseteq I^k$ \\


Ça serait un cas où leurs algorithme vas dire 
"On n'a pas $(\uparrow post^* (s)) \cap J \subseteq I^k$ donc on n'a pas la $k$-résilience "

Alors qu'on aurait la $k$-résilience (i.e. $post^* (s) \cap J \subseteq I^k$)

Pour $I^k$, le contre exemple prendrait du coup $pred(\uparrow \{ (q,b)\})$ qui n'est pas clos par le haut.

On voudrait

$post^*(s) \cap J \subseteq pred(\uparrow \{ (q,b)\})$


Je propose de prendre $J$ = ensemble des configurations je pense le truc ici doit venir de
$\uparrow post^* (s)$ being too big
donc autant pas se restreindre du tout sur $J$


$post^*(s) \subseteq pred(\uparrow \{ (q,b)\})$

Si $s=(q,b)$ par exemple, et que on peut aller de $q$ vers $q$ en écrivant un 
$b$ ? et évidement en perdant le dernier message (parce qu'on est lossy). Et c'est tout, disons.

Déjà par lossiness $(q,\epsilon)$ serait dans $post^*((q,b))$.

Et aussi dans $pred(\uparrow \{ (q,b)\})$ puisque $(q,\epsilon) \to (q,b)$.

Donc on aurait:

$post^*((q,b))  \subseteq pred(\uparrow \{ (q,b)\})$

Et on aurait $(q,aaaaaa)$ dans 
$\uparrow post^* ((q,b))$ puisque $(q,\epsilon)$ est dans $post^* ((q,b))$.

Hors $(q,aaaaa)$ est pas dans $pred(\uparrow \{ (q,b)\})$
(même si il est dans $pred^*(\uparrow \{ (q,b)\})$).



}
 
 
 
 
\end{proof}

\begin{proposition}(à prouver)
{\sc bounded resilience} is undecidable for WSTS with upward compatibility withd an effective basis of $post^*$.
\end{proposition}

\begin{proof}
 \textcolor{red}{pas certain ! à faire Mathieu}
\end{proof}

%
%si on garde strong mais on enlève effective basis of $post^*$:
%
%\begin{proposition}(à prouver)
%{\sc $k$-resilience}, hence {\sc bounded resilience}, is undecidable for effective WSTS with strong upward compatibility.
%\end{proposition}
%
%\begin{proof}
%pas de effective basis of $post^*$. leur algorithme n'est plus un algo. ex reset PN ?  \textcolor{red}{pas certain ! à faire Mathieu}
%\end{proof}


\begin{theorem}
{\sc Resilience} is decidable for (not necessarily strong) WSTS with effective $\uparrow$ $\post^*$ basis.
\end{theorem}


\begin{proof}
Let $B_p$ be a basis of $\uparrow \post^*(s)$, $B_I$ a basis of $I$
and $J$ decidable downward-closed.
By applying Lemma~\ref{Lemma intersection} twice, we obtain

$$ post^*(s) \cap J \subseteq \pred^*(I) \equiv B_p \cap J \subseteq \pred^*(I)$$


Since $B_p$ is finite and $J$ is decidable, we can directly compute $ B_p \cap J$.
% For every element of $ B_p \cap J$, checking that it is in $\pred^*(I)$ is the same as checking coverability of I, and thus decidable.
We can compute a basis of $\pred^*(I)$ from $B_I$, and hence check that $B_p \cap J \setminus \pred^*(I) = \emptyset$. 
\end{proof}

% In the proof of Thm. 1, it was crucial that we have strong compatibility. This approach does not work
% for WSTSs in general. We loose precision when we only demand compatibility. Thus, we conjecture
% that both resilience problems are undecidable for WSTSs in general, but this question remains still open.
%
% Well, maybe their version of resilience but resilience with * seems fine.
%

Unfortunately, resilience is undecidable for (general) WSTS even with strong monotony.
This stems from the fact that $t$-liveness is undecidable for reset Petri-Nets.


\begin{theorem}[Adjusting Theorem~5.5 from \cite{peterson1981petri}]
The  Reset Petri Net zero reachability problem can be reduced to the liveness problem.
\end{theorem}

\begin{proof}
If we wish to determine if $0 \in R(C_1, \mu_1)$ in any
reset Petri net $C_1 = (P_1, T_1, I_1, O_1, R_1 )$ (where $R_1(t) =$ places that are reset when t fires off), with initial marking $\mu_1$, we construct a
reset petri net $C_2 = (P_2, T_2, I_2, O_2, R_2 )$ with initial marking
$\mu_2$ which is live if and only if the zero marking is not reachable from $\mu_1$.

The reset PN $C_2$ is constructed from $C_1$ by the addition of two places $r_1$ and $r_2$
and $|P_1| +2$ transitions $s$, $\{ s_p | p \in P_1 \}$ and $s'$.

We first modify all transitions of $T_1$ to include $r_1$ as both an input and an output.
The initial marking $\mu_2$ will include a token in $r_1$. 


The place $r_1$ is a 'run' place; as long as the token remains in $r_1$ the transitions of $T_1$ can fire normally.
Thus any marking which is reachable in the places of $P_1$ is also reachable in $C_2$.



Transition $s$ is defined to have $r_1$ as an input and a null output.
This allows the token in $r_1$ to be removed, disabling all transitions in $T_1$
and "freezing" the marking of $P_1$.

Note that all transitions of $T_1$ are in conflict and, by construction if not by definition, no more than one transition can fire at a time.

The place $r_1$ and transition $s$ allow the net $C_1$ to reach any reachable marking and then for $s$ to fire and freeze the net at that marking.

Now we need to see if that marking is zero.

We introduce a new place $r_2$ and new transitions $s_p$ which have $p$ as input and $r_2$ as output.

If $p$ can ever become zero, $s_p$ is not live. The entire net is dead if
$s$ fires in the marking zero.

If we can always have some $p_i$ not be zero, then we can always fire some $s_{p_i}$, putting a token in $r_2$.

In this case we must put a token back in $r_1$ and assure that all transitions in $C_2$ are live.
We must be sure that $C_2$ is live even if $C_1$ is {\em not} live.
This is accomplished by a transition $s'$ which  "floods" the net $C_2$ with tokens, assuring that every transition is live if a token is ever put in $r_2$.
Transition $s'$ has $r_2$ as its input and every place of $C_2$ as output

Now if $0$ is reachable
in $R(C_1, \mu_1)$ 
then the net $C_2$ can also reach this marking
in the place of $P_1$ by executing the same sequence of transition firings.
Then $s$ can fire, freezing the $C_1$ subset.
Since $mu(p_i) = 0$ for all $p_i \in P_1$, no transition $s_{p_i}$ with $p_i \in P_1$ can fire, and $C_2$ is dead.
Thus if $0$ is reachable then $C_2$ is not live.

Conversey, if $C_2$ is not live, then a marking $\mu$ must be reachable in which 
$\mu(r_2) = 0$ and there is no reachable state in which $r_2$ has a token
(in particular, since we do not allow token removal from $r_2$, the marking $\mu$ must be reached in a sequence of transitions that do not place any token in $r_2$.)
% (Rappel: tant que il y a un token quelque part, on peut mettre un token en r_2
% et r_2 commence sans token)
If $r_2$ has no token and cannot get any, then the transitions $s_p$ are not live, and the markings of $p$ for $p \in P_1$ must be zero. 
Thus if $C_2$ is not live then a marking is reachable in which the marking of each $p$ in $P_1$ is zero. 


\end{proof}

\begin{corollary}
Reset Petri Net liveness is undecidable 
\end{corollary}

\begin{corollary}
Reset Petri Net $t$-liveness is undecidable 
\end{corollary}


Since $t$-liveness is undecidable for reset Petri-Nets that are WSTS with strong monotony, from Proposition \ref{reductions},  we deduce that resilience is undecidable for WSTS with strong monotony.

\begin{theorem}
{\sc resilience} is undecidable for WSTS with strong monotony.
\end{theorem}






\begin{proposition}
In WSTS with strong monotony, the bounded resilience problem is
reducible to the $k$-resilience problem.
\end{proposition}

\begin{proof}
strong WSTS
$\pred^*(I) = \pred^{k}(I)$
Check all $k$ until $\pred^*(I) = \pred^{k}(I)$.
\end{proof}

\begin{proposition}
In WSTS with strong monotony, the bounded resilience problem is
equivalent to the resilience problem.
\end{proposition}

\begin{proof}
strong WSTS

$\pred^*(I) = \pred^{k}(I)$

\end{proof}

\begin{corollary}
Bounded Resilience is undecidable for WSTS with strong monotony.
\end{corollary}

\begin{proof} 
  From
  cf
the two  theorems above. 
\end{proof}



Let us deduce that effective basis of $post^*$ ex reset PN ?

Since VASS are WSTS with strong upward compatibility and since there is an algorithm that computes a finite basis of  $\uparrow post^*(S)$, \cite{DBLP:conf/gg/Ozkan22} deduced that bounded resilience is decidable for VASS, hence resilience is decidable for VASS. \alain{resilience decidable for LCS ? LCS sont-ils strong (quelle sémantique ?) et $\uparrow post^*(S)$ is computable ?}

We may consider context-free grammars.

\begin{corollary}
{\sc resilience} is decidable for finite automata and context-free grammars (seen as transition systems)
\end{corollary}

\begin{proof}

  \textcolor{red}{à faire Mathieu}
Context-free grammars are WSTS \cite{DBLP:journals/tcs/FinkelS01} \alain{monotony ne semble pas strong...}avec $\uparrow post^*$ calculable (because the reachability set of a pushdown machine is an effectively computable regular language) ???? resilience decidable.
\end{proof}


\cite{DBLP:journals/ipl/BouajjaniEFMRWW00}

